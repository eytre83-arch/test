  \item $88+76$ ifodani hisoblang.\\
A) 164\\
B) 124\\
C) 111\\
D) 156
  \item $43+65$ ifodani hisoblang.\\
A) 144\\
B) 135\\
C) 108\\
D) 106
  \item $18+66$ ifodani hisoblang.\\
A) 84\\
B) 74\\
C) 171\\
D) 116
  \item 87+77 ifodani hisoblang.\\
A) 166\\
B) 124\\
C) 111\\
D) 164
  \item $88+14$ ifodani hisoblang.\\
A) 104\\
B) 102\\
C) 111\\
D) 176
  \item $17+94$ ifodani hisoblang.\\
A) 164\\
B) 124\\
C) 111\\
D) 156
  \item $98+26$ ifodani hisoblang.\\
A) 164\\
B) 124\\
C) 111\\
D) 156
  \item $88+23$ ifodani hisoblang.\\
A) 164\\
B) 124\\
C) 111\\
D) 156
  \item $97+59$ ifodani hisoblang.\\
A) 164\\
B) 124\\
C) 111\\
D) 156
  \item $85+39$ ifodani hisoblang.\\
A) 164\\
B) 124\\
C) 111\\
D) 156
  \item $188+276$ ifodani hisoblang.\\
A) 464\\
B) 455\\
C) 312\\
D) 346
  \item $413+265$ ifodani hisoblang.\\
A) 644\\
B) 435\\
C) 678\\
D) 106
  \item $148+616$ ifodani hisoblang.\\
A) 684\\
B) 74\\
C) 764\\
D) 116
  \item $817+177$ ifodani hisoblang.\\
A) 966\\
B) 994\\
C) 711\\
D) 964
  \item $818+214$ ifodani hisoblang.\\
A) 1032 B) 1732 C) 1121 D) 1176
  \item $117+194$ ifodani hisoblang.\\
A) 364\\
B) 224\\
C) 311\\
D) 156
  \item $298+226$ ifodani hisoblang.\\
A) 164\\
B) 524\\
C) 111\\
D) 156
  \item $817+213$ ifodani hisoblang.\\
A) 1030\\
B) 1024\\
C) 1511\\
D) 1524
  \item $197+159$ ifodani hisoblang.\\
A) 364\\
B) 324\\
C) 311\\
D) 356
  \item $185+319$ ifodani hisoblang.\\
A) 164\\
B) 504\\
C) 111\\
D) 156
  \item $1188+2276$ ifodani hisoblang.\\
A) 3464\\
B) 3124\\
C) 3111\\
D) 3156
  \item 1143+4565 ifodani hisoblang.\\
A) 5144\\
B) 5135\\
C) 5708\\
D) 5106
  \item 4418+4566 ifodani hisoblang.\\
A) 8984\\
B) 8874\\
C) 8171\\
D) 8116
  \item $7787+1277$ ifodani hisoblang.\\
A) 9166\\
B) 9124\\
C) 9111\\
D) 9064
  \item $4188+4114$ ifodani hisoblang.\\
A) 8104\\
B) 8302\\
C) 8111\\
D) 8176
  \item 1417+2194 ifodani hisoblang.\\
A) 3164\\
B) 3124\\
C) 3611\\
D) 3156
  \item $4198+5426$ ifodani hisoblang.\\
A) 9164\\
B) 9624\\
C) 9111\\
D) 9156
  \item $4288+4523$ ifodani hisoblang.\\
A) 8164\\
B) 8124\\
C) 8811\\
D) 8156
  \item 1497+1159 ifodani hisoblang.\\
A) 3164\\
B) 2124\\
C) 2111\\
D) 2656
  \item $1485+5739$ ifodani hisoblang.\\
A) 6164\\
B) 7224\\
C) 8111\\
D) 7156
  \item $18,8+27,6$ ifodani hisoblang.\\
A) 46,4\\
B) 45,5\\
C) 31,2\\
D) 34,6\\
  \item $41,3+26,5$ ifodani hisoblang.\\
A) 64,4\\
B) 43,5\\
C) 67,8\\
D) 10,6
  \item $14,8+61,6$ ifodani hisoblang.\\
A) 68,4\\
B) 74\\
C) 76,4\\
D) 11,6
  \item $81,8+21,4$ ifodani hisoblang.\\
A) 103,2\\
B) 173,2\\
C) 112,1\\
D) 117,6
  \item $11,7+19,4$ ifodani hisoblang.\\
A) 36,4\\
B) 22,4\\
C) 31,1\\
D) 15,6
  \item $29,8+22,6$ ifodani hisoblang.\\
A) 16,4\\
B) 52,4\\
C) 11,1\\
D) 15,6
  \item $81,7+21,3$ ifodani hisoblang.\\
A) 103\\
B) 102\\
C) 151\\
D) 152
  \item $19,7+15,9$ ifodani hisoblang.\\
A) 364\\
B) 32,4\\
C) 311\\
D) 35,6
  \item $18,5+31,9$ ifodani hisoblang.\\
A) 164\\
B) 50,4\\
C) 111\\
D) 15,6
  \item 1376-1248 ifodani hisoblang.\\
A) 128\\
B) 130\\
C) 311\\
D) 142
  \item 1536-1287 ifodani hisoblang.\\
A) 514\\
B) 135\\
C) 249\\
D) 106
  \item 1714-1699 ifodani hisoblang.\\
A) 15\\
B) 74\\
C) 71\\
D) 16
  \item $7314-5978$ ifodani hisoblang.\\
A) 1166\\
B) 1124\\
C) 2111\\
D) 1336
  \item 6111-5987 ifodani hisoblang.\\
A) 104\\
B) 124\\
C) 111\\
D) 176
  \item 5009-4716 ifodani hisoblang.\\
A) 164\\
B) 293\\
C) 211\\
D) 256
  \item 7111-6554 ifodani hisoblang.\\
A) 564\\
B) 524\\
C) 557\\
D) 556
  \item 1779-1596 ifodani hisoblang.\\
A) 164\\
B) 124\\
C) 183\\
D) 156
  \item 7465-5846 ifodani hisoblang.\\
A) 1164\\
B) 2124\\
C) 2111\\
D) 1619
  \item $7414 \cdot 6512$ ifodani hisoblang.\\
A) 164\\
B) 902\\
C) 811\\
D) 856
  \item $1376 \cdot 1248$ ifodani hisoblang.\\
A) 1717248\\
B) 3124244\\
C) 3111757\\
D) 1156588
  \item $1536 \cdot 1287$ ifodani hisoblang.\\
A) 51475745\\
B) 13575577\\
C) 1976832\\
D) 10644454
  \item $1714 \cdot 1699$ ifodani hisoblang.\\
A) 2912086\\
B) 7445647\\
C) 7866441\\
D) 1675455
  \item $7314 \cdot 5978$ ifodani hisoblang.\\
A) 11754466\\
B) 11654424\\
C) 21789511\\
D) 43723092
  \item $6111 \cdot 5987$ ifodani hisoblang.\\
A) 76575104\\
B) 36586557\\
C) 15548511\\
D) 85577176
  \item $5009 \cdot 4716$ ifodani hisoblang.\\
A) 44557164\\
B) 23622444\\
C) 45447211\\
D) 25655774
  \item 7111•6554 ifodani hisoblang.\\
A) 45555564\\
B) 52455774\\
C) 46605494\\
D) 55611257
  \item $1779 \cdot 1596$ ifodani hisoblang.\\
A) 2445164\\
B) 1244456\\
C) 2839284\\
D) 1567896
  \item $7465 \cdot 5846$ ifodani hisoblang.\\
A) 47581164\\
B) 21245445\\
C) 21114578\\
D) 43640390
  \item $7414 \cdot 6512$ ifodani hisoblang.\\
A) 45228164\\
B) 48279968\\
C) 81175547\\
D) 85644514
62. 264966:559 ifodani hisoblang.\\
A) 474\\
B) 372\\
C) 287\\
D) 175\\
63. 62415:365 ifodani hisoblang.\\
A) 171\\
B) 312\\
C) 287\\
D) 115\\
64. 37375:325 ifodani hisoblang.\\
A) 171\\
B) 312\\
C) 287\\
D) 115\\
65. 128232:411 ifodani hisoblang.\\
A) 171\\
B) 312\\
C) 287\\
D) 115\\
66. 70794:414 ifodani hisoblang.\\
A) 171\\
B) 312\\
C) 287\\
D) 115\\
67. 90405:315 ifodani hisoblang.\\
A) 171\\
B) 312\\
C) 287\\
D) 115\\
68. 97344:312 ifodani hisoblang.\\
A) 171\\
B) 312\\
C) 287\\
D) 115\\
69. 59225:515 ifodani hisoblang.\\
A) 171\\
B) 312\\
C) 287\\
D) 115\\
70. 29241:171 ifodani hisoblang.\\
A) 171\\
B) 312\\
C) 287\\
D) 115
  \item 112:22.4 ifodani hisoblang.\\
A) 5\\
B) 8\\
C) 2\\
D) 12
  \item $89,6: 11,2$ ifodani hisoblang.\\
A) 5\\
B) 8\\
C) 2\\
D) 12
  \item $67,2: 33,6$ ifodani hisoblang.\\
A) 5\\
B) 8\\
C) 2\\
D) 12
  \item $134,4: 11,2$ ifodani hisoblang.\\
A) 5\\
B) 8\\
C) 2\\
D) 12
  \item 89,6:22.4 ifodani hisoblang.\\
A) 4\\
B) 6\\
C) 1\\
D) 12
  \item 134,4:22,4 ifodani hisoblang.\\
A) 5\\
B) 8\\
C) 6\\
D) 12
  \item 56:22,4 ifodani hisoblang.\\
A) 5,2\\
B) 8\\
C) 2,5\\
D) 12
  \item $44,8: 5,6$ ifodani hisoblang.\\
A) 5\\
B) 8\\
C) 2\\
D) 12
  \item 179.2:11.2 ifodani hisoblang.\\
A) 5\\
B) 8\\
C) 16\\
D) 12
  \item $224: 11,2$ ifodani hisoblang.\\
A) 5\\
B) 8\\
C) 20\\
D) 12
  \item $120,4 \cdot 10^{22:} 6,02 \cdot 10^{23}$ ifodani hisoblang.\\
A) 5\\
B) 8\\
C) 2\\
D) 12
  \item $602 \cdot 10^{22:} 6,02 \cdot 10^{23}$ ifodani hisoblang.\\
A) 15\\
B) 10\\
C) 2\\
D) 12
  \item $722,4 \cdot 10^{22:} 6,02 \cdot 10^{23}$ ifodani hisoblang.\\
A) 5\\
B) 8\\
C) 2 
D) 12
  \item $90,3 \cdot 10^{23}: 6,02 \cdot 10^{23}$ ifodani hisoblang.\\
A) 15\\
B) 8\\
C) 2\\
D) 12
  \item $180,6 \cdot 10^{22:} 6,02 \cdot 10^{23}$ ifodani hisoblang.\\
A) 4\\
B) 6\\
C) 3\\
D) 12
  \item $120,4 \cdot 10^{23:} 6,02 \cdot 10^{23}$ ifodani hisoblang.\\
A) 5\\
B) 8\\
C) 60\\
D) 20
  \item $240,8 \cdot 10^{22}: 6,02 \cdot 10^{23}$ ifodani hisoblang.\\
A) 5,2\\
B) 8\\
C) 4\\
D) 12
  \item $30,1 \cdot 10^{23}: 6,02 \cdot 10^{23}$ ifodani hisoblang.\\
A) 5\\
B) 8\\
C) 2\\ 
D) 12
  \item $963,2 \cdot 10^{22}: 6,02 \cdot 10^{23}$ ifodani hisoblang.\\
A) 5\\
B) 8\\
C) 16\\
D) 12
  \item $481,6 \cdot 10^{22:} 6,02 \cdot 10^{23}$ ifodani hisoblang.\\
A) 5\\
B) 8\\
C) 20\\
D) 12
 \item Quyidagi proporsiyani hisoblang va noma'lum $X$ ning qiymatini toping:
\[
\frac{54}{X} = \frac{27}{6}
\]
A) 5 \\ 
B) 8 \\ 
C) 24 \\ 
D) 12
\item Quyidagi proporsiyani hisoblang va noma'lum $X$ ning qiymatini toping:
\[
\frac{33{,}6}{X} = \frac{11{,}2}{8}
\]
A) 5 \\ 
B) 8 \\ 
C) 24 \\ 
D) 12
\item Quyidagi proporsiyani hisoblang va noma'lum $X$ ning qiymatini toping:
\[
\frac{89{,}6}{X} = \frac{11{,}2}{1}
\]
A) 5 \\ 
B) 8 \\ 
C) 24 \\ 
D) 12
\item Quyidagi proporsiyani hisoblang va noma'lum $X$ ning qiymatini toping:
\[
\frac{6{,}02}{X} = \frac{12{,}04}{12}
\]
A) 5 \\ 
B) 8 \\ 
C) 2 \\ 
D) 6
\item Quyidagi proporsiyani hisoblang va noma'lum $X$ ning qiymatini toping:
\[
\frac{44}{X} = \frac{11}{12}
\]
A) 14 \\ 
B) 48 \\ 
C) 24 \\ 
D) 12
\item Quyidagi proporsiyani hisoblang va noma'lum $X$ ning qiymatini toping:
\[
\frac{33}{X} = \frac{66}{12}
\]
A) 5 \\ 
B) 8 \\ 
C) 2 \\ 
D) 6
\item Quyidagi proporsiyani hisoblang va noma'lum $X$ ning qiymatini toping:
\[
\frac{35}{X} = \frac{70}{16}
\]
A) 5 \\ 
B) 8 \\ 
C) 24 \\ 
D) 12
\item Quyidagi proporsiyani hisoblang va noma'lum $X$ ning qiymatini toping:
\[
\frac{10}{X} = \frac{5}{12}
\]
A) 5 \\ 
B) 8 \\ 
C) 24 \\ 
D) 12
  \item Quyidagi ifodani hisoblang.\\
$\frac{10 \cdot 8,314 \cdot 50}{10 \cdot 83.14}$\\
A) 6\\
B) 8\\
C) 5\\
D) 2
102. Quyidagi ifodani hisoblang.\\
$\frac{4 \cdot 22,4 \cdot 30}{11,2 \cdot 40}$\\
A) 6\\
B) 8\\
C) 5\\
D) 2\\
103. Quyidagi ifodani hisoblang.\\
$\frac{10 \cdot 80 \cdot 5}{10 \cdot 8}$\\
A) 6\\
B) 8\\
C) 50\\
D) 2\\
104. Quyidagi ifodani hisoblang.\\
$\frac{10 \cdot 8,314 \cdot 70}{35 \cdot 83.14}$\\
A) 6\\
B) 8\\
C) 5\\
D) 2\\
105. Quyidagi ifodani hisoblang.\\
$\frac{10 \cdot 8,314 \cdot 45}{90 \cdot 83.14}$\\
A) 6\\
B) 8\\
C) 0,5\\
D) 0,2\\
106. Quyidagi ifodani hisoblang.\\
$\frac{10 \cdot 22,4 \cdot 60}{10 \cdot 120}$\\
A) 11,2\\
B) 89,6\\
C) 5\\
D) 2\\
107. Quyidagi ifodani hisoblang.\\
$\frac{10 \cdot 8 \cdot 12,04}{10 \cdot 6,02}$\\
A) 16\\
B) 18\\
C) 15\\
D) 12\\
108. Quyidagi ifodani hisoblang.\\
$\frac{5 \cdot 16,6 \cdot 50}{10 \cdot 8,3}$\\
A) 60\\
B) 80\\
C) 50\\
D) 20\\
109. Quyidagi ifodani hisoblang.\\
$\frac{10 \cdot 49 \cdot 25}{10 \cdot 24,5}$\\
A) 60\\
B) 80\\
C) 50\\
D) 20\\
110. Quyidagi ifodani hisoblang.\\
$\frac{10 \cdot 2 \cdot 101,3}{10 \cdot 202,6}$\\
A) 1\\
B) 8\\
C) 5\\
D) 2
  \item Quyidagi tenglamani hisoblang va noma'lum X ning qiymatini toping.\\
$20-\mathrm{X}=15$\\
A) 1\\
B) 8\\
C) 5\\
D) 2\\
  \item Quyidagi tenglamani hisoblang va noma'lum X ning qiymatini toping. $18-\mathrm{X}=10$\\
A) 1\\
B) 8\\
C) 5\\
D) 2
  \item Quyidagi tenglamani hisoblang va noma'lum X ning qiymatini toping. $16 \cdot \mathrm{X}=15$\\
A) 1\\
B) 8\\
C) 5\\
D) 2
  \item Quyidagi tenglamani hisoblang va noma'lum X ning qiymatini toping. $18-X=16$\\
A) 1\\
B) 8\\
C) 5\\
D) 2
  \item Quyidagi tenglamani hisoblang va noma'lum X ning qiymatini toping. $20-\mathrm{X}=12$\\
A) 1\\
B) 8\\
C) 5\\
D) 2
  \item Quyidagi tenglamani hisoblang va noma'lum X ning qiymatini toping. $23-\mathrm{X}=15$\\
A) 1\\
B) 8\\
C) 5\\
D) 2
  \item Quyidagi tenglamani hisoblang va noma'lum X ning qiymatini toping. $25-\mathrm{X}=15$\\
A) 10\\
B) 8\\
C) 50\\
D) 2
  \item Quyidagi tenglamani hisoblang va noma'lum X ning qiymatini toping. $28-\mathrm{X}=27$\\
A) 1\\
B) 8\\
C) 5\\
D) 2
  \item Quyidagi tenglamani hisoblang v̇a noma'lum X ning qiymatini toping. $20-\mathrm{X}=14$\\
A) 1\\
B) 8\\
C) 6\\
D) 2
  \item Quyidagi tenglamani hisoblang va noma'lum X ning qiymatini toping. $27-\mathrm{X}=15$\\
A) 12\\
B) 8\\
C) 5\\
D) 2
  \item Quyidagi tenglamani hisoblang va noma'lum X ning qiymatini toping. $20-\mathrm{X}=15+\mathrm{X}$\\
A) 1\\
B) 1,8\\
C) 2,5\\
D) 2
  \item Quyidagi tenglamani hisoblang va noma'lum X ning qiymatini toping. $18-\mathrm{X}=10+\mathrm{X}$\\
A) 1\\
B) 4\\
C) 5\\
D) 2
  \item Quyidagi tenglamani hisoblang va noma'lum X ning qiymatini toping. $16-X=15+X$\\
A) 0,1\\
B) 8\\
C) 0,5\\
D) 2
  \item Quyidagi tenglamani hisoblang va noma'lum X ning qiymatini toping.\\
$18-X=16+X$\\
A) 1\\
B) 8\\
C) 5\\
D) 2
  \item Quyidagi tenglamani hisoblang va noma'lum X ning qiymatini toping.\\
$20 \cdot \mathrm{X}=12+\mathrm{X}$\\
A) 1\\
B) 6\\
C) 5\\
D) 4
  \item Quyidagi tenglamani hisoblang va noma'lum X ning qiymatini toping. $23-\mathrm{X}=15+\mathrm{X}$\\
A) 1\\
B) 6\\
C) 5\\
D) 4
  \item Quyidagi tenglamani hisoblang va noma'lum X ning qiymatini toping. $25-\mathrm{X}=15+\mathrm{X}$\\
A) 10\\
B) 8\\
C) 5\\
D) 2
  \item Quyidagi tenglamani hisoblang va noma'lum X ning qiymatini toping. $28-\mathrm{X}=27+\mathrm{X}$\\
A) 0,1\\
B) 8\\
C) 0,5\\
D) 2
  \item Quyidagi tenglamani hisoblang va noma'lum X ning qiymatini toping. $20-\mathrm{X}=14+\mathrm{X}$\\
A) 1\\
B) 8\\
C) 3\\
D) 2
  \item Quyidagi tenglamani hisoblang va noma'lum X ning qiymatini toping. $27-\mathrm{X}=15+\mathrm{X}$\\
A) 12\\
B) 8\\
C) 5\\
D) 6
  \item Quyidagi tenglamani hisoblang va noma'lum X ning qiymatini toping.\\
A) 10\\
B) 8\\
C) 5\\
D) 6
  \item Quyidagi tenglamani hisoblang va noma'lum X ning qiymatini toping.\\
$1 \cdots--\cdots-\cdots(5+X)$\\
$10-\cdots-\cdots-(104+\mathrm{X})$\\
A) 10\\
B) 8\\
C) 5\\
D) 6
  \item Quyidagi tenglamani hisoblang va noma'lum X ning qiymatini toping.\\
$4-\cdots-\cdots-\cdots-(5+\mathrm{X})$\\
$12 \cdots \cdots \cdots(20+\mathrm{X})$\\
A) 10\\
B) 8,2\\
C) 2,5\\
D) 6
  \item Quyidagi tenglamani hisoblang va noma'lum $X$ ning qiymatini toping.\\
$8-\cdots-\cdots--(5+X)$\\
$24-\cdots-\cdots-(27+X)$\\
A) 10\\
B) 8\\
C) 5\\
D) 6
  \item Quyidagi tenglamani hisoblang va noma'lum X ning qiymatini toping.\\
$27-\cdots-\cdots-\cdots-(5+X)$\\
$54-\cdots-\cdots-(15+\mathrm{X})$\\
A) 10\\
B) 8\\
C) 5\\
D) 6
  \item Quyidagi tenglamani hisoblang va noma'lum X ning qiymatini toping.\\
$12--------(5+\mathrm{X})$\\
$36-\cdots-\cdots-(25+X)$\\
A) 10\\
B) 8\\
C) 5\\
D) 6
  \item Quyidagi tenglamani hisoblang va noma'lum X ning qiymatini toping.\\
$4-\cdots-\cdots-\cdots(5+X)$\\
$20-\cdots-\cdots-(49+X)$\\
A) 10\\
B) 8\\
C) 5\\
D) 6
  \item Quyidagi tenglamani hisoblang va noma'lum X ning qiymatini toping. $2=\frac{10+\mathrm{x}}{10}$\\
A) 10\\
B) 8\\
C) 5\\
D) 6
  \item Quyidagi tenglamani hisoblang va noma'lum X ning qiymatini toping.\\
$3=\frac{12+x}{6}$\\
A) 10\\
B) 8\\
C) 5\\
D) 6
  \item Quyidagi tenglamani hisoblang va noma'lum X ning qiymatini toping.\\
$2=\frac{10+x}{8}$\\
A) 10\\
B) 8\\
C) 5\\
D) 6
  \item Quyidagi tenglamani hisoblang va noma'lum X ning qiymatini toping.\\
$5=\frac{10+x}{4}$\\
A) 10\\
B) 8\\
C) 5\\
D) 6
  \item Quyidagi tenglamani hisoblang va noma'lum X ning qiymatini toping.\\
$4=\frac{35+x}{10}$\\
A) 10\\
B) 8\\
C) 5\\
D) 6
  \item Quyidagi tenglamani hisoblang va noma'lum X ning qiymatini toping.\\
$6=\frac{30+x}{6}$\\
A) 10\\
B) 8\\
C) 5\\
D) 6
  \item Quyidagi tenglamani hisoblang va noma'lum X ning qiymatini toping.\\
$4=\frac{24+x}{8}$\\
A) 10\\
B) 8\\
C) 5\\
D) 6
  \item Quyidagi tenglamani hisoblang va noma'lum $X$ ning qiymatini toping. $3=\frac{16+x}{8}$\\
A) 10\\
B) 8\\
C) 5\\
D) 6
  \item Quyidagi tenglamani hisoblang va noma'lum X ning qiymatini toping.\\
$6=\frac{54+x}{10}$\\
A) 10\\
B) 8\\
C) 5\\
D) 6
  \item Quyidagi tenglamani hisoblang va noma'lum X ning qiymatini toping.\\
$8=\frac{40+x}{6}$\\
A) 10\\
B) 8\\
C) 5\\
D) 6
  \item Quyidagi tenglamani hisoblang va noma'lum X ning qiymatini toping.\\
$5=\frac{20+\mathrm{x}}{10-\mathrm{x}}$\\
A) 10\\
B) 8\\
C) 5\\
D) 6\\
  \item Quyidagi tenglamani hisoblang va noma'lum X ning qiymatini toping.\\
$3=\frac{12+\mathrm{x}}{12-\mathrm{x}}$\\
A) 10\\
B) 8\\
C) 5\\
D) 6
  \item Quyidagi tenglamani hisoblang va noma'lum X ning qiymatini toping.\\
$2=\frac{6+x}{12-x}$\\
A) 10\\
B) 8\\
C) 5\\
D) 6
  \item Quyidagi tenglamani hisoblang va noma'lum $X$ ning qiymatini toping.\\
$5=\frac{10+\mathrm{x}}{14-\mathrm{x}}$\\
A) 10\\
B) 8\\
C) 5\\
D) 6
  \item Quyidagi tenglamani hisoblang va noma'lum X ning qiymatini toping.\\
$4=\frac{35+x}{10-x}$\\
A) 10\\
B) 8\\
C) 1\\
D) 6
  \item Quyidagi tenglamani hisoblang va noma'lum $X$ ning qiymatini toping.\\
$6=\frac{30+\mathrm{x}}{12-\mathrm{x}}$\\
A) 10\\
B) 8\\
C) 5\\
D) 6
  \item Quyidagi tenglamani hisoblang va noma'lum X ning qiymatini toping.\\
$4=\frac{24+x}{16-x}$\\
A) 10\\
B) 8\\
C) 5\\
D) 6
  \item Quyidagi tenglamani hisoblang va noma'lum X ning qiymatini toping.\\
$3=\frac{25+x}{15-x}$\\
A) 10\\
B) 8\\
C) 5\\
D) 6
  \item Quyidagi tenglamani hisoblang va noma'lum X ning qiymatini toping.\\
$6=\frac{24+x}{11-x}$\\
A) 10\\
B) 8\\
C) 5\\
D) 6
  \item Quyidagi tenglamani hisoblang va noma'lum X ning qiymatini toping. $10=\frac{72+x}{16-x}$\\
A) 10\\
B) 8\\
C) 5\\
D) 6
  \item Quyidagi tenglamani hisoblang va noma'lum X va Y ning qiymatini toping. $\left\{\begin{array}{c}x+y=5 \\ 2 x+3 y=13\end{array}\right.$\\
A) $2 ; 3$\\
B) $4 ; 5$\\
C) $5 ; 3$\\
D) $3 ; 6$
  \item Quyidagi tenglamani hisoblang va noma'lum X va Y ning qiymatini toping. $\left\{\begin{array}{c}2 x+y=13 \\ 2 x+3 y=23\end{array}\right.$\\
A) $2 ; 3$\\
B) $4 ; 5$\\
C) $5: 3$\\
D) 3;6
  \item Quyidagi tenglamani hisoblang va noma'lum X va Y ning qiymatini toping. $\left\{\begin{array}{c}x+y=8 \\ 2 x+y=13\end{array}\right.$\\
A) $2 ; 3$\\
B) $4 ; 5$\\
C) $5 ; 3$\\
D) $3: 6$
  \item Quyidagi tenglamani hisoblang va noma'lum X va Y ning qiymatini toping. $\{4 x+y=18$\\
$\{2 x+2 y=18$\\
A) $2 ; 3$\\
B) $4 ; 5$\\
C) $5 ; 3$\\
D) $3 ; 6$
  \item Quyidagi tenglamani hisoblang va noma'lum X va Y ning qiymatini toping. $\{3 x+3 y=15$\\
$\{2 x+4 y=16$\\
A) $2 ; 3$\\
B) $4 ; 5$\\
C) $5 ; 3$\\
D) $3 ; 6$
  \item Quyidagi tenglamani hisoblang va noma'lum $X$ va $Y$ ning qiymatini toping. $\left\{\begin{array}{l}5 x+y=28 \\ x+5 y=20\end{array}\right.$\\
A) $2 ; 3$\\
B) $4 ; 5$\\
C) $5 ; 3$\\
D) $3 ; 6$
  \item Quyidagi tenglamani hisoblang va noma'lum X va Y ning qiymatini toping. $\{8 x+y=19$\\
$\{2 x+y=7$\\
A) $2 ; 3$\\
B) $4 ; 5$\\
C) $5 ; 3$\\
D) 3;6
  \item Quyidagi tenglamani hisoblang va noma'lum X va Y ning qiymatini toping. $\left\{\begin{array}{c}3 x+3 y=15 \\ 2 x+y=7\end{array}\right.$\\
A) $2 ; 3$\\
B) $4 ; 5$\\
C) $5 ; 3$\\
D)3;6
  \item Quyidagi tenglamani hisoblang va noma'lum X va Y ning qiymatini toping. $\left\{\begin{array}{c}x-y=2 \\ 2 x+3 y=19\end{array}\right.$\\
A) $2 ; 3$\\
B) $4 ; 5$\\
C) $5 ; 3$\\
D) $3 ; 6$
  \item Bir gektar bug'doy dalasini 5 ta dehqon 10 kunda o'rib tugatsa, 25 ta dehqon neeha kunda o'ra oladi?\\
A) 10\\
B) 2\\
C) 5\\
D) 6
  \item Quyidagi tenglamani hisoblang va noma'lum X va Y ning qiymatini toping. $\left\{\begin{array}{c}x-y=3 \\ x+3 y=15\end{array}\right.$\\
A) $2 ; 3$\\
B) $4 ; 5$\\
C) $5 ; 3$\\
D) $6 ; 3$
  \item Quyidagi tenglamani hisoblang va noma'lum X va Y ning qiymatini toping. $\left\{\begin{array}{c}2 x-y=3 \\ 2 x+3 y=23\end{array}\right.$\\
A) $2 ; 3$\\
B) $4 ; 5$\\
C) $5 ; 3$\\
D) $3 ; 6$
  \item Quyidagi tenglamani hisoblang va noma'lum X va Y ning qiymatini toping. $\left\{\begin{array}{c}x-y=1 \\ 2 x+y=11\end{array}\right.$\\
A) $2 ; 3$\\
B) $4 ; 3$\\
C) $3 ; 5$\\
D) $3 ; 6$
  \item Quyidagi tenglamani hisoblang va noma'lum X va Y ning qiymatini toping. $\left\{\begin{array}{c}2 x-5 y=5 \\ 2 x+3 y=13\end{array}\right.$\\
A) $2 ; 3$\\
B) $4 ; 5$\\
C) $5 ; 1$\\
D) $3 ; 6$
  \item Quyidagi tenglamani hisoblang va noma'lum X va Y ning qiymatini toping. $\left\{\begin{array}{c}4 x-y=5 \\ 2 x+2 y=10\end{array}\right.$\\
A) $2 ; 3$\\
B) $4 ; 5$\\
C) $5 ; 3$\\
D) $3 ; 6$
  \item Quyidagi tenglamani hisoblang va noma'lum X va Y ning qiymatini toping. $\left\{\begin{array}{c}3 x-3 y=6 \\ 2 x+4 y=22\end{array}\right.$\\
A) $2 ; 3$\\
B) $4 ; 5$\\
C) $5 ; 3$\\
D) $3 ; 6$
  \item Quyidagi tenglamani hisoblang va noma'lum X va Y ning qiymatini toping. $\left\{\begin{array}{c}5 x-y=7 \\ x+5 y=17\end{array}\right.$\\
A) $2 ; 3$\\
B) $4 ; 5$\\
C) $5 ; 3$\\
D) $3 ; 6$
  \item Quyidagi tenglamani hisoblang va noma'lum X va Y ning qiymatini toping. $\{8 x-y=18$ $\{2 x+y=12$\\
A) $2 ; 3$\\
B) $4 ; 5$\\
C) $5 ; 3$\\
D) $3 ; 6$
  \item Quyidagi tenglamani hisoblang va noma'lum X va Y ning qiymatini toping. $\left\{\begin{array}{l}4 x-7 y=1 \\ 2 x+3 y=7\end{array}\right.$\\
A) $2 ; 1$\\
B) $4 ; 5$\\
C) $5 ; 1$\\
D) $3 ; 6$
182. Bir gektar bug'doy dalasini 25 ta dehqon 10 kunda o'rib tugatsa, 5 ta dehqon necha kunda o'ra oladi?\\
A) 50\\
B) 20\\
C) 25\\
D) 60\\
183. Bir gektar bug'doy dalasini 8 ta dehqon 10 kunda o'rib tugatsa, 40 ta dehqon necha kunda o'ra oladi?\\
A) 10\\
B) 2\\
C) 5\\
D) 6\\
184. Bir gektar bug'doy dalasini 15 ta dehqon 15 kunda o'rib tugatsa, 45 ta dehqon necha kunda o'ra oladi?\\
A) 10\\
B) 2\\
C) 5\\
D) 6\\
185. Bir gektar bug'doy dalasini 7 ta dehqon 18 kunda o'rib tugatsa, 21 ta dehqon necha kunda o'ra oladi?\\
A) 10\\
B) 2\\
C) 5\\
D) 6\\
186. Bir gektar bug'doy dalasini 14 ta dehqon 10 kunda o'rib tugatsa, 7 ta dehqon necha kunda o'ra oladi?\\
A) 10\\
B) 20\\
C) 25\\
D) 6\\
187. Bir gektar bug'doy dalasini 28 ta dehqon 21 kunda o'rib tugatsa, 84 ta dehqon necha kunda o'ra oladi?\\
A) 7\\
B) 2\\
C) 5\\
D) 4\\
188. Bir gektar bug'doy dalasini 12 ta dehqon 10 kunda o'rib tugatsa, 4 ta dehqon necha kunda o'ra oladi?\\
A) 10\\
B) 30\\
C) 50\\
D) 16\\
189. Bir gektar bug'doy dalasini 50 ta dehqon 10 kunda o'rib tugatsa, 25 ta dehqon necha kunda o'ra oladi?\\
A) 10\\
B) 2\\
C) 20\\
D) 6\\
190. Bir gektar bug'doy dalasini 5 ta dehqon 6 kunda o'rib tugatsa, 30 ta dehqon necha kunda o'ra oladi?\\
A) 1\\
B) 2\\
C) 3\\
D) 4
  \item Bir ko'za suvni Feruzaning o'zi ichsa, 10 kunga yetadi singlisi ichsa, 30 kunga yetadi ular birgalikda ichishsa, necha kunga yetadi?\\
A) 1,5\\
B) 2\\
C) 3\\
D) 7,5
192. Bir ko'za suvni Feruzaning o'zi ichsa, 10 kunga yetadi singlisi ichsa, 40 kunga yetadi ular birgalikda ichishsa, necha kunga yetadi?\\
A) 8\\
B) 7\\
C) 6\\
D) 5\\
193. Bir ko'za suvni Feruzaning o'zi ichsa, 10 kunga yetadi singlisi ichsa, 15 kunga yetadi ular birgalikda ichishsa, necha kunga yetadi?\\
A) 8\\
B) 7\\
C) 6\\
D) 5\\
194. Bir ko'za suvni Feruzaning o'zi ichsa, 20 kunga yetadi singlisi ichsa, 80 kunga yetadi ular birgalikda ichishsa, necha kunga yetadi?\\
A) 16\\
B) 20\\
C) 30\\
D) 7,5\\
195. Bir ko'za suvni Feruzaning o'zi ichsa, 20 kunga yetadi singlisi ichsa, 30 kunga yetadi ular birgalikda ichishsa, necha kunga yetadi?\\
A) 15\\
B) 12\\
C) 13\\
D) 7,5\\
196. Bir ko'za suvni Feruzaning o'zi ichsa, 4 kunga yetadi singlisi ichsa, 16 kunga yetadi ular birgalikda ichishsa, necha kunga yetadi?\\
A) 1,5\\
B) 2\\
C) 3\\
D) 3,2\\
197. Bir ko'za suvni Feruzaning o'zi ichsa, 20 kunga yetadi singlisi ichsa, 60 kunga yetadi ular birgalikda ichishsa, necha kunga yetadi?\\
A) 15\\
B) 2\\
C) 3\\
D) 7,5\\
198. Bir ko'za suvni Feruzaning o'zi ichsa, 5 kunga yetadi singlisi ichsa, 20 kunga yetadi ular birgalikda ichishsa, necha kunga yetadi?\\
A) 1,5\\
B) 5\\
C) 4\\
D) 7,5\\
199. Bir ko'za suvni Feruzaning o'zi ichsa, 6 kunga yetadi singlisi ichsa, 18 kunga yetadi ular birgalikda ichishsa, necha kunga yetadi?\\
A) 1.5\\
B) 2\\
C) 3\\
D) 4,5\\
200. Bir ko'za suvni Feruzaning o'zi ichsa, 7 kunga yetadi singlisi ichsa, 28 kunga yetadi ular birgalikda ichishsa, necha kunga yetadi?\\
A) 5,6\\
B) 2\\
C) 3,2\\
D) 7,5
  \item 160 litr suv bilan 32 ta ko'zani to'ldirish mumkin. 300 litr suv bilan nechta ko'zani to'ldirish mumkin?\\
A) 40\\
B) 50\\
C) 60\\
D) 70
202. 16 litr suv bilan 32 ta ko'zani to'ldirish mumkin. 8 litr suv bilan nechta ko'zani to'ldirish mumkin?\\
A) 16\\
B) 5\\
C) 15\\
D) 7\\
203. 30 litr suv bilan 40 ta ko'zani to'ldirish mumkin. 7,5 litr suv bilan nechta ko'zani to'ldirish mumkin?\\
A) 4\\
B) 5\\
C) 10\\
D) 7\\
204. 50 litr suv bilan 30 ta ko'zani to'ldirish mumkin. 25 litr suv bilan nechta ko'zani to'ldirish mumkin?\\
A) 16\\
B) 15\\
C) 5\\
D) 7\\
205. 250 litr suv bilan 25 ta ko'zani to'ldirish mumkin. 100 litr suv bilan nechta ko'zani to'ldirish mumkin?\\
A) 4\\
B) 5\\
C) 10\\
D) 7\\
206. 48 litr suv bilan 16 ta ko'zani to'ldirish mumkin. 8 ta ko'zani to'ldirish uchun necha litr suv kerak?\\
A) 72\\
B) 96\\
C) 78\\
D) 24\\
207. 72 litr suv bilan 24 ta ko'zani to'ldirish mumkin. 36 ta ko'zani to'ldirish uchun necha litr suv kerak?\\
A) 54\\
B) 108\\
C ) 36\\
D) 24\\
208. 96 litr suv bilan 16 ta ko'zani to'ldirish mumkin. 4 ta ko'zani to'ldirish uchun necha litr suv kerak?\\
A) 24\\
B) 96\\
C) 78\\
D) 84\\
209. 16 litr suv bilan 6 ta ko'zani to'ldirish mumkin. 18 ta ko'zani to'ldirish uchun necha litr suv kerak?\\
A) 72\\
B) 96\\
C) 48\\
D) 84\\
210. 8 litr suv bilan 5 ta ko'zani to'ldirish mumkin. 15 ta ko'zani to'ldirish uchun necha litr suv kerak?\\
A) 72\\
B) 96\\
C) 78\\
D) 24\\
211. Yo'lovchi 7 soatda 28 km yo'lni bosib' o'tdi. U shunday tezlik bilan 8 km yo'lni necha minutda bosib o'tadi?\\
A) 90\\
B) 120\\
C) 180\\
D) 240
  \item Yo'lovchi 6 soatda 30 km yo'lni bosib o'tdi. U shunday tezlik bilan 1 km yo'lni necha minutda bosib o'tadi?\\
A) 12\\
B) 18\\
C) 24\\
D) 30
  \item Yo'lovchi 10 soatda 30 km yo'lni bosib o'tdi. U shunday tezlik bilan 3 km yo'lni necha minutda bosib o'tadi?\\
A) 60\\
B) 180\\
C) 240\\
D) 30
  \item Yo'lovchi 4 soatda 30 km yo'lni bosib o'tdi. U shunday tezlik bilan 6 km yo'lni necha minutda bosib o'tadi?\\
A) 12\\
B) 18\\
C) 48\\
D) 30
  \item Yo'lovchi 5 soatda 30 km yo'lni bosib o'tdi. U shunday tezlik bilan 12 km yo'lni necha minutda bosib o'tadi?\\
A) 120\\
B) 180\\
C) 240\\
D) 300
  \item Yo'lovchi 5 soatda 20 km yo'lni bosib o'tdi. U shunday tezlik bilan 45 minutda necha km yo'lni bosib o'tadi?\\
A) 1\\
B) 2\\
C) 3\\
D) 4
  \item Yotlovchi 4 soatda 12 km yo'lni bosib o'tdi. U shunday tezlik bilan 80 minutda necha km yo'lni bosib o'tadi?\\
A) 1\\
B) 2\\
C) 3\\
D) 4
  \item Yo'lovchi 6 soatda 18 km yo'lni bosib o'tdi. U shunday tezlik bilan 120 minutda necha km yo'lni bosib o'tadi?\\
A) 1\\
B) 5\\
C) 6\\
D) 3
  \item Yo'lovchi 4 soatda 20 km yo'lni bosib o'tdi. U shunday tezlik bilan 180 minutda necha km yo'lni bosib o'tadi?\\
A) 11\\
B) 15\\
C) 16\\
D) 13
  \item Yo'lovchi 4 soatda 100 km yo'lni bosib o'tdi. U shunday tezlik bilan 60 minutda necha km yo'lni bosib o'tadi?\\
A) 21\\
B) 25\\
C) 26\\
D) 32
  \item Quyidagi xususiyatlardan atomga xos bo'lganini toping.\\
A) Rang\\
B) Yadro zaryadi\\
C) Hid\\
D) Agregat xolat
  \item Quyidagi xususiyatlardan atomga xos bo'lganini toping.\\
A) Elektromanfiylik\\
B) Zichlik\\
C) Protonlar bo'lishi\\
D) Agregat xolat
  \item Quyidagi xususiyatlardan atomga xos bo'lganini toping.\\
A) Guruh raqami\\
B) Uchuvchanlik\\
C) Atom massa\\
D) Agregat xolat
  \item Quyidagi xususiyatlardan molekulaga xos bo'lganini toping.\\
A) Kimyoviy bog'\\
B) Tartib raqami\\
C) Atom massa\\
D) Agregat xolat
  \item Quyidagi xususiyatlardan molekulaga xos bo'lganini toping.\\
A) Guruh raqami\\
B) Nisbiy massa\\
C) Atom massa\\
D) Agregat xolat
  \item Quyidagi xususiyatlardan molekulaga xos bo'lganini toping.\\
A) Rang\\
B) Zichlik\\
C) Atom massa\\
D) Harakatlanish
  \item Quyidagi xususiyatlardan moddaga xos bo'lganini toping.\\
A) Shakl\\
B) Guruh raqami\\
C) Rang\\
D) Elektromanfiylik
  \item Quyidagi xususiyatlardan jismga xos bo'lganini toping.\\
A) Shakl\\
B) Guruh raqami\\
C) Molyar massa\\
D) Elektromanfiylik
  \item Quyidagi xususiyatlardan jismga xos bo'lganini toping.\\
A) Hajm\\
B) Guruh raqami\\
C) Davr raqami\\
D) Elektromanfiylik
  \item Quyidagi xususiyatlardan jismga xos bo'lganini toping.\\
A) Oksidlovchilik\\
B) Guruh raqami\\
C) Massa\\
D) Elektromanfiylik
  \item Quyidagilardan qaysi biri modda?\\
A) Mis sim\\
B) Siyohli ruchka\\
C) Tuz\\
D) Qaychi
  \item Quyidagilardan qaysi biri modda?\\
A) Cho'yan pechka\\
B) Ohaktosh\\
C) Tuzlangan karam\\
D) Tillo zirak
  \item Quyidagilardan qaysi biri modda?\\
A) Grafitli qalam\\
B) Rezina o'chirgich\\
C) Dengiz suvi\\
D) Qorbobo
  \item Quyidagilardan qaysi biri modda?\\
A) Temir biton\\
B) Siment\\
C) Rezina shar\\
D) Pichoq
  \item Quyidagilardan qaysi biri modda?\\
A) Mis\\
B) Siyohli ruchka\\
C) Plastik idish\\
D) Siyohdon
  \item Quyidagilardan qaysi biri modda?\\
A) Mis trubka\\
B) Siyoh\\
C) Plastik idish\\
D) Sharikli ruchka
  \item Quyidagilardan qaysi biri modda?\\
A) Tosh\\
B) Siyohli ruchka\\
C) Sulfat kislota\\
D) Probirka
  \item Quyidagilardan qaysi biri modda?\\
A) Pirit\\
B) Muz bo'lagi\\
C) Daraxt\\
D) Quyosh
  \item Quyidagilardan qaysi biri modda?\\
A) Mis sim\\
B) Siyohli ruchka\\
C) Polietilen\\
D) Grafit elektrod
  \item Quyidagilardan qaysi biri modda?\\
A) Temir kubik\\
B) Siyohli ruchka\\
C) Plastik idish\\
D) Qum
21. Quyidagi javoblardagi elementlardan qaysi birida allotropiya hodisasi uchraydi?\\
A) He\\
B) Ne\\
C) O\\
D) Cl\\
22. Quyidagi javoblardagi elementlardan qaysi birida allotropiya hodisasi uchraydi?\\
A) Ar\\
B) S\\
C) Br\\
D) Cl\\
23. Quyidagi javoblardagi elementlardan qaysi birida allotropiya hodisasi uchraydi?\\
A) Kr\\
B) Ne\\
C) I\\
D) C\\
24. Quyidagi javoblardagi elementlardan qaysi birida allotropiya hodisasi uchraydi?\\
A) He\\
B) Ar\\
C) P\\
D) Cl\\
25. Quyidagi javoblardagi elementlardan qaysi birida allotropiya hodisasi uchraydi?\\
A) Cl\\
B) Ne\\
C) Se\\
D) I\\
26. Quyidagi javoblardagi elementlardan qaysi birida allotropiya hodisasi uchraydi?\\
A) He\\
B) Ne\\
C) Cl\\
D) Sn\\
27. Quyidagi javoblardagi elementlardan qaysi birida allotropiya hodisasi uchraydi?\\
A) Sb\\
B) Ne\\
C) He\\
D) Cl\\
28. Quyidagi javoblardagi elementlardan qaysi birida allotropiya hodisasi uchraydi?\\
A) He\\
B) Ne\\
C) Ar\\
D) C\\
29. Quyidagi javoblardagi elementlardan qaysi birida allotropiya hodisasi uchraydi?\\
A) I\\
B) N\\
C) K\\
D) P\\
30. Quyidagi javoblardagi elementlardan qaysi birida allotropiya hodisasi uchraydi?\\
A) H\\
B) Ne\\
C) S\\
D) Cl
31. Quyidagi moddalardan qaysi birida allotropiya hodisasi kuzatiladigan element bor?\\
A) NaOH\\
B) $\mathrm{N}_{2}$\\
C) NaI\\
D) $\mathrm{CaCl}_{2}$\\
32. Quyidagi moddalardan qaysi birida allotropiya hodisasi kuzatiladigan element bor?\\
A) $\mathrm{H}_{2}$\\
B) Ne\\
C) $\mathrm{Na}_{2} \mathrm{~S}$\\
D) NaCl\\
33. Quyidagi moddalardan qaysi birida allotropiya hodisasi kuzatiladigan element bor?\\
A) HCl\\
B) CO\\
C) $\mathrm{N}_{2}$\\
D) $\mathrm{MgBr}_{2}$\\
34. Quyidagi moddalardan qaysi birida allotropiya hodisasi kuzatiladigan element bor?\\
A) $\mathrm{H}_{2} \mathrm{O}$\\
B) Ne\\
C) Mg\\
D) NaI\\
35. Quyidagi moddalardan qaysi birida allotropiya hodisasi kuzatiladigan element bor?\\
A) $\mathrm{H}_{2} \mathrm{~S}$\\
B) Ne\\
C) NaI\\
D) NaCl\\
36. Quyidagi moddalardan qaysi birida allotropiya hodisasi kuzatiladigan element bor?\\
A) Kr\\
B) $\mathrm{CaCO}_{3}$\\
C) Na\\
D) NaCl\\
37. Quyidagi moddalardan qaysi birida allotropiya hodisasi kuzatiladigan element bor?\\
A) NaCl\\
B) KH\\
C) $\mathrm{KHCO}_{3}$\\
D) $\mathrm{MgBr}_{2}$\\
38. Quyidagi moddalardan qaysi birida allotropiya hodisasi kuzatiladigan element bor?\\
A) $\mathrm{AlCl}_{3}$\\
B) Ne\\
C) $\mathrm{K}_{2} \mathrm{~S}$\\
D) NaCl\\
39. Quyidagi moddalardan qaysi birida allotropiya hodisasi kuzatiladigan element bor?\\
A) $\mathrm{H}_{2} \mathrm{Te}$\\
B) $\mathrm{AlBr}_{3}$\\
C) $\mathrm{N}_{2}$\\
D) NaCl\\
40. Quyidagi moddalardan qaysi birida allotropiya hodisasi kuzatiladigan element bor?\\
A) $\mathrm{H}_{2} \mathrm{SO}_{4}$\\
B) HCl\\
C) Na\\
D) NaCl
  \item Quyidagi moddalar orasidan oddiy moddani toping.\\
A) $\mathrm{H}_{2} \mathrm{SO}_{4}$\\
B) HCl\\
C) Na\\
D) NaCl
  \item Quyidagi moddalar orasidan oddiy moddani toping.\\
A) $\mathrm{CaCO}_{3}$\\
B) Olmos\\
C) Osh tuzi\\
D) KCl
  \item Quyidagi moddalar orasidan oddiy moddani toping.\\
A) $\mathrm{H}_{3} \mathrm{AsO}_{4}$\\
B) $\mathrm{HClO}_{2}$\\
C) $\mathrm{P}_{4}$\\
D) NaCl
  \item Quyidagi moddalar orasidan oddiy moddani toping.\\
A) $\mathrm{H}_{2} \mathrm{SO}_{3}$\\
B) Ishqor\\
C) NaH\\
D) Grafit
  \item Quyidagi moddalar orasidan oddiy moddani toping.\\
A) $\mathrm{H}_{2} \mathrm{SO}_{4}$\\
B) Al\\
C) NaCN\\
D) NaCl
  \item Quyidagi moddalar orasidan oddiy moddani toping.\\
A) $\mathrm{H}_{2} \mathrm{~S}$\\
B) HCl\\
C) $\mathrm{NaNO}_{2}$\\
D) Ar
  \item Quyidagi moddalar orasidan oddiy moddani toping.\\
A) $\mathrm{H}_{2}$\\
B) HCl\\
C) NaI\\
D) NaCl
  \item Quyidagi moddalar orasidan oddiy moddani toping.\\
A) S\\
B) HCl\\
C) NaBr\\
D) NaCl
  \item Quyidagi moddalar orasidan oddiy moddani toping.\\
A) $\mathrm{H}_{2} \mathrm{SO}_{4}$\\
B) HCl\\
C) Mg\\
D) NaCl
  \item Quyidagi moddalar orasidan oddiy moddani toping.\\
A) $\mathrm{H}_{2} \mathrm{SO}_{4}$\\
B) HCl\\
C) Cr\\
D) NaCl
  \item Mis simning bukilishi
A)Kimyoviy\\
B)Fizikaviy\\
  \item Qo'rgoshinning erib suyuq holga o'tishi
A)Kimyoviy\\
B)Fizikaviy\\
  \item Temirning zanglashi
A)Kimyoviy\\
B)Fizikaviy\\
  \item Suvning qaynashi
A)Kimyoviy\\
B)Fizikaviy\\
  \item Spirtning bug'lanishi
A)Kimyoviy\\
B)Fizikaviy\\
  \item Bertole tuzini parchalab kislorod olish Kimyoviy Fizikaviy
A)Kimyoviy\\
B)Fizikaviy\\
  \item Havoni suyultirib azot olish
A)Kimyoviy\\
B)Fizikaviy\\
  \item Qog'ozning yonishi
A)Kimyoviy\\
B)Fizikaviy\\
  \item Aluminiy simining qizdirilganda qorayishi
A)Kimyoviy\\
B)Fizikaviy\\
  \item Kimyoviy hodisani tanlang.\\
A) Bertolle tuzini parchalash\\
B) suv bug'ining kondensatlash\\
C) quruq muzning sublimatlanishi\\
D) galliy qo'lga olinganda suyuqlanishi
  \item Kimyoviy hodisani tanlang.\\
A) azotdan kislorod olish\\
B) temiming zanglashi\\
C) qo'rg'oshinning temperatura ta'sirida suyuqlanishi\\
D) qaldiroq gazdan vodorod olish
  \item Quyidagilardan fizikaviy hodisalami aniqlang.\\
A) qirovning hosil bo'lishi\\
B) shamning yonishi\\
C) qizdirilgan shakarning qorayishi\\
D) temirning zanglashi
  \item Quyidagilardan fizikaviy hodisalami aniqlang.\\
A) havoning suyuqlanishi\\
B) shamning yonishi\\
C) qizdirilgan shakarning qorayishi\\
D) temirning zanglashi
  \item Quyidagilardan kimyoviy hodisani aniqlang.\\
A) azotning suyuqlanishi\\
B) havodan kislorod olish\\
C) qalayning suyuqlanishi\\
D) qatiqning ivishi
  \item Quyidagilardan kimyoviy hodisani aniqlang.\\
A) azotning suyuqlanishi\\
B) havodan kislorod olish\\
C) qalayning suyuqlanishi\\
D) sutning achishi
  \item Fizikaviy hodisalarga kiradigan jarayonni ko'rsating,\\
A) sutning achishi\\
B) uzumdan sirka olish\\
C) ohakning suvda erishi\\
D) shamning erishi
  \item Kimyoviy hodisaga kiradigan jarayonni ko'rsating.\\
A) yodning sublimatlanishi\\
B) uzum sharbatining bijg'ishi\\
C) temiming magnitga tortilishi\\
D) yog'ning sovuqda qotishi
  \item Kimvoviy hodisalarga kirmaydigan jarayonlarni ko'rsating.\\
A) uzum sharbatining bijg"ishi\\
B) benzinning yonishi\\
C) yog'ning sovuqda qotishi\\
D) ohakning so'ndirilishi
  \item Kimvoviy hodisalarga kirmaydigan jarayonlarni ko'rsating.\\
A) benzinning yonishi\\
B) uzum sharbatining bijg'ishi\\
C) ohakning so'ndirilishi\\
D) suyuq azotda aluminiyning mo'rt bo'lib qolishi
  \item Qattiq $\rightarrow$ suyuq ushbu holat uchun qaysi javob to'g'ri?\\
A) kondetsatlanish\\
B) sublimatlanish\\
C) cho'kish\\
D) erish
  \item Gaz $\rightarrow$ suyuq ushbu holat uchun qaysi javob to'g'ri?\\
A) kondetsatlanish\\
B) sublimatlanish\\
C) cho'kish\\
D) erish
  \item Qattiq $\rightarrow$ gaz ushbu holat uchun qaysi javob to'g'ri?\\
A) kondetsatlanish\\
B) sublimatlanish\\
C) cho'kish\\
D) erish
  \item Gaz $\rightarrow$ qattiq ushbu holat uchun qaysi javob to'g'ri?\\
A) kondetsatlanish\\
B) sublimatlanish\\
C) cho'kish\\
D) erish
  \item Suyuq $\rightarrow$ qattiq ushbu holat uchun qaysi javob to'g'ri?\\
A) bug'lanish\\
B) sublimatlanish\\
C) qotish\\
D) erish
  \item Suyuq $\rightarrow$ gaz ushbu holat uchun qaysi javob to'g'ri?\\
A) bug'lanish\\
B) sublimatlanish\\
C) qotish\\
D) erish
  \item Kondensatlanish ushbu holatlardan qaysi biriga to'g'ri keladi?\\
A) suyuq $\rightarrow \mathrm{gaz}$\\
B) gaz $\rightarrow$ suyuq\\
C) suyuq $\rightarrow$ qattiq\\
D) qattiq $\rightarrow$ gaz
  \item Sublimatlanish ushbu holatlardan qaysi biriga to'g'ri keladi?\\
A) suyuq $\rightarrow \mathrm{gaz}$\\
B) gaz $\rightarrow$ suyuq\\
C) suyuq $\rightarrow$ qattiq\\
D) qattiq $\rightarrow$ gaz
  \item Bug'lanish ushbu holatlardan qaysi biriga to'g'ri keladi?\\
A) suyuq $\rightarrow$ gaz\\
B) gaz $\rightarrow$ suyuq\\
C) suyuq $\rightarrow$ qattiq\\
D) qattiq $\rightarrow \mathrm{gaz}$
  \item Qotish ushbu holatlardan qaysi biriga to'g'ri keladi?\\
A) suyuq $\rightarrow$ gaz\\
B) gaz $\rightarrow$ suyuq\\
C) suyuq $\rightarrow$ qattiq\\
D) qattiq $\rightarrow$ gaz
  \item Al atomining absolyut massasini kg da hisoblang.\\
A) $35.6 \cdot 10^{-26}$\\
B) $53.334 \cdot 10^{.27}$\\
C) $44.82 \cdot 10^{-27}$\\
D) $48.334 \cdot 10^{\cdot 26}$
  \item Ca atomining absolyut massasini kg da hisoblang.\\
A) $66.4 \cdot 10^{-27}$\\
B) $53.334 \cdot 10^{-27}$\\
C) $44.82 \cdot 10^{-27}$\\
D) $48.334 \cdot 10^{-26}$
  \item Mg atomining absolyut massasini kg da hisoblang.\\
A) $35.6 \cdot 10^{-26}$\\
B) $39.84 \cdot 10^{-27}$\\
C) $44.82 \cdot 10^{-27}$\\
D) $48.334 \cdot 10^{-26}$
  \item Fe atomining absolyut massasini kg da hisoblang.\\
A) $35.6 \cdot 10^{-26}$\\
B) $53.334 \cdot 10^{-27}$\\
C) $92.96 \cdot 10^{\cdot 27}$\\
D) $48.334 \cdot 10^{\cdot 26}$
  \item Na atomining absolyut massasini kg da hisoblang.\\
A) $35.6 \cdot 10^{\cdot 26}$\\
B) $53.334 \cdot 10^{-27}$\\
C) $44.82 \cdot 10^{-27}$\\
D) $38.18 \cdot 10^{-27}$
  \item Cr atomining absolyut massasini g da hisoblang.\\
A) $86.32 \cdot 10^{.24}$\\
B) $53.334 \cdot 10^{-27}$\\
C) $44.82 \cdot 10^{-27}$\\
D) $48.334 \cdot 10^{.24}$
  \item Cu atomining absolyut massasini g da hisoblang.\\
A) $86.32 \cdot 10 \cdot 24$\\
B) $53.334 \cdot 10^{-27}$\\
C) $44.82 \cdot 10^{-27}$\\
D) $106.24 \cdot 10^{.24}$
  \item S atomining absolyut massasini g da hisoblang.\\
A) $86.32 \cdot 10^{-24}$\\
B) $53.12 \cdot 10^{.24}$\\
C) $44.82 \cdot 10^{\cdot 27}$\\
D) $106.24 \cdot 10^{.24}$
  \item Ne atomining absolyut massasini g da hisoblang.\\
A) $33.2 \cdot 10^{-24}$\\
B) $53.334 \cdot 10^{-27}$\\
C) $44.82 \cdot 10 \cdot 27$\\
D) $106.24 \cdot 10 \cdot 24$
  \item Ar atomining absolyut massasini g da hisoblang.\\
A) $86.32 \cdot 10^{.24}$\\
B) $53.334 \cdot 10^{-27}$\\
C) $66.4 \cdot 10^{-24}$\\
D) $106.24 \cdot 10^{-24}$
  \item Agarda B atoming absolyut atom massasi qiymati $18.26 \cdot 10^{-27} \mathrm{~kg}$ bo'lsa, uning nisbiy atom massasi nechaga teng.\\
A) 40\\
B) 24\\
C) 11\\
D) 56\\
  \item Agarda Br atoming absolyut atom massasi qiymati $132.8 \cdot 10^{27} \mathrm{~kg}$ bo'lsa, uning nisbiy atom massasi nechaga teng.\\
A) 80\\
B) 24\\
C) 27\\
D) 56
  \item Agarda F atoming absolyut atom massasi qiymati $31.54 \cdot 10^{-27} \mathrm{~kg}$ bo'lsa, uning nisbiy atom massasi nechaga teng.\\
A) 19\\
B) 24\\
C) 27\\
D) 56
  \item Agarda N atoming absolyut atom massasi qiymati $23.24 \cdot 10^{.27} \mathrm{~kg}$ bo'lsa, uning nisbiy atom massasi nechaga teng.\\
A) 19\\
B) 14\\
C) 27\\
D) 56
  \item Agarda K atoming absolyut atom massasi qiymati $64.74 \cdot 10^{-27} \mathrm{~kg}$ bo'lsa, uning nisbiy atom massasi nechaga teng.\\
A) 19\\
B) 24\\
C) 39\\
D) 56
  \item Agarda Fe atoming absolyut atom massasi qiymati $92.96 \cdot 10^{-24} \mathrm{~g}$ bo'lsa, uning nisbiy atom massasi nechaga teng.\\
A) 19\\
B) 24\\
C) 27\\
D) 56
  \item Agarda P atoming absolyut atom massasi qiymati $51.46 \cdot 10^{-24} \mathrm{~g}$ bo'lsa, uning nisbiy atom massasi nechaga teng.\\
A) 31\\
B) 24\\
C) 28\\
D) 56
  \item Agarda Si atoming absolyut atom massasi qiymati $46.48 \cdot 10^{-2.4} \mathrm{~g}$ bo'lsa, uning nisbiy atom massasi nechaga teng.\\
A) 19\\
B) 24\\
C) 28\\
D) 56
  \item Agarda I atoming absolyut atom massasi qiymati $210,82 \cdot 10^{-24} \mathrm{~g}$ bo'lsa, uning nisbiy atom massasi nechaga teng.\\
A) 19\\
B) 24\\
C) 127\\
D) 56
  \item Agarda Ti atoming absolyut atom massasi qiymati $79.68 \cdot 10^{-24} \mathrm{~g}$ bo'lsa, uning nisbiy atom massasi nechaga teng.\\
A) 19\\
B) 48\\
C) 27\\
D) 56
  \item Al atomining absolyut massasini uglerod birligida hisoblang.\\
A) $27 u$\\
B) $53.334 \cdot 10^{-27}$\\
C) $44.82 \cdot 10^{-27}$\\
D) $19 u$
  \item Ca atomining absolyut massasini uglerod birligida hisoblang.\\
A) $66.4 \cdot 10^{-27}$\\
B) 27 u\\
C) $40 u$\\
D) $48.334 \cdot 10^{-26}$
  \item Mg atomining absolyut massasini uglerod birligida hisoblang.\\
A) $24 u$\\
B) $39.84 \cdot 10^{-27}$\\
C) $40 u$\\
D) $48.334 \cdot 10^{-26}$
  \item Fe atomining absolyut massasini uglerod birligida hisoblang.\\
A) $24 u$\\
B) $56 u$\\
C) $92.96 \cdot 10^{-27}$\\
D) $48.334 \cdot 10^{-26}$
  \item Na atomining absolyut massasini uglerod birligida hisoblang.\\
A) $35.6 \cdot 10^{26}$\\
B) 35.5 u\\
C) $23 u$\\
D) $38.18 \cdot 10^{-27}$
  \item Cr atomining absolyut massasini uglerod birligida hisoblang.\\
A) $86.32 \cdot 10^{-24}$\\
B) $52 u$\\
C) $44.82 \cdot 10^{.27}$\\
D) $48 u$
  \item Cu atomining absolyut massasini uglerod birligida hisoblang.\\
A) $64 u$\\
B) $53.334 \cdot 10^{-27}$\\
C) $44 u$\\
D) $106.24 \cdot 10^{-24}$
  \item S atomining absolyut massasini uglerod birligida hisoblang.\\
A) $86.32 \cdot 10^{-24}$\\
B) $53.12 \cdot 10^{-24}$\\
C) $32 u$\\
D) $20 u$
  \item Ne atomining absolyut massasini uglerod birligida hisoblang.\\
A) $86.32 \cdot 10^{.24}$\\
B) $53.12 \cdot 10^{-24}$\\
C) $32 u$\\
D) $20 u$
  \item Ar atomining absolyut massasini uglerod birligida hisoblang.\\
A) 40 u\\
B) $53.334 \cdot 10^{-27}$\\
C) $66.4 \cdot 10^{-24}$\\
D) 39 u
  \item $\mathrm{H}_{2} \mathrm{O}$ molekulasining absolyut massasini kg hisoblang.\\
A) $29.88 \cdot 10^{\cdot 27}$\\
B) $53.334 \cdot 10^{\cdot 27}$\\
C) $66.4 \cdot 10^{-24}$\\
D) 18
  \item $\mathrm{Ca}(\mathrm{OH})_{2}$ molekulasining absolyut massasini kg hisoblang.\\
A) $29.88 \cdot 10^{.27}$\\
B) $53.334 \cdot 10^{-27}$\\
C) 74\\
D) $122.84 \cdot 10^{-27}$
  \item $\mathrm{Mg}(\mathrm{OH})_{2}$ molekulasining absolyut massasini kg hisoblang.\\
A) $29.88 \cdot 10^{-27}$\\
B) $96.28 \cdot 10^{-27}$\\
C) 58\\
D) $122.84 \cdot 10^{-27}$
  \item $\mathrm{Al}_{2}\left(\mathrm{SO}_{4}\right)_{3}$ molekulasining absolyut massasini kg hisoblang.\\
A) $29.88 \cdot 10^{-27}$\\
B) $53.334 \cdot 10^{-27}$\\
C) 342\\
D) $567.72 \cdot 10^{-27}$
  \item $\mathrm{K}_{3}\left[\mathrm{Fe}(\mathrm{CN})_{6}\right]$ molekulasining absolyut massasini kg hisoblang.\\
A) $546.14 \cdot 10^{-27}$\\
B) 329\\
C) 74\\
D) $122.84 \cdot 10^{-27}$
  \item $\mathrm{H}_{2} \mathrm{SO}_{4}$ molekulasining absolyut massasini g hisoblang.\\
A) $546.14 \cdot 10^{-27}$\\
B) $162.68 \cdot 10^{-24}$\\
C) 98 g\\
D) $122.84 \cdot 10^{-27}$
  \item $\mathrm{Cr}(\mathrm{OH})_{2}$ molekulasining absolyut massasini g hisoblang.\\
A) $142.76 \cdot 10^{-24}$\\
B) $162.68 \cdot 10^{-24}$\\
C) 86 g\\
D) $124.84 \cdot 10^{-27}$
  \item $\mathrm{K}_{4}\left[\mathrm{Fe}(\mathrm{CN})_{6}\right]$ molekulasining absolyut massasini g hisoblang.\\
A) $546.14 \cdot 10^{-27}$\\
B) $610.88 \cdot 10^{-24}$\\
C) 368\\
D) $122.84 \cdot 10^{-27}$
  \item $\mathrm{HNO}_{3}$ molekulasining absolyut massasini g hisoblang.\\
A) $546.14 \cdot 10^{-27}$\\
B) $162.68 \cdot 10^{-24}$\\
C) 63 g\\
D) $104.58 \cdot 10^{-2.4}$
  \item $\mathrm{Cr}(\mathrm{OH})_{3}$ molekulasining absolyut massasini g hisoblang.\\
A) $546.14 \cdot 10^{-27}$\\
B) $170.98 \cdot 10^{-24}$\\
C) 103 g\\
D) $14.58 \cdot 10^{-24}$
  \item NaOH molekulasining $\mathrm{M}_{\mathrm{r}}$ qiymatini hisoblang.
A) 40\\
B) $40 u$\\
C) $40 \mathrm{~g} / \mathrm{mol}$\\
D) $66.4 \cdot 10^{.21}$\\
42. $\mathrm{H}_{2} \mathrm{SO}_{4}$ molekulasining $\mathrm{M}_{r}$ qiymatini hisoblang.\\
A) $98 u$\\
B) $98 \mathrm{~g} / \mathrm{mol}$\\
C) 98\\
D) $162.68 \cdot 10^{-24}$\\
43. $\mathrm{Al}_{2}\left(\mathrm{SO}_{4}\right)_{3}$ molekulasining $\mathrm{M}_{\mathrm{r}}$ qiymatini hisoblang.\\
A) 342\\
B) $288 \mathrm{~g} / \mathrm{mol}$\\
C) 288\\
D) $162.68 \cdot 10^{-24}$\\
44. $\mathrm{K}_{3}\left[\mathrm{Fe}(\mathrm{CN})_{6}\right]$ molekulasining $\mathrm{M}_{\mathrm{r}}$ qiymatini hisoblang.\\
A) $368 / \mathrm{N}_{\mathrm{A}}$\\
B) 329\\
C) 288\\
D) $62.68 \cdot 10^{.24}$\\
45. $\mathrm{Cr}(\mathrm{OH})_{2}$ molekulasining $\mathrm{M}_{\mathrm{r}}$ qiymatini hisoblang.\\
A) $86 / \mathrm{N}_{\mathrm{A}}$\\
B) 88\\
C) 86\\
D) $142.76 \cdot 10^{-24}$\\
46. $\mathrm{Na}_{2} \mathrm{CO}_{3} \cdot 10 \mathrm{H}_{2} \mathrm{O}$ molekulasining $\mathrm{M}_{\mathrm{r}}$ qiymatini hisoblang.\\
A) $286 / \mathrm{N}_{\mathrm{A}}$\\
B) 286\\
C) 288\\
D) $474.76 \cdot 10^{\cdot 24}$\\
47. $\mathrm{K}_{4}\left[\mathrm{Fe}(\mathrm{CN})_{6}\right]$ molekulasining $\mathrm{M}_{\mathrm{r}}$ qiymatini hisoblang.\\
A) 368 kg\\
B) 368 g\\
C) 368\\
D) $610.88 \cdot 10^{\cdot 24}$\\
48. $\mathrm{CuSO}_{4} \cdot 5 \mathrm{H}_{2} \mathrm{O}$ molekulasining $\mathrm{M}_{\mathrm{r}}$ qiymatini hisoblang.\\
A) 250\\
B) 345 g\\
C) 245\\
D) $415 \cdot 10^{-24}$\\
49. $\mathrm{Cr}(\mathrm{OH})_{3}$ molekulasining $\mathrm{M}_{\mathrm{r}}$ qiymatini hisoblang.\\
A) 102\\
B) 103\\
C) $102 / \mathrm{N}_{\mathrm{A}}$\\
D) $415 \cdot 10^{-24}$\\
50. HF molekulasining $\mathrm{M}_{\mathrm{r}}$ qiymatini hisoblang.\\
A) 20\\
B) 10\\
C) $10 / \mathrm{N}_{\mathrm{A}}$\\
D) $40 \cdot 10^{-24}$
  \item NaOH ning M qiymatini hisoblang.\\
A) 20\\
B) $30 \mathrm{~g} / \mathrm{mol}$\\
C) $40 \mathrm{~g} / \mathrm{mol}$\\
D) 80
  \item $\mathrm{H}_{2} \mathrm{SO}_{4}$ ning M qiymatini hisoblang.\\
A) $49 \mathrm{~g} / \mathrm{mol}$\\
B) $196 \mathrm{~g} / \mathrm{mol}$\\
C) $56 \mathrm{~g} / \mathrm{mol}$\\
D) $98 \mathrm{~g} / \mathrm{mol}$
  \item $\mathrm{HNO}_{3}$ ning M qiymatini hisoblang.\\
A) $63 \mathrm{~g} / \mathrm{mol}$\\
B) $31 u$\\
C) $45 \mathrm{~g} / \mathrm{mol}$\\
D) 56
  \item $_{2} \mathrm{Fe}\left(\mathrm{NO}_{3}\right)_{2}$ ning M qiymatini hisoblang.\\
A) $180 \mathrm{~g} / \mathrm{mol}$\\
B) $180 \mathrm{~g} / \mathrm{mol}$\\
C) $86 \mathrm{~g} / \mathrm{mol}$\\
D) 200 u
  \item $\mathrm{Al}_{2}\left(\mathrm{SO}_{4}\right)_{3}$ ning M qiymatini hísoblang,\\
A) $123 \mathrm{~g} / \mathrm{mol}$\\
B) 75\\
C) $342 \mathrm{~g} / \mathrm{mol}$\\
D) 342
  \item $\mathrm{K}_{3} \mathrm{PO}_{4}$ ning M qiymatini hisoblang.\\
A) $212 \mathrm{~g} / \mathrm{mol}$\\
B) $86 u$\\
C) $121 \mathrm{~g} / \mathrm{mol}$\\
D) 39
  \item $\mathrm{Fe}\left(\mathrm{NO}_{3}\right)_{3}$ ning M giymatini hisoblang.\\
A) $242 \mathrm{~g} / \mathrm{mol}$\\
B) $180 \mathrm{~g} / \mathrm{mol}$\\
C) $86 \mathrm{~g} / \mathrm{mol}$\\
D) 40 g
  \item $\mathrm{Na}_{2} \mathrm{CO}_{3} \cdot 10 \mathrm{H}_{2} \mathrm{O}$ ning M qiymatiní hisoblang.\\
A) $180 \mathrm{~g} / \mathrm{mol}$\\
B) $286 \mathrm{~g} / \mathrm{mol}$\\
C) $142 u$\\
D) 124 g
  \item $\mathrm{FeSO}_{4} \cdot 7 \mathrm{H}_{2} \mathrm{O}$ molekulasining M qiymatini hisoblang.\\
A) $128 \mathrm{~g} / \mathrm{mol}$\\
B) $278 \mathrm{~g} / \mathrm{mol}$\\
C) $156 \mathrm{~g} / \mathrm{mol}$\\
D) 87 kg
  \item Mis kuprosi $\mathrm{CuSO}_{4} \cdot 5 \mathrm{H}_{2} \mathrm{O}$ ning M qiymatini hisoblang.\\
A) $120 \mathrm{~g} / \mathrm{mol}$\\
B) 25\\
C) $250 \mathrm{~g} / \mathrm{mol}$\\
D) 125
  \item $\mathrm{CO}_{2}$ molekulasidagi elemetlarni massasini kichik butun son nisbatini aniqlang\\
A) $3: 8$\\
B) $1: 2,32$\\
C) $2: 3$\\
D) $2: 6$
  \item $\mathrm{H}_{2} \mathrm{SO}_{4}$ molekulasidagi elemetlarni massasini kichik butun son nisbatini aniqlang\\
A) $1: 2: 3$\\
B) $1: 4: 6$\\
C) $1: 16: 32$\\
D) $1: 32: 49$
  \item $\mathrm{SO}_{3}$ molekulasidagi elemetlarni massasini kichik butun son nisbatini aniqlang\\
A) $1: 3$\\
B) $2: 4$\\
C) $2: 5$\\
D) $2: 3$
  \item $\mathrm{SO}_{2}$ molekulasidagi elemetlarni massasini kichik butun son nisbatini aniqlang\\
A) $1: 1$\\
B) $2: 4$\\
C) $2: 5$\\
D) $1: 1,5$
  \item $\mathrm{Na}_{2} \mathrm{CO}_{3}$ molekulasidagi elemetlarni massasini kichik butun son nisbatini aniqlang\\
A) $3: 6: 16$\\
B) $23: 6: 24$\\
C) $4: 2: 8$\\
D) $2: 2: 1$
  \item $\mathrm{K}_{2} \mathrm{SO}_{4}$ molekulasidagi elemetlarni massasini kichik butun son nisbatini aniqlang\\
A) $39: 32: 16$\\
B) $78: 32: 32$\\
C) $39: 16: 32$\\
D) $3: 5: 1$
  \item $\mathrm{Ca}(\mathrm{OH})_{2}$ molekulasidagi elemetlarni massasini kichik butun son nisbatini aniqlang\\
A) $20: 16: 1$\\
B) $10: 8: 1$\\
C) $25: 12: 10$\\
D) $1: 2: 2$
  \item $\mathrm{NH}_{4} \mathrm{OH}$ molekulasidagi elemetlarni massasini kichik butun son nisbatini aniqlang\\
A) $3: 2: 1$\\
B) $2: 1: 1$\\
C) $14: 5: 16$\\
D) $2: 1: 5$
  \item $\mathrm{K}_{3} \mathrm{PO}_{4}$ molekulasidági elementlarni massa nisbatlarini aniqlang.\\
A) $3: 6: 16$\\
B) $8: 1: 4$\\
C) $117: 31: 64$\\
D) $2: 2: 1$
  \item $\mathrm{FeSO}_{4}$ molekulasidagi elementlarni massa nisbatlarini aniqlang.\\
A) $14: 8: 16$\\
B) $3,83: 1: 4$\\
C) $3,8: 1: 2$\\
D) $2: 2: 1$
  \item 2 mol CO va $8 \mathrm{~mol} \mathrm{CO}_{2}$ gazlari tarkibidagi uglerod atomlarining son nisbatini toping.\\
A) $1: 4$\\
B) $1,5: 1$\\
C) $1: 1$\\
D) $1,5: 3,5$
  \item $2 \mathrm{~mol} \mathrm{P}_{2} \mathrm{O}_{5}$ va $1 \mathrm{~mol} \mathrm{P}_{2} \mathrm{O}_{3}$ molekulalari tarkibidagi kislorod atomlari son nisbatini toping.\\
A) $10: 3$\\
B) $10: 1,5$\\
C) $10: 1$\\
D) $1,5: 1$
  \item $2 \mathrm{~mol} \mathrm{H}_{2} \mathrm{O}$ va $1 \mathrm{~mol} \mathrm{H}_{2} \mathrm{O}_{2}$ molekulalari tarkibidagi kislorod atomlari son nisbatini toping.\\
A) $10: 3$\\
B) $10: 1,5$\\
C) $1: 1$\\
D) $1,5: 1$
  \item $1,5 \mathrm{~mol} \mathrm{Cl}_{2} \mathrm{O}$ va $3 \mathrm{~mol} \mathrm{NO}_{2}$ gazlaridagi atomlarining son nisbatini toping.\\
A) $1: 2$\\
B) $1: 3$\\
C) $1: 4$\\
D) $1: 5$
  \item $1,5 \mathrm{~mol} \mathrm{Cl}_{2} \mathrm{O}$ va $3 \mathrm{~mol} \mathrm{~N}_{2} \mathrm{O}$ gazlaridagi atomlarining son nisbatini toping.\\
A) $1: 2$\\
B) $1: 3$\\
C) $1: 4$\\
D) $1: 5$
  \item 2 mol CO va $8 \mathrm{~mol} \mathrm{CO}_{2}$ gazlari tarkibidagi atomlarining son nisbatini toping.\\
A) $1: 4$\\
B) $1,5: 1$\\
C) $1: 6$\\
D) $1,5: 3,5$
  \item $2 \mathrm{~mol} \mathrm{P}_{2} \mathrm{O}_{5}$ va $1 \mathrm{~mol} \mathrm{P}_{2} \mathrm{O}_{3}$ molekulalari tarkibidagi atomlaring son nisbatini toping.\\
A) $10: 3$\\
B) $10: 2,5$\\
C) $7: 2,5$\\
D) $1,5: 1$
  \item $2 \mathrm{~mol} \mathrm{H}_{2} \mathrm{O}$ va $1 \mathrm{~mol} \mathrm{H}_{2} \mathrm{O}_{2}$ molekulalari tarkibidagi atomlaring son nisbatini toping.\\
A) $3: 2$\\
B) $10: 1,5$\\
C) $1: 1$\\
D) $1,5: 1$
  \item $\mathrm{H}_{2} \mathrm{SO}_{4}$ molekulasi tarkibidagi\\
kislorodning massa ulushini toping.\\
A) $\frac{32}{49}$\\
B) $\frac{64}{49}$\\
C) $\frac{32}{40}$\\
D) $\frac{32}{16}$
  \item $\mathrm{Na}_{2} \mathrm{SO}_{4}$ molekulasi tarkibidagi natriyning massa ulushini toping.\\
A) $\frac{32}{49}$\\
B) $\frac{64}{49}$\\
C) $\frac{32}{142}$\\
D) $\frac{23}{71}$
  \item $\mathrm{NH}_{3}$ molekulasi tarkibidagi azotning massa ulushini toping.\\
A) $\frac{7}{17}$\\
B) $\frac{14}{17}$\\
C) $\frac{32}{40}$\\
D) $\frac{15}{16}$
  \item $\mathrm{CO}_{2}$ molekulasi tarkibidagi uglerodning massa ulushini toping.\\
A) $\frac{6}{22}$\\
B) $\frac{64}{49}$\\
C) $\frac{32}{4-4}$\\
D) $\frac{32}{16}$
  \item $\mathrm{H}_{3} \mathrm{PO}_{4}$ molekulasi tarkibidagi vadorodning massa ulushini toping.\\
A) $\frac{32}{49}$\\
B) $\frac{3}{49}$\\
C) $\frac{3}{98}$\\
D) $\frac{32}{16}$
  \item $\mathrm{Na}_{2} \mathrm{CO}_{3} \cdot 10 \mathrm{H}_{2} \mathrm{O}$ molekulasi tarkibidagi kislorodning massa ulushini toping.\\
A) $\frac{48}{286}$\\
B) $\frac{104}{143}$\\
C) $\frac{32}{40}$\\
D) $\frac{32}{16}$
  \item $\mathrm{CuSO}_{4} \cdot 5 \mathrm{H}_{2} \mathrm{O}$ molekulasi tarkibidagi oltingugurtning massa ulushini toping.\\
A) $\frac{48}{250}$\\
B) $\frac{64}{250}$\\
C) $\frac{16}{125}$\\
D) $\frac{32}{16}$
  \item $\mathrm{Ca}(\mathrm{OH})_{2}$ molekulasi tarkibidagi vodorodning massa ulushini toping.\\
A) $\frac{2}{74}$\\
B) $\frac{2}{37}$\\
C) $\frac{2}{40}$\\
D) $\frac{2}{16}$
  \item $\mathrm{SO}_{3}$ molekulasi tarkibidagi kislorodning massa ulushini toping.\\
A) $\frac{48}{86}$\\
B) $\frac{48}{143}$\\
C) $\frac{24}{40}$\\
D) $\frac{32}{16}$
  \item $\mathrm{CaCO}_{3} \cdot \mathrm{MgCO}_{3}$ tarkibidagi magniyning massa ulushini hisoblang\\
A) $\frac{6}{46}$\\
B) $\frac{12}{184}$\\
C) $\frac{24}{40}$\\
D) $\frac{32}{16}$
  \item Noma'lum modda tarkibida elementlarning massa ulushlari nisbati quyidagicha $\mathrm{Ca}-40$ : $\mathrm{C}-12$ : 0 - 48 bo'lsa, noma'lum moddaning formulasini toping.\\
A) $\mathrm{Ca}_{2} \mathrm{CO}_{2}$\\
B) $\mathrm{CaC}_{2} \mathrm{O}_{6}$\\
C) $\mathrm{CaCO}_{2}$\\
D) $\mathrm{CaCO}_{3}$
  \item Noma'lum modda tarkibida elementlarning massa ulushlari nisbati quyidagicha $\mathrm{Na}-46$ :Cl-71:O-32 bo'lsa, noma'lum moddaning formulasini toping.\\
A) NaClO\\
B) $\mathrm{NaClO}_{2}$\\
C) $\mathrm{NaClO}_{3}$\\
D) $\mathrm{NaClO}_{4}$
  \item Noma'lum modda tarkibida elementlarning massa ulushlari nisbati quyidagicha Na-46:Cl-71:O-64 bo'lsa, noma'lum moddaning formulasini toping.\\
A) NaClO\\
B) $\mathrm{NaClO}_{2}$\\
C) $\mathrm{NaClO}_{3}$\\
D) $\mathrm{NaClO}_{4}$
  \item Noma'lum modda tarkibida elementlarning massa ulushlari nisbati quyidagicha $\mathrm{Na}-46: \mathrm{Cl}-71: 0-96$ bo'lsa, noma'lum moddaning formulasini toping.\\
A) NaClO\\
B) $\mathrm{NaClO}_{2}$\\
C) $\mathrm{NaClO}_{3}$\\
D) $\mathrm{NaClO}_{4}$
  \item Noma’lum modda tarkibida elementlarning massa ulushlari nisbati quyidagicha $\mathrm{Na}-46$ :Cl-71:0-128 bo'lsa, noma'lum moddaning formulasini toping.\\
A) NaClO\\
B) $\mathrm{NaClO}_{2}$\\
C) $\mathrm{NaClO}_{3}$\\
D) $\mathrm{NaClO}_{4}$
  \item Noma'lum modda tarkibida elementlarning massa ulushlari nisbati quyidagicha $\mathrm{Na}-23: \mathrm{S}-16: \mathrm{O}-32$ bo'lsa, noma'lum moddaning formulasini toping.\\
A) $\mathrm{Na}_{2} \mathrm{~S}_{2} \mathrm{O}_{8}$\\
B) $\mathrm{Na}_{2} \mathrm{~S}_{2} \mathrm{O}_{7}$\\
C) $\mathrm{Na}_{2} \mathrm{SO}_{4}$\\
D) $\mathrm{Na}_{2} \mathrm{~S}_{2} \mathrm{O}_{3}$
  \item Noma'lum modda tarkibida elementlarning massa ulushlari nisbati quyidagicha K-39:H-1:S-32:O-64 bo'lsa, noma'lum moddaning formulasini toping.\\
A) $\mathrm{K}_{2} \mathrm{HSO}_{4}$\\
B) $\mathrm{KHSO}_{4}$\\
C) $\mathrm{KHSO}_{3}$\\
D) $\mathrm{K}_{2} \mathrm{H}_{2} \mathrm{~S}_{2} \mathrm{O}_{8}$
  \item Noma'lum modda tarkibida elementlarning massa ulushlari nisbati quyidagicha K-39:Mn-55:O-64 bo'lsa, noma'lum moddaning formulasini toping.\\
A) $\mathrm{KMnO}_{3}$\\
B) $\mathrm{K}_{2} \mathrm{Mn}_{2} \mathrm{O}_{4}$\\
C) $\mathrm{K}_{2} \mathrm{MnO}_{4}$\\
D) $\mathrm{KMnO}_{4}$
  \item Noma'lum modda tarkibida elementlarning massa ulushlari nisbati quyidagicha $\mathrm{Cu} \cdot 16: \mathrm{N}-7: \mathrm{O}-24$ bo'lsa, noma'lum moddaning formulasini toping.\\
A) $\mathrm{CuNO}_{3}$\\
B) $\mathrm{Cu}\left(\mathrm{NO}_{3}\right)_{2}$\\
C) $\mathrm{Cu}_{2} \mathrm{NO}_{3}$\\
D) $\mathrm{Cu}\left(\mathrm{NO}_{2}\right)_{2}$
  \item Noma’lum modda tarkibida elementlarning massa ulushlari nisbati quyidagicha $\mathrm{Cu} \cdot 16: \mathrm{N}-7: \mathrm{O}-16$ bo'lsa, noma'lum moddaning formulasini toping.\\
A) $\mathrm{CuNO}_{3}$\\
B) $\mathrm{Cu}\left(\mathrm{NO}_{3}\right)_{2}$\\
C) $\mathrm{Cu}_{2} \mathrm{NO}_{3}$\\
D) $\mathrm{Cu}\left(\mathrm{NO}_{2}\right)_{2}$
  \item 60 g noma'lum moddaning 28 g miqdori Fe qolgan qismi S dan iborat bo'lsa, noma'lum modda formulasini toping.\\
A) $\mathrm{FeS}_{2}$\\
B) $\mathrm{Fe}_{2} \mathrm{~S}_{3}$\\
C) $FeS$\\
D) $\mathrm{Fe}_{2} \mathrm{~S}$\\
  \item 44 g noma'lum moddaning 28 g miqdori Fe qolgan qismi S dan iborat bo'lsa, noma'lum modda formulasini toping.\\
A) $\mathrm{FeS}_{2}$\\
B) $\mathrm{Fe}_{2} \mathrm{~S}_{3}$\\
C) $FeS$\\
D) $\mathrm{Fe}_{2} \mathrm{~S}$
  \item 104 g noma'lum moddaning 56 g miqdori Fe qolgan qismi S dan iborat bo'lsa, noma'lum modda formulasini toping.\\
A) $\mathrm{FeS}_{2}$\\
B) $\mathrm{Fe}_{2} \mathrm{~S}_{3}$\\
C) FeS\\
D) $\mathrm{Fe}_{2} \mathrm{~S}$
  \item 71 g noma'lum moddaning 31 g miqdori P qolgan qismi O dan iborat bo'lsa, noma'lum modda formulasini toping.\\
A) $\mathrm{P}_{2} \mathrm{O}_{5}$\\
B) $\mathrm{P}_{2} \mathrm{O}_{4}$\\
C) $\mathrm{P}_{2} \mathrm{O}_{3}$\\
D) $\mathrm{P}_{2} \mathrm{O}_{7}$
  \item 55 g noma'lum moddaning 31 g miqdori P qolgan qismi O dan iborat bo'lsa, noma'lum modda formulasini toping.\\
A) $\mathrm{P}_{2} \mathrm{O}_{5}$\\
B) $\mathrm{P}_{2} \mathrm{O}_{4}$\\
C) $\mathrm{P}_{2} \mathrm{O}_{3}$\\
D) $\mathrm{P}_{2} \mathrm{O}_{7}$
  \item 103 g noma'lum moddaning 32 g miqdori S qolgan qismi Cl dan iborat bo'lsa, noma'lum modda formulasini toping.\\
A) $\mathrm{SCl}_{6}$\\
B) $\mathrm{SCl}_{2}$\\
C) $\mathrm{SCl}_{4}$\\
D) $\mathrm{SCl}_{5}$
  \item 245 g noma'lum moddaning 32 g miqdori S qolgan qismi Cl dan iborat bo'lsa, noma'lum modda formulasini toping.\\
A) $\mathrm{SCl}_{6}$\\
B) $\mathrm{SCl}_{2}$\\
C) $\mathrm{SCl}_{4}$\\
D) $\mathrm{SCl}_{5}$
  \item 87 g noma'lum moddaning 16 g miqdori S qolgan qismi Cl dan iborat bo'lsa, noma'lum modda formulasini toping.\\
A) $\mathrm{SCl}_{6}$\\
B) $\mathrm{SCl}_{2}$\\
C) $\mathrm{SCl}_{4}$\\
D) SCl
  \item 35 g noma'lum moddaning 16 g miqdori S qolgan qismi F dan iborat bo'lsa, noma'lum modda formulasini toping.\\
A) $\mathrm{SF}_{6}$\\
B) $\mathrm{SF}_{2}$\\
C) $\mathrm{SF}_{4}$\\
D) $\mathrm{SF}_{5}$
  \item 73 g noma'lum moddaning 16 g miqdori S qolgan qismi F dan iborat bo'lsa, noma'lum modda formulasini toping.\\
A) $\mathrm{SF}_{6}$\\
B) $\mathrm{SF}_{2}$\\
C) $\mathrm{SF}_{4}$\\
D) $\mathrm{SF}_{5}$
  \item Birinchi guruh elementining gidridi tarkibida noma'lum element va vodorodning massa nisbati 14:2 bo'lsa, bu qaysi elemet?\\
A) Na\\
B) K\\
C) Li\\
D) Rb
  \item Ikkinchi guruh elementining gidridi tarkibida noma'lum element va vodorodning massa nisbati $12: 1$ bo'lsa, bu qaysi elemet?\\
A) Mg\\
B) Ca\\
C) Be\\
D) Sr
  \item Uchinchi guruh elementining gidridi tarkibida noma'lum element va vodorodning massa nisbati 11:3 bo'lsa, bu qaysi elemet?\\
A) Al\\
B) B\\
C) Ga\\
D) In
  \item To'rtinchi guruh elementining gidridi tarkibida noma'lum element va vodorodning massa nisbati 14:2 bo'lsa, bu qaysi elemet?\\
A) C\\
B) Ge\\
C) Si\\
D) Sn
  \item Beshinchi guruh elementining gidridi tarkibida noma'lum element va vodorodning massa nisbati 31:3 bo'lsa, bu qaysi elemet?\\
A) P\\
B) N\\
C) As\\
D) Sb
  \item Yettinchi guruh elementining gidridi tarkibida noma'lum element va vodorodning massa nisbati 19:1 bo'lsa, bu qaysi elemet?\\
A) Cl\\
B) Br\\
C) F\\
D) I
  \item To'rtinchi guruh elementining yuqori oksidi tarkibida noma'lum element va kislorodning massa nisbati 7:8 bo'lsa, bu qaysi element?\\
A) C\\
B) Si\\
C) Ge\\
D) Pb
  \item Beshinchi guruh elementining yuqori oksidi tarkibida noma'lum element va kislorodning massa nisbati 7:20 bo'lsa, bu qaysi element?\\
A) N\\
B) P\\
C) As\\
D) Sb
  \item Oltinchi guruh elementining yuqori oksidi tarkibida noma'lum element va kislorodning massa nisbati $32: 12$ bo'lsa, bu qaysi element?\\
A) O\\
B) S\\
C) Te\\
D) Po
  \item Yettinchi guruh elementining yuqori oksidi tarkibida noma'lum element va kislorodning massa nisbati 40:28 bo'lsa, bu qaysi element?\\
A) F\\
B) Cl\\
C) I\\
D) Br
  \item 1.7 g nomalum modda yonganda 3 g NO va $2,7 \mathrm{~g} \mathrm{H}_{2} \mathrm{O}$ hosil bo'ldi. Noma'lum moddani formulasini aniqlang.\\
A) $\mathrm{NH}_{3}$\\
B) $\mathrm{NH}_{4} \mathrm{OH}$\\
C) $\mathrm{NH}_{2} \mathrm{OH}$\\
D) $\mathrm{NH}_{4} \mathrm{NO}_{3}$
62. 32 g noma'lum modda yonganda 88 g $\mathrm{CO}_{2}$ va $72 \mathrm{~g} \mathrm{H}_{2} \mathrm{O}$ hosil bo'ldi. Noma'lum moddani formulasini toping.\\
A) $\mathrm{CH}_{4}$\\
B) $\mathrm{C}_{2} \mathrm{H}_{6}$\\
C) $\mathrm{CH}_{3} \mathrm{OH}$\\
D) $\mathrm{C}_{3} \mathrm{H}_{6}$\\
63. 13 g noma'lum modda yonganda 44 g $\mathrm{CO}_{2}$ va 9 g $\mathrm{H}_{2} \mathrm{O}$ hosil bo'ldi. Noma'lum moddani formulasini toping.\\
A) $\mathrm{C}_{2} \mathrm{H}_{4}$\\
B) $\mathrm{CH}_{4}$\\
C) $\mathrm{C}_{2} \mathrm{H}_{2}$\\
D) $\mathrm{C}_{3} \mathrm{H}_{8}$\\
64. 23 g noma'lum modda yonganda 44 g $\mathrm{CO}_{2}$ va $27 \mathrm{~g} \mathrm{H}_{2} \mathrm{O}$ hosil bo'ldi. Noma'lum moddani formulasini toping.\\
A) $\mathrm{CH}_{3} \mathrm{OH}$\\
B) HCHO\\
C) HCOOH\\
D) $\mathrm{C}_{2} \mathrm{H}_{5} \mathrm{OH}$\\
65. 32 g noma'lum modda yonganda 44 g $\mathrm{CO}_{2}$ va $36 \mathrm{~g} \mathrm{H}_{2} \mathrm{O}$ hosil bo'ldi. Noma'lum moddani formulasini toping.\\
A) $\mathrm{C}_{2} \mathrm{H}_{5} \mathrm{OH}$\\
B) $\mathrm{CH}_{3} \mathrm{OH}$\\
C) $\mathrm{CH}_{3} \mathrm{COOH}$\\
D) $HCHO$\\
66. 24 g noma'lum modda yonganda 16 g $\mathrm{Fe}_{2} \mathrm{O}_{3}$ va $25,6 \mathrm{~g} \mathrm{SO}_{2}$ hosil bo'ldi. Noma'lum moddani formulasini toping.\\
A) FeS\\
B) $\mathrm{FeSO}_{4}$\\
C) $\mathrm{FeSO}_{3}$\\
D) $\mathrm{FeS}_{2}$\\
67. 34 g noma'lum modda yonganda 64 g $\mathrm{SO}_{2}$ va $18 \mathrm{~g} \mathrm{H}_{2} \mathrm{O}$ hosil bo'ldi. Noma'lum moddani formulasini toping.\\
A) $\mathrm{H}_{2} \mathrm{~S}$\\
B) $\mathrm{H}_{2} \mathrm{SO}_{3}$\\
C) $\mathrm{H}_{2} \mathrm{SO}_{4}$\\
D) $\mathrm{H}_{2} \mathrm{~S}_{2} \mathrm{O}_{7}^{-}$\\
68. 34 g noma'lum modda yonganda 28 g $\mathrm{N}_{2}$ va $54 \mathrm{~g} \mathrm{H}_{2} \mathrm{O}$ hosil bo'ldi. Noma'lum moddani formulasini toping.\\
A) $\mathrm{NH}_{2} \mathrm{OH}$\\
B) $\mathrm{NH}_{4} \mathrm{OH}$\\
C) $\mathrm{NH}_{3}$\\
D) $\mathrm{NH}_{4} \mathrm{NO}_{3}$\\
69. 15 g noma'lum modda yonganda $2,8 \mathrm{~g}$ $\mathrm{N}_{2}, 17,6 \mathrm{~g} \mathrm{CO}_{2}$ va $9 \mathrm{~g} \mathrm{H}_{2} \mathrm{O}$ hosil bo'ldi.\\
Noma'lum moddani formulasini toping.\\
A) $\mathrm{CO}\left(\mathrm{NH}_{2}\right)_{2}$\\
B) $\mathrm{CH}_{3} \mathrm{NO}_{2}$\\
C) $\mathrm{C}_{2} \mathrm{H}_{5} \mathrm{NO}_{2}$\\
D) $\mathrm{C}_{3} \mathrm{H}_{8} \mathrm{NO}_{2}$\\
70. 54 g noma'lum modda yonganda 88 g $\mathrm{CO}_{2}, 28 \mathrm{~g} \mathrm{~N}_{2}$ va $18 \mathrm{~g} \mathrm{H}_{2} \mathrm{O}$ hosil bo'ldi. Noma'lum moddani formulasini toping.\\
A) $\mathrm{CH}_{3} \mathrm{NO}_{2}$\\
B) $\mathrm{CO}\left(\mathrm{NH}_{2}\right)_{2}$\\
C) HOCN\\
D) HCN
  \item 9 g suv tarkibida necha g kislorod mavjud?\\
A) 1\\
B) 8\\
C) 16\\
D) 32\\
  \item 36 g suv tarkibida necha g vodorod mavjud?\\
A) 1\\
B) 2\\
C) 36\\
D) 4
  \item 72 g suv tarkibida necha g kislorod mavjud?\\
A) 64\\
B) 32\\
C) 48\\
D) 74
  \item $5,6 \mathrm{~g} \mathrm{KOH}$ tarkíbida necha g kaliy mavjud?\\
A) 5,2\\
B) 3,9\\
C) 1,6\\
D) 0,1
  \item 80 g NaOH tarkibida necha g natriy mavjud?\\
A) 23\\
B) 36\\
C) 46\\
D) 56
  \item $4,9 \mathrm{~g} \mathrm{H}_{2} \mathrm{SO}_{4}$ tarkibida necha g kislorod mavjud?\\
A) 3,0\\
B) 6,4\\
C) 16\\
D) 3,2
  \item $500 \mathrm{~g} \mathrm{CuSO}{ }_{4} \cdot 5 \mathrm{H}_{2} \mathrm{O}$ tarkibida necha g kislorod mavjud?\\
A) 144\\
B) 250\\
C) 64\\
D) 288
  \item $22,2 \mathrm{~g}(\mathrm{CuOH})_{2} \mathrm{CO}_{3}$ tarkibida necha g kislorod mavjud?\\
A) 2\\
B) 4\\
C) 8\\
D) 7
  \item $121 \mathrm{~g} \mathrm{FeSO}_{4} \cdot 5 \mathrm{H}_{2} \mathrm{O}$ tarkibida necha g kislorod mavjud?\\
A) 12,6\\
B) 36\\
C) 77\\
D) 72
  \item $572 \mathrm{~g} \mathrm{Na}_{2} \mathrm{CO}_{3} \cdot 10 \mathrm{H}_{2} \mathrm{O}$ tarkibida necha g kislorod mavjud?\\
A) 416\\
B) 208\\
C) 180\\
D) 360
  \item 32 g kislorod saqlagan suvni massasini aniqlang.\\
A) 9\\
B) 18\\
C) 32\\
D) 36
  \item 8 g kislorod saqlagan KOH massasini aniqlang.\\
A) 56\\
B) 28\\
C) 40\\
D) 5,6
  \item 23 g natriy necha $\mathrm{g} \mathrm{Na}_{2} \mathrm{CO}_{3}$ tarkibida bo'ladi?\\
A) 106\\
B) 212\\
C) 53\\
D) 48
  \item $6,4 \mathrm{~g}$ kislorod necha $\mathrm{g} \mathrm{H}_{2} \mathrm{SO}_{4}$ tarkibida bo'ladi?\\
A) 98\\
B) 49\\
C) 4,9\\
D) 9,8
  \item 31 g fosfor necha g fosfat kislota $\mathrm{H}_{3} \mathrm{PO}_{4}$ tarkibida bo'ladi?\\
A) 98\\
B) 49\\
C) 4,9\\
D) 9,8
  \item 14 g azot necha $\mathrm{g} \mathrm{NH}_{3}$ tarkibida bo'ladi?\\
A) 34\\
B) 49\\
C) 4,9\\
D) 17
  \item 6 g uglerod necha $\mathrm{g} \mathrm{CaCO}_{3}$ tarkibida bo'ladi?\\
A) 50\\
B) 100\\
C) 10\\
D) 150
  \item 8 g kislorod necha $\mathrm{g} \mathrm{H}_{2} \mathrm{O}_{2}$ tarkibida bo'ladi?\\
A) 34\\
B) 4.9\\
C) 8.5\\
D) 17
  \item 2 g vodorod necha $\mathrm{g} \mathrm{H}_{2} \mathrm{SO}_{4}$ tarkibida bo'ladi?\\
A) 98\\
B) 49\\
C) 4,9\\
D) 17
  \item 24 g uglerod necha $\mathrm{g} \mathrm{H}_{2} \mathrm{CO}_{3}$ tarkibida bo'ladi?\\
A) 62\\
B) 49\\
C) 124\\
D)
  \item 36 g suv tarkibidagidek kislorod necha $\mathrm{g} \mathrm{H}_{2} \mathrm{SO}_{4}$ tarkibida bo'ladi?\\
A) 98\\
B) 4,9\\
C) 49\\
D) 196
  \item $34 \mathrm{~g} \mathrm{NH}_{3}$ tarkibidagidek vodorod necha g suv tarkibida bo'ladi?\\
A) 54\\
B) 36\\
C) 96\\
D) 72
  \item $1,84 \mathrm{~g} \mathrm{CaCO}_{3} \cdot \mathrm{MgCO}_{3}$ tarkibidagidek kislorod necha $\mathrm{g} \mathrm{H}_{3} \mathrm{PO}_{4}$ tarkibida bo'ladi?\\
A) 1,47\\
B) 9.8\\
C) 4,9\\
D) 40
  \item $37 \mathrm{~g} \mathrm{Ca}(\mathrm{OH})_{2}$ tarkibidagidek vodorod necha g NaOH tarkibida bo'ladi?\\
A) 20\\
B) 40\\
C) 60\\
D) 80
  \item $25 \mathrm{~g} \mathrm{CaCO}_{3}$ tarkibidagidek kislorod necha g suv tarkibida bo'ladi?\\
A) 18\\
B) 16,2\\
C) 13.5\\
D) 36
  \item $18 \mathrm{~g} \mathrm{H}_{2} \mathrm{O}$ tarkibidagidek vodorod necha $\mathrm{g} \mathrm{H}_{2} \mathrm{SO}_{4}$ tarkibida bo'ladi?\\
A) 34\\
B) 98\\
C) 4,9\\
D) 17
  \item $50 \mathrm{~g} \mathrm{CaCO}_{3}$ tarkibidagidek kalsiy necha $\mathrm{g} \mathrm{Ca}(\mathrm{OH})_{2}$ tarkibida bo'ladi?\\
A) 37\\
B) 74\\
C) 22.5\\
D) 18
  \item $126 \mathrm{~g} \mathrm{HNO}_{3}$ tarkibidagidek vodorod necha $\mathrm{g} \mathrm{H}_{2} \mathrm{~S}$ tarkibida bo'ladi?\\
A) 35\\
B) 34\\
C) 20\\
D) 17
  \item $98 \mathrm{~g} \mathrm{H}_{3} \mathrm{PO}_{4}$ tarkibidagidek fosfor necha $\mathrm{g} \mathrm{P}_{2} \mathrm{O}_{5}$ tarkibida bo'ladi?\\
A) 45\\
B) 142\\
C) 35\\
D) 71
  \item 56 g KOH tarkibidagidek kaliy necha g $\mathrm{K}_{2} \mathrm{CO}_{3}$ tarkibida bo'ladi?\\
A) 69\\
B) 49\\
C) 124\\
D) 31
  \item $\mathrm{A}+\mathrm{B} \rightarrow \mathrm{C}+\mathrm{D}$ reaksiyasi bo'yicha 10 g A modda 20 g B modda bilan to'liq reaksiyaga kirishdi natijada 12 g C modda hosil bo'lsa, necha g D modda olingan?\\
A) 18\\
B) 22\\
C) 12\\
D) 3\\
  \item $\mathrm{A}+\mathrm{B} \rightarrow \mathrm{C}+\mathrm{D}$ reaksivasi bo'yicha 10 g A modda 20 g B modda bilan to'liq reaksiyaga kirishdi natijada 18 g C modda hosil bo'lsa, necha g D modda olingan?\\
A) 18\\
B) 22\\
C) 12\\
D) 3
  \item $\mathrm{A}+\mathrm{B} \rightarrow \mathrm{C}+\mathrm{D}$ reaksiyasi bo'yicha 10 g A modda 20 g B modda bilan to'liq reaksiyaga kirishdi natijada 27 g C modda hosil bo'lsa, necha g D modda olingan?\\
A) 18\\
B) 22\\
C) 12\\
D) 3
  \item $\mathrm{A}+\mathrm{B} \rightarrow \mathrm{C}+\mathrm{D}$ reaksiyasi bo'yicha 10 g A modda 20 g B modda bilan to'liq reaksiyaga kirishdi natijada 8 g C modda hosil bo'lsa, necha g D modda olingan?\\
A) 18\\
B) 22\\
C) 12\\
D) 3
  \item $\mathrm{A}+\mathrm{B} \rightarrow \mathrm{C}+\mathrm{D}$ reaksiyasi bo'yicha 20 g A modda 20 g B modda bilan to'liq reaksiyaga kirishdi natijada 12 g C modda hosil bo'lsa, necha g D modda olingan?\\
A) 28\\
B) 22\\
C) 18\\
D) 33
  \item $\mathrm{A}+\mathrm{B} \rightarrow \mathrm{C}+\mathrm{D}$ reaksiyasi bo'yicha 20 g A modda 20 g B modda bilan to'liq reaksiyaga kirishdi natijada 30 g C modda hosil bo'lsa, necha g D modda olingan?\\
A) 15\\
B) 21\\
C) 10\\
D) 3
  \item $\mathrm{A}+\mathrm{B} \rightarrow \mathrm{C}+\mathrm{D}$ reaksiyasi bo'yicha 20 g A modda 20 g B modda bilan to'liq reaksiyaga kirishdi natijada 22 g C modda hosil bo'lsa, necha g D modda olingan?\\
A) 18\\
B) 22\\
C) 12\\
D) 3
  \item $\mathrm{A}+\mathrm{B} \rightarrow \mathrm{C}+\mathrm{D}$ reaksiyasi bo'yicha 20 g A modda 20 g B modda bilan to'liq reaksiyaga kirishdi natijada 37 g C modda hosil bo'lsa, necha g D modda olingan?\\
A) 18\\
B) 22\\
C) 12\\
D) 3
  \item $\mathrm{A}+\mathrm{B} \rightarrow \mathrm{C}+\mathrm{D}$ reaksiyasi bo'yicha 30 g A modda 20 g B modda bilan to'liq reaksiyaga kirishdi natijada 25 g C modda hosil bo'lsa, necha g D modda olingan?\\
A) 18\\
B) 25\\
C) 12\\
D) 3
  \item $\mathrm{A}+\mathrm{B} \rightarrow \mathrm{C}+\mathrm{D}$ reaksiyasi bo'yicha 10 g A modda 20 g , B modda bilan to'liq reaksiyaga kirishdi natijada 19 g C modda hosil bo'lsa, necha g D modda olingan?\\
A) 18\\
B) 32\\
C) 12\\
D) 11
  \item $50 \mathrm{~g} \mathrm{CaCO}_{s}$ necha mol keladi?\\
A) 0.3\\
B) 0.5\\
C) 1.2\\
D) 3\\
  \item $36 \mathrm{~g} \mathrm{H}-\mathrm{O}$ necha mol keladi"?\\
A) 0,3\\
B) 0.5\\
C) 2\\
D) 3
  \item 71 g Clo necha mol keladi?\\
A) 3\\
B) 5\\
C) 2\\
D) 1
  \item $132 \mathrm{~g} \mathrm{~N}_{2} \mathrm{O}$ necha mol keladi?\\
A) 0,3\\
B) 0.5\\
C) 2\\
D) 3
  \item $31.5 \mathrm{~g} \mathrm{HNO}_{3}$ necha mol keladi?\\
A) 0.3\\
B) 0.5\\
C) 2\\
D) 3
  \item $171 \mathrm{~g} \mathrm{Alo}_{2}\left(\mathrm{SO}_{4}\right)_{3}$ necha mol keladi?\\
A) 0,3\\
B) 0,5\\
C) 2\\
D) 3
  \item $468 \mathrm{~g} \mathrm{Al}_{2}\left(\mathrm{CO}_{3}\right)_{3}$ necha mol keladi?\\
A) 0.3\\
B) 0.5\\
C) 2\\
D) 3
  \item $186 \mathrm{~g} \mathrm{H}_{2} \mathrm{CO}_{3}$ necha mol keladi?\\
A) 0,3\\
B) 0,5\\
C) 2\\
D) 3
  \item 16.8 g Fe necha mol keladi?\\
A) 0.3\\
B) 0,5\\
C) 2\\
D) 3
  \item 20 g Ne necha mol keladi?\\
A) 0.3\\
B) 0,5\\
C) 1\\
D) 3
  \item $0,3 \mathrm{~mol}$ miqdordagi $\mathrm{Ca}(\mathrm{OH})_{2}$ massasini (g) hisoblang.\\
A) $22,2 \mathrm{~g}$\\
B) 30 g\\
C) $16,8 \mathrm{~g}$\\
D) 167 g
  \item $0,11 \mathrm{~mol}$ miqdordagi $\mathrm{MgSO}_{4} \cdot 7 \mathrm{H}_{2} \mathrm{O}$ massasini (g) hisoblang.\\
A) $27,06 \mathrm{~g}$\\
B) 100 g\\
C) $67,06 \mathrm{~g}$\\
D) 44 g 13. $0,4 \mathrm{~mol}$ miqdordagi $\mathrm{CuSO}_{4} \cdot 5 \mathrm{H}_{2} \mathrm{O}$ massasini (g) hisoblang.\\
A) $27,06 \mathrm{~g}$\\
B) 64 g\\
C) 100 g\\
D) $66,33 \mathrm{~g}$
  \item $0,2 \mathrm{~mol}$ miqdordagi $\mathrm{Na}_{2} \mathrm{SO}_{4} \cdot 10 \mathrm{H}_{2} \mathrm{O}$ massasini (g) hisoblang.\\
A) $27,06 \mathrm{~g}$\\
B) $23,04 \mathrm{~g}$\\
C) 44 g\\
D) $64,4 \mathrm{~g}$
  \item 2 mol miqdordagi $\mathrm{Cu}(\mathrm{OH})_{2} \cdot \mathrm{CuCO}_{3}$ massasini (g) hisoblang.\\
A) 44 g\\
B) 444 g\\
C) 222 g\\
D) 22 g
  \item $0,3 \mathrm{~mol}$ miqdordagi CaO massasini (g) hisoblang.\\
A) $22,2 \mathrm{~g}$\\
B) 30 g\\
C) $16,8 \mathrm{~g}$\\
D) 167 g
  \item $0,4 \mathrm{~mol}$ miqdordagi $\mathrm{KClO}_{3}$ massasini (g) hisoblang\\
A) 49 g\\
B) 40.9 g\\
C) 9.4 g\\
D) 90.4 g
  \item 0.22 mol miqdordagi H .0 O massasini hisoblang\\
A) 39.6 g\\
B) 3.96 g\\
C) 19.8 g\\
D) 1.98 g
  \item 1.1 mol miqdordagi $\mathrm{H}_{2} \mathrm{O}$ massasini hisoblange\\
A) 39.6 g\\
B) 3.96 g\\
C) 19.8 g\\
D) $1,98 \mathrm{~g}$
  \item 1.9 mol miqdordagi $\mathrm{CH}_{3} \mathrm{COOH}$ massasini (g) hisoblang\\
A) 114 g\\
B) 118 g\\
C) 60 g\\
D) 120 g
  \item $0,2 \mathrm{~mol}$ miqdorining massasi 8.8 g bo'lsa, ushbu moddaning molyar massasini toping.\\
A) $8.6 \mathrm{~g} / \mathrm{mol}$\\
B) $1,76 \mathrm{~g} / \mathrm{mol}$\\
C) $44 \mathrm{~g} / \mathrm{mol}$\\
D) $66 \mathrm{~g} / \mathrm{mol}$
  \item $0,25 \mathrm{~mol}$ miqdordagi moddaning massasi 11 g bo'lsa, ushbu moddaning molyar massasini toping.\\
A) $10,75 \mathrm{~g} / \mathrm{mol}$\\
B) $2,75 \mathrm{~g} / \mathrm{mol}$\\
C) $44 \mathrm{~g} / \mathrm{l}$\\
D) $22 \mathrm{~g} / \mathrm{mol}$
  \item 0,2 mol miqdorining massasi $3,6 \mathrm{~g}$ bo'lsa, ushbu moddaning molyar massasini toping.\\
A) $3.4 \mathrm{~g} / \mathrm{mol}$\\
B) $0,72 \mathrm{~g} / \mathrm{mol}$\\
C) $18 \mathrm{~g} / \mathrm{mol}$\\
D) $36 \mathrm{~g} / \mathrm{mol}$
  \item $0,47 \mathrm{~mol}$ miqdorining massasi 47 g bo'lgan moddani toping.\\
A) $\mathrm{H}_{2} \mathrm{SO}_{4}$\\
B) $\mathrm{H}_{3} \mathrm{PO}_{3}$\\
C) $\mathrm{HNO}_{3}$\\
D) $\mathrm{CaCO}_{3}$
  \item $0,5 \mathrm{~mol}$ miqdorining massasi 49 g bo'lgan moddani toping.\\
A) $\mathrm{H}_{2} \mathrm{SO}_{4}$\\
B) $\mathrm{H}_{3} \mathrm{PO}_{3}$\\
C) $\mathrm{HNO}_{3}$\\
D) $\mathrm{CaCO}_{3}$
  \item $0,2 \mathrm{~mol}$ modda $6,4 \mathrm{~g}$ bo'lsa, shu moddaning molyar massasini toping.\\
A) $32 \mathrm{~g} / \mathrm{mol}$\\
B) $44 \mathrm{~g} / \mathrm{mol}$\\
C) $22 \mathrm{~g} / \mathrm{mol}$\\
D) $16 \mathrm{~g} / \mathrm{mol}$
  \item $20,4 \mathrm{~g}$ gaz moddaning miqdori $0,6 \mathrm{~mol}$ bolsa, uning molyar massasini toping.\\
A) $17 \mathrm{~g} / \mathrm{mol}$\\
B) $32 \mathrm{~g} / \mathrm{mol}$\\
C) $34 \mathrm{~g} / \mathrm{mol}$\\
D) $44 \mathrm{~g} / \mathrm{mol}$
  \item $2,5 \mathrm{~mol}$ gazning massasi 110 g ga teng ekanligi malum bo'lsa,, shu gazning molyar massasi qanchaga teng bo'lishini hisoblang.\\
A) $17 \mathrm{~g} / \mathrm{mol}$\\
B) $44 \mathrm{~g} / \mathrm{mol}$\\
C) $64 \mathrm{~g} / \mathrm{mol} \quad$ D) $32 \mathrm{~g} / \mathrm{mol}$
  \item $9,8 \mathrm{~g}$ modda $0,1 \mathrm{~mol}$ bo'lsa, uning $\mathrm{M}_{\mathrm{r}}$ qiymatini aninqlang.\\
A) 80\\
B) 64\\
C) 85\\
D) 98
  \item 8 g qattiq modda miqdori $0,2 \mathrm{~mol}$. Shu moddaning $\mathrm{M}_{\mathrm{r}}$ qiymatini aninqlang.\\
A) 40\\
B) 44\\
C) 64\\
D) 32
  \item 11.2 litrda o'lchangan gaz necha mol keladi?\\
A) 0.5\\
B) 1\\
C) 0.3\\
D) 2\\
  \item 6,72 litrda o'lchangan gaz necha mol keladi?\\
A) 0,5\\
B) 1\\
C) 0,3\\
D) 2
  \item 22,4 litrda o'lchangan gaz necha mol keladi?\\
A) 0,5\\
B) 1\\
C) 0,3\\
D) 2
  \item 44,8 litrda o'lchangan gaz necha mol keladi?\\
A) 0,5\\
B) 1\\
C) 0,3\\
D) 2
  \item 11,2 litrda o'lchangan $\mathrm{H}_{2}$ gazi necha mol keladi?\\
A) 0,5\\
B) 1\\
C) 0,3\\
D) 2
  \item 11,2 litrda o'lchangan $\mathrm{Cl}_{2}$ gazi necha mol keladi?\\
A) 0,5\\
B) 1\\
C) 0.3\\
D) 2
  \item 67,2 litrda o'lchangan $\mathrm{F}_{2}$ gazi necha mol keladi?\\
A) 5\\
B) 1\\
C) 3\\
D) 2
  \item 112 litrda o'lchangan $\mathrm{N}_{2}$ gazi necha mol keladi?\\
A) 5\\
B) 1\\
C) 3\\
D) 2
  \item 224 litrda o'lchangan $\mathrm{Br}_{2}$ gazi necha mol keladi?\\
A) 5\\
B) 10\\
C) 3\\
D) 2
  \item 44,8 litrda o'lchangan $\mathrm{N}_{2}$ gazi necha mol keladi?\\
A) 5\\
B) 1\\
C) 3\\
D) 2
  \item 67,2 litr $\mathrm{Cl}_{2}$ massasini (g) toping.\\
A) 71\\ 
B) 106,5\\
C) 213\\
D) 426
  \item 44,8 litr $\mathrm{Cl}_{2}$ massasini (g) toping.\\
A) 71\\
B) 142\\
C) 213\\
D) 426
  \item 134,4 litr $\mathrm{Cl}_{2}$ massasini (g) toping.\\
A) 71\\
B) 106,5\\
C) 213\\
D) 426
  \item 6,72 litr $\mathrm{N}_{2}$ massasini (g) toping.\\
A) 28\\
B) 84\\
C) 8,4\\
D) 42
  \item 67,2 litr $\mathrm{N}_{2}$ massasini (g) toping.\\
A) 28\\
B) 84\\
C) 8,4\\
D) 42
  \item 2,24 litr $\mathrm{NH}_{3}$ massasini (g) toping.\\
A) 17\\
B) 3,4\\
C) 1,7\\
D) 5,1
  \item 4,48 litr $\mathrm{NH}_{3}$ massasini (g) toping.\\
A) 17\\
B) 3,4\\
C) 1,7\\
D) 5,1
  \item 2,24 litr $\mathrm{CO}_{2}$ massasini (g) toping.\\
A) 4,4\\
B) 44\\
C) 2,2\\
D) 28
  \item 22,4 litr CO massasini (g) toping.\\
A) 28\\
B) 44 C) 2,8\\
D) 5,1
  \item 3,36 litr $\mathrm{H}_{2}$ massasini (g) toping.\\
A) 3\\
B) 1,5\\
C) 0,15\\
D) 0,3
  \item Agar n.sh.da $3,2 \mathrm{~g}$ gaz 2,24 litr hajmni egallasa, shu gazning nisbiy molekulyar massasini aniqlang.\\
A) 71\\
B) 32\\
C) 28\\
D) 64
  \item Agar n.sh.da $2,8 \mathrm{~g}$ gaz 2.24 litr hajmni egallasa, shu gazning nisbiy molekulyar massasini aniqlang.\\
A) 71\\
B) 32\\
C) 28\\
D) 64
  \item Agar n.sh.da $6,4 \mathrm{~g}$ gaz 4,48 litr hajmni egallasa, shu gazning nisbiy molekulyar massasini aniqlang.\\
A) 71\\
B) 32\\
C) 28\\
D) 64
  \item Agar n.sh.da $8,4 \mathrm{~g}$ gaz 6,72 litr hajmni egallasa, shu gazning nisbiy molekulyar massasini aniqlang.\\
A) 71\\
B) 32\\
C) 28\\
D) 64
  \item Agar n.sh.da $3,2 \mathrm{~g}$ gaz 1,12 litr hajmni egallasa, shu gazning nisbiy molekulyar massasini aniqlang.\\
A) 71\\
B) 32\\
C) 28\\
D) 6 -
  \item Agar n.sh.da 16 g gaz 11.2 litr hajmni egallasa, shu gazning nisbiy molekulyar massasini aniqlang.\\
A) 71\\
B) 32\\
C) 28\\
D) 64
  \item Agar n.sh.da $35,5 \mathrm{~g}$ gaz $11 ; 2$ litr hajmni egallasa, shu gazning nisbiy molekulyar massasini aniqlang.\\
A) 71\\
B) 32\\
C) 28\\
D) 64
  \item Agar n.sh.da 7 g gaz 5,6 litr hajmni egallasa, shu gazning nisbiy molekulyar massasini aniqlang.\\
A) 71\\
B) 32\\
C) 28\\
D) 64
  \item Agar n.sh.da 16 g gaz 5,6 litr hajmni egallasa, shu gazning nisbiy molekulyar massasini aniqlang.\\
A) 71\\
B) 32\\
C) 28\\
D) 64
  \item Agar n.sh.da 8 g gaz 5,6 litr hajmni egallasa, shu gazning nisbiy molekulyar massasini aniqlang.\\
A) 71\\
B) 32\\
C) 28\\
D) 64
  \item $25,6 \mathrm{~g} \mathrm{SO}_{2}$ hajmini litrda toping.\\
A) 4,48\\
B) 2,24\\
C) 8,96\\
D) 11,2
  \item $32 \mathrm{~g} \mathrm{SO}_{2}$ hajmini litrda toping.\\
A) 4,48\\
B) 2,24\\
C) 8,96\\
D) 11,2
  \item $8,8 \mathrm{~g} \mathrm{CO}_{2}$ hajmini litrda toping.\\
A) 4,48\\
B) 2,24\\
C) 8,96\\
D) 11,2
  \item $23 \mathrm{~g} \mathrm{NO}_{2}$ hajmini litrda toping.\\
A) 4,48\\
B) 2,24\\
C) 8,96\\
D) 11,2
  \item $25,6 \mathrm{~g} \mathrm{O}_{2}$ hajmini litrda toping.\\
A) 4,48\\
B) 22,4\\
C) 8,96\\
D) 17,92
  \item $15,4 \mathrm{~g} \mathrm{CO}_{2}$ hajmini litrda toping.\\
A) 4,47\\
B) 22,4\\
C) 7.84\\
D) 17,92
  \item $8,5 \mathrm{~g} \mathrm{H}_{2} \mathrm{~S}$ necha litr (\href{http://n.sh}{n.sh}) hajmni egallaydi?\\
A) 5,6\\
B) 22,4\\
C) 0,56\\
D) 4,48
  \item $11 \mathrm{~g} \mathrm{CO}_{2}$ hajmini litrda toping.\\
A) 1,12\\
B) 5,6\\
C) 8,96\\
D) 11,2
  \item $2,8 \mathrm{~g} \mathrm{CO}$ hajmni litrda toping.\\
A) 1,12\\
B) 2,24\\
C) 8,96\\
D) 5,6
  \item $6,4 \mathrm{~g}$ kislorod hajmni litrda toping.\\
A) 4,48\\
B) 2,24\\
C) 33,6\\
D) 8,96
  \item $\mathrm{SiH}_{4}$ ning 2 mol miqdorida nechta molekula bor?\\
A) $3,01 \cdot 10^{24}$\\
B) $12,04 \cdot 10^{23}$\\
C) $24,08 \cdot 10^{23}$\\
D) $3,01 \cdot 10^{23}$\\
  \item NaCl ning 2 mol miqdorida nechta molekula bor?\\
A) $3,01 \cdot 10^{2,4}$\\
B) $12,04 \cdot 10^{23}$\\
C) $24,08 \cdot 10^{23}$\\
D) $3,01 \cdot 10^{23}$
  \item KCl ning 4 mol miqdorida nechta molekula bor?\\
A) $3,01 \cdot 10^{24}$\\
B) $12,04 \cdot 10^{23}$\\
C) $24,08 \cdot 10^{23}$\\
D) $3,01 \cdot 10^{23}$
  \item NaBr ning $0,5 \mathrm{~mol}$ miqdorida nechta molekula bor?\\
A) $3,01 \cdot 10^{24}$\\
B) $12,04 \cdot 10^{23}$\\
C) $24,08 \cdot 10^{23}$\\
D) $3,01 \cdot 10^{23}$
  \item $1,2 \mathrm{~mol}$ suvdagi molekulalar sonini toping.\\
A) $7,224 \cdot 10^{23}$\\
B) $3,01 \cdot 10^{23}$\\
C) $6,02 \cdot 10^{23}$\\
D) $7,224 \cdot 10^{24}$
  \item $0,5 \mathrm{~mol}$ ammiakdagi $\left(\mathrm{NH}_{3}\right)$ molekulalar sonini aniqlang.\\
A) $0,602 \cdot 10^{23}$\\
B) $12,04 \cdot 10^{23}$\\
C) $3,01 \cdot 10^{23}$\\
D) $6,02 \cdot 10^{23}$
  \item $0,4 \mathrm{~mol}$ metanda $\left(\mathrm{CH}_{4}\right)$ nechta molekula bor?\\
A) $6,02 \cdot 10^{23}$\\
B) $3,01 \cdot 10^{23}$\\
C) $12,04 \cdot 10^{23}$\\
D) $2,408 \cdot 10^{23}$
  \item 4 mol natriy oksiddagi molekulalar sonini toping.\\
A) $0,602 \cdot 10^{23}$\\
B) $24,08 \cdot 10^{23}$\\
C) $3,01 \cdot 10^{23}$\\
D) $7,224 \cdot 10^{22}$
  \item $0,1 \mathrm{~mol}$ kisloroddagi molekulalar sonini aniqlang.\\
A) $3,01 \cdot 10^{23}$\\
B) $12,04 \cdot 10^{23}$\\
C) $0.602 \cdot 10^{23}$\\
D) $6,02 \cdot 10^{23}$
  \item $0.8 \mathrm{~mol} \mathrm{SO}_{2}$ dagi molekulalar sonini aniqlang.\\
A) $3,01 \cdot 10^{23}$\\
B) $4,816 \cdot 10^{23}$\\
C) $0,602 \cdot 10^{23}$\\
D) $6,02 \cdot 10^{23}$
  \item NaBr ning 2 molida nechta atom bor?\\
A) $3,01 \cdot 10^{24}$\\
B) $12,04 \cdot 10^{23}$\\
C) $24,08 \cdot 10^{23}$\\
D) $3,01 \cdot 10^{23}$
  \item $\mathrm{SO}_{3}$ ning 1,5 molida nechta atom bor?\\
A) $9,03 \cdot 10^{23}$\\
B) $42,14 \cdot 10^{23}$\\
C) $24,08 \cdot 10^{23}$\\
D) $36.12 \cdot 10^{23}$
  \item $\mathrm{HNO}_{3}$ ning 4 molida nechta atom bor?\\
A) $3,01 \cdot 10^{24}$\\
B) $12,04 \cdot 10^{24}$\\
C) $24,08 \cdot 10^{23}$\\
D) $3.01 \cdot 10^{23}$
  \item $2,5 \mathrm{~mol}$ suvda nechta vodorod atomi bor?\\
A) $3,01 \cdot 10^{24}$\\
B) $12,04 \cdot 10^{23}$\\
C) $24,08 \cdot 10^{23}$\\
D) $3,01 \cdot 10^{23}$
  \item $2 \mathrm{~mol} \mathrm{P}_{4} \mathrm{O}_{10}$ da nechta kislorod atomi bor?\\
A) $6,02 \cdot 10^{23}$\\
B) $12,04 \cdot 10^{24}$\\
C) $18,06 \cdot 10^{23}$\\
D) $9,03 \cdot 10^{24}$
  \item 5 mol suv tarkibidagi atomlar sonini aniqlang.\\
A) $24,08 \cdot 10^{23}$\\
B) $30,1 \cdot 10^{23}$\\
C) $60,2 \cdot 10^{22}$\\
D) $90,3 \cdot 10^{23}$
  \item $0,2 \mathrm{~mol} \mathrm{NH}_{3}$ tarkibidagi vadorod atomlarining sonini aniqlang.\\
A) $3,01 \cdot 10^{23}$\\
B) $1,204 \cdot 10^{23}$\\
C) $3,612 \cdot 10^{23}$\\
D) $6,02 \cdot 10^{23}$
  \item $2,5 \mathrm{~mol} \mathrm{O}_{2}$ da nechta atom bor.\\
A) $30,1 \cdot 10^{23}$\\
B) $60,2 \cdot 10^{23}$\\
C) $24,08 \cdot 10^{23}$\\
D) $90,3 \cdot 10^{23}$
  \item $0,8 \mathrm{~mol} \mathrm{CO}_{2}$ tarkibidagi kislorod atomlarining sonini aniqlang.\\
A) $4,25 \cdot 10^{23}$\\
B) $6,02 \cdot 10^{23}$\\
C) $9,632 \cdot 10^{23}$\\
D) $12,04 \cdot 10^{23}$
  \item $0,1 \mathrm{~mol} \mathrm{O}_{2}$ ning atomlar sonini aniqlang.\\
A) $1,204 \cdot 10^{23}$\\
B) $3,01 \cdot 10^{23}$\\
C) $12,04 \cdot 10^{23}$\\
D) $1,806 \cdot 10^{23}$
  \item $245 \mathrm{~g} \mathrm{KClO}_{3}$ da nechta xlor atomi bor?\\
A) $6,02 \cdot 10^{24}$\\
B) $12,04 \cdot 10^{23}$\\
C) $18,06 \cdot 10^{23}$\\
D) $9,03 \cdot 10^{24}$
  \item $290 \mathrm{~g} \mathrm{Fe}_{2}\left(\mathrm{SO}_{4}\right)_{3} \cdot 10 \mathrm{H}_{2} \mathrm{O}$ da nechta kislorod atomi bor?\\
A) $36,12 \cdot 10^{23}$\\
B) $30,1 \cdot 10^{23}$\\
C) $66,22 \cdot 10^{23}$\\
D) $132,44 \cdot 10^{23}$
  \item $290 \mathrm{~g} \mathrm{Fe}_{2}\left(\mathrm{SO}_{4}\right)_{3} \cdot 10 \mathrm{H}_{2} \mathrm{O}$ da nechta vodorod atomi bor?\\
A) $36,12 \cdot 10^{23}$\\
B) $30,1 \cdot 10^{23}$\\
C) $66,22 \cdot 10^{23}$\\
D) $60,2 \cdot 10^{23}$
  \item 22 g karbonat angidrid tarkibidagi kislorod atomlarining sonini aniqlang.\\
A) $3,01 \cdot 10^{23}$\\
B) $12,04 \cdot 10^{23}$\\
C) $6,02 \cdot 10^{23}$\\
D) $18,06 \cdot 10^{23}$
  \item $1,8 \mathrm{~g} \mathrm{H}_{2} \mathrm{O}$ dagi barcha atomlar sonini aniqlang.\\
A) $1,806 \cdot 10^{23}$\\
B) $1,12 \cdot 10^{23}$\\
C) $2,408 \cdot 10^{23}$\\
D) $1,204 \cdot 10^{23}$
  \item $0,32 \mathrm{~g} \mathrm{SO}_{2}$ tarkibidagi oltingugurt atomlari sonini aniqlang.\\
A) $0,0602 \cdot 10^{2: 3}$\\
B) $0,1204 \cdot 10^{23}$\\
C) $0,0301 \cdot 10^{23}$\\
D) $0,2408 \cdot 10^{23}$
  \item $3,4 \mathrm{~g} \mathrm{NH}_{3}$ dagi jami atomlar sonini aniqlang.\\
A) $2,408 \cdot 10^{23}$\\
B) $4,816 \cdot 10^{23}$\\
C) $1,204 \cdot 10^{23}$\\
D) $1,12 \cdot 10^{23}$
  \item $200 \mathrm{~g} \mathrm{CaCO}_{3}$ tarkibidagi jami atomlar sonini aniqlang.\\
A) $60,2 \cdot 10^{23}$\\
B) $90,3 \cdot 10^{23}$\\
C) $48,16 \cdot 10^{23}$\\
D) $30,1 \cdot 10^{23}$
  \item $17,1 \mathrm{~g} \mathrm{Al}_{2}\left(\mathrm{SO}_{4}\right)_{3}$ jami nechta atomdan tashkil topgan?\\
A) $0,25 \cdot 10^{24}$\\
B) $5,117 \cdot 10^{23}$\\
C) $2,25 \cdot 10^{24}$\\
D) $6,02 \cdot 10^{23}$
  \item $122.5 \mathrm{~g} \mathrm{KClO}_{3}$ da nechta xlor atomi bor?\\
A) $6,02 \cdot 10^{23}$\\
B) $12,04 \cdot 10^{23}$\\
C) $18,06 \cdot 10^{23}$\\
D) $9,03 \cdot 10^{24}$
  \item $3,01 \cdot 10^{23}$ ta vodorod molekulasi n.sh.da necha litr hajmni egallaydi?\\
A) 11,2\\
B) 22,4\\
C) 33,6\\
D) 5,6
102. $9,03 \cdot 10^{23}$ ta vodorod molekulasi n.sh.da necha litr hajmni egallaydi?\\
A) 11,2\\
B) 22,4\\
C) 33,6\\
D) 5,6\\
103. $3,01 \cdot 10^{23}$ ta kislorod molekulasi n.sh.da necha litr hajmni egallaydi?\\
A) 11,2\\
B) 22,4\\
C) 33,6\\
D) 5,6\\
104. $12,04 \cdot 10^{23}$ ta azot molekulasi n.sh.da necha litr hajmni egallaydi?\\
A) 11,2\\
B) 22,4\\
C) 33,6\\
D) 44,8\\
105. $36,12 \cdot 10^{23}$ ta ammiak molekulasi n.sh.da necha litr hajmni egallaydi?\\
A) 134.4\\
B) 22,4\\
C) 33,6\\
D) 5,6\\
106. $3,01 \cdot 10^{23}$ ta molekula saqlagan $\mathrm{H}_{2}$ gazi n.sh.da necha litr hajmni egallaydi?\\
A) 11,2\\
B) 22,4\\
C) 33,6\\
D) 5,6\\
107. $6,02 \cdot 10^{23}$ ta molekula saqlagan $\mathrm{H}_{2}$ gazi n.sh.da necha litr hajmni egallaydi?\\
A) 11,2\\
B) 22,4\\
C) 33,6\\
D) 5,6\\
108. $12,04 \cdot 10^{23}$ ta molekula saqlagan $\mathrm{O}_{2}$ gazi n.sh.da necha litr hajmni egallaydi?\\
A) 11,2\\
B) 44,8\\
C) 33,6\\
D) 5,6\\
109. 18,06 $\cdot 10^{23}$ ta molekula saqlagan $\mathrm{N}_{2}$ gazi n.sh.da necha litr hajmni egallaydi?\\
A) 11,2\\
B) 22,4\\
C) 33,6\\
D) 67,2\\
110. $3,01 \cdot 10^{23}$ ta molekula saqlagan $\mathrm{Cl}_{2}$ gazi n.sh.da necha litr hajmni egallaydi?\\
A) 11,2\\
B) 22,4\\
C) 33,6\\
D) 5,6
  \item $\mathrm{Na}_{2} \mathrm{CO}_{3} \cdot \mathrm{nH}_{2} \mathrm{O}$ tarkibli kristallogidrat tarkibida uglerodning massa ulushi $6 / 143$ bo'lsa, (n) qiymatini aniqlang.\\
A) 5\\
B) 10\\
C) 8\\
D) 9\\
  \item $\mathrm{Na}_{2} \mathrm{SO}_{4} \cdot \mathrm{nH}_{2} \mathrm{O}$ tarkibli kristallogidrat tarkibida oltingugurtning massa ulushi 16/116 bo'lsa, (n) qiymatini aniqlang.\\
A) 5\\
B) 10\\
C) 8\\
D) 9
  \item $\mathrm{Li}_{2} \mathrm{CO}_{3} \cdot \mathrm{nH}_{2} \mathrm{O}$ tarkibli kristallogidrat tarkibida litiyning massa ulushi 7/127 bo'lsa, (n) qiymatini aniqlang.\\
A) 5\\
B) 10\\
C) 8\\
D) 9
  \item $\mathrm{Al}\left(\mathrm{NO}_{3}\right)_{3} \cdot \mathrm{nH}_{2} \mathrm{O}$ tarkibli kristallogidrat tarkibida azotning massa ulushi 14/125 bo'lsa, (n) qiymatini aniqlang.\\
A) 5\\
B) 10\\
C) 8\\
D) 9
  \item $\mathrm{K}_{2} \mathrm{CO}_{3} \cdot \mathrm{nH}_{2} \mathrm{O}$ tarkibli kristallogidrat tarkibida uglerodning massa ulushi 3/57 bo'lsa, (n) qiymatini aniqlang.\\
A) 5\\
B) 10\\
C) 8\\
D) 9
  \item $\mathrm{Na}_{2} \mathrm{CO}_{3} \cdot \mathrm{nH}_{2} \mathrm{O}$ tarkibli kristallogidrat tarkibida natriyning massa ulushi 23/125 bo'lsa, (n) qiymatini aniqlang.\\
A) 5\\
B) 10\\
C) 8\\
D) 9
  \item $\mathrm{Al}_{2}\left(\mathrm{SO}_{4}\right)_{3} \cdot \mathrm{nH}_{2} \mathrm{O}$ tarkibli kristallogidrat tarkibida alyuminiyning massa ulushi 18/144 bo'lsa, (n) qiymatini aniqlang.\\
A) 5\\
B) 10\\
C) 8\\
D) 9
  \item $\mathrm{Na}_{2} \mathrm{SO}_{4} \cdot \mathrm{nH}_{2} \mathrm{O}$ tarkibli kristallogidrat tarkibida oltingugurtning massa ulushi 8/76 bo'lsa, (n) qiymatini aniqlang.\\
A) 5\\
B) 10\\
C) 8\\
D) 9
  \item $\mathrm{MgSO}_{4} \cdot \mathrm{nH}_{2} \mathrm{O}$ tarkibli kristallogidrat tarkibida magniyning massa ulushi $8 / 70$ bo'lsa, (n) qiymatini aniqlang.\\
A) 5\\
B) 10\\
C) 8\\
D) 9
  \item $\mathrm{Al}_{2}\left(\mathrm{SO}_{4}\right)_{3} \cdot \mathrm{nH}_{2} \mathrm{O}$ tarkibli kristallogidrat tarkibida oltingugurtning massa ulushi 24/108 bo'lsa, (n) qiymatini aniqlang.\\
A) 5\\
B) 10\\
C) 8\\
D) 9
  \item $\mathrm{Na}_{2} \mathrm{CO}_{3} \cdot \mathrm{nH}_{2} \mathrm{O}$ tarkibli kristallogidrat tarkibida uglerod va natriyning massa ulushi 29/98 bo'lsa, (n) qiymatini aniqlang.\\
A) 5\\
B) 10\\
C) 8\\
D) 9
12. $\mathrm{Na}_{2} \mathrm{SO}_{4} \cdot \mathrm{nH}_{2} \mathrm{O}$ tarkibli kristallogidrat tarkibida oltingugurt va natriyning massa ulushi 39/161 bo'lsa, (n) qiymatini aniqlang.\\
A) 5\\
B) 10\\
C) 8\\
D) 9\\
13. $\mathrm{Li}_{2} \mathrm{CO}_{3} \cdot \mathrm{nH}_{2} \mathrm{O}$ tarkibli kristallogidrat tarkibida litiy va uglerodning massa ulushi 13/118 bo'lsa, (n) qiymatini aniqlang.\\
A) 5\\
B) 10\\
C) 8\\
D) 9\\
14. $\mathrm{Al}\left(\mathrm{NO}_{3}\right)_{3} \cdot \mathrm{nH}_{2} \mathrm{O}$ tarkibli kristallogidrat tarkibida azot va alyuminiyning massa ulushi 23/101 bo'lsa, (n) qiymatini aniqlang.\\
A) 5\\
B) 10\\
C) 8\\
D) 9\\
15. $\mathrm{K}_{2} \mathrm{CO}_{3} \cdot \mathrm{nH}_{2} \mathrm{O}$ tarkibli kristallogidrat tarkibida uglerod va kaliyning massa ulushi 30/94 bo'lsa, (n) qiymatini aniqlang.\\
A) 5\\
B) 10\\
C) 8\\
D) 9\\
16. $\mathrm{Na}_{2} \mathrm{CO}_{3} \cdot \mathrm{nH}_{2} \mathrm{O}$ tarkibli kristallogidrat tarkibida natriy va uglerodning 29/143 massa bo'lsa, (n) qiymatini aniqlang.\\
A) 5\\
B) 10\\
C) 8\\
D) 9\\
17. $\mathrm{Al}_{2}\left(\mathrm{SO}_{4}\right)_{3} \cdot \mathrm{nH}_{2} \mathrm{O}$ tarkibli kristallogidrat tarkibida alyuminiy va oltingugurtning\\
massa ulushi 50/168 bo'lsa, (n) qiymatini aniqlang.\\
A) 5\\
B) 10\\
C) 8\\
D) 9\\
18. $\mathrm{Na}_{2} \mathrm{SO}_{4} \cdot \mathrm{nH}_{2} \mathrm{O}$ tarkibli kristallogidrat tarkibida oltingugurt va natriyning massa ulushi 39/143 bo'lsa, (n) qiymatini aniqlang.\\
A) 5\\
B) 10\\
C) 8\\
D) 9\\
19. $\mathrm{MgSO}_{4} \cdot \mathrm{nH}_{2} \mathrm{O}$ tarkibli kristallogidrat tarkibida magniy va oltingugurtning massa ulushi 14/75 bo'lsa, (n) qiymatini aniqlang.\\
A) 5\\
B) 10\\
C) 8\\
D) 9\\
20. $\mathrm{Al}_{2}\left(\mathrm{SO}_{4}\right)_{3} \cdot \mathrm{nH}_{2} \mathrm{O}$ tarkibli kristallogidrat tarkibida alyuminiy va oltingugurtning massa ulushi 75/216 bo'lsa, (n) qiymatini aniqlang.\\
A) 5\\
B) 10\\
C) 8\\
D) 9
  \item $\mathrm{NaHCO}_{3} \cdot \mathrm{nH}_{2} \mathrm{O}$ tarkibli kristallogidrat tarkibida vodorodning massa ulushi 11/174 bo'lsa, (n) qiymatini aniqlang.\\
A) 5\\
B) 10\\
C) 8\\
D) 9\\
  \item $\mathrm{NaHSO}_{4} \cdot \mathrm{nH}_{2} \mathrm{O}$ tarkibli kristallogidrat tarkibida vodorodning massa ulushi 7/100 bo'lsa, (n) qiymatini aniqlang.\\
A) 5\\
B) 10\\
C) 8\\
D) 9
  \item $\mathrm{LiHCO}_{3} \cdot \mathrm{nH}_{2} \mathrm{O}$ tarkibli kristallogidrat tarkibida vodorodning massa ulushi 17/212 bo'lsa, (n) qiymatini aniqlang.\\
A) 5\\
B) 10\\
C) 8\\
D) 9
  \item $\mathrm{Al}\left(\mathrm{HSO}_{3}\right)_{3} \cdot \mathrm{nH}_{2} \mathrm{O}$ tarkibli kristallogidrat tarkibida vodorodning\\
massa ulushi 7/144 bo'lsa, (n) qiymatini aniqlang.\\
A) 5\\
B) 10\\
C) 8\\
D) 9
  \item $\mathrm{KHCO}_{3} \cdot \mathrm{nH}_{2} \mathrm{O}$ tarkibli kristallogidrat tarkibida vodorodning massa ulushi 21/280 bo'lsa, (n) qiymatini aniqlang.\\
A) 5\\
B) 10\\
C) 8\\
D) 9
  \item $\mathrm{Na}_{2} \mathrm{CO}_{3} \cdot \mathrm{nH}_{2} \mathrm{O}$ tarkibli kristallogidrat tarkibida kislorodning massa ulushi 88/125 bo'lsa, (n) qiymatini aniqlang.\\
A) 5\\
B) 10\\
C) 8\\
D) 9
  \item $\mathrm{Al}_{2}\left(\mathrm{SO}_{4}\right)_{3} \cdot \mathrm{nH}_{2} \mathrm{O}$ tarkibli kristallogidrat tarkibida kislorodning massa ulushi 136/216 bo'lsa, (n) qiymatini aniqlang.\\
A) 5\\
B) 10\\
C) 8\\
D) 9
  \item $\mathrm{Na}_{2} \mathrm{SO}_{4} \cdot \mathrm{nH}_{2} \mathrm{O}$ tarkibli kristallogidrat tarkibida kislorodning massa ulushi 112/161 bo'lsa, (n) qiymatini aniqlang.\\
A) 5\\
B) 10\\
C) 8\\
D) 9
  \item $\mathrm{MgSO}_{4} \cdot \mathrm{nH}_{2} \mathrm{O}$ tarkibli kristallogidrat tarkibida kislorodning massa ulushi 48/70 bo'lsa, (n) qiymatini aniqlang.\\
A) 5\\
B) 10\\
C) 8\\
D) 9
  \item $\mathrm{Al}_{2}\left(\mathrm{SO}_{4}\right)_{3} \cdot \mathrm{nH}_{2} \mathrm{O}$ tarkibli kristallogidrat tarkibida kislorodning massa ulushi 192/279 bo'lsa, (n) qiymatini aniqlang.\\
A) 5\\
B) 10\\
C) 12\\
D) 9
  \item $\mathrm{NaHCO} \mathrm{H}_{3} \cdot \mathrm{nH}_{2} \mathrm{O}$ tarkibli kristallogidrat tarkibidagi natriy va vodorod atomlari son nisbati $1: 11$ ga teng bo'lsa, (n) qiymatini aniqlang.\\
A) 5\\
B) 10\\
C) 8\\
D) 9




  \setcounter{enumi}{31}
  \item $\mathrm{NaHSO}_{4} \cdot \mathrm{nH}_{2} \mathrm{O}$ tarkibli kristallogidrat tarkibidagi natriy va kislorod atomlari son nisbati 1:12 ga teng bo'lsa, (n) qiymatini aniqlang.\\
A) 5\\
B) 10\\
C) 8\\
D) 9
  \item $\mathrm{LiHCO}_{3} \cdot \mathrm{nH}_{2} \mathrm{O}$ tarkibli kristallogidrat tarkibidagi litiy va vodorod atomlari son\\
nisbati 1:19 ga teng bo'lsa, (n) qiymatini aniqlang.\\
A) 5\\
B) 10\\
C) 8\\
D) 9
  \item $\mathrm{Al}\left(\mathrm{HSO}_{3}\right)_{3} \cdot \mathrm{nH}_{2} \mathrm{O}$ tarkibli kristallogidrat tarkibidagi alyuminiy va vodorod atomlari son nisbati $1: 13$ ga teng bo'lsa, (n) qiymatini aniqlang.\\
A) 5\\
B) 10\\
C) 8\\
D) 9
  \item $\mathrm{KHCO}_{3} \cdot \mathrm{nH}_{2} \mathrm{O}$ tarkibli kristallogidrat tarkibidagi kaliy va vodorod atomlari son nisbati $1: 21$ ga teng bo'lsa, (n) qiymatini aniqlang.\\
A) 5\\
B) 10\\
C) 8\\
D) 9
  \item $\mathrm{Na}_{2} \mathrm{CO}_{3} \cdot \mathrm{nH}_{2} \mathrm{O}$ tarkibli kristallogidrat tarkibidagi uglerod va vodorod atomlari son nisbati 1:10 ga teng bo'lsa, (n) qiymatini aniqlang.\\
A) 5\\
B) 10\\
C) 8\\
D) 9
  \item $\mathrm{Al}_{2}\left(\mathrm{SO}_{4}\right)_{3} \cdot \mathrm{nH}_{2} \mathrm{O}$ tarkibli kristallogidrat tarkibidagi oltingugurt va vodorod atomlari son nisbati $3: 10$ ga teng bo'lsa, (n) qiymatini aniqlang.\\
A) 5\\
B) 10\\
C) 8\\
D) 9
  \item $\mathrm{Na}_{2} \mathrm{SO}_{4} \cdot \mathrm{nH}_{2} \mathrm{O}$ tarkibli kristallogidrat tarkibidagi natriy va kislorod atomlari son nisbati $1: 10$ teng bo'lsa, (n) qiymatini aniqlang.\\
A) 5\\
B) 10\\
C) 16\\
D) 9
  \item $\mathrm{MgSO}_{4} \cdot \mathrm{nH}_{2} \mathrm{O}$ tarkibli tarkibidagi magniy va vodorod atomlari son nisbati $1: 10$ teng bo'lsa, (n) qiymatini aniqlang.\\
A) 5\\
B) 10\\
C) 8\\
D) 9
  \item $\mathrm{Al}_{2}\left(\mathrm{SO}_{4}\right)_{3} \cdot \mathrm{nH}_{2} \mathrm{O}$ tarkibli kristallogidrat tarkibidagi alyuminiy va vodorod atomlari son nisbati $1: 12$ teng bo'lsa, ( n ) qiymatini aniqlang.\\
A) 5\\
B) 10\\
C) 12\\
D) 9


\section*{5-tip}

  \setcounter{enumi}{40}
  \item $\mathrm{Na}_{2} \mathrm{CO}_{3} \cdot \mathrm{nH}_{2} \mathrm{O}$ tarkibli kristallogidratning 0.3 mol miqdorining massasi 85.8 g kelsa (n) qiymatini aniqlang.\\
A) 5\\
B) 10\\
C) 8\\
D) 9\\

  \item $\mathrm{Na}_{2} \mathrm{SO}_{4} \cdot \mathrm{nH}_{2} \mathrm{O}$ tarkibli\\
kristallogidratning 0.1 mol miqdorining massasi 23.2 g kelsa (n) qiymatini aniqlang.\\
A) 5\\
B) 10\\
C) 8\\
D) 9
  \item $\mathrm{Li}_{2} \mathrm{CO}_{3} \cdot \mathrm{nH}_{2} \mathrm{O}$ tarkibli kristallogidratning 0.2 mol miqdorining massasi 50.8 g kelsa (n) qiymatini aniqlang.\\
A) 5\\
B) 10\\
C) 8\\
D) 9
  \item $\mathrm{Al}\left(\mathrm{NO}_{3}\right)_{3} \cdot \mathrm{nH}_{2} \mathrm{O}$ tarkibli kristallogidratning 0.5 mol miqdorining massasi 187.5 g kelsa (n) qiymatini aniqlang.\\
A) 5\\
B) 10\\
C) 8\\
D) 9
  \item $\mathrm{K}_{2} \mathrm{CO}_{3} \cdot \mathrm{nH}_{2} \mathrm{O}$ tarkibli kristallogidratning 0.3 mol miqdorining massasi 68.4 g kelsa (n) qiymatini aniqlang.\\
A) 5\\
B) 10\\
C) 8\\
D) 9
  \item $\mathrm{Na}_{2} \mathrm{CO}_{3} \cdot \mathrm{nH}_{2} \mathrm{O}$ tarkibli kristallogidratning 0.4 mol miqdorining massasi 100 g kelsa (n) qiymatini aniqlang..\\
A) 5\\
B) 10\\
C) 8\\
D) 9
  \item $\mathrm{Al}_{2}\left(\mathrm{SO}_{4}\right)_{3} \cdot \mathrm{nH}_{2} \mathrm{O}$ tarkibli kristallogidratning 0.3 mol miqdorining massasi 129.6 g kelsa (n) qiymatini aniqlang.\\
A) 5\\
B) 10\\
C) 8\\
D) 9
  \item $\mathrm{Na}_{2} \mathrm{SO}_{4} \cdot \mathrm{nH}_{2} \mathrm{O}$ tarkibli kristallogidratning 0.25 mol miqdorining massasi 76 g kelsa ( n ) qiymatini aniqlang.\\
A) 5\\
B) 10\\
C) 8\\
D) 9
  \item $\mathrm{MgSO}_{4} \cdot \mathrm{nH}_{2} \mathrm{O}$ tarkibli\\
kristallogidratning 0.45 mol miqdorining massasi 94.5 g kelsa (n) qiymatini aniqlang.\\
A) 5\\
B) 10\\
C) 8\\
D) 9
  \item $\mathrm{Al}_{2}\left(\mathrm{SO}_{4}\right)_{3} \cdot \mathrm{nH}_{2} \mathrm{O}$ tarkibli kristallogidratning 0.3 mol miqdorining massasi 145.8 g kelsa (n) qiymatini aniqlang.\\
A) 5\\
B) 10\\
C) 8\\
D) 9


\section*{6-tip}

  \setcounter{enumi}{50}
  \item $\mathrm{Na}_{2} \mathrm{CO}_{3} \cdot \mathrm{nH}_{2} \mathrm{O}$ tarkibli kristallogidrat tarkibidagi kislorodning massa ulushi vodorodning massa ulushidan 11 marta katta bo'lsa, (n) qiymatini aniqlang.\\
A) 5\\
B) 10\\
C) 8\\
D) 9\\

  \item $\mathrm{Na}_{2} \mathrm{SO}_{4} \cdot \mathrm{nH}_{2} \mathrm{O}$ tarkibli kristallogidrat tarkibidagi kislorodning massa ulushi vodorodning massa ulushidan 14,4 marta katta bo'lsa, (n) qiymatini aniqlang.\\
A) 5\\
B) 10\\
C) 8\\
D) 9
  \item $\mathrm{Li}_{2} \mathrm{CO}_{3} \cdot \mathrm{nH}_{2} \mathrm{O}$ tarkibli kristallogidrat tarkibidagi kislorodning massa ulushi vodorodning massa ulushidan 10.4 marta katta bo'lsa, (n) qiymatini aniqlang.\\
A) 5\\
B) 10\\
C) 8\\
D) 9
  \item $\mathrm{Al}\left(\mathrm{NO}_{3}\right)_{3} \cdot \mathrm{nH}_{2} \mathrm{O}$ tarkibli kristallogidrat tarkibidagi kislorodning massa ulushi vodorodning massa ulushidan 22.4 marta katta bo'lsa, (n) qiymatini aniqlang.\\
A) 5\\
B) 10\\
C) 8\\
D) 9
  \item $\mathrm{K}_{2} \mathrm{CO}_{3} \cdot \mathrm{nH}_{2} \mathrm{O}$ tarkibli kristallogidrat tarkibidagi kislorodning massa ulushi vodorodning massa ulushidan 10.67 marta katta bo'lsa, (n) qiymatini aniqlang.\\
A) 5\\
B) 10\\
C) 8\\
D) 9
  \item $\mathrm{Na}_{2} \mathrm{CO}_{3} \cdot \mathrm{nH}_{2} \mathrm{O}$ tarkibli kristallogidrat tarkibidagi kislorodning massa ulushi vodorodning massa ulushidan 12.8 marta katta bo'lsa, (n) qiymatini aniqlang.\\
A) 5\\
B) 10\\
C) 8\\
D) 9
  \item $\mathrm{Al}_{2}\left(\mathrm{SO}_{4}\right)_{3} \cdot \mathrm{nH}_{2} \mathrm{O}$ tarkibli kristallogidrat tarkibidagi kislorodning massa ulushi vodorodning massa ulushidan 17.6 marta katta bo'lsa, (n) qiymatini aniqlang.\\
A) 5\\
B) 10\\
C) 8\\
D) 9
  \item $\mathrm{Na}_{2} \mathrm{SO}_{4} \cdot \mathrm{nH}_{2} \mathrm{O}$ tarkibli kristallogidrat tarkibidagi kislorodning massa ulushi vodorodning massa ulushidan 12 marta katta bo'lsa, (n) qiymatini aniqlang.\\
A) 5\\
B) 10\\
C) 8\\
D) 9
  \item $\mathrm{MgSO}_{4} \cdot \mathrm{nH}_{2} \mathrm{O}$ tarkibli kristallogidrat tarkibidagi kislorodning massa ulushi vodorodning massa ulushidan 11.56 marta katta bo'lsa, (n) qiymatini aniqlang.\\
A) 5\\
B) 10\\
C) 8\\
D) 9
  \item $\mathrm{Al}_{2}\left(\mathrm{SO}_{4}\right)_{3} \cdot \mathrm{nH}_{2} \mathrm{O}$ tarkibli kristallogidrat tarkibidagi kislorodning massa ulushi vodorodning massa ulushidan 18.67 marta katta bo'lsa, (n) qiymatini aniqlang.\\
A) 5\\
B) 10\\
C) 8\\
D) 9
  \item $\mathrm{P}_{2} \mathrm{O}_{5}$ molekulasida fosfor atomi necha valentli bo'ladi?\\
A) 4\\
B) 3\\
C) 5\\
D) 2
  \item $\mathrm{Cl}_{2} \mathrm{O}_{7}$ molekulasida xlor atomi necha valentli bo'ladi?\\
A) 7\\
B) 3\\
C) 6\\
D) 2
  \item $\mathrm{Na}_{2} \mathrm{O}$ molekulasida natriy atomi necha valentli bo'ladi?\\
A) 1\\
B) 3\\
C) 5\\
D) 2
  \item CaO molekulasida kalsiy atomi necha valentli bo'ladi?\\
A) 7\\
B) 3\\
C) 6\\
D) 2
  \item $\mathrm{Al}_{2} \mathrm{O}_{3}$ molekulasida alyuminiy atomi necha valentli bo'ladi?\\
A) 7\\
B) 3\\
C) 6\\
D) 2
  \item $\mathrm{Cl}_{2} \mathrm{O}_{7}$ molekulasida xlor atomi necha valentli bo'ladi?\\
A) 7\\
B) 3\\
C) 6\\
D) 2
  \item $\mathrm{SO}_{3}$ molekulasida oltingugurt atomi necha valentli bo'ladi?\\
A) 7\\
B) 3\\
C) 6\\
D) 2
  \item $\mathrm{CO}_{2}$ molekulasida uglerod atomi necha valentli bo'ladi?\\
A) 5\\
B) 4\\
C) 6\\
D) 3
  \item $\mathrm{PtO}_{4}$ molekulasida platina atomi necha valentli bo'ladi?\\
A) 7\\
B) 5\\
C) 6\\
D) 8
  \item $\mathrm{Cl}_{2} \mathrm{O}$ molekulasida xlor atomi necha valentli bo'ladi?\\
A) 1\\
B) 3\\
C) 5\\
D) 2
  \item NaOH molekulasida Na atomi necha valentli bo'ladi?\\
A) 1\\
B) 3\\
C) 5\\
D) 2
12. $\mathrm{Ca}(\mathrm{OH})_{2}$ molekulasida Ca atomi necha valentli bo'ladi?\\
A) 1\\
B) 3\\
C) 5\\
D) 2\\
13. $\mathrm{Al}(\mathrm{OH})_{3}$ molekulasida Al atomi necha valentli bo'ladi?\\
A) 1\\
B) 3\\
C) 5\\
D) 2\\
14. $\mathrm{Cr}(\mathrm{OH})_{3}$ molekulasida Cr atomi necha valentli bo'ladi?\\
A) 1\\
B) 3\\
C) 5\\
D) 2\\
15. $\mathrm{Sn}(\mathrm{OH})_{4}$ molekulasida Sn atomi necha valentli bo'ladi?\\
A) 2\\
B) 3\\
C) 5\\
D) 4\\
16. $\mathrm{Mg}(\mathrm{OH})_{2}$ molekulasida Mg atomi necha valentli bo'ladi?\\
A) 1\\
B) 3\\
C) 5\\
D) 2\\
17. KOH molekulasida K atomi necha valentli bo'ladi?\\
A) 1\\
B) 3\\
C) 5\\
D) 2\\
18. $\mathrm{Tl}(\mathrm{OH})_{3}$ molekulasida Ti atomi necha valentli bo'ladi?\\
A) 1\\
B) 3\\
C) 5\\
D) 2\\
19. $\mathrm{Be}(\mathrm{OH})_{2}$ molekulasida Be atomi necha valentli bo'ladi?\\
A) 1\\
B) 3\\
C) 5\\
D) 2\\
20. $\mathrm{Pb}(\mathrm{OH})_{4}$ molekulasida Pb atomi necha valentli bo'ladi?\\
A) 2\\
B) 3\\
C) 5\\
D) 4
  \item HCl molekulasi tarkibidagi Cl atomi necha valentli.\\
A) 1\\
B) 3\\
C) 5\\
D) 4
22. $\mathrm{H}_{2} \mathrm{~S}$ molckulasi tarkibidagi S atomi necha valentli.\\
A) 2\\
B) 3\\
C) 5\\
D) 4\\
23. HClO molekulasi tarkibidagi Cl atomi necha valentli.\\
A) 1\\
B) 3\\
C) 5\\
D) 4\\
24. $\mathrm{HClO}_{4}$ molekulasi tarkibidagi Cl atomi necha valentli.\\
A) 1\\
B) 7\\
C) 5\\
D) 3\\
25. $\mathrm{H}_{2} \mathrm{SO}_{3}$ molekulasi tarkibidagi S atomi necha valentli.\\
A) 2\\
B) 3\\
C) 5\\
D) 4\\
26. $\mathrm{H}_{2} \mathrm{SO}_{4}$ molekulasi tarkibidagi S atomi necha valentli.\\
A) 6\\
B) 4\\
C) 2\\
D) 5\\
27. $\mathrm{H}_{2} \mathrm{Cr}_{2} \mathrm{O}_{7}$ molekulasi tarkibidagi Cr atomi necha valentli.\\
A) 6\\
B) 4\\
C) 2\\
D) 5\\
28. $\mathrm{H}_{4} \mathrm{P}_{2} \mathrm{O}_{7}$ molekulasi tarkibidagi P atomi necha valentli,\\
A) 6\\
B) 4\\
C) 2\\
D) 5\\
29. HCN molekulasi tarkibidagi N atomi necha valentli.\\
A) 6\\
B) 3\\
C) 2\\
D) 5\\
30. $\mathrm{H}_{2} \mathrm{CrO}_{4}$ molekulasi tarkibidagi Cr atomi necha valentli.\\
A) 6\\
B) 3\\
C) 2\\
D) 5
  \item NaCl molekulasi tarkibidagi Na atomi necha valentli.\\
A) 1\\
B) 3\\
C) 5\\
D) 4
32. $\mathrm{Al}_{2}\left(\mathrm{SO}_{4}\right)_{3}$ molekulasi tarkíbidagi S atomi necha valentli.\\
A) 2\\
B) 3\\
C) 5\\
D) 6\\
33. $\mathrm{Al}(\mathrm{ClO})_{3}$ molekulasi tarkibidagi Al atomi necha valentli.\\
A) 1\\
B) 3\\
C) 5\\
D) 4\\
84. $\mathrm{Cu}\left(\mathrm{ClO}_{4}\right)_{y}$ mologkulnai turkibídagi ('l ntomi nochn vilontli.\\
A) 1\\
B) 7\\
C) (")\\
1)) 3\\
35. Na2 $\mathrm{SO}_{0}$ molekulasi tarkibídagi \& ntomi nocha valontli.\\
A) 2\\
B) 3\\
C) $\bar{b}$\\
D) 1\\
36. $\mathrm{Na}_{4} \mathrm{P}_{2} \mathrm{O}_{7}$ molokulani tarkibjdngj ${ }^{\prime}$ atomi nocha valontli,\\
A) 6\\
B) 4\\
C) 2\\
D) 5\\
87. $\mathrm{K}_{2} \mathrm{Cr}_{2} \mathrm{O}_{7}$ molekulasi tarkibidagi Cr atomi necha valontli.\\
A) 6\\
B) 4\\
C) 2\\
D) 5\\
38. $\mathrm{MgCl}_{2}$ molekulasi tarkíbídagi Mg atomi necha valentli.\\
A) 6\\
B) 4\\
C) 2\\
D) 5\\
39. $\mathrm{Cd}(\mathrm{CN})_{2}$ molekulasi tarkibidagi N atomi necha valentlí.\\
A) 6\\
B) 3\\
C) 2\\
D) 5

40, $\mathrm{Li}_{2} \mathrm{CrO}_{4}$ molekulasi tarkibidagi Cr atomi necha valentli.\\
A) 6\\
B) 3\\
C) 2\\
D) 5
  \item $\mathrm{Br}_{2}$ molekulasi tarkibidagi Br ning ekvivalentini aniqlang.
A) 160\\
B) 120\\
C) 127\\
D) 80\\
42. $\mathrm{F}_{2}$ molekulasi tarkibidagi F ninig ekvivalentini aniqlang.\\
A) 160\\
B) 19\\
C) 38\\
D) 80\\
43. Ig molekulasi tarkibidagi I ning ekvivalentni aniqlang.\\
A) 127\\
B) 19\\
C) 38\\
D) 254\\
44. $\mathrm{SO}_{2}$ tarkibidagi oltingugurtninig ekvivalentini aniqlang.\\
A) 64\\
B) 32\\
C) 16\\
D) 8\\
45. $\mathrm{SiO}_{2}$ tarkibidagi kremniyning ekvivalentini aniqlang.\\
A) 60\\
B) 28\\
C) 17\\
D) 7\\
46. $\mathrm{Cl}_{2} \mathrm{O}_{7}$ tarkibidagi xlorninig ekvivalentini aniqlang.\\
A) 183\\
B) 71\\
C) 35,5\\
D) 5,07\\
47. $\mathrm{SO}_{3}$ tarkibidagi oltingugurtninig ekvivalentini aniqlang.\\
A) 80\\
B) 13,33\\
C) 16\\
D) 5,33\\
48. $\mathrm{N}_{2} \mathrm{O}_{3}$ tarkibidagi azotninig ekvivalentini aniqlang.\\
A) 12,66\\
B) 28\\
C) 14\\
D) 4,67\\
49. HClO tarkibidagi xlornining ekvivalentini aniqlang.\\
A) 183\\
B) 71\\
C) 35,5\\
D) 5,07\\
50. $\mathrm{HIO}_{3}$ tarkibidagi yodning ekvivalentini aniqlang.\\
A) 127\\
B) 19\\
C) 38\\
D) 25,4
  \item $\mathrm{SO}_{3}$ ning ekvivalent massa qiymatini aniqlang.\\
A) 80\\
B) 13,33\\
C) 16\\
D) 5,33\\
  \item $\mathrm{NO}_{2}$ ning ekvivalent massasini aniqlang.\\
A) 46\\
B) 11,5\\
C) 14\\
D) 4,67
  \item $\mathrm{P}_{2} \mathrm{O}_{6}$ ning ekvivalent massasini aniqlang.\\
A) 142\\
B) 14,2\\
C) 71\\
D) 6,2
  \item $\mathrm{Na}_{2} \mathrm{O}$ ning ekvivalent massasini aniqlang.\\
A) 62\\
B) 155\\
C) 31\\
D) 6,2
  \item CaO ning ekvivalent massasini aniqlang.\\
A) 56\\
B) 28\\
C) 20\\
D) 8
  \item $\mathrm{Cl}_{2} \mathrm{O}_{7}$ ning ekvivalent massasini aniqlang.\\
A) 183\\
B) 71\\
C) 13,07\\
D) 5,07
  \item $\mathrm{P}_{2} \mathrm{O}_{3}$ ning okvivalent mussasini aniqlang.\\
A) 10\\
B) 14,2\\
C) 62\\
D) 18,33
  \item MgO ning ekvivalent massasini aniqlang.\\
A) 56\\
B) 20\\
C) 40\\
D) 8
  \item $\mathrm{Cr}_{3} \mathrm{O}_{7}$ ning ekvivalent massasini aniqlang.\\
A) 183\\
B) 71\\
C) 15.43\\
D) 5,07
60, $\mathrm{Al}_{9} \mathrm{O}_{3}$ ning ekvivalent massasini aniqlang.\\
A) 10\\
B) 14,2\\
C) 17\\
D) 18,33
  \item NaOH ning ekvivalent massasini aniqlang.\\
A) 40\\
B) 23\\
C) 8\\
D) 17\\
  \item $\mathrm{Al}(\mathrm{OH})_{3}$ ning ekvivalent massasini aniqlang.\\
A) 78\\
B) 26\\
C) 8\\
D) 9
  \item $\mathrm{Cr}(\mathrm{OH})_{3}$ ning ekvivalent massasini aniqlang.\\
A) 103\\
B) 34,33\\
C) 51\\
D) 17,33
  \item $\mathrm{Fe}(\mathrm{OH})_{3}$ ning ekvivalent massasini aniqlang.\\
A) 103\\
B) 35.67\\
C) 51\\
D) 17,33
  \item $\mathrm{Mg}(\mathrm{OH})_{2}$ ning ekvivalent massasini aniqlang.\\
A) 29\\
B) 34\\
C) 51\\
D) 17
  \item LiOH ning ekvivalent massasini aniqlang.\\
A) 19\\
B) 24\\
C) 51\\
D) 27
  \item $\mathrm{Zn}(\mathrm{OH})_{2}$ ning ekvivalent massasini aniqlang.\\
A) 29\\
B) 34.5\\
C) 24.5\\
D) 49.5
  \item $\mathrm{Ca}(\mathrm{OH})_{2}$ ning ekvivalent massasini aniqlang.\\
A) 29\\
B) 37\\
C) 51\\
D) 17
69, KOH ning ekvivalent massasini aniqlang.\\
A) 29\\
B) 36\\
C) 56\\
D) 17\\
70. $\mathrm{Cr}(\mathrm{OH})_{2}$ ning ekvivalent massasini aniqlang.\\
A) 43\\
B) 37\\
C) 50\\
D) 19\\
71. $\mathrm{H}_{2} \mathrm{SO}_{3}$ ning ekvivalent massasini aniqlang.\\
A) 82\\
B) 41\\
C) 8\\
D) 16\\
72. $\mathrm{H}_{2} \mathrm{CO}_{3}$ ning ekvivalent massasini aniqlang.\\
A) 62\\
B) 12\\
C) 8\\
D) 31\\
73. $\mathrm{HClO}_{4}$ ning ekvivalent massasini aniqlang.\\
A) 35,5\\
B) 8\\
C) 71\\
D) 100,5\\
74. $\mathrm{HCIO}_{2}$ ning ekvivalent massasini aniqlang.\\
A) 35,5\\
B) 68,5\\
C) 71\\
D) 100,5\\
75. $\mathrm{H}_{3} \mathrm{PO}_{4}$ ning ekvivalent massasini aniqlang.\\
A) 32,67\\
B) 31\\
C) 98\\
D) 24,5\\
76. $\mathrm{H}_{4} \mathrm{P}_{2} \mathrm{O}_{7}$ ning ekvivalent massasini aniqlang.\\
A) 44,5\\
B) 31\\
C) 178\\
D) 24,5\\
77. $\mathrm{H}_{3} \mathrm{PO}_{2}$ ning ekvivalent massasini aniqlang.\\
A) 32,67\\
B) 44\\
C) 66\\
D) 24,5\\
78. HCN ning ekvivalent massasini aniqlang.\\
A) 32\\
B) 44\\
C) 66\\
D) 27\\
79. $\mathrm{H}_{3} \mathrm{PO}_{3}$ ning ekvivalent massasini aniqlang.\\
A) 82\\
B) 41\\
C) 8\\
D) 16\\
80. $\mathrm{H}_{2} \mathrm{CrO}_{4}$ ning ekvivalent massasini aniqlang.\\
A) 32\\
B) 31\\
C) 59\\
D) 24
  \item $\mathrm{Na}_{2} \mathrm{SO}_{4}$ ning ekvivalent massasini aniqlang.\\
A) 71\\
B) 32\\
C) 8\\
D) 142\\
  \item $\mathrm{LiNO}_{3}$ ning ekvivalent massasini aniqlang.\\
A) 69\\
B) 7\\
C) 8\\
D) 34,5
  \item $\mathrm{Ba}_{3}\left(\mathrm{PO}_{4}\right)_{2}$ ning ekvivalent massasini aniqlang.\\
A) 601\\
B) 137\\
C) 200,33\\
D) 100,167
  \item $\mathrm{Ca}\left(\mathrm{NO}_{3}\right)_{2}$ ning ekvivalent massasini aniqlang.\\
A) 164\\
B) 82\\
C) 8\\
D) 63
  \item $\mathrm{CdSO}_{4}$ ning ekvivalent massasini aniqlang.\\
A) 208\\
B) 104\\
C) 96\\
D) 56
  \item $\mathrm{Cr}_{2}\left(\mathrm{SiO}_{3}\right)_{3}$ ning ekvivalent massasini aniqlang.\\
A) 332\\
B) 110,66\\
C) 55,33\\
D) 17,33
  \item $\mathrm{MgSO}_{4}$ ning ekvivalent massasini aniqlang.\\
A) 208\\
B) 104\\
C) 60\\
D) 56
  \item $\mathrm{Be}\left(\mathrm{NO}_{3}\right)_{2}$ ning ekvivalent massasini aniqlang.\\
A) 164\\
B) 82\\
C) 8\\
D) 66.5
  \item $\mathrm{Al}_{2}\left(\mathrm{SiO}_{3}\right)_{3}$ ning ekvivalent massasini aniqlang.\\
A) 47\\
B) 110,66\\
C) 55,33\\
D) 45
  \item $\mathrm{Ca}_{3}\left(\mathrm{PO}_{4}\right)_{2}$ ning ekvivalent massasini aniqlang.\\
A) 60\\
B) 137\\
C) 20\\
D) 51.67
  \item 2.4 g metall oksidlanganda 4 g oksid hosil qilsa. Bu qaysi metall?\\
A) Mg\\
B) Ca\\
C) Na\\
D) Zn
  \item 12 g metall oksidlanganda 16.8 g oksid hosil qilsa. Bu qaysi metall?\\
A) Mg\\
B) Ca\\
C) Be\\
D) K
  \item 5.4 g metall oksidlanganda 10.2 g oksid hosil qilsa. Bu qaysi metall?\\
A) Be\\
B) Al\\
C) Fr\\
D) Mg
  \item 4.6 g metall oksidlanganda 6.2 g oksid hosil qilsa. Bu qaysi metall?\\
A) Na\\
B) K\\
C) Sc\\
D) Ga
  \item 15.6 g metall oksidlanganda 18.8 g oksid hosil qilsa. Bu qaysi metall?\\
A) K\\
B) Ca\\
C) Zn\\
D) Mg
  \item 4.11 g metall oksidlanganda 4.59 g oksid hosil qilsa. Bu qaysi metall?\\
A) Zn\\
B) Ga\\
C) Ba\\
D) Te
  \item 2 g metall oksidlanganda 2.8 g oksid hosil qilsa. Bu qaysi metall?\\
A) Ca\\
B) Si\\
C) Zn\\
D) Be
  \item 2.8 g metall oksidlanganda 3.6 g oksid hosil qilsa. Bu qaysi metall?\\
A) Fe\\
B) Si\\
C) Zn\\
D) Ca
  \item 3.2 g metall oksidlanganda 4 g oksid hosil qilsa. Bu qaysi metall?\\
A) Zn\\
B) Cu\\
C) Ba\\
D) Ca
  \item 1.2 g metall oksidlanganda 2 g oksid hosil qilsa. Bu qaysi metall?\\
A) Zn\\
B) Mg\\
C) Ca\\
D) Na
  \item Element yonganda uning massasi 36/28 martaga oshdi. Element toping.\\
A) S\\
B) Fe\\
C) Mn\\
D) Si\\
  \item Element yonganda uning massasi 76/52 martaga oshdi. Element toping.\\
A) CI\\
B) Fe\\
C) Cr\\
D) Mn
  \item Element yonganda uning massasi 51/27 martaga oshdi. Element toping.\\
A) Be\\
B) Mg\\
C) Al\\
D) Na
  \item Element xlorlanganda uning massasi 104/68,5 martaga ortdi. Elementni aniqlang.\\
A) Fe\\
B) Ba\\
C) Mg\\
D) Ca
  \item Element xlorlanganda uning massasi 190/48 martaga ortdi. Elementni aniqlang.\\
A) Fe\\
B) Ba\\
C) Mg\\
D) Ca
  \item Element yonganda uning massasi 162/130 martaga oshdi. Element toping.\\
A) Cu\\
B) Fe\\
C) Zn\\
D) Mn
  \item Element xlorlanganda uning massasi 63,5/28 martaga ortdi. Elementni aniqlang.\\
A) Fe\\
B) Ba\\
C) Mg\\
D) Ca
  \item Element sulfidlanganda uning massasi 36/20 martaga ortdi. Elementni aniqlang.\\
A) Fe\\
B) Ba\\
C) Mg\\
D) Ca
  \item Element sulfidlanganda uning massasi 28/12 martaga ortdi. Elementni aniqlang.\\
A) Fe\\
B) Ba\\
C) Mg\\
D) Ca
  \item Element sulfidlanganda uning massasi 338/274 martaga ortdi. Elementni aniqlang.\\
A) Fe\\
B) Ba\\
C) Mg\\
D) Ca
  \item 8 g NaOH bilan to'liq reaksiyaga kirisha oladigan kislotaning massasi $9,8 \mathrm{~g}$ kelsa bu qaysi kislota.\\
A) $\mathrm{H}_{2} \mathrm{SO}_{4}$\\
B) $\mathrm{HNO}_{3}$\\
C) HCl\\
D) $\mathrm{H}_{2} \mathrm{CO}_{3}$
  \item $14,8 \mathrm{~g} \mathrm{Ca}(\mathrm{OH})_{2}$ bilan to'liq reaksiyaga kirisha oladigan kislotaning massasi 14,6 g kelsa bu qaysi kislota.\\
A) $\mathrm{H}_{2} \mathrm{SO}_{4}$\\
B) $\mathrm{HNO}_{3}$\\
C) HCl\\
D) $\mathrm{H}_{2} \mathrm{CO}_{3}$
  \item 28 g KOH bilan to'liq reaksiyaga kirisha oladigan kislotaning massasi 31,5 g kelsa bu qaysi kislota.\\
A) $\mathrm{H}_{2} \mathrm{SO}_{4}$\\
B) $\mathrm{HNO}_{3}$\\
C) HCl\\
D) $\mathrm{H}_{2} \mathrm{CO}_{3}$
  \item $42,75 \mathrm{~g} \mathrm{Ba}(\mathrm{OH})_{2}$ bilan to'liq reaksiyaga kirisha oladigan kislotaning massasi $15,5 \mathrm{~g}$ kelsa bu qaysi kislota.\\
A) $\mathrm{H}_{2} \mathrm{SO}_{4}$\\
B) $\mathrm{HNO}_{3}$\\
C) HCl\\
D) $\mathrm{H}_{2} \mathrm{CO}_{3}$
  \item 16 g NaOH bilan to'liq reaksiyaga kirisha oladigan kislotaning massasi 14,6 g kelsa bu qaysi kislota.\\
A) $\mathrm{H}_{2} \mathrm{SO}_{4}$\\
B) $\mathrm{HNO}_{3}$\\
C) HCl\\
D) $\mathrm{H}_{2} \mathrm{CO}_{3}$
  \item $6,3 \mathrm{~g} \mathrm{HNO}_{3}$ bilan to'liq reaksiyaga kirisha oladigan asos massasi 4 g kelsa bu qaysi asos\\
A) NaOH\\
B) $\mathrm{Ba}(\mathrm{OH})_{2}$\\
C) KOH\\
D) $\mathrm{Ca}(\mathrm{OH})_{2}$
  \item $9,8 \mathrm{~g} \mathrm{H}_{2} \mathrm{SO}_{4}$ bilan to'liq reaksiyaga kirisha oladigan asos massasi $7,4 \mathrm{~g}$ kelsa bu qaysi asos\\
A) NaOH\\
B) $\mathrm{Ba}(\mathrm{OH})_{2}$\\
C) KOH\\
D) $\mathrm{Ca}(\mathrm{OH})_{2}$
  \item $31 \mathrm{~g} \mathrm{H}_{2} \mathrm{CO}_{3}$ bilan to'liq reaksiyaga kirisha oladigan asos massasi 56 g kelsa bu qaysi asos\\
A) NaOH\\
B) $\mathrm{Ba}(\mathrm{OH})_{2}$\\
C) KOH\\
D) $\mathrm{Ca}(\mathrm{OH})_{2}$
  \item $3,65 \mathrm{~g} \mathrm{HCl}$ bilan to'liq reaksiyaga kirisha oladigan asos massasi $8,55 \mathrm{~g}$ kelsa bu qaysi asos\\
A) NaOH\\
B) $\mathrm{Ba}(\mathrm{OH})_{2}$\\
C) KOH\\
D) $\mathrm{Ca}(\mathrm{OH})_{2}$
  \item $12,6 \mathrm{~g} \mathrm{HNO}_{3}$ bilan to'liq reaksiyaga kirisha oladigan asos massasi $7,4 \mathrm{~g}$ kelsa bu qaysi asos\\
A) NaOH\\
B) $\mathrm{Ba}(\mathrm{OH})_{2}$\\
C) KOH\\
D) $\mathrm{Ca}(\mathrm{OH})_{2}$
121. Noma'lum metal xloridi va xloratining molyar massa nisbati $1,17: 2,13$ bo'lsa, metalni toping.\\
A) Zn\\
B) Mg\\
C) Ca\\
D) Na
122, Noma'lum metal karbonati va oksidining molyar massa nisbati 50:28 bo'lsa, metalni toping.\\
A) Zn\\
B) Mg\\
C) Ca\\
D) Na\\
123. Noma'lum metal xloridi va gidroksidining molyar massa nisbati $11,1: 7,4$ bo'lsa, metalni toping.\\
A) Zn\\
B) Mg\\
C) Ca\\
D) Na\\
124. Noma'lum metal oksidi va karbonatining molyar massa nisbati $4: 8,4$ bo'lsa, metalni toping.\\
A) Cr\\
B) Mg\\
C) Ca\\
D) Na\\
125. Noma'lum metal xloridi va perxloratining molyar massa nisbati 17:33 bo'lsa, metalni toping.\\
A) Zn\\
B) Mg\\
C) Fe\\
D) Na\\
126. Noma'lum metal xloridi va karbonatining molyar massa nisbati 19:16,8 bo'lsa, metalni toping.\\
A) Zn\\
B) Mg\\
C) Ca\\
D) Na\\
127. Noma'lum metal xloridi va xloratining molyar massa nisbati 12.7:22,3 bo'lsa, metalni toping.\\
A) Zn\\
B) Mg\\
C) Fe\\
D) Na\\
128. Noma'lum metal oksidi va karbonatining molyar massa nisbati\\
15.5:26,5 bo'lsa, metalni toping.\\
A) Cr\\
B) Mg\\
C) Ca\\
D) Na\\
129. Noma'lum metal oksidi va nitratining molyar massa nisbati 5:18,5 bo'lsa, metalni toping.\\
A) Cr\\
B) Mg\\
C) Ca\\
D) Na\\
130. Noma'lum metal oksidi va karbonatining molyar massa nisbati 7:12,5 bo'lsa, metalni toping.\\
A) Zn\\
B) Mg\\
C) Ca\\
D) Na
  \item $2 \mathrm{Mg}+\mathrm{O}_{2} \rightarrow 2 \mathrm{MgO}$ ushbu reaksiya bo'yicha $2,4 \mathrm{~g}$ magniy necha g kislorod bilan reaksiyaga kirishadi?\\
A) 3,2\\
B) 1,6\\
C) 6,4\\
D) 4,8
  \item $4 \mathrm{Al}+3 \mathrm{O}_{2} \rightarrow 2 \mathrm{Al}_{2} \mathrm{O}_{3}$ ushbu reaksiya bo'yicha $5,4 \mathrm{~g} \mathrm{Al}$ necha g kislorod bilan reaksiyaga kirishadi?\\
A) 3,2\\
B) 1,6\\
C) 6,4\\
D) 4,8
  \item $2 \mathrm{Zn}+\mathrm{O}_{2} \rightarrow 2 \mathrm{ZnO}$ ushbu reaksiya bo'yicha $6,5 \mathrm{~g} \mathrm{Zn}$ necha g kislorod bilan reaksiyaga kirishadi?\\
A) 3,2\\
B) 1,6\\
C) 6,4\\
D) 4,8
  \item $4 \mathrm{Li}+\mathrm{O}_{2} \rightarrow 2 \mathrm{Li}_{2} \mathrm{O}$ ushbu reaksiya bo'yicha $5,6 \mathrm{~g}$ Li necha g kislorod bilan reaksiyaga kirishadi?\\
A) 3,2\\
B) 1,6\\
C) 6,4\\
D) 4,8
  \item $3 \mathrm{Fe}+2 \mathrm{O}_{2} \rightarrow \mathrm{Fe}_{3} \mathrm{O}_{4}$ ushbu reaksiya bo'yicha $8,4 \mathrm{~g}$ Fe necha g kislorod bilan reaksiyaga kirishadi?\\
A) 3,2\\
B) 1,6\\
C) 6,4\\
D) 4,8
  \item $\mathrm{CaO}+\mathrm{SO}_{3} \rightarrow \mathrm{CaSO}_{4}$ ushbu reaksiya bo'yicha $5,6 \mathrm{~g}$ CaO necha g $\mathrm{SO}_{3}$ bilan reaksiyaga kirishadi?\\
A) 3\\
B) 6\\
C) 4\\
D) 8
  \item $\mathrm{Na}_{2} \mathrm{O}+\mathrm{H}_{2} \mathrm{O} \rightarrow 2 \mathrm{NaOH}$ ushbu reaksiya bo'yicha $3,1 \mathrm{~g} \mathrm{Na}{ }_{2} \mathrm{O}$ necha $\mathrm{g} \mathrm{H}_{2} \mathrm{O}$ bilan reaksiyaga kirishadi?\\
A) 1,8\\
B) 1,6\\
C) 6,4\\
D) 0,9
  \item $\mathrm{Al}_{2} \mathrm{O}_{3}+3 \mathrm{SO}_{3} \rightarrow \mathrm{Al}_{2}\left(\mathrm{SO}_{4}\right)_{3}$ ushbu reaksiya bo'yicha $10,2 \mathrm{~g} \mathrm{Al}_{2} \mathrm{O}_{3}$ necha $\mathrm{g} \mathrm{SO}_{3}$ bilan reaksiyaga kirishadi?\\
A) 32\\
B) 16\\
C) 24\\
D) 48
  \item $\mathrm{P}_{2} \mathrm{O}_{5}+3 \mathrm{H}_{2} \mathrm{O} \rightarrow 2 \mathrm{H}_{3} \mathrm{PO}_{4}$ ushbu reaksiya bo'yicha $71 \mathrm{~g} \mathrm{P}_{2} \mathrm{O}_{5}$ necha $\mathrm{g} \mathrm{H}_{2} \mathrm{O}$ bilan reaksiyaga kirishadi?\\
A) 27\\
B) 18\\
C) 54\\
D) 36
  \item $\mathrm{SO}_{3}+\mathrm{H}_{2} \mathrm{O} \rightarrow \mathrm{H}_{2} \mathrm{SO}_{4}$ ushbu reaksiya bo'yicha $8 \mathrm{~g} \mathrm{SO}_{3}$ necha g $\mathrm{H}_{2} \mathrm{O}$ bilan reaksiyaga kirishadi?\\
A) 2,7\\
B) 1,8\\
C) 5,4\\
D) 3,6
  \item $2 \mathrm{AgNO}_{3}+\mathrm{Fe} \rightarrow \mathrm{Fe}\left(\mathrm{NO}_{3}\right)_{2}+2 \mathrm{Ag}$ ushbu reaksiya bo'yicha $340 \mathrm{~g} \mathrm{AgNO}_{3}$ dan foydalanib necha mol Ag olish mumkin?\\
A) 2\\
B) 1\\
C) 5\\
D) 3
  \item $\mathrm{CuSO}_{4}+\mathrm{Mn} \rightarrow \mathrm{MnSO}_{4}+\mathrm{Cu}$ ushbu reaksiya bo'yicha $80 \mathrm{~g} \mathrm{CuSO}_{4}$ dan foydalanib necha mol Cu olish mumkin?\\
A) 0,2\\
B) 1\\
C) 0,5\\
D) 3
  \item $3 \mathrm{AgNO}_{3}+\mathrm{Al} \rightarrow \mathrm{Al}\left(\mathrm{NO}_{3}\right)_{3}+3 \mathrm{Ag}$ ushbu reaksiya bo'yicha $170 \mathrm{~g} \mathrm{AgNO}_{3}$ dan foydalanib necha mol Ag olish mumkin?\\
A) 2\\
B) 1\\
C) 5\\
D) 3
  \item $2 \mathrm{H}_{3} \mathrm{PO}_{4}+6 \mathrm{Na} \rightarrow 2 \mathrm{Na}_{3} \mathrm{PO}_{4}+3 \mathrm{H}_{2}$ ushbu reaksiya bo'yicha $98 \mathrm{~g} \mathrm{H}_{3} \mathrm{PO}_{4}$ dan foydalanib necha mol $\mathrm{Na}_{3} \mathrm{PO}_{4}$ olish mumkin?\\
A) 2\\
B) 1\\
C) 5\\
D) 3
  \item $3 \mathrm{H}_{2} \mathrm{SO}_{4}+2 \mathrm{Al} \rightarrow \mathrm{Al}_{2}\left(\mathrm{SO}_{4}\right)_{3}+3 \mathrm{H}_{2}$ ushbu reaksiya bo'yicha $294 \mathrm{~g} \mathrm{H}_{2} \mathrm{SO}_{4}$ dan foydalanib necha mol $\mathrm{Al}_{2}\left(\mathrm{SO}_{4}\right)_{3}$ olish mumkin?\\
A) 2\\
B) 1\\
C) 5\\
D) 3
  \item $2 \mathrm{HCl}+\mathrm{Ca} \rightarrow \mathrm{CaCl}_{2}+\mathrm{H}_{2}$ ushbu reaksiya boyicha 146 g HCl dan foydalanib necha mol $\mathrm{H}_{2}$ olish mumkin?\\
A) 2\\
B) 1\\
C) 5\\
D) 3
  \item $2 \mathrm{H}_{3} \mathrm{PO}_{4}+2 \mathrm{Al} \rightarrow 2 \mathrm{AlPO}_{4}+3 \mathrm{H}_{2}$ ushbu reaksiya bo'yicha $294 \mathrm{~g} \mathrm{H}_{3} \mathrm{PO}_{4}$ dan foydalanib necha $\mathrm{mol} \mathrm{AlPO}_{4}$ olish mumkin?\\
A) 2\\
B) 1\\
C) 5\\
D) 3
  \item $\mathrm{Cr}_{2} \mathrm{O}_{3}+2 \mathrm{Al} \rightarrow \mathrm{Al}_{2} \mathrm{O}_{3}+2 \mathrm{Cr}$ ushbu reaksiya bo'yicha $304 \mathrm{~g} \mathrm{Cr}_{2} \mathrm{O}_{3}$ dan foydalanib necha $\mathrm{mol} \mathrm{Al}_{2} \mathrm{O}_{3}$ olish mumkin?\\
A) 2\\
B) 1\\
C) 5\\
D) 3
  \item $\mathrm{Fe}_{3} \mathrm{O}_{4}+2 \mathrm{C} \rightarrow 3 \mathrm{Fe}+2 \mathrm{CO}_{2}$ ushbu reaksiya bo'yicha $116 \mathrm{~g} \mathrm{Fe}_{3} \mathrm{O}_{4}$ dan foydalanib necha $\mathrm{mol} \mathrm{CO}_{2}$ olish mumkin?\\
A) 2\\
B) 1\\
C) 5\\
D) 3
  \item $\mathrm{Fe}_{2} \mathrm{O}_{3}+2 \mathrm{Al} \rightarrow \mathrm{Al}_{2} \mathrm{O}_{3}+2 \mathrm{Fe}$ ushbu reaksiya bo'yicha $320 \mathrm{~g} \mathrm{Fe}_{2} \mathrm{O}_{3}$ dan foydalanib necha mol Fe olish mumkin?\\
A) 2\\
B) 1\\
C) 5\\
D) 4
  \item $\mathrm{AgNO}_{3}+\mathrm{NaCl} \rightarrow \mathrm{NaNO}_{3}+\mathrm{AgCl}$ ushbu reaksiya bo'yicha 117 g NaCl sarflangan bo'lsa, necha g $\mathrm{NaNO}_{3}$ hamda necha mol AgCl hosil bo'lgan?
A) $85 ; 1$\\
B) $170 ; 2$\\
C) $80 ; 2$\\
D) $17 ; 1$\\
22. $3 \mathrm{AgF}+\mathrm{AlCl}_{3} \rightarrow 3 \mathrm{AgCl}+\mathrm{AlF}_{3}$ ushbu reaksiya bo'yicha $133,5 \mathrm{~g} \mathrm{AlCl}_{3}$ sarflangan bo'lsa, necha $\mathrm{g} \mathrm{AlF}_{3}$ hamda necha mol AgCl hosil bo'lgan?\\
A) $84 ; 3$\\
B) $126 ; 2$\\
C) $180 ; 2$\\
D) $117 ; 1$\\
23. $\mathrm{CuSO}_{4}+\mathrm{K}_{2} \mathrm{~S} \rightarrow \mathrm{CuS}+\mathrm{K}_{2} \mathrm{SO}_{4}$ ushbu reaksiya bo'yicha $80 \mathrm{~g} \mathrm{CuSO}_{4}$ sarflangan bo'lsa, necha g CuS hamda necha mol $\mathrm{K}_{2} \mathrm{SO}_{4}$ hosil bo'lgan?\\
A) $52 ; 1$\\
B) $96 ; 0,2$\\
C) $480 ; 0,5$\\
D) $48 ; 0,5$\\
24. $\mathrm{H}_{3} \mathrm{PO}_{4}+3 \mathrm{NaOH} \rightarrow \mathrm{Na}_{3} \mathrm{PO}_{4}+3 \mathrm{H}_{2} \mathrm{O}$ ushbu reaksiya bo'yicha $196 \mathrm{~g} \mathrm{H}_{3} \mathrm{PO}_{4}$ sarflangan bo'lsa, necha g $\mathrm{Na}_{3} \mathrm{PO}_{4}$ hamda necha mol $\mathrm{H}_{2} \mathrm{O}$ hosil bo'lgan?\\
A) $656 ; 2$\\
B) $164 ; 1$\\
C) $328 ; 6$\\
D) $114 ; 2$\\
25. $3 \mathrm{H}_{2} \mathrm{SO}_{4}+2 \mathrm{Al}(\mathrm{OH})_{3} \rightarrow \mathrm{Al}_{2}\left(\mathrm{SO}_{4}\right)_{3}+6 \mathrm{H}_{2} \mathrm{O}$ ushbu reaksiya bo'yicha $156 \mathrm{~g} \mathrm{Al}(\mathrm{OH})_{3}$ sarflangan bo'lsa, necha $\mathrm{g} \mathrm{Al}_{2}\left(\mathrm{SO}_{4}\right)_{3}$ hamda necha mol $\mathrm{H}_{2} \mathrm{O}$ hosil bo'lgan?\\
A) $342 ; 2$\\
B) 85,$5 ; 1$\\
C) $171 ; 2$\\
D) $342 ; 6$\\
26. $3 \mathrm{HCl}+\mathrm{Al}(\mathrm{OH})_{3} \rightarrow \mathrm{AlCl}_{3}+3 \mathrm{H}_{2} \mathrm{O}$ ushbu reaksiya bo'yicha 219 g HCl sarflangan bo'lsa, necha $\mathrm{g} \mathrm{AlCl}_{3}$ hamda necha mol $\mathrm{H}_{2} \mathrm{O}$ hosil bo'lgan?\\
A) $267 ; 6$\\
B) 133,$5 ; 1$\\
C) $85.5 ; 2$\\
D) $18 ; 5$\\
27. $\mathrm{H}_{3} \mathrm{PO}_{4}+\mathrm{Al}(\mathrm{OH})_{3} \rightarrow \mathrm{AlPO}_{4}+3 \mathrm{H}_{2} \mathrm{O}$ ushbu reaksiya bo'yicha $196 \mathrm{~g} \mathrm{H}_{3} \mathrm{PO}_{4}$ sarflangan bo'lsa, necha $\mathrm{g} \mathrm{AlPO}_{4}$ hamda necha mol $\mathrm{H}_{2} \mathrm{O}$ hosil bo'lgan?\\
A) $122 ; 3$\\
B) $61 ; 1$\\
C) $244 ; 6$\\
D) $366 ; 6$\\
28. $\mathrm{H}_{2} \mathrm{SO}_{4}+2 \mathrm{NaOH} \rightarrow \mathrm{Na}_{2} \mathrm{SO}_{4}+2 \mathrm{H}_{2} \mathrm{O}$ ushbu reaksiya bo'yicha 160 g NaOH sarflangan bo'lsa, necha g $\mathrm{Na}_{2} \mathrm{SO}_{4}$ hamda necha mol $\mathrm{H}_{2} \mathrm{O}$ hosil bo'lgan?\\
A) $284 ; 4$\\
B) $71 ; 1$\\
C) $142 ; 2$\\
D) 35,$5 ; 1$\\
29. $\mathrm{CuSO}_{4}+2 \mathrm{NaOH} \rightarrow \mathrm{Na}_{2} \mathrm{SO}_{4}+\mathrm{Cu}(\mathrm{OH})_{2}$ ushbu reaksiya bo'yicha $160 \mathrm{~g} \mathrm{CuSO}_{4}$ sarflangan bo'lsa, necha g $\mathrm{Na}_{2} \mathrm{SO}_{4}$ hamda necha mol $\mathrm{Cu}(\mathrm{OH})_{2}$ hosil bo'lgan?\\
A) $284 ; 2$\\
B) $142 ; 1$\\
C) $71 ; 4$\\
D) $142 ; 4$\\
30. $3 \mathrm{AgNO}_{3}+\mathrm{H}_{3} \mathrm{PO}_{4} \rightarrow \mathrm{Ag}_{3} \mathrm{PO}_{4}+3 \mathrm{HNO}_{3}$ ushbu reaksiya bo'yicha $98 \mathrm{~g} \mathrm{H}_{3} \mathrm{PO}_{4}$ sarflangan bo'lsa, necha $g \mathrm{Ag}_{3} \mathrm{PO}_{4}$ hamda necha mol $\mathrm{HNO}_{3}$ hosil bo'lgan?\\
A) $419 ; 3$\\
B) $115 ; 7$\\
C) $152 ; 3$\\
D) $114 ; 2$
  \item $2 \mathrm{KClO}_{3} \longrightarrow 2 \mathrm{KCl}+3 \mathrm{O}_{2}$ ushbu reaksiya bo'yicha $245 \mathrm{~g} \mathrm{KClO}_{3}$ sarflangan bo'lsa, necha mol KCl hamda necha litr (\href{http://n.sh}{n.sh}) $\mathrm{O}_{2}$ hosil bo'lgan?\\
A) $1 ; 22,4$\\
B) $2 ; 67,2$\\
C) $5 ; 89,6$\\
D) $3 ; 44,8$
  \item $2 \mathrm{KMnO}_{4} \rightarrow \mathrm{~K}_{2} \mathrm{MnO}_{4}+\mathrm{MnO}_{2}+\mathrm{O}_{2}$ ushbu reaksiya bo'yicha $632 \mathrm{~g} \mathrm{KMnO}_{4}$ sarflangan bo'lsa, necha $\mathrm{mol} \mathrm{MnO}_{2}$ hamda necha litr (\href{http://n.sh}{n.sh}) $\mathrm{O}_{2}$ hosil bo'lgan?\\
A) $2 ; 44,8$\\
B) $2 ; 67,2$\\
C) $5 ; 89,6$\\
D) $3 ; 44,8$
  \item $2 \mathrm{HgO} \rightarrow 2 \mathrm{Hg}+\mathrm{O}_{2}$ ushbu reaksiya bo'yicha 434 g HgO sarflangan bo'lsa, necha mol Hg hamda necha litr (\href{http://n.sh}{n.sh}) $\mathrm{O}_{2}$ hosil bo'lgan?\\
A) $2 ; 67,2$\\
B) $4 ; 89,6$\\
C) $2 ; 22,4$\\
D) $3 ; 44,8$
  \item $2 \mathrm{NaHCO}_{3} \rightarrow \mathrm{Na}_{2} \mathrm{CO}_{2}+\mathrm{CO}_{2}+\mathrm{H}_{2} \mathrm{O}$ ushbu reaksiya bo'yicha $84 \mathrm{~g} \mathrm{NaHCO}_{3}$ sarflangan bo'lsa, necha mol $\mathrm{H}_{2} \mathrm{O}$ hamda necha litr (\href{http://n.sh}{n.sh}) $\mathrm{CO}_{2}$ hosil bo'lgan?\\
A) 0,$5 ; 11,2$\\
B) $1 ; 89,6$\\
C) $1 ; 22,4$\\
D) $2 ; 44,8$
  \item $(\mathrm{CuOH})_{2} \mathrm{CO}_{3} \rightarrow 2 \mathrm{CuO}+\mathrm{CO}_{2}+\mathrm{H}_{2} \mathrm{O}$ ushbu reaksiya bo'yicha 333 g $(\mathrm{CuOH})_{2} \mathrm{CO}_{2}$ sarflangan bo'lsa, necha mol CuO hamda necha litr (\href{http://n.sh}{n.sh}) $\mathrm{CO}_{2}$ hosil bo'lgan?\\
A) $2 ; 67,2$\\
B) $4 ; 89,6$\\
C) $2 ; 22,4$\\
D) $3 ; 33,6$
  \item $\mathrm{NH}_{4} \mathrm{NO}_{3} \rightarrow \mathrm{~N}_{2} \mathrm{O}+2 \mathrm{H}_{2} \mathrm{O}$ ushbu reaksiya bo'yicha $160 \mathrm{~g} \mathrm{NH}_{4} \mathrm{NO}_{3}$ sarflangan bo'lsa, necha mol $\mathrm{H}_{2} \mathrm{O}$ hamda necha litr (\href{http://n.sh}{n.sh}) $\mathrm{N}_{2} \mathrm{O}$ hosil bo'lgan?\\
A) $2 ; 67,2$\\
B) $4 ; 44,8$\\
C) $2 ; 22,4$\\
D) $1 ; 33,6$
  \item $\mathrm{NH}_{4} \mathrm{NO}_{2} \rightarrow \mathrm{~N}_{2}+2 \mathrm{H}_{2} \mathrm{O}$ ushbu reaksiya bo'yicha $192 \mathrm{~g} \mathrm{NH}_{4} \mathrm{NO}_{2}$ sarflangan bo'lsa, necha mol $\mathrm{H}_{2} \mathrm{O}$ hamda necha litr (\href{http://n.sh}{n.sh}) $\mathrm{N}_{2}$ hosil bo'lgan?\\
A) $6 ; 67,2$\\
B) $6 ; 89,6$\\
C) $2 ; 22,4$\\
D) $3 ; 33,6$
  \item $2 \mathrm{Cu}\left(\mathrm{NO}_{3}\right)_{2} \rightarrow 2 \mathrm{CuO}+4 \mathrm{NO}_{2}+\mathrm{O}_{2}$ ushbu reaksiya bo'yicha $188 \mathrm{~g} \mathrm{Cu}^{\prime}\left(\mathrm{NO}_{3}\right)_{2}$ sarflangan bo'lsa, necha mol CuO hamda necha litr (\href{http://n.sh}{n.sh}) $\mathrm{O}_{2}$ hosil bo'lgan?\\
A) $2 ; 67,2$\\
B) $1 ; 22,4$\\
C) $2 ; 22,4$\\
D) $1 ; 11,2$
  \item $2 \mathrm{AgNO}_{3} \rightarrow 2 \mathrm{Ag}+2 \mathrm{NO}_{2}+\mathrm{O}_{2}$ ushbu reaksiya bo'yicha $170 \mathrm{~g} \mathrm{AgNO}_{3}$ sarflangan bo'lsa, necha mol Ag hamda necha litr (\href{http://n.sh}{n.sh}) $\mathrm{O}_{2}$ hosil bo'lgan?\\
A) $2 ; 67,2$\\
B) $1 ; 22,4$\\
C) $2 ; 22,4$\\
D) $1 ; 11,2$
  \item $4 \mathrm{FeSO}_{4} \rightarrow 2 \mathrm{Fe}_{2} \mathrm{O}_{3}+4 \mathrm{SO}_{2}+\mathrm{O}_{2}$ ushbu reaksiya bo'yicha $304 \mathrm{~g} \mathrm{FeSO}_{4}$ sarflangan bo'lsa, necha mol $\mathrm{Fe}_{2} \mathrm{O}_{3}$ hamda necha litr (\href{http://n.sh}{n.sh}) $\mathrm{O}_{2}$ hosil bo'lgan?\\
A) $2 ; 67,2$\\
B) $4 ; 22,4$\\
C) $1 ; 11,2$\\
D) $1 ; 33,6$
  \item $2 \mathrm{AgNO}_{3}+\mathrm{Fe} \rightarrow \mathrm{Fe}\left(\mathrm{NO}_{3}\right)_{2}+2 \mathrm{Ag}$ ushbu reaksiya bo'yicha $340 \mathrm{~g} \mathrm{AgNO}_{3}$ hamda 112 g Fe dan foydalanib necha mol Ag olish mumkin?\\
A) 2\\
B) 1\\
C) 5\\
D) 3
42. $\mathrm{CuSO}_{4}+\mathrm{Mu}_{4} \mathrm{MnSO}_{4}+\mathrm{Cu}_{4}$ ushbu $_{4}$ roaksiya boyicha $320 \mathrm{~g} \mathrm{CuSO}_{4}$ hamdn 55 g Mn fordalanib necha mol Cu olish mumkin?\\
B) 1\\
C) 0.5\\
D) 3\\
N) 0,2\\
43,3 $\mathrm{Ag} \mathrm{NO}_{3}+\mathrm{Al} \rightarrow \mathrm{Al}\left(\mathrm{NO}_{3}\right)_{3}+3 \mathrm{Ag}$ ushbu reaksiya boyicha $170 \mathrm{~g} \mathrm{AgNO}_{3}$ hamda 27 g Al dan foydalanib necha mol Ag olish mumkin?\\
八) 2\\
B) 3\\
C) 5\\
D) 1\\
44. $2 \mathrm{H}_{3} \mathrm{PO}_{4}+6 \mathrm{Na} \rightarrow 2 \mathrm{Na}_{3} \mathrm{PO}_{4}+3 \mathrm{H}_{2}$ ushbu reaksival boyicha $98 \mathrm{~g} \mathrm{H}_{3} \mathrm{PO}_{4}$ hamda $34,5 \mathrm{~g}$ Na dan foydalanib necha mol $\mathrm{Na}_{3} \mathrm{PO}_{4}$ olish mumkin?\\
A) 0.2\\
B) 1\\
C) 0,5\\
D) 3\\
45. $3 \mathrm{H}_{2} \mathrm{SO}_{4}+2 \mathrm{Al} \rightarrow \mathrm{Al}_{2}\left(\mathrm{SO}_{4}\right)_{3}+3 \mathrm{H}_{2}$ ushbu reaksiya bo'yicha $294 \mathrm{~g} \mathrm{H}_{2} \mathrm{SO}_{4}$ hamda 27 g Al dan foydalanib necha $\mathrm{mol} \mathrm{Al}_{2}\left(\mathrm{SO}_{4}\right)_{3}$ olish mumkin?\\
A) 0,2\\
B) 1\\
C) 0,5\\
D) 3\\
46. $2 \mathrm{HCl}+\mathrm{Ca} \rightarrow \mathrm{CaCl}_{2}+\mathrm{H}_{2}$ ushbu reaksiya bo'yicha 146 g HCl hamda 40 g Ca dan foydalnuib necha $\mathrm{mol} \mathrm{H}_{2}$ olish mumkin?\\
A) 2\\
B) 1\\
C) 5\\
D) 3\\
47. $2 \mathrm{H}_{3} \mathrm{PO}_{4}+2 \mathrm{Al} \rightarrow 2 \mathrm{AlPO}_{4}+3 \mathrm{H}_{2}$ ushbu reaksiya bo'yicha $294 \mathrm{~g} \mathrm{H}_{3} \mathrm{PO}_{4}$ hamda 54 g Al dan foydalanib necha mol $\mathrm{AlPO}_{4}$ olish mumkin?\\
A) 2\\
B) 1\\
C) 5\\
D) 3\\
48. $\mathrm{Cr}_{2} \mathrm{O}_{3}+2 \mathrm{Al} \rightarrow \mathrm{Al}_{2} \mathrm{O}_{3}+2 \mathrm{Cr}$ ushbu reaksiya bo'yicha $304 \mathrm{~g} \mathrm{Cr}_{2} \mathrm{O}_{3}$ hamda 27 g Al dan foydalanib necha mol $\mathrm{Al}_{2} \mathrm{O}_{3}$ olish mumkin?\\
A) 0,2\\
B) 1\\
C) 0,5\\
D) 3\\
49. $\mathrm{Fe}_{3} \mathrm{O}_{4}+2 \mathrm{C} \rightarrow 3 \mathrm{Fe}+2 \mathrm{CO}_{2}$ ushbu reaksiya bo'yicha $116 \mathrm{~g} \mathrm{Fe}_{3} \mathrm{O}_{4}$ hamda 24 g C dan foydalanib necha $\mathrm{mol}_{2} \mathrm{CO}_{2}$ olish mumkin?\\
A) 2\\
B) 1\\
C) 5\\
D) 3\\
50. $\mathrm{Fe}_{2} \mathrm{O}_{3}+2 \mathrm{Al} \rightarrow \mathrm{Al}_{2} \mathrm{O}_{3}+2 \mathrm{Fe}$ ushbu reaksiya boyicha $320 \mathrm{~g} \mathrm{Fe}_{2} \mathrm{O}_{3}$ hamda 27 g Al dan foydalanib necha mol Fe olish mumkin?\\
A) 2\\
B) 1\\
C) 5\\
D) 4
  \item $2 \mathrm{Mg}+\mathrm{O}_{2} \rightarrow 2 \mathrm{MgO}$ unhbu reaksiya bo'yicha 2.4 g magniy va $4,8 \mathrm{~g}$ kislorod reaksiyaga kirishsa nocha g qaysi moddudun ortib qoladi?
A) $3,2 \mathrm{O}_{2}$\\
B) $1,6 \mathrm{O}_{2}$\\
C) $6,4 \mathrm{Mg}$\\
D) $4,8 \mathrm{Mg}$\\
52. $4 \mathrm{Al}+3 \mathrm{O}_{2} \rightarrow 2 \mathrm{Al}_{2} \mathrm{O}_{3}$ ushbu reaksiya bo'yicha $5,4 \mathrm{~g} \mathrm{Al}$ va $1,6 \mathrm{~g}$ kislorod reaksiyaga kirishsa necha g qaysi moddadan ortib qoladi?\\
A) $2,7 \mathrm{Al}$\\
B) $3,6 \mathrm{Al}$\\
C) $3,4 \mathrm{O}_{2}$\\
D) $4,8 \mathrm{O}_{2}$\\
53. $2 \mathrm{Zn}+\mathrm{O}_{2} \rightarrow 2 \mathrm{ZnO}$ ushbu reaksiya bo'yicha $6,5 \mathrm{~g} \mathrm{Zn}$ va 16 g kislorod reaksiyaga kirishsa necha g qaysi moddadan ortib qoladi?\\
A) $3,25 \mathrm{Zn}$\\
B) $3,65 \mathrm{Zn}$\\
C) $6,4 \mathrm{O}_{2}$\\
D) $14,4 \mathrm{O}_{2}$\\
54. $4 \mathrm{Li}+\mathrm{O}_{2} \rightarrow 2 \mathrm{Li}_{2} \mathrm{O}$ ushbu reaksiya bo'yicha $5,6 \mathrm{~g} \mathrm{Li}$ va $3,2 \mathrm{~g}$ kislorod reaksiyaga kirishsa necha g qaysi moddadan ortib qoladi?\\
A) $2,8 \mathrm{Li}$\\
B) $1,4 \mathrm{Li}$\\
C) $6,4 \mathrm{O}_{2}$\\
D) $14,4 \mathrm{O}_{2}$\\
55. $3 \mathrm{Fe}+2 \mathrm{O}_{2} \rightarrow \mathrm{Fe}_{3} \mathrm{O}_{4}$ ushbu reaksiya bo'yicha $8,4 \mathrm{~g}$ Fe va 16 g kislorod reaksiyaga kirishsa necha g qaysi moddadan ortib qoladi?\\
A) $2,8 \mathrm{Fe}$\\
B) $1,4 \mathrm{Fe}$\\
C) $6,4 \mathrm{O}_{2}$\\
D) $12,8 \mathrm{O}_{2}$\\
56. $\mathrm{CaO}+\mathrm{SO}_{3} \rightarrow \mathrm{CaSO}_{4}$ ushbu reaksiya bo'yicha $5,6 \mathrm{~g}$ CaO va $16 \mathrm{~g} \mathrm{SO}_{3}$ reaksiyaga kirishsa necha g qaysi moddadan ortib qoladi?\\
A) $8 \mathrm{SO}_{3}$\\
B) $6 \mathrm{SO}_{3}$\\
C) 4 CaO\\
D) 3 CaO\\
57. $\mathrm{Na} 2 \mathrm{O}+\mathrm{H}_{2} \mathrm{O} \rightarrow 2 \mathrm{NaOH}$ ushbu reaksiya bo'yicha $6.2 \mathrm{~g} \mathrm{Na}_{2} \mathrm{O}$ va $3,6 \mathrm{~g} \mathrm{H}_{2} \mathrm{O}$\\
reaksiyaga kirishsa necha g qaysi moddadan ortib qoladi?\\
A) $1,8 \mathrm{H}_{2} \mathrm{O}$\\
B) $1,6 \mathrm{H}_{2} \mathrm{O}$\\
C) $3,1 \mathrm{Na}_{2} \mathrm{O}$\\
D) $0,9 \mathrm{Na}_{2} \mathrm{O}$\\
58. $\mathrm{Al}_{2} \mathrm{O}_{3}+3 \mathrm{SO}_{3} \rightarrow \mathrm{Al}_{2}\left(\mathrm{SO}_{4}\right)_{3}$ ushbu reaksiya bo'yicha $10,2 \mathrm{~g} \mathrm{Al}_{2} \mathrm{O}_{3}$ va $48 \mathrm{~g} \mathrm{SO}_{3}$ reaksiyaga kirishsa necha g qaysi moddadan ortib qoladi?\\
A) $32 \mathrm{Al}_{2} \mathrm{O}_{3}$\\
B) $16 \mathrm{Al}_{2} \mathrm{O}_{3}$\\
C) $24 \mathrm{SO}_{3}$\\
D) $48 \mathrm{SO}_{3}$\\
59. $\mathrm{P}_{2} \mathrm{O}_{5}+3 \mathrm{H}_{2} \mathrm{O} \rightarrow 2 \mathrm{H}_{3} \mathrm{PO}_{4}$ ushbu reaksiya bo'yicha $71 \mathrm{~g} \mathrm{P}_{2} \mathrm{O}_{5}$ va $36 \mathrm{~g} \mathrm{H}_{2} \mathrm{O}$ reaksiyaga kirishsa necha g qaysi moddadan ortib qoladi?\\
A) $27 \mathrm{P}_{2} \mathrm{O}_{5}$\\
B) $9 \mathrm{H}_{2} \mathrm{O}$\\
C) $54 \mathrm{P}_{2} \mathrm{O}_{5}$\\
D) $36 \mathrm{H}_{2} \mathrm{O}$\\
60. $\mathrm{SO}_{3}+\mathrm{H}_{2} \mathrm{O} \rightarrow \mathrm{H}_{2} \mathrm{SO}_{4}$ ushbu reaksiya bo'yicha $8 \mathrm{~g} \mathrm{SO}_{3}$ va $3,6 \mathrm{~g} \mathrm{H}_{2} \mathrm{O}$ reaksiyaga kirishsa necha g qaysi moddadan ortib qoladi?\\
A) $2,7 \mathrm{H}_{2} \mathrm{O}$\\
B) $1,8 \mathrm{H}_{2} \mathrm{O}$\\
C) $5,4 \mathrm{SO}_{3}$\\
D) $3,6 \mathrm{SO}_{3}$
  \item $2 \mathrm{Mg}+\mathrm{O}_{2} \rightarrow 2 \mathrm{MgO}$ ushbu reaksiya bo'yicha $2,4 \mathrm{~g} \mathrm{Mg}$ kislorod bilan reaksiyaga kirishganda 2 g MgO hosil bo'lsa, reaksiya unumi necha \% teng?\\
A) 100\\
B) 50\\
C) 40\\
D) 60\\
  \item $4 \mathrm{Al}+3 \mathrm{O}_{2} \rightarrow 2 \mathrm{Al}_{2} \mathrm{O}_{3}$ ushbu reaksiya bo'yicha $5,4 \mathrm{~g} \mathrm{Al}$ kislorod bilan reaksiyaga kirishganda $4,08 \mathrm{~g} \mathrm{Al}_{2} \mathrm{O}_{3}$ hosil bo'lsa, reaksiya unumi necha \% teng?\\
A) 100\\
B) 50\\
C) 40\\
D) 60
  \item $2 \mathrm{Zn}+\mathrm{O}_{2} \rightarrow 2 \mathrm{ZnO}$ ushbu reaksiya bo'yicha $6,5 \mathrm{~g} \mathrm{Zn}$ kislorod bilan reaksiyaga kirishganda $4,86 \mathrm{~g} \mathrm{ZnO}$ hosil bo'lsa, reaksiya unumi necha \% teng?\\
A) 100\\
B) 50\\
C) 40\\
D) 60
  \item $4 \mathrm{Li}+\mathrm{O}_{2} \rightarrow 2 \mathrm{Li}_{2} \mathrm{O}$ ushbu reaksiya bo'yicha $5,6 \mathrm{~g} \mathrm{Li}$ kislorod bilan reaksiyaga kirishganda $12 \mathrm{~g} \mathrm{Li}_{2} \mathrm{O}$ hosil bo'lsa, reaksiya unumi necha \% teng?\\
A) 100\\
B) 50\\
C) 40\\
D) 60
  \item $3 \mathrm{Fe}+2 \mathrm{O}_{2} \rightarrow \mathrm{Fe}_{3} \mathrm{O}_{4}$ ushbu reaksiya bo'yicha $8,4 \mathrm{~g}$ Fe kislorod bilan reaksiyaga kirishganda $4,64 \mathrm{~g} \mathrm{Fe}_{3} \mathrm{O}_{4}$ hosil bo'lsa, reaksiya unumi necha \% teng?\\
A) 100\\
B) 50\\
C) 40\\
D) 60
  \item $\mathrm{CaO}+\mathrm{SO}_{3} \rightarrow \mathrm{CaSO}_{4}$ ushbu reaksiya bo'yicha $5,6 \mathrm{~g} \mathrm{CaO}$ sulfat angidirid bilan reaksiyaga kirishganda $8,16 \mathrm{~g} \mathrm{CaSO}_{4}$ hosil bo'lsa, reaksiya unumi necha \% teng?\\
A) 100\\
B) 50\\
C) 40\\
D) 60
  \item $\mathrm{Na}_{2} \mathrm{O}+\mathrm{H}_{2} \mathrm{O} \rightarrow 2 \mathrm{NaOH}$ ushbu reaksiya bo'yicha $3,1 \mathrm{~g} \mathrm{Na}_{2} \mathrm{O}$ suv bilan reaksiyaga kirishganda 2 g NaOH hosil bo'lsa, reaksiya unumi necha \% teng?\\
A) 100\\
B) 50\\
C) 40\\
D) 60
  \item $\mathrm{Al}_{2} \mathrm{O}_{3}+3 \mathrm{SO}_{3} \rightarrow \mathrm{Al}_{2}\left(\mathrm{SO}_{4}\right)_{3}$ ushbu reaksiya bo'yicha $10,2 \mathrm{~g} \mathrm{Al}_{2} \mathrm{O}_{3}$ sulfat angidirid bilan reaksiyaga kirishganda $17,1 \mathrm{~g}$ $\mathrm{Al}_{2}\left(\mathrm{SO}_{4}\right)_{3}$ hosil bo'lsa, reaksiya unumi necha \% teng?\\
A) 100\\
B) 50\\
C) 40\\
D) 60
  \item $\mathrm{P}_{2} \mathrm{O}_{5}+3 \mathrm{H}_{2} \mathrm{O} \rightarrow 2 \mathrm{H}_{3} \mathrm{PO}_{4}$ ushbu reaksiya bo'yicha $71 \mathrm{~g} \mathrm{P}_{2} \mathrm{O}_{5}$ suv bilan reaksiyaga kirishganda $39,2 \mathrm{~g} \mathrm{H}_{3} \mathrm{PO}_{4}$ hosil bo'lsa, reaksiya unumi necha \% teng?\\
A) 100\\
B) 50\\
C) 40\\
D) 60
  \item $\mathrm{SO}_{3}+\mathrm{H}_{2} \mathrm{O} \rightarrow \mathrm{H}_{2} \mathrm{SO}_{4}$ ushbu reaksiya bo'yicha $8 \mathrm{~g} \mathrm{SO}_{3}$ suv bilan reaksiyaga\\
kirishyanda $9.8 \mathrm{~g} \mathrm{H}_{4} \mathrm{SO}_{4}$ hosil bo'les. resksym unumi necha 96 teng?\\
A) 100\\
B) 50\\
C) 40\\
D) 60
  \item $2 \mathrm{AgNO}_{s}+\mathrm{Fe} \rightarrow \mathrm{Fo}\left(\mathrm{NO}_{a}\right)_{g}+2 \mathrm{Ag}$ ushbu reaksiya bo'yicha $340 \mathrm{~g} \mathrm{AgNO}_{*}$ dan foydalanib necha mol Ag olish mumkin? ( $\eta=50 \%$ )\\
A) 2\\
B) 1\\
C) 5\\
D) 3
  \item $\mathrm{CuSO}_{4}+\mathrm{Mn} \rightarrow \mathrm{MnSO}_{4}+\mathrm{Cu}$ ushbu reaksiya bo'yicha $80 \mathrm{~g} \mathrm{CuSO}_{4}$ dan foydalanib necha mol Cu olish mumkin? ( $n=100 \%$ )\\
A) 0,2\\
B) 1\\
C) 0,5\\
D) 3
  \item $3 \mathrm{AgNO} \mathrm{N}_{3}+\mathrm{Al} \rightarrow \mathrm{Al}\left(\mathrm{NO}_{3}\right)_{3}+3 \mathrm{Ag}$ ushbu reaksiya bo'yicha $170 \mathrm{~g} \mathrm{AgNO}_{3}$ dan foydalanib necha mol Ag olish mumkin? ( $n=40 \%$ )\\
A) 0,2\\
B) 0,4\\
C) 0,25\\
D) 0,3
  \item $2 \mathrm{H}_{3} \mathrm{PO}_{4}+6 \mathrm{Na} \rightarrow 2 \mathrm{Na}_{3} \mathrm{PO}_{4}+3 \mathrm{H}_{2}$ ushbu reaksiya bo'yicha $98 \mathrm{~g} \mathrm{H}_{3} \mathrm{PO}_{4}$ dan foydalanib necha mol $\mathrm{Na}_{3} \mathrm{PO}_{4}$ olish mumkin? ( $\mathrm{n}=80 \%$ )\\
A) 0,8\\
B) 1\\
C) 0,5\\
D) 3
  \item $3 \mathrm{H}_{2} \mathrm{SO}_{4}+2 \mathrm{Al} \rightarrow \mathrm{Al}_{2}\left(\mathrm{SO}_{4}\right)_{3}+3 \mathrm{H}_{2}$ ushbu reaksiya bo'yicha $294 \mathrm{~g} \mathrm{H}_{2} \mathrm{SO}_{4}$ dan foydalanib necha $\mathrm{mol} \mathrm{Al}_{2}\left(\mathrm{SO}_{4}\right)_{3}$ olish mumkin? ( $\mathrm{n}=70 \%$ )\\
A) 0,2\\
B) 0,7\\
C) 0,5\\
D) 0,3
  \item $2 \mathrm{HCl}+\mathrm{Ca} \rightarrow \mathrm{CaCl}_{2}+\mathrm{H}_{2}$ ushbu reaksiya bo'yicha 146 g HCl dan foydalanib necha mol $\mathrm{H}_{2}$ olish mumkin? ( $n=50 \%$ )\\
A) 2\\
B) 1\\
C) 5\\
D) 3
  \item $2 \mathrm{H}_{3} \mathrm{PO}_{4}+2 \mathrm{Al} \rightarrow 2 \mathrm{AlPO}_{4}+3 \mathrm{H}_{2}$ ushbu reaksiya bo'yicha $294 \mathrm{~g} \mathrm{H}_{3} \mathrm{PO}_{4}$ dan foydalanib necha $\mathrm{mol} \mathrm{AlPO}_{4}$ olish mumkin? ( $\eta=100 \%$ )\\
A) 2\\
B) 1\\
C) 5\\
D) 3
  \item $\mathrm{Cr}_{2} \mathrm{O}_{3}+2 \mathrm{Al} \rightarrow \mathrm{Al}_{2} \mathrm{O}_{3}+2 \mathrm{Cr}$ ushbu reaksiya boyicha $304 \mathrm{~g} \mathrm{Cr}_{2} \mathrm{O}_{3}$ dan foydalanib necha $\mathrm{mol} \mathrm{Al}_{4} \mathrm{O}_{3}$ olish mumkin? ( $n=40 \%$ )\\
A) 0.8\\
B) 1\\
C) 0,5\\
D) 3
  \item $\mathrm{Fe}_{4} \mathrm{O}_{4}+2 \mathrm{C} \rightarrow 3 \mathrm{Fe}+2 \mathrm{CO}_{2}$ ushbu reaksiya boyicha $116 \mathrm{~g} \mathrm{Fe}_{3} \mathrm{O}_{4}$ dan foydalanib necha mol $\mathrm{CO}_{2}$ olish mumkin? $(n=60 \%)$\\
A) 0,2\\
B) 1\\
C) 0,5\\
D) 0,6
  \item $\mathrm{Fe}_{2} \mathrm{O}_{3}+2 \mathrm{Al} \rightarrow \mathrm{Al}_{2} \mathrm{O}_{3}+2 \mathrm{Fe}$ ushbu reaksiya bo'yicha $320 \mathrm{~g} \mathrm{Fe}_{2} \mathrm{O}_{3}$ dan foydalanib necha mol Fe olish mumkin? ( $n=25 \%$ )\\
A) 2\\
B) 1\\
C) 5\\
D) 4
  \item $2 \mathrm{AgNO}_{3}+\mathrm{Fe} \rightarrow \mathrm{Fe}\left(\mathrm{NO}_{3}\right)_{2}+2 \mathrm{Ag}$ ushbu reaksiya bo'yicha necha g $\mathrm{AgNO}_{3}$ dan foydalanib 1 mol Ag olish mumkin? ( $n=50 \%$ )\\
A) 340\\
B) 170\\
C) 510\\
D) 333
  \item $\mathrm{CuSO}_{4}+\mathrm{Mn} \rightarrow \mathrm{MnSO}_{4}+\mathrm{Cu}$ ushbu reaksiya bo'yicha necha g $\mathrm{CuSO}_{4}$ dan foydalanib $0,5 \mathrm{~mol} \mathrm{Cu}$ olish mumkin? ( $\mathrm{n}=100 \%$ )\\
A) 160\\
B) 16\\
C) 80\\
D) 32
  \item $3 \mathrm{AgNO}_{3}+\mathrm{Al} \rightarrow \mathrm{Al}\left(\mathrm{NO}_{3}\right)_{3}+3 \mathrm{Ag}$ ushbu reaksiya bo'yicha necha g $\mathrm{AgNO}_{3}$ dan foydalanib $0,4 \mathrm{~mol} \mathrm{Ag}$ olish mumkin? ( $\mathrm{n}=40 \%$ )\\
A) 17\\
B) 170\\
C) 34\\
D) 340
  \item $2 \mathrm{H}_{3} \mathrm{PO}_{4}+6 \mathrm{Na} \rightarrow 2 \mathrm{Na}_{3} \mathrm{PO}_{4}+3 \mathrm{H}_{2}$ ushbu reaksiya bo'yicha nechag $\mathrm{H}_{3} \mathrm{PO}_{4}$ dan foydalanib $1,6 \mathrm{~mol} \mathrm{Na}_{3} \mathrm{PO}_{4}$ olish mumkin? ( $\mathrm{n}=80 \%$ )\\
A) 196\\
B) 98\\
C) 98\\
D) 39,2
  \item $3 \mathrm{H}_{2} \mathrm{SO}_{4}+2 \mathrm{Al} \rightarrow \mathrm{Al}_{2}\left(\mathrm{SO}_{4}\right)_{3}+3 \mathrm{H}_{2}$ ushbu reaksiya bo'yicha necha $\mathrm{g} \mathrm{H}_{2} \mathrm{SO}_{4}$ dan foydalanib $2,8 \mathrm{~mol} \mathrm{Al}_{2}\left(\mathrm{SO}_{-1}\right)_{3}$ olish mumkin? ( $\eta=70 \%$ )\\
A) 1176\\
B) 1980\\
C) 3920\\
D) 980
  \item $2 \mathrm{HCl}+\mathrm{Ca} \rightarrow \mathrm{CaCl}_{2}+\mathrm{H}_{2}$ ushbu reaksiya bo'yicha necha g HCl dan foydalanib 2 mol $\mathrm{H}_{2}$ olish mumkin? ( $\mathrm{n}=50 \%$ )\\
A) 365\\
B) 292\\
C) 730\\
D) 36,5
  \item $2 \mathrm{H}_{3} \mathrm{PO}_{4}+2 \mathrm{Al} \rightarrow 2 \mathrm{AlPO}_{4}+3 \mathrm{H}_{2}$ ushbu reaksiya bo'yicha necha $\mathrm{g} \mathrm{H}_{3} \mathrm{PO}_{4}$ dan foydalanib necha $2 \mathrm{~mol} \mathrm{AlPO}_{4}$ olish mumkin? ( $\mathrm{n}=100 \%$ )\\
A) 98\\
B) 19,2\\
C) 39,2\\
D) 196
  \item $\mathrm{Cr}_{2} \mathrm{O}_{3}+2 \mathrm{Al} \rightarrow \mathrm{Al}_{2} \mathrm{O}_{3}+2 \mathrm{Cr}$ ushbu reaksiya bo'yicha necha $\mathrm{g} \mathrm{Cr}_{2} \mathrm{O}_{3}$ dan foydalanib necha $1,6 \mathrm{~mol} \mathrm{Al}_{2} \mathrm{O}_{3}$ olish mumkin? ( $\mathrm{n}=40 \%$ )\\
A) 608\\
B) 152\\
C) 60,8\\
D) 304
  \item $\mathrm{Fe}_{3} \mathrm{O}_{4}+2 \mathrm{C} \rightarrow 3 \mathrm{Fe}+2 \mathrm{CO}_{2}$ ushbu reaksiya bo'yicha necha $\mathrm{g} \mathrm{Fe}_{3} \mathrm{O}_{4}$ dan foydalanib 3 mol $\mathrm{CO}_{2}$ olish mumkin? ( $n=60 \%$ )\\
A) 290\\
B) 464\\
C) 232\\
D) 580
  \item $\mathrm{Fe}_{2} \mathrm{O}_{3}+2 \mathrm{Al} \rightarrow \mathrm{Al}_{2} \mathrm{O}_{3}+2 \mathrm{Fe}$ ushbu reaksiya bo'yicha necha $\mathrm{g} \mathrm{Fe}_{2} \mathrm{O}_{3}$ dan foydalanib necha $0,25 \mathrm{~mol}$ Fe olish mumkin? ( $n=25 \%$ )\\
A) 80\\
B) 160\\
C) 40\\
D) 240
  \item $2 \mathrm{KClO}_{3} \longrightarrow 2 \mathrm{KCl}+3 \mathrm{O}_{2}$ ushbu reaksiya bo'yicha $50 \%$ qo'shimchasi bo'lgan $490 \mathrm{~g} \mathrm{KClO}_{3}$ sarflangan bo'lsa, necha mol KCl hamda necha litr (\href{http://n.sh}{n.sh}) $\mathrm{O}_{2}$ hosil bo'lgan?\\
A) $1 ; 22,4$\\
B) $2 ; 67,2$\\
C) $5 ; 89,6$\\
D) $3 ; 44,8$
33. $2 \mathrm{HgO} \rightarrow 2 \mathrm{Hg}+\mathrm{O}_{2}$ ushbu reaksiya bo'yicha $50 \%$ qo'shimchasi bo'lgan 868 g HgO sarflangan bo'lsa, necha mol Hg hamda necha litr (\href{http://n.sh}{n.sh}) $\mathrm{O}_{2}$ hosil bo'lgan?\\
A) $2 ; 67,2$\\
B) $4 ; 89,6$\\
C) $2 ; 22,4$\\
D) $3 ; 44,8$\\
34. $2 \mathrm{NaHCO}_{3} \rightarrow \mathrm{Na}_{2} \mathrm{CO}_{3}+\mathrm{CO}_{2}+\mathrm{H}_{2} \mathrm{O}$ ushbu reaksiya bo'yicha $50 \%$ qo'shimchasi bo'lgan $168 \mathrm{~g} \mathrm{NaHCO}_{3}$ sarflangan bo'lsa, necha mol $\mathrm{H}_{2} \mathrm{O}$ hamda necha litr (\href{http://n.sh}{n.sh}) $\mathrm{CO}_{2}$ hosil bo'lgan?\\
A) 0,$5 ; 11,2$\\
B) $1 ; 89,6$\\
C) $1 ; 22,4$\\
D) $2 ; 44,8$\\
35. $(\mathrm{CuOH})_{2} \mathrm{CO}_{3} \rightarrow 2 \mathrm{CuO}+\mathrm{CO}_{2}+\mathrm{H}_{2} \mathrm{O}$ ushbu reaksiya bo'yicha $50 \%$ qo'shimchasi bo'lgan $666 \mathrm{~g}(\mathrm{CuOH})_{2} \mathrm{CO}_{3}$ sarflangan bo'lsa, necha mol CuO hamda necha litr (\href{http://n.sh}{n.sh}) $\mathrm{CO}_{2}$ hosil bo'lgan?\\
A) $2 ; 67,2$\\
B) $4 ; 89,6$\\
C) $2 ; 22,4$\\
D) $3 ; 33,6$\\
36. $\mathrm{NH}_{4} \mathrm{NO}_{3} \rightarrow \mathrm{~N}_{2} \mathrm{O}+2 \mathrm{H}_{2} \mathrm{O}$ ushbu reaksiya bo'yicha $20 \%$ qo'shimchasi bo'lgan 200 g $\mathrm{NH}_{4} \mathrm{NO}_{3}$ sarflangan bo'lsa, necha mol $\mathrm{H}_{2} \mathrm{O}$ hamda necha litr (\href{http://n.sh}{n.sh}) $\mathrm{N}_{2} \mathrm{O}$ hosil bo'lgan?\\
A) $2 ; 67,2$\\
B) $4 ; 44,8$\\
C) $2 ; 22,4$\\
D) $1 ; 33,6$\\
37. $\mathrm{NH}_{4} \mathrm{NO}_{2} \rightarrow \mathrm{~N}_{2}+2 \mathrm{H}_{2} \mathrm{O}$ ushbu reaksiya bo'yicha $20 \%$ qo'shimchasi bo'lgan 240 g $\mathrm{NH}_{4} \mathrm{NO}_{2}$ sarflangan bo'lsa, necha mol $\mathrm{H}_{2} \mathrm{O}$ hamda necha litr (\href{http://n.sh}{n.sh}) $\mathrm{N}_{2}$ hosil bo'lgan?\\
A) $6 ; 67,2$\\
B) $6 ; 89,6$\\
C) $2 ; 22,4$\\
D) $3 ; 33,6$\\
38. $2 \mathrm{Cu}\left(\mathrm{NO}_{3}\right)_{2} \rightarrow 2 \mathrm{CuO}+4 \mathrm{NO}_{2}+\mathrm{O}_{2}$ ushbu reaksiya bo'yicha $6 \%$ qo'shimchasi bo'lgan $200 \mathrm{~g} \mathrm{Cu}\left(\mathrm{NO}_{3}\right)_{2}$ sarflangan bo'lsa, necha mol CuO hamda necha litr (\href{http://n.sh}{n.sh}) $\mathrm{O}_{2}$ hosil bo'lgan?\\
A) $2 ; 67,2$\\
B) $1 ; 22,4$\\
C) $2 ; 22,4$\\
D) $1 ; 11,2$\\
39. $2 \mathrm{AgNO}_{3} \rightarrow 2 \mathrm{Ag}+2 \mathrm{NO}_{2}+\mathrm{O}_{2}$ ushbu reaksiya bo'yicha $15 \%$ qo'shimchasi bo'lgan $200 \mathrm{~g} \mathrm{AgNO}_{3}$ sarflangan bo'lsa, necha mol Ag hamda necha litr (\href{http://n.sh}{n.sh}) $\mathrm{O}_{2}$ hosil bo'lgan?\\
A) $2 ; 67,2$\\
B) $1 ; 22,4$\\
C) $2 ; 22,4$\\
D) $1 ; 11,2$\\
40. $4 \mathrm{FeSO}_{4} \rightarrow 2 \mathrm{Fe}_{2} \mathrm{O}_{3}+4 \mathrm{SO}_{2}+\mathrm{O}_{2}$ ushbu reaksiya bo'yicha $24 \%$ qo'shimchasi bo'lgan $400 \mathrm{~g} \mathrm{FeSO}_{4}$ sarflangan bo'lsa, necha mol $\mathrm{Fe}_{2} \mathrm{O}_{3}$ hamda necha litr (\href{http://n.sh}{n.sh}) $\mathrm{O}_{2}$ hosil bo'lgan?\\
A) $2: 67,2$\\
B) $4: 22,4$\\
C) $1 ; 11,2$\\
D) $1: 33,6$
42. $2 \mathrm{KMnO}_{4} \rightarrow \mathrm{~K}_{2} \mathrm{MnO}_{4}+\mathrm{MnO}_{2}+\mathrm{O}_{2}$ ushbu reaksiya bo'yicha $80 \%$ sof moddasi bo'lgan $395 \mathrm{~g} \mathrm{KMnO}_{4}$ sarflangan bo'lsa, necha mol $\mathrm{MnO}_{2}$ hamda necha litr (\href{http://n.sh}{n.sh}) $\mathrm{O}_{2}$ hosil bo'lgan?\\
A) $2 ; 44,8$\\
B) $2 ; 67,2$\\
C) $1: 22,4$\\
D) $3: 44,8$\\
43. $2 \mathrm{HgO} \rightarrow 2 \mathrm{Hg}+\mathrm{O}_{2}$ ushbu reaksiya bo'yicha 50\% sof moddasi bo'lgan 434 g HgO sarflangan bo'lsa, necha mol Hg hamda necha litr (\href{http://n.sh}{n.sh}) $\mathrm{O}_{2}$ hosil bo'lgan?\\
A) $2 ; 67,2$\\
B) $1: 11,2$\\
C) $2 ; 22,4$\\
D) $3 ; 44,8$\\
44. $2 \mathrm{NaHCO}_{3} \rightarrow \mathrm{Na}_{2} \mathrm{CO}_{3}+\mathrm{CO}_{2}+\mathrm{H}_{2} \mathrm{O}$ ushbu reaksiya bo'yicha $50 \%$ sof moddasi bo'lgan $336 \mathrm{~g} \mathrm{NaHCO}_{3}$ sarflangan bo'lsa, necha mol $\mathrm{H}_{2} \mathrm{O}$ hamda necha litr (\href{http://n.sh}{n.sh}) $\mathrm{CO}_{2}$ hosil bo'lgan?\\
A) $0,5,11,2$\\
B) $1 ; 89,6$\\
C) $1 ; 22,4$\\
D) $2 ; 44,8$\\
45. $(\mathrm{CuOH})_{2} \mathrm{CO}_{3} \rightarrow 2 \mathrm{CuO}+\mathrm{CO}_{2}+\mathrm{H}_{2} \mathrm{O}$ ushbu reaksiya bo'yicha $50 \%$ sof moddasi bo'lgan $666 \mathrm{~g}(\mathrm{CuOH})_{2} \mathrm{CO}_{3}$ sarflangan bo'lsa, necha mol CuO hamda necha litr (\href{http://n.sh}{n.sh}) $\mathrm{CO}_{2}$ hosil bo'lgan?\\
A) $2 ; 67,2$\\
B) $4 ; 89,6$\\
C) $2 ; 22,4$\\
D) $3 ; 33,6$\\
46. $\mathrm{NH}_{4} \mathrm{NO}_{3} \rightarrow \mathrm{~N}_{2} \mathrm{O}+2 \mathrm{H}_{2} \mathrm{O}$ ushbu reaksiya bo'yicha $80 \%$ sof moddasi bo'lgan 100 g\\
$\mathrm{NH}_{4} \mathrm{NO}_{3}$ aarflangan bo'lsa, nocha mol $\mathrm{H}_{2} \mathrm{O}$ hamda necha litr (\href{http://n.sh}{n.sh}) $\mathrm{N}_{2} \mathrm{O}$ homil bo'lgan?\\
A) $2 ; 67,2$\\
B) $2: 22,4$\\
C) $2: 11,2$\\
D) $1 ; 33,6$\\
47. $\mathrm{NH}_{1} \mathrm{NO}_{2} \rightarrow \mathrm{~N}_{2}+2 \mathrm{H}_{2} \mathrm{O}$ ushbu reaksiya bo'yicha 80\% rof moddasi bo'lgan 120 g $\mathrm{NH}_{4} \mathrm{NO}_{2}$ вагflangan bo'lsa, necha mol $\mathrm{H}_{2} \mathrm{O}$ hamda necha litr ( $\mathrm{n}, \mathrm{sh}$ ) $\mathrm{N}_{2}$ hosil bo'lgan?\\
A) $6: 67,2$\\
B) $6: 89,6$\\
C) $2 ; 22,4$\\
D) $3 ; 33,6$\\
48. $2 \mathrm{Cu}\left(\mathrm{NO}_{3}\right)_{2} \rightarrow 2 \mathrm{CuO}+4 \mathrm{NO}_{2}+\mathrm{O}_{2}$ ushbu reaksiya bo'yicha $94 \%$ sof moddasi bo'lgan $400 \mathrm{~g} \mathrm{Cu}\left(\mathrm{NO}_{3}\right)_{2}$ sarflangan bo'lsa, necha mol CuO hamda necha litr (\href{http://n.sh}{n.sh}) $\mathrm{O}_{2}$ hosil bo'lgan?\\
A) $2 ; 67,2$\\
B) $1 ; 22,4$\\
C) $2 ; 22,4$\\
D) $1 ; 11,2$\\
49. $2 \mathrm{AgNO}_{3} \rightarrow 2 \mathrm{Ag}+2 \mathrm{NO}_{2}+\mathrm{O}_{2}$ ushbu reaksiya bo'yicha $85 \%$ sof moddasi bo'lgan $400 \mathrm{~g} \mathrm{AgNO}_{3}$ sarflangan bo'lsa, necha mol Ag hamda necha litr (\href{http://n.sh}{n.sh}) $\mathrm{O}_{2}$ hosil bo'lgan?\\
A) $2 ; 67,2$\\
B) $1 ; 22,4$\\
C) $2 ; 22,4$\\
D) $1 ; 11,2$

50, $4 \mathrm{FeSO}_{4} \rightarrow 2 \mathrm{Fe}_{2} \mathrm{O}_{3}+4 \mathrm{SO}_{2}+\mathrm{O}_{2}$ ushbu reaksiya bo'yicha $76 \%$ sof moddasi bo'lgan $200 \mathrm{~g} \mathrm{FeSO}_{4}$ sarflangan bo'lsa, necha mol $\mathrm{Fe}_{2} \mathrm{O}_{3}$ hamda necha litr (\href{http://n.sh}{n.sh}) $\mathrm{O}_{2}$ hosil bo'lgan?\\
A) $2 ; 67,2$\\
B) $4 ; 22,4$\\
C) 0,$5 ; 5,6$\\
D) $1 ; 33,6$
  \item $2 \mathrm{KClO}_{3} \longrightarrow 2 \mathrm{KCl}+3 \mathrm{O}_{2}$ ushbu reaksiya bo'yicha 245 g qo'shimchasi bo'lgan $75 \%$ sof modda saqlagan $\mathrm{KClO}_{3}$ sarflangan bo'lsa, necha mol KCl hamda necha litr (\href{http://n.sh}{n.sh}) $\mathrm{O}_{2}$ hosil bo'lgan?\\
A) $4 ; 33,6$\\
B) $2 ; 67,2$\\
C) $2 ; 89,6$\\
D) $6 ; 201,6$
  \item $2 \mathrm{KMnO}_{4} \rightarrow \mathrm{~K}_{2} \mathrm{MnO}_{4}+\mathrm{MnO}_{2}+\mathrm{O}_{2}$ ushbu reaksiya bo'yicha 316 g qo'shimchasi bo'lgan $80 \%$ sof modda saqlagan $\mathrm{KMnO}_{4}$ sarflangan bo'lsa, necha mol $\mathrm{MnO}_{2}$ hamda necha litr (\href{http://n.sh}{n.sh}) $\mathrm{O}_{2}$ hosil bo'lgan?\\
A) $4 ; 89.6$\\
B) $2 ; 67,2$\\
C) $1 ; 22,4$\\
D) $3 ; 44,8$
  \item $2 \mathrm{HgO} \rightarrow 2 \mathrm{Hg}+\mathrm{O}_{2}$ ushbu reaksiya bo'yicha 217 g qo'shimchasi bo'lgan $50 \%$ sof modda saqlagan HgO sarflangan bo'lsa, necha mol Hg hamda necha litr (\href{http://n.sh}{n.sh}) $\mathrm{O}_{2}$ hosil bo'lgan?\\
A) $2 ; 67,2$\\
B) $1 ; 11,2$\\
C) $2 ; 22,4$\\
D) $3 ; 44,8$
  \item $2 \mathrm{NaHCO}_{3} \rightarrow \mathrm{Na}_{2} \mathrm{CO}_{3}+\mathrm{CO}_{2}+\mathrm{H}_{2} \mathrm{O}$ ushbu reaksiya bo'yicha 84 g qo'shimchasi bo'lgan $60 \%$ sof modda saqlagan $\mathrm{NaHCO}_{3}$ sarflangan bo'lsa, necha mol $\mathrm{H}_{2} \mathrm{O}$ hamda necha litr (\href{http://n.sh}{n.sh}) $\mathrm{CO}_{2}$ hosil bo'lgan?\\
A) 0,$75 ; 16,8$\\
B) $1 ; 89,6$\\
C) $1 ; 22,4$\\
D) $2 ; 44,8$
  \item $(\mathrm{CuOH})_{2} \mathrm{CO}_{3} \rightarrow 2 \mathrm{CuO}+\mathrm{CO}_{2}+\mathrm{H}_{2} \mathrm{O}$ ushbu reaksiya bo'yicha 222 gr qo'shimchasi bo'lgan $50 \%$ sof modda saqlagan $(\mathrm{CuOH})_{2} \mathrm{CO}_{3}$ sarflangan bo'lsa, necha mol CuO hamda necha litr (\href{http://n.sh}{n.sh}) $\mathrm{CO}_{2}$ hosil bo'lgan?\\
A) $2 ; 67,2$\\
B) $4 ; 89,6$\\
C) $2 ; 22,4$\\
D) $3 ; 33,6$
  \item $\mathrm{NH}_{4} \mathrm{NO}_{3} \rightarrow \mathrm{~N}_{2} \mathrm{O}+2 \mathrm{H}_{2} \mathrm{O}$ ushbu reaksiya bo'yicha 160 gr qo'shimchasi bo'lgan $80 \%$ sof modda saqlagan $\mathrm{NH}_{4} \mathrm{NO}_{3}$ sarflangan bo'lsa, necha mol $\mathrm{H}_{2} \mathrm{O}$ hamda necha litr (\href{http://n.sh}{n.sh}) $\mathrm{N}_{2} \mathrm{O}$ hosil bo'lgan?\\
A) $2 ; 67,2$\\
B) $4 ; 22,4$\\
C) $12 ; 112$\\
D) $16 ; 179,2$
  \item $\mathrm{NH}_{4} \mathrm{NO}_{2} \rightarrow \mathrm{~N}_{2}+2 \mathrm{H}_{2} \mathrm{O}$ ushbu reaksiya bo'yicha 64 gr qo'shimchasi bo'lgan $80 \%$ sof modda saqlagan $\mathrm{NH}_{4} \mathrm{NO}_{2}$ sarflangan bo'lsa, necha mol $\mathrm{H}_{2} \mathrm{O}$ hamda necha litr (\href{http://n.sh}{n.sh}) $\mathrm{N}_{2}$ hosil bo'lgan?\\
A) $8 ; 89,6$\\
B) $4 ; 22,4$\\
C) $12 ; 112$\\
D) $16 ; 179,2$
  \item $2 \mathrm{Cu}\left(\mathrm{NO}_{3}\right)_{2} \rightarrow 2 \mathrm{CuO}+4 \mathrm{NO}_{2}+\mathrm{O}_{2}$ ushbu reaksiya 376 g qo'shimchasi bo'yicha $50 \%$ sof modda saqlagan $\mathrm{Cu}\left(\mathrm{NO}_{3}\right)_{2}$ sarflangan bo'lsa, necha mol CuO hamda necha litr (\href{http://n.sh}{n.sh}) $\mathrm{O}_{2}$ hosil bo'lgan?\\
A) $2: 67,2$\\
B) $1: 22,4$\\
C) $2: 22,4$\\
D) $1: 11,2$
  \item $2 \mathrm{AgNO}_{3} \rightarrow 2 \mathrm{Ag}+2 \mathrm{NO}_{2}+\mathrm{O}_{2}$ ushbu reaksiya bo'yicha 170 g qo'shimchasi bo'lgan 50\% sof modda saqlagan $\mathrm{AgNO}_{3}$ sarflangan bo'lsa, necha mol Ag hamda necha litr (\href{http://n.sh}{n.sh}) $\mathrm{O}_{2}$ hosil bo'lgan?\\
A) $2 ; 67,2$\\
B) $1 ; 22,4$\\
C) $2 ; 22,4$\\
D) $1 ; 11,2$
  \item $4 \mathrm{FeSO}_{4} \rightarrow 2 \mathrm{Fe}_{2} \mathrm{O}_{3}+4 \mathrm{SO}_{2}+\mathrm{O}_{2}$ ushbu reaksiya bo'yicha 152 g qo'shimchasi bo'lgan $75 \%$ sof modda saqlagan $\mathrm{FeSO}_{4}$ sarflangan bo'lsa, necha mol $\mathrm{Fe}_{2} \mathrm{O}_{3}$ hamda necha litr (\href{http://n.sh}{n.sh}) $\mathrm{O}_{2}$ hosil bo'lgan?\\
A) $2 ; 67,2$\\
B) $4 ; 22.4$\\
C) 1,$5 ; 16,8$\\
D) $1: 33,6$
  \item Quyidagi moddalarning qaysi biri suvda eriydigan asosli oksidga misol bo'la oladi?\\
A) $\mathrm{Na}_{2} \mathrm{O}$\\
B) MgO\\
C) $\mathrm{Cu}_{2} \mathrm{O}$\\
D) ZnO
  \item Quyidagi moddalarning qaysi biri kislotali oksidga misol bo'la oladi?\\
A) $\mathrm{SO}_{2}$\\
B) MgO\\
C) $\mathrm{Cu}_{2} \mathrm{O}$\\
D) NO
  \item Quyidagi moddalarning qaysi biri befarq oksidga misol bo'la oladi?\\
A) $\mathrm{SO}_{2}$\\
B) MgO\\
C) $\mathrm{Cu}_{2} \mathrm{O}$\\
D) NO
  \item Quyidagi moddalarning qaysi biri peroksidga misol bo'la oladi?\\
A) $\mathrm{Na}_{2} \mathrm{O}_{2}$\\
B) $\mathrm{CrO}_{3}$\\
C) $\mathrm{Cu}_{2} \mathrm{O}$\\
D) ZnO
  \item Quyidagi moddalarning qaysi biri ishqorlarga misol bo'la oladi?\\
A) $\mathrm{Zn}(\mathrm{OH})_{2}$\\
B) $\mathrm{Mg}(\mathrm{OH})_{2}$\\
C) $\mathrm{Cu}(\mathrm{OH})_{2}$\\
D) NaOH
  \item Quyidagi moddalarning qaysi biri amfoter asosga misol bo'la oladi?\\
A) $\mathrm{Zn}(\mathrm{OH})_{2}$\\
B) $\mathrm{Mg}(\mathrm{OH})_{2}$\\
C) $\mathrm{Cu}(\mathrm{OH})_{2}$\\
D) NaOH
  \item Quyidagi moddalarning qaysi biri asosli tuzga misol bo'la oladi?\\
A) $\mathrm{Na}_{2} \mathrm{SO}_{4}$\\
B) $\mathrm{MgSO}_{4}$\\
C) CuOHCl\\
D) $\mathrm{Zn}\left(\mathrm{HSO}_{4}\right)_{2}$
  \item Quyidagi moddalarning qaysi biri nordon tuzga misol bo'la oladi?\\
A) $\mathrm{Na}_{2} \mathrm{SO}_{4}$\\
B) $\mathrm{MgSO}_{4}$\\
C) CuOHCl\\
D) $\mathrm{Zn}\left(\mathrm{HSO}_{4}\right)_{2}$
  \item Quyidagi moddalarning qaysi biri o'rta tuzga misol bo'la oladi?\\
A) $\mathrm{Na}_{2} \mathrm{SO}_{4}$\\
B) $\mathrm{MgCO}_{3} \cdot \mathrm{CaCO}_{3}$\\
C) CuOHCl\\
D) $\mathrm{Zn}\left(\mathrm{HSO}_{4}\right)_{2}$
  \item Quyidagi moddalarning qaysi biri qo'sh tuzga misol bo'la oladi?\\
A) $\mathrm{Na}_{2} \mathrm{SO}_{4}$\\
B) $\mathrm{KCl} \cdot \mathrm{NaCl}$\\
C) CuOHCl\\
D) $\mathrm{Zn}\left(\mathrm{HSO}_{4}\right)_{2}$
  \item Quyidagi qaysi reaksiyada asosli oksid hosil bo'ladi?\\
A) $\mathrm{S}+\mathrm{O}_{2} \rightarrow \mathrm{SO}_{2}$\\
B) $\mathrm{Zn}+\mathrm{O}_{2} \rightarrow \mathrm{ZnO}$\\
C) $\mathrm{Cu}+\mathrm{O}_{2} \rightarrow \mathrm{CuO}$\\
D) $\mathrm{P}+\mathrm{O}_{2} \rightarrow \mathrm{P}_{2} \mathrm{O}_{5}$
  \item Quyidagi qaysi reaksiyada amfoter oksid hosil bo'ladi?\\
A) $\mathrm{S}+\mathrm{O}_{2} \rightarrow \mathrm{SO}_{2}$\\
B) $\mathrm{Zn}+\mathrm{O}_{2} \rightarrow \mathrm{ZnO}$\\
C) $\mathrm{Cu}+\mathrm{O}_{2} \rightarrow \mathrm{CuO}$\\
D) $\mathrm{P}+\mathrm{O}_{2} \rightarrow \mathrm{P}_{2} \mathrm{O}_{5}$
  \item Quyidagi qaysi reaksiyada kislotali oksid hosil bo'ladi?\\
A) $\mathrm{S}+\mathrm{O}_{2} \rightarrow \mathrm{SO}_{2}$\\
B) $\mathrm{Zn}+\mathrm{O}_{2} \rightarrow \mathrm{ZnO}$\\
C) $\mathrm{Cu}+\mathrm{O}_{2} \rightarrow \mathrm{CuO}$\\
D) $\mathrm{Na}+\mathrm{O}_{2} \rightarrow \mathrm{Na}_{2} \mathrm{O}$
  \item Quyidagi qaysi reaksiyada ishqor hosil bo'ladi?\\
A) $\mathrm{Na}+\mathrm{S} \rightarrow \mathrm{Na}_{2} \mathrm{~S}$\\
B) $\mathrm{K}+\mathrm{H}_{2} \mathrm{O} \rightarrow \mathrm{KOH}+\mathrm{H}_{2}$\\
C) $\mathrm{CuCl}_{2}+\mathrm{NaOH} \rightarrow \mathrm{Cu}(\mathrm{OH})_{2}+\mathrm{NaCl}$\\
D) $\mathrm{P}+\mathrm{O}_{2} \rightarrow \mathrm{P}_{2} \mathrm{O}_{5}$
  \item Quyidagi qaysi reaksiyada asosli tuz hosil bo'ladi?\\
A) $\mathrm{NaOH}+\mathrm{SO}_{2} \rightarrow \mathrm{NaHSO}_{3}$\\
B) $\mathrm{Zn}+\mathrm{O}_{2} \rightarrow \mathrm{ZnO}$\\
C) $\mathrm{Cu}(\mathrm{OH})_{2}+\mathrm{HCl} \rightarrow \mathrm{CuOHCl} \quad+\mathrm{H}_{2} \mathrm{O}$\\
D) $\mathrm{P}+\mathrm{O}_{2} \rightarrow \mathrm{P}_{2} \mathrm{O}_{6}$
  \item Quyidagi qaysi reaksiyada nordon tuz hosil bo'ladi?\\
A) $\mathrm{NaOH}+\mathrm{SO}_{2} \rightarrow \mathrm{NaHSO}_{3}$\\
B) $\mathrm{Zn}+\mathrm{O}_{2} \rightarrow \mathrm{ZnO}$\\
C) $\mathrm{Cu}(\mathrm{OH})_{2}+\mathrm{HCl} \rightarrow \mathrm{CuOHCl}+\mathrm{H}_{2} \mathrm{O}$\\
D) $\mathrm{P}+\mathrm{O}_{2} \rightarrow \mathrm{P}_{2} \mathrm{O}_{5}$
  \item Quyidagi qaysi reaksiyada kislota hosil bo'ladi?\\
A) $\mathrm{SO}_{2} \quad+\mathrm{H}_{2} \mathrm{O} \rightarrow \mathrm{H}_{2} \mathrm{SO}_{3}$\\
B) $\mathrm{S}+\mathrm{O}_{2} \rightarrow \mathrm{SO}_{2}$\\
C) $\mathrm{Cu}+\mathrm{O}_{2} \rightarrow \mathrm{CuO}$\\
D) $\mathrm{P}+\mathrm{O}_{2} \rightarrow \mathrm{P}_{2} \mathrm{O}_{5}$
  \item Quyidagi qaysi reaksiyada amfoter gidroksid hosil bo'ladi?\\
A) $\mathrm{Cu}(\mathrm{OH})_{2}+\mathrm{HCl} \rightarrow \mathrm{CuOHCl}+\mathrm{H}_{2} \mathrm{O}$\\
B) $\mathrm{ZnCl}_{2}+\mathrm{NaOH} \rightarrow \mathrm{Zn}(\mathrm{OH})_{2}+\mathrm{NaCl}$\\
C) $\mathrm{CuCl}_{2}+\mathrm{NaOH} \rightarrow \mathrm{Cu}(\mathrm{OH})_{2}+\mathrm{NaCl}$\\
D) $\mathrm{P}+\mathrm{O}_{2} \rightarrow \mathrm{P}_{2} \mathrm{O}_{5}$
  \item Quyidagi qaysi reaksiyada asos hosil bo'ladi?\\
A) $\mathrm{Cu}(\mathrm{OH})_{2}+\mathrm{HCl} \rightarrow \mathrm{CuOHCl}+\mathrm{H}_{2} \mathrm{O}$\\
B) $\mathrm{ZnCl}_{2}+\mathrm{NaOH} \rightarrow \mathrm{Zn}(\mathrm{OH})_{2}+\mathrm{NaCl}$\\
C) $\mathrm{CuCl}_{2}+\mathrm{NaOH} \rightarrow \mathrm{Cu}(\mathrm{OH})_{2}+\mathrm{NaCl}$\\
D) $\mathrm{P}+\mathrm{O}_{2} \rightarrow \mathrm{P}_{2} \mathrm{O}_{5}$
  \item Quyidagi qaysi reaksiyada o'rta tuz hosil bo'ladi?\\
A) $\mathrm{NaOH}+\mathrm{SO}_{2} \rightarrow \mathrm{Na}_{2} \mathrm{SO}_{3}+\mathrm{H}_{2} \mathrm{O}$\\
B) $\mathrm{Zn}+\mathrm{O}_{2} \rightarrow \mathrm{ZnO}$\\
C) $\mathrm{Cu}(\mathrm{OH})_{2}+\mathrm{HCl} \rightarrow \mathrm{CuOHCl}+\mathrm{H}_{2} \mathrm{O}$\\
D) $\mathrm{P}+\mathrm{O}_{2} \rightarrow \mathrm{P}_{2} \mathrm{O}_{5}$
  \item Quyidagi moddalardan qaysi biri $\mathrm{H}_{2} \mathrm{O}$ bilan reaksiyaga kirishadi?\\
A) $\mathrm{Na}_{2} \mathrm{O}$\\
B) CdO\\
C) $\mathrm{Cu}_{2} \mathrm{O}$\\
D) ZnO
  \item Quyidagi moddalardan qaysi biri $\mathrm{H}_{2} \mathrm{O}$ bilan reaksiyaga kirishadi?\\
A) $\mathrm{Ag}_{2} \mathrm{O}$\\
B) $\mathrm{CaH}_{2}$\\
C) $\mathrm{Cu}_{2} \mathrm{O}$\\
D) $\mathrm{Zn}(\mathrm{OH})_{2}$
  \item Quyidagi moddalardan qaysi biri $\mathrm{H}_{2} \mathrm{O}$ bilan reaksiyaga kirishadi?\\
A) $\mathrm{Al}_{2} \mathrm{O}_{3}$\\
B) ZnO\\
C) CoO\\
D) NaH
  \item Quyidagi moddalardan qaysi biri $\mathrm{SO}_{3}$ bilan reaksiyaga kirishadi?\\
A) $\mathrm{CrO}_{3}$\\
B) $\mathrm{Mn}_{2} \mathrm{O}_{7}$\\
C) CuO\\
D) $\mathrm{CO}_{2}$
  \item Quyidagi moddalardan qaysi biri $\mathrm{SO}_{3}$ bilan reaksiyaga kirishadi?\\
A) $\mathrm{Cl}_{2} \mathrm{O}_{7}$\\
B) $\mathrm{MgSO}_{4}$\\
C) $\mathrm{Na}_{2} \mathrm{SO}_{4}$\\
D) ZnO
  \item Quyidagi moddalardan qaysi biri $\mathrm{SO}_{3}$ bilan reaksiyaga kirishadi?\\
A) $\mathrm{SO}_{2}$\\
B) $\mathrm{Al}_{2}\left(\mathrm{SO}_{4}\right)_{3}$\\
C) $\mathrm{N}_{2} \mathrm{O}_{5}$\\
D) $\mathrm{Li}_{2} \mathrm{O}$
  \item Quyidagi moddalardan qaysi biri $\mathrm{Na}_{2} \mathrm{O}$ bilan reaksiyaga kirishadi?\\
A) $\mathrm{Na}_{2} \mathrm{O}$\\
B) MgO\\
C) $\mathrm{Cu}_{2} \mathrm{O}$\\
D) ZnO
  \item Quyidagi moddalardan qaysi biri $\mathrm{K}_{2} \mathrm{O}$ bilan reaksiyaga kirishadi?\\
A) $\mathrm{SO}_{2}$\\
B) $\mathrm{Al}_{2}\left(\mathrm{SO}_{4}\right)_{3}$\\
C) MgO\\
D) $\mathrm{Li}_{2} \mathrm{O}$
  \item Quyidagi moddalardan qaysi biri $\mathrm{Li}_{2} \mathrm{O}$ bilan reaksiyaga kirishadi?\\
A) CrO\\
B) $\mathrm{Mn}_{2} \mathrm{O}_{7}$\\
C) CuO\\
D) FeO
  \item Quyidagi moddalardan qaysi biri $\mathrm{Rb}_{2} \mathrm{O}$ bilan reaksiyaga kirishadi?\\
A) $\mathrm{CrO}_{3}$\\
B) MnO\\
C) CuO\\
D) FeO
  \item Quyidagi qaysi reaksiya boradi?\\
A) $\mathrm{Cu}(\mathrm{OH})_{2}+\mathrm{NaOH} \rightarrow$\\
B) $\mathrm{ZnCl}_{2}+\mathrm{CuCl}_{2} \rightarrow$\\
C) $\mathrm{CuCl}_{2}+\mathrm{NaOH} \rightarrow$\\
D) $\mathrm{Cu}(\mathrm{OH})_{2}+\mathrm{Fe}(\mathrm{OH})_{2} \rightarrow$
  \item Quyidagi qaysi reaksiya boradi?\\
A) $\mathrm{KOH}+\mathrm{NaOH} \rightarrow$\\
B) $\mathrm{ZnCl}_{2}+\mathrm{AlCl}_{3} \rightarrow$\\
C) $\mathrm{NaCl}+\mathrm{NaOH} \rightarrow$\\
D) $\mathrm{AlCl}_{3}+\mathrm{NaOH} \rightarrow$
  \item Quyidagi qaysi reaksiya boradi?\\
A) $\mathrm{Cu}(\mathrm{OH})_{2}+\mathrm{CuSO} 4 \rightarrow$\\
B) $\mathrm{ZnCl}_{2}+\mathrm{CuCl}_{2} \rightarrow$\\
C) $\mathrm{HCl}+\mathrm{NaOH} \rightarrow$\\
D) $\mathrm{Cu}(\mathrm{OH})_{2}+\mathrm{Fe}(\mathrm{OH})_{2} \rightarrow$
  \item Quyidagi qaysi reaksiya boradi?\\
A) $\mathrm{Ca}(\mathrm{OH})_{2}+\mathrm{RbOH} \rightarrow$\\
B) $\mathrm{ZnCl}_{2}+\mathrm{NaOH} \rightarrow$\\
C) $\mathrm{ZnCl}_{2}+\mathrm{CuCl}_{2} \rightarrow$\\
D) $\mathrm{Zn}(\mathrm{OH})_{2}+\mathrm{Fe}(\mathrm{OH})_{3} \rightarrow$
  \item Quyidagi qaysi reaksiya boradi?\\
A) $\mathrm{Be}(\mathrm{OH})_{2}+\mathrm{NaOH} \rightarrow$\\
B) $\mathrm{CaCl}_{2}+\mathrm{CuCl}_{2} \rightarrow$\\
C) $\mathrm{BaCl}_{2}+\mathrm{NaOH} \rightarrow$\\
D) $\mathrm{Cu}(\mathrm{OH})_{2}+\mathrm{Fe}(\mathrm{OH})_{2} \rightarrow$
  \item Quyidagi qaysi reaksiya boradi?\\
A) $\mathrm{Fe}(\mathrm{OH})_{2}+\mathrm{NaOH} \rightarrow$\\
B) $\mathrm{ZnCl}_{2}+\mathrm{CuCl}_{2} \rightarrow$\\
C) $\mathrm{CaCl}_{2}+\mathrm{NaOH} \rightarrow$\\
D) $\mathrm{Be}(\mathrm{OH})_{2}+\mathrm{Ca}(\mathrm{OH})_{2} \rightarrow$
  \item Quyidagi qaysi reaksiya boradi?\\
A) $\mathrm{Fe}(\mathrm{OH})_{2}+\mathrm{CuOH} \rightarrow$\\
B) $\mathrm{ZnCl}_{2}+\mathrm{NiCl}_{2} \rightarrow$\\
C) $\mathrm{FeCl}_{3}+\mathrm{NaOH} \rightarrow$\\
D) $\mathrm{Cu}(\mathrm{OH})_{2}+\mathrm{Fe}(\mathrm{OH})_{2} \rightarrow$
  \item Quyidagi qaysi reaksiya boradi?\\
A) $\mathrm{Ba}(\mathrm{OH})_{2}+\mathrm{Sn}(\mathrm{OH})_{2} \rightarrow$\\
B) $\mathrm{CoCl}_{2}+\mathrm{CuCl}_{2} \rightarrow$\\
C) $\mathrm{CuCl}_{2}+\mathrm{NaCl} \rightarrow$\\
D) $\mathrm{Cu}(\mathrm{OH})_{2}+\mathrm{Fe}(\mathrm{OH})_{2} \rightarrow$
  \item Quyidagi qaysi reaksiya boradi?\\
A) $\mathrm{Mg}(\mathrm{OH})_{2}+\mathrm{NaOH} \rightarrow$\\
B) $\mathrm{ZnCl}_{2}+\mathrm{CuCl}_{2} \rightarrow$\\
C) $\mathrm{MgCl}_{2}+\mathrm{Fe}(\mathrm{OH})_{2} \rightarrow$\\
D) $\mathrm{Cu}(\mathrm{OH})_{2}+\mathrm{HCl} \rightarrow$
  \item Quyidagi qaysi reaksiya boradi?\\
A) $\mathrm{Cu}(\mathrm{OH})_{2}+\mathrm{NaOH} \rightarrow$\\
B) $\mathrm{ZnCl}_{2}+\mathrm{CuCl}_{2} \rightarrow$\\
C) $\mathrm{NaCl}+\mathrm{NaOH} \rightarrow$\\
D) $\mathrm{Ca}(\mathrm{OH})_{2}+\mathrm{Fe}(\mathrm{OH})_{3} \rightarrow$
  \item 1 mol HCl va $2 \mathrm{~mol} \mathrm{CO}_{2}$ iborat gazlar aralashmasining o'rtacha molyar massasini $\mathrm{g} / \mathrm{mol}$ da hisoblang.\\
A) 41,5\\
B) 44,4\\
C) 74,2\\
D) 12,5
  \item 1 mol CO va $2 \mathrm{~mol} \mathrm{CO}_{2}$ iborat gazlar aralashmasining o'rtacha molyar massasini $\mathrm{g} / \mathrm{mol}$ da hisoblang.\\
A) 61,33\\
B) 33,33\\
C) 74\\
D) 38,67
  \item $2 \mathrm{~mol} \mathrm{SO}_{2}$ va $3 \mathrm{~mol} \mathrm{CO}_{2}$ iborat gazlar aralashmasining o'rtacha molyar massasini $\mathrm{g} / \mathrm{mol}$ da hisoblang.\\
A) 52\\
B) 44\\
C) 64\\
D) 48
  \item $4 \mathrm{~mol} \mathrm{H}_{2}$ va 2 mol CO iborat gazlar aralashmasining o'rtacha molyar massasini $\mathrm{g} / \mathrm{mol}$ da hisoblang.\\
A) 12,25\\
B) 14,4\\
C) 10,67\\
D) 12,5
  \item $1 \mathrm{~mol} \mathrm{Cl}_{2}$ va $2 \mathrm{~mol} \mathrm{CO}_{2}$ iborat gazlar aralashmasining o'rtacha molyar massasini $\mathrm{g} / \mathrm{mol}$ da hisoblang.\\
A) 72,25\\
B) 53\\
C) 74,7\\
D) 42,5
  \item 2 mol He va 2 mol Ar iborat gazlar aralashmasining o'rtacha molyar massasini $\mathrm{g} / \mathrm{mol}$ da hisoblang.\\
A) 22\\
B) 44\\
C) 74\\
D) 12
  \item 3 mol Ar va $2 \mathrm{~mol} \mathrm{~F}_{2}$ iborat gazlar aralashmasining o'rtacha molyar massasini $\mathrm{g} / \mathrm{mol}$ da hisoblang.\\
A) 62,25\\
B) 34,4\\
C) 39,2\\
D) 32,5
  \item 1 mol Ne va $2 \mathrm{~mol} \mathrm{CO}_{2}$ iborat gazlar aralashmasining o'rtacha molyar massasini $\mathrm{g} / \mathrm{mol}$ da hisoblang.\\
A) 25\\
B) 36\\
C) 44,2\\
D) 12,5
  \item 2 mol Ne va 3 mol CO iborat gazlar aralashmasining o'rtacha molyar massasini $\mathrm{g} / \mathrm{mol}$ da hisoblang.\\
A) 25,2\\
B) 14,4\\
C) 24,8\\
D) 22,5
  \item 1 mol NO va $2 \mathrm{~mol} \mathrm{SO}_{2}$ iborat gazlar aralashmasining o'rtacha molyar massasini $\mathrm{g} / \mathrm{mol}$ da hisoblang.\\
A) 42,25\\
B) 52,67\\
C) 74,7\\
D) 72,5
  \item 10 litr HCl va 5 litr $\mathrm{CO}_{2}$ iborat gazlar aralashmasining o'rtacha molyar massasini g/mol da hisoblang. (hajmlar bir xil sharoitda o'lchangan)\\
A) 62\\
B) 44\\
C) 39\\
D) 25
  \item 3 litr $\overline{\mathrm{CO}}$ va 2 litr $\overline{\mathrm{CO}}_{2}$ iborat gazlar aralashmasining o'rtacha molyar massasini g/mol da hisoblang.(hajmlar bir xil sharoitda o'lchangan)\\
A) 62,5\\
B) 34,4\\
C) 74\\
D) 58
  \item 8 litr $\mathrm{SO}_{2}$ va 3 litr $\mathrm{CO}_{2}$ iborat gazlar aralashmasining o'rtacha molyar massasini $\mathrm{g} / \mathrm{mol}$ da hisoblang. (hajmlar bir xil sharoitda o'lchangan)\\
A) 52,2\\
B) 34,4\\
C) 64\\
D) 58,5
  \item 4 litr $\mathrm{H}_{2}$ va 2 litr CO iborat gazlar aralashmasining o'rtacha molyar massasini g/mol da hisoblang. (hajmlar bir xil sharoitda o'lchangan)\\
A) 12,25\\
B) 14,4\\
C) 10,67\\
D) 12,5
  \item 10 litr $\mathrm{Cl}_{2}$ va 2 litr $\mathrm{CO}_{2}$ iborat gazlar aralashmasining o'rtacha molyar massasini $\mathrm{g} / \mathrm{mol}$ da hisoblang. (hajmlar bir xil sharoitda o'lchangan)\\
A) 62,25\\
B) 69,5\\
C) 66,5\\
D) 62,5
  \item 12 litr Ne va 2 litr Ar iborat gazlar aralashmasining o'rtacha molyar\\
massasini g/mol da hisoblang. (hajmlar bir xil sharoitda o'lchangan)\\
A) 22,86\\
B) 44,4\\
C) 24,2\\
D) 21,2
  \item 30 litr Ar va 20 litr $\mathrm{F}_{2}$ iborat gazlar aralashmasining o'rtacha molyar massasini g/mol da hisoblang.(hajmlar bir xil sharoitda o'lchangan)\\
A) 72,27\\
B) 34,4\\
C) 21,6\\
D) 39,2
  \item 21 litr Ne va 12 litr $\mathrm{CO}_{2}$ iborat gazlar aralashmasining o'rtacha molyar massasini $\mathrm{g} / \mathrm{mol}$ da hisoblang. (hajmlar bir xil sharoitda o'lchangan)\\
A) 25,2\\
B) 36,6\\
C) 28,73\\
D) 12,5
  \item 5 litr Ne va 3 litr CO iborat gazlar aralashmasining o'rtacha molyar massasini g/mol da hisoblang. (hajmlar bir xil sharoitda o'lchangan)\\
A) 25\\
B) 23\\
C) 24,8\\
D) 22,5
  \item 7 litr NO va 2 litr $\mathrm{SO}_{2}$ iborat gazlar aralashmasining o'rtacha molyar massasini g/mol da hisoblang. (hajmlar bir xil sharoitda o'lchangan)\\
A) 42,25\\
B) 52,67\\
C) 37,5\\
D) 42,5
  \item 73 g HCl va $88 \mathrm{~g} \mathrm{CO}_{2}$ iborat gazlar aralashmasining o'rtacha molyar massasini $\mathrm{g} / \mathrm{mol}$ da hisoblang.\\
A) 42,5\\
B) 40,25\\
C) 49\\
D) 45
  \item 56 g CO va $44 \mathrm{~g} \mathrm{CO}_{2}$ iborat gazlar aralashmasining o'rtacha molyar massasini $\mathrm{g} / \mathrm{mol}$ da hisoblang.\\
A) 32,5\\
B) 34,4\\
C) 33,3\\
D) 38
  \item $128 \mathrm{~g} \mathrm{SO}_{2}$ va $132 \mathrm{~g} \mathrm{CO}_{2}$ iborat gazlar aralashmasining o'rtacha molyar massasini $\mathrm{g} / \mathrm{mol}$ da hisoblang.\\
A) 52\\
B) 34\\
C) 64\\
D) 58
  \item $4 \mathrm{~g} \mathrm{H}_{2}$ va 28 g CO iborat gazlar aralashmasining o'rtacha molyar massasini g/mol da hisoblang.\\
A) 12,25\\
B) 14,4\\
C) 10,67\\
D) 12,5
  \item $142 \mathrm{~g} \mathrm{Cl}_{2}$ va $66 \mathrm{~g} \mathrm{CO}_{2}$ iborat gazlar aralashmasining o'rtacha molyar massasini g/mol da hisoblang.\\
A) 42,25\\
B) 59,42\\
C) 66,53\\
D) 12,5
  \item 12 g He va 40 g Ar iborat gazlar aralashmasining o'rtacha molyar massasini $\mathrm{g} / \mathrm{mol}$ da hisoblang.\\
A) 13\\
B) 24,4\\
C) 14\\
D) 21,2
  \item 80 g Ar va $38 \mathrm{~g} \mathrm{~F}_{2}$ iborat gazlar aralashmasining o'rtacha molyar massasini $\mathrm{g} / \mathrm{mol}$ da hisoblang.\\
A) 32,27\\
B) 34,5\\
C) 31,6\\
D) 39,3
  \item 20 g Ne va $88 \mathrm{~g} \mathrm{CO}_{2}$ iborat gazlar aralashmasining o'rtacha molyar massasini $\mathrm{g} / \mathrm{mol}$ da hisoblang.\\
A) 45,2\\
B) 36\\
C) 28\\
D) 22,5
  \item 60 g Ne va 84 g CO iborat gazlar aralashmasining o'rtacha molyar massasini $\mathrm{g} / \mathrm{mol}$ da hisoblang.\\
A) 25\\
B) 23\\
C) 24\\
D) 22
  \item 60 g NO va $128 \mathrm{~g} \mathrm{SO}_{2}$ iborat gazlar aralashmasining o'rtacha molyar massasini $\mathrm{g} / \mathrm{mol}$ da hisoblang.\\
A) 32,25\\
B) 32,67\\
C) 47\\
D) 32,5
  \item 4 dona HCl va 5 dona $\mathrm{CO}_{2}$ molekulalaridan iborat gazlar aralashmasining o'rtacha molyar massasini $\mathrm{g} / \mathrm{mol}$ da hisoblang.\\
A) 62\\
B) 40,67\\
C) 39,2\\
D) 25\\
  \item 6 dona CO va 4 dona $\mathrm{CO}_{2}$ molekulalaridan iborat gazlar aralashmasining o'rtacha molyar massasini g/mol da hisoblang.\\
A) 62,5\\
B) 34,4\\
C) 74\\
D) 58
  \item 8 dona $\mathrm{SO}_{2}$ va 2 dona $\mathrm{CO}_{2}$ molekulalaridan iborat gazlar aralashmasining o'rtacha molyar massasini g/mol da hisoblang.\\
A) 52\\
B) 34\\
C) 60\\
D) 58
  \item 3 dona $\mathrm{H}_{2}$ va 7 dona CO molekulalaridan iborat gazlar aralashmasining o'rtacha molyar massasini $\mathrm{g} / \mathrm{mol}$ da hisoblang.\\
A) 12,25\\
B) 24,4\\
C) 20,2\\
D) 12,5
  \item 8 dona $\mathrm{Cl}_{2}$ va 2 dona $\mathrm{CO}_{2}$ molekulalaridan iborat gazlar aralashmasining o'rtacha molyar massasini $\mathrm{g} / \mathrm{mol}$ da hisoblang.\\
A) 62,6\\
B) 69,5\\
C) 66,5\\
D) 65,6
  \item 2 dona He va 8 dona Ar molekulalaridan iborat gazlar aralashmasining o'rtacha molyar massasini $\mathrm{g} / \mathrm{mol}$ da hisoblang.\\
A) 12,86\\
B) 44,7\\
C) 32,8\\
D) 21,2
  \item 1 dona Ar va 9 dona $\mathrm{F}_{2}$ molekulalaridan iborat gazlar aralashmasining o'rtacha molyar massasini g/mol da hisoblang.\\
A) 72,27\\
B) 34,4\\
C) 21,6\\
D) 38,2
  \item 5 dona Ne va 5 dona $\mathrm{CO}_{2}$ molekulalaridan iborat gazlar aralashmasining o'rtacha molyar massasini g/mol da hisoblang.\\
A) 25,2\\
B) 32\\
C) 28\\
D) 12,5
  \item 4 dona Ne va 6 dona CO molekulalaridan iborat gazlar aralashmasining o'rtacha molyar massasini g/mol da hisoblang.\\
A) 25\\
B) 23\\
C) 24,8\\
D) 22,5
  \item 7 dona NO va 3 dona $\mathrm{SO}_{2}$ molekulalaridan iborat gazlar aralashmasining o'rtacha molyar massasini $\mathrm{g} / \mathrm{mol}$ da hisoblang.\\
A) 22,25\\
B) 52,67\\
C) 37,5\\
D) 40,2
  \item $12,04 \cdot 10^{23}$ dona HCl va $48,16 \cdot 10^{23}$ dona $\mathrm{CO}_{2}$ molekulalaridan iborat gazlar aralashmasining o'rtacha molyar massasini $\mathrm{g} / \mathrm{mol}$ da hisoblang.\\
A) 32\\
B) 30,67\\
C) 42,5\\
D) 25
  \item 18,06 $\cdot 10^{23}$ dona CO va $42,14 \cdot 10^{23}$ dona $\mathrm{CO}_{2}$ molekulalaridan iborat gazlar aralashmasining o'rtacha molyar massasini $\mathrm{g} / \mathrm{mol}$ da hisoblang.\\
A) 22,5\\
B) 24,4\\
C) 39,2\\
D) 28
  \item $6,02 \cdot 10^{23}$ dona $\mathrm{SO}_{2}$ va $54,18 \cdot 10^{23}$ dona $\mathrm{CO}_{2}$ molekulalaridan iborat gazlar aralashmasining o'rtacha molyar massasini g/mol da hisoblang.\\
A) 46\\
B) 44\\
C) 60\\
D) 48
  \item $24,08 \cdot 10^{23}$ dona $\mathrm{H}_{2}$ va $36,12 \cdot 10^{23}$ dona CO molekulalaridan iborat gazlar aralashmasining o'rtacha molyar massasini g/mol da hisoblang.\\
A) 17,6\\
B) 44,4\\
C) 40,2\\
D) 22,5
  \item $48,16 \cdot 10^{23}$ dona $\mathrm{Cl}_{2}$ va $12,04 \cdot 10^{23}$ dona $\mathrm{CO}_{2}$ molekulalaridan iborat gazlar aralashmasining o'rtacha molyar massasini $\mathrm{g} / \mathrm{mol}$ da hisoblang.\\
A) 62,6\\
B) 69,5\\
C) 65,6\\
D) 62,4
  \item $12,04 \cdot 10^{23}$ dona He va $48,16 \cdot 10^{23}$ dona Ar molekulalaridan iborat gazlar aralashmasining o'rtacha molyar massasini g/mol da hisoblang.\\
A) 12,86\\
B) 44,7\\
C) 32,8\\
D) 21,2
  \item $30,1 \cdot 10^{23}$ dona Ar va $30,1 \cdot 10^{23}$ dona $\mathrm{F}_{2}$ molekulalaridan iborat gazlar aralashmasining o'rtacha molyar massasini g/mol da hisoblang.\\
A) 39\\
B) 34\\
C) 22\\
D) 38
  \item $18,06 \cdot 10^{23}$ dona Ne va $42,14 \cdot 10^{23}$ dona $\mathrm{CO}_{2}$ molekulalaridan iborat gazlar aralashmasining o'rtacha molyar massasini $\mathrm{g} / \mathrm{mol}$ da hisoblang.\\
A) 35,2\\
B) 32,3\\
C) 36,8\\
D) 12,5
  \item $24,08 \cdot 10^{23}$ dona Ne va $36,12 \cdot 10^{23}$ dona CO molekulalaridan iborat gazlar aralashmasining o'rtacha molyar massasini $\mathrm{g} / \mathrm{mol}$ da hisoblang.\\
A) 25\\
B) 23\\
C) 24,8\\
D) 22,5
  \item $42,14 \cdot 10^{23}$ dona NO va $18,06 \cdot 10^{23}$ dona $\mathrm{SO}_{2}$ molekulalaridan iborat gazlar\\
aralashmasining o'rtacha molyar massasini $\mathrm{g} / \mathrm{mol}$ da hisoblang.\\
A) 22,25\\
B) 52,67\\
C) 37,5\\
D) 40,2
  \item $6 \mathrm{~mol} \mathrm{HCl}, 44 \mathrm{~g} \mathrm{CO}_{2}$ va $18,06 \cdot 10^{23}$ dona NO dan iborat gazlar aralashmasining o'rtacha molyar massasini $\mathrm{g} / \mathrm{mol}$ da hisoblang.\\
A) 35,3\\
B) 34.4\\
C) 14,2\\
D) 32,5\\
  \item $4 \mathrm{~mol} \mathrm{He}, 56 \mathrm{~g} \mathrm{CO}$ va $24,08 \cdot 10^{2 s}$ dona $\mathrm{NO}_{2}$ dan iborat gazlar aralashmasining o'rtacha molyar massasini g/mol da hisoblang.\\
A) 42,25\\
B) 24,4\\
C) 25.6\\
D) 62,5
  \item $2 \mathrm{~mol} \mathrm{Ar}, 320 \mathrm{~g} \mathrm{SO}$ va $18,06 \cdot 10^{23}$ dona CO dan iborat gazlar aralashmasining o'rtacha molyar massasini $\mathrm{g} / \mathrm{mol}$ da hisoblang.\\
A) 48,4\\
B) 74.4\\
C) 44,2\\
D) 32,5
54. $5 \mathrm{~mol} \mathrm{Ne}, 84 \mathrm{~g} \mathrm{CO}$ va $12,04 \cdot 10^{29}$ dona NO dan iborat gazlar aralashmasining o'rtacha molyar massasini g/mol da hisoblang.\\
A) 32,7\\
B) 44,7\\
C) 24,4\\
D) 32,4\\
55. $6 \mathrm{~mol} \mathrm{H}_{2}, 44 \mathrm{~g} \mathrm{CO}_{2}$ va $18,06 \cdot 10^{23}$ dona $\mathrm{N}_{2}$ dan iborat gazlar aralashmasining ortacha molyar massasini g/mol da hisoblang.\\
A) 12,25\\
B) 14\\
C) 24\\
D) 12\\
56. $4 \mathrm{~mol} \mathrm{HCl}, 44 \mathrm{~g} \mathrm{CO}_{2}$ va $30,1 \cdot 10^{29}$ dona NO dan iborat gazlar aralashmasining ortacha molyar massasini $\mathrm{g} / \mathrm{mol}$ da hisoblang.\\
A) 32\\
B) 44\\
C) 34\\
D) 12\\
57. $3 \mathrm{~mol} \mathrm{Cl}_{3}, 34 \mathrm{~g} \mathrm{H}_{2} \mathrm{~S}$ va $18,06 \cdot 10^{29}$ dona HBr dan iborat gazlar aralashmasining\\
o'rtacha molyar massasini $\mathrm{g} / \mathrm{mol}$ da hisoblang.\\
A) 62\\
B) 49\\
C) 74\\
D) 32\\
58. $2 \mathrm{~mol} \mathrm{HF}, 68 \mathrm{~g} \mathrm{H}_{2} \mathrm{~S}$ va $36,12 \cdot 10^{23}$ dona NO dan iborat gazlar aralashmasining o'rtacha molyar massasini $\mathrm{g} / \mathrm{mol}$ da hisoblang.\\
A) 28,8\\
B) 44,4\\
C) 24,2\\
D) 72,5\\
59. $5 \mathrm{~mol} \mathrm{HCl}, 132 \mathrm{~g} \mathrm{CO}_{2}$ va $12,04 \cdot 10^{23}$ dona Ne dan iborat gazlar aralashmasining o'rtacha molyar massasini $\mathrm{g} / \mathrm{mol}$ da hisoblang.\\
A) 42,25\\
B) 35,45\\
C) 74,25\\
D) 32,5\\
60. $1 \mathrm{~mol} \mathrm{Ar} ,32 \mathrm{~g} \mathrm{SiH}_{4}$ va $48,16 \cdot 10^{23}$ dona NO dan iborat gazlar aralashmasining o'rtacha molyar massasini $\mathrm{g} / \mathrm{mol}$ da hisoblang.\\
A) 72,8\\
B) 44,4\\
C) 74,2\\
D) 31,2
  \item $\mathrm{CO}_{2}$ gaziningdagi zichligini $\mathrm{g} / \mathrm{l}$ da hisoblang.\\
A) $22 / 11,2$\\
B) $22 / 22,4$\\
C) $44 / 5,6$\\
D) $44 / 11,2$\\
  \item $\mathrm{NO}_{2}$ gaziningdagi zichligini $\mathrm{g} / \mathrm{l}$ da hisoblang.\\
A) $46 / 11,2$\\
B) $23 / 22,4$\\
C) $11,5 / 5,6$\\
D) $46 / 11,2$
  \item $\mathrm{SO}_{2}$ gaziningdagi zichligini $\mathrm{g} / \mathrm{l}$ da hisoblang.\\
A) $64 / 11,2$\\
B) $64 / 22,4$\\
C) $32 / 5,6$\\
D) $16 / 11,2$
  \item №O gaziningdagi zichligini $\mathrm{g} / \mathrm{l}$ da hisoblang.\\
A) $22 / 11,2$\\
B) $22 / 22,4$\\
C) $44 / 5,6$\\
D) $44 / 11,2$
  \item CO gaziningdagi zichligini $\mathrm{g} / \mathrm{l}$ da hisoblang.\\
A) $7 / 11,2$\\
B) $28 / 22,4$\\
C) $56 / 5,6$\\
D) $28 / 11,2$
  \item $\mathrm{N}_{2}$ gaziningdagi zichligini $\mathrm{g} / \mathrm{l}$ da hisoblang.\\
A) $14 / 11,2$\\
B) $14 / 22,4$\\
C) $7 / 56$\\
D) $44 / 11,2$
  \item $\mathrm{H}_{2} \mathrm{~S}$ gaziningdagi zichligini $\mathrm{g} / \mathrm{l}$ da hisoblang.\\
A) $8,5 / 11,2$\\
B) $68 / 22,4$\\
C) $34 / 5,6$\\
D) $17 / 11,2$
  \item HCl gaziningdagi zichligini $\mathrm{g} / \mathrm{l}$ da hisoblang.\\
A) $36,5 / 11,2$\\
B) $73 / 22,4$\\
C) $73 / 5,6$\\
D) $18,25 / 11,2$
  \item $\mathrm{SiH}_{4}$ gaziningdagi zichligini $\mathrm{g} / \mathrm{l}$ da hisoblang.\\
A) $32 / 11,2$\\
B) $64 / 22,4$\\
C) $16 / 5,6$\\
D) $16 / 11,2$
  \item $\mathrm{CH}_{4}$ gaziningdagi zichligini $\mathrm{g} / \mathrm{l}$ da hisoblang.\\
A) $8 / 11,2$\\
B) $32 / 22,4$\\
C) $32 / 5,6$\\
D) $16 / 11,2$
  \item $\mathrm{CO}_{2}$ gazining He ga nisbatan zichligini hisoblang.\\
A) 11\\
B) 22\\
C) 44\\
D) 33
  \item $\mathrm{NO}_{2}$ gazining $\mathrm{H}_{2}$ ga nisbatan zichligini hisoblang.\\
A) 46\\
B) 12\\
C) 23\\
D) 24
  \item $\mathrm{SO}_{2}$ gazining Ne ga nisbatan zichligini hisoblang.\\
A) 1,1\\
B) 1,2\\
C) 3,2\\
D) 1,8
  \item $\mathrm{N}_{2} \mathrm{O}$ gazining HF ga nisbatan zichligini hisoblang.\\
A) 1,1\\
B) 2,2\\
C) 4,4\\
D) 3,3
  \item CO gazining Ar ga nisbatan zichligini hisoblang.\\
A) 0,7\\
B) 0,8\\
C) 1,4\\
D) 1,2
  \item $\mathrm{N}_{2}$ gazining $\mathrm{CH}_{4}$ ga nisbatan zichligini hisoblang.\\
A) 1,12\\
B) 1,25\\
C) 0,875\\
D) 1,75
  \item $\mathrm{H}_{2} \mathrm{~S}$ gazining $\mathrm{H}_{2}$ ga nisbatan zichligini hisoblang.\\
A) 17\\
B) 18\\
C) 22\\
D) 33
  \item HCl gazining NO ga nisbatan zichligini hisoblang.\\
A) 0,7\\
B) 0,8\\
C) 1,4\\
D) 1,2
  \item $\mathrm{SiH}_{4}$ gazining $\mathrm{C}_{2} \mathrm{H}_{4}$ ga nisbatan zichligini hisoblang.\\
A) 1,14\\
B) 0,8\\
C) 1,41\\
D) 1,2
  \item $\mathrm{CH}_{4}$ gazining HF ga nisbatan zichligini hisoblang.\\
A) 0,7\\
B) 0,8\\
C) 1,4\\
D) 1,2
  \item $\mathrm{H}_{2}$ ga nisbatan zichligi 10 ga teng bo'lgan gazni aniqlang\\
A) $\mathrm{F}_{2}$\\
B) $\mathrm{N}_{2}$\\
C) Ne\\
D) He\\
  \item He ga nisbatan zichligi 10 ga teng bo'lgan gazni aniqlang\\
A) Ar\\
B) $\mathrm{N}_{2} \mathrm{O}$\\
C) Ne\\
D) $\mathrm{CO}_{2}$
  \item Ne ga nisbatan zichligi 1,4 ga teng bo'lgan gazni aniqlang\\
A) NO\\
B) $\mathrm{N}_{2}$\\
C) $\mathrm{CO}_{2}$\\
D) He
  \item HF ga nisbatan zichligi 1 ga teng bo'lgan gazni aniqlang\\
A) $\mathrm{F}_{2}$\\
B) $\mathrm{N}_{2}$\\
C) Ne\\
D) He
  \item Havoga nisbatan zichligi 2 ga teng bo'lgan gazni aniqlang\\
A) $\mathrm{C}_{4} \mathrm{H}_{10}$\\
B) $\mathrm{C}_{3} \mathrm{H}_{8}$\\
C) $\mathrm{C}_{2} \mathrm{H}_{6}$\\
D) $\mathrm{CH}_{4}$
  \item $\mathrm{H}_{2}$ ga nisbatan zichligi 32 ga teng bo'lgan gazni aniqlang\\
A) $\mathrm{SO}_{2}$\\
B) $\mathrm{N}_{2}$\\
C) Ar\\
D) $\mathrm{N}_{2} \mathrm{O}_{5}$
  \item $\mathrm{H}_{2}$ ga nisbatan zichligi 16 ga teng bo'lgan gazni aniqlang\\
A) $\mathrm{O}_{2}$\\
B) $\mathrm{N}_{2}$\\
C) Ar\\
D) $\mathrm{N}_{2} \mathrm{O}_{5}$
  \item $\mathrm{CH}_{4}$ ga nisbatan zíchligi 2 ga teng bo'lgan gazni aniqlang\\
A) $\mathrm{Br}_{2}$\\
B) $\mathrm{O}_{2}$\\
C) Ar\\
D) He
  \item $\mathrm{SiH}_{4}$ ga nisbatan zichligi 2 ga teng bo'lgan gazni aniqlang\\
A) $\mathrm{SO}_{2}$\\
B) $\mathrm{N}_{2}$\\
C) Ar\\
D) $\mathrm{N}_{2} \mathrm{O}_{5}$
  \item Ne ga nisbatan zichligi 2,2 ga teng bo'lgan gazni aniqlang\\
A) CO\\
B) $\mathrm{N}_{2} \mathrm{O}_{3}$\\
C) $\mathrm{CO}_{2}$\\
D) $\mathrm{NO}_{2}$
  \item He ga nisbatan zichligi 5 bolgan gazning $\mathrm{H}_{2}$ ga nisbatan zichligini aniqlang\\
A) 2,5\\
B) 10\\
C) 5\\
D) 8\\
  \item Ne ga nisbatan zichligi 1,6 bo'lgan gazning He ga nisbatan zichligini aniqlang\\
A) 2,5\\
B) 9\\
C) 5\\
D) 8
  \item $\mathrm{O}_{2}$ ga nisbatan zichligi 2 bo'lgan gazning $\mathrm{H}_{2}$ ga nisbatan zichligini toping\\
A) 32\\
B) 1,36\\
C) 5\\
D) 8
  \item $\mathrm{O}_{2}$ ga nisbatan zichligi 0,5 bo'lgan gazning $\mathrm{H}_{2}$ ga nisbatan zichligini toping\\
A) 8\\
B) 7\\
C) 6\\
D) 5
  \item He ga nisbatan zichligi 10 bo'lgan moddaning $\mathrm{H}_{2}$ ga nisbatan zichligini aniqlang\\
A) 20\\
B) 10\\
C) 5\\
D) 8
  \item Ne ga nisbatan zichligi 3,2 bo'lgan gazning He ga nisbatan zichligini aniqlang\\
A) 25\\
B) 9\\
C) 16\\
D) 18
  \item $\mathrm{O}_{2}$ ga nisbatan zichligi 1 bo'lgan gazning $\mathrm{H}_{2}$ ga nisbatan zichliginí toping\\
A) 32\\
B) 36\\
C) 16\\
D) 8
  \item $\mathrm{CH}_{4}$ ga nisbatan zichligi 0,5 bolgan gazning $\mathrm{H}_{2}$ ga nisbatan zichligini toping\\
A) 4\\
B) 7\\
C) 6\\
D) 5
  \item $\mathrm{N}_{2}$ ga nisbatan zichligi 2 bo'lgan gazning $\mathrm{H}_{2}$ nisbatan zichligini toping\\
A) 28\\
B) 26\\
C) 15\\
D) 18
  \item $\mathrm{NO}_{2}$ ga nisbatan zichligi 2 bo'lgan gazning He nisbatan zichligini toping\\
A) 18\\
B) 17\\
C) 23\\
D) 15
101. 1 mol HCl va $2 \mathrm{~mol} \mathrm{CO}_{2}$ dan iborat gazlar aralashmasining $\mathrm{H}_{2}$ ga nisbatan zichligini toping.\\
A) $41,5 / 2$\\
B) $44,4 / 2$\\
C) $74,2 / 2$\\
D) $12,5 / 2$
  \item 1 mol CO va $2 \mathrm{~mol} \mathrm{CO}_{2}$ dan iborat gazlar aralashmasining He ga nisbatan zichligini toping.\\
A) $62 / 4$\\
B) $44 / 4$\\
C) $74 / 4$\\
D) $58 / 6$
  \item $2 \mathrm{~mol} \mathrm{SO}_{2}$ va $3 \mathrm{~mol} \mathrm{CO}_{2}$ dan iborat gazlar aralashmasining CO ga nisbatan zichligini toping.\\
A) $52 / 28$\\
B) $44 / 28$\\
C) $64 / 28$\\
D) $48 / 28$
  \item $4 \mathrm{~mol} \mathrm{H}_{2}$ va 2 mol CO dan iborat gazlar aralashmasining $\mathrm{H}_{2}$ ga nisbatan zichligini toping.\\
A) $12,25 / 2$\\
B) $14,4 / 2$\\
C) $10,67 / 2$\\
D) $12,5 / 2$
  \item $1 \mathrm{~mol} \mathrm{Cl}_{2}$ va $2 \mathrm{~mol} \mathrm{CO}_{2}$ dan iborat gazlar aralashmasining $\mathrm{N}_{2}$ ga nisbatan zichligini toping.\\
A) $72,25 / 28$\\
B) $53 / 28$\\
C) $74,7 / 28$\\
D) $42,5 / 28$
  \item 2 mol He va 2 mol Ar dan iborat gazlar aralashmasining $\mathrm{F}_{2}$ ga nisbatan zichliginí toping.\\
A) $22 / 38$\\
B) $44 / 19$\\
C) $74 / 19$\\
D) $12 / 38$
  \item 3 mol Ar va $2 \mathrm{~mol} \mathrm{~F}_{2}$ dan iborat gazlar aralashmasining $\mathrm{O}_{2}$ ga nisbatan zichligini toping.\\
A) $62,25 / 16$\\
B) $34,4 / 16$\\
C) $39,2 / 32$\\
D) $32,5 / 32$
  \item 1 mol Ne va $2 \mathrm{~mol} \mathrm{CO}_{2}$ dan iborat gazlar aralashmasining $\mathrm{H}_{2}$ ga nisbatan zichligini toping.\\
A) $25 / 2$\\
B) $36 / 2$\\
C) $44,2 / 2$\\
D) $12,5 / 2$
  \item 2 mol Ne va 3 mol CO dan iborat gazlar aralashmasining $\mathrm{Br}_{2}$ ga nisbatan zichligini toping.\\
A) $25,2 / 160$\\
B) $14,4 / 80$\\
C) $24,8 / 160$\\
D) $22,5 / 80$
  \item 1 mol NO va $2 \mathrm{~mol} \mathrm{SO}_{2}$ dan iborat gazlar aralashmasining $\mathrm{N}_{2}$ ga nisbatan zichligini toping.\\
A) $42,25 / 28$\\
B) $52,67 / 28$\\
C) $74,7 / 14$\\
D) $72,5 / 14$
  \item 5 litr $\mathrm{H}_{2}$ va 5 litr $\mathrm{N}_{2}$ dan iborat gazlar aralashmasidagi $\mathrm{H}_{2}$ ning hajmiy ulushini \% da toping. (hajmlar bir xil sharoitda o'lchangan)\\
A) 50\\
B) 40\\
C) 30\\
D) 20\\
  \item 5,6 litr $\mathrm{H}_{2} \mathrm{~S}$ va 16,8 litr $\mathrm{N}_{2} \mathrm{O}$ dan\\
iborat gazlar aralashmasidagi $\mathrm{H}_{2} \mathrm{~S}$ ning hajmiy ulushini \% da toping. (hajmlar bir xil sharoitda o'lchangan)\\
A) 20\\
B) 30\\
C) 25\\
D) 40
  \item 8,96 litr NO va 13,44 litr CO dan iborat gazlar aralashmasidagi CO ning hajmiy ulushini \% da toping. (hajmlar bir xil sharoitda o'lchangan)\\
A) 70\\
B) 60\\
C) 25\\
D) 40
  \item 6, 72 litr $\mathrm{Cl}_{2}$ va 15,68 litr Ar dan iborat gazlar aralashmasidagi $\mathrm{Cl}_{2}$ ning hajmiy ulushini \% da toping. (hajmlar bir xil sharoitda o'lchangan)\\
A) 20\\
B) 30\\
C) 25\\
D) 40
  \item 2,24 litr $\mathrm{NO}_{2}$ va 20,16 litr $\mathrm{N}_{2}$ dan iborat gazlar aralashmasidagi $\mathrm{N}_{2}$ ning hajmiy ulushini \% da toping. (hajmlar bir xil sharoitda o'lchangan)\\
A) 90\\
B) 10\\
C) 20\\
D) 80
  \item 11,2 litr NO va 11,2 litr $\mathrm{N}_{2} \mathrm{O}$ dan iborat gazlar aralashmasidagi $\mathrm{N}_{2} \mathrm{O}$ ning hajmiy ulushini \% da toping. (hajmlar bir xil sharoitda o'lchangan)\\
A) 55,5\\
B) 17,77\\
C) 50\\
D) 20
  \item 13,44 litr Kr va 8,96 litr $\mathrm{N}_{2}$ dan iborat gazlar aralashmasidagi $\mathrm{N}_{2}$ ning hajmiy ulushini \% da toping. (hajmlar bir xil sharoitda o'lchangan)\\
A) 55\\
B) 40\\
C) 60\\
D) 45
  \item 2,8 litr Ar va 19,6 litr He dan iborat gazlar aralashmasidagi He ning hajmiy ulushini \% da toping. (hajmlar bir xil sharoitda o'lchangan)\\
A) 65,5\\
B) 27,77\\
C) 23,33\\
D) 87,5
  \item 7,84 litr CO va 14,56 litr $\mathrm{N}_{2} \mathrm{O}_{5}$ dan iborat gazlar aralashmasidagi CO ning hajmiy ulushini \% da toping. (hajmlar bir xil sharoitda o'lchangan)\\
A) 65\\
B) 45\\
C) 35\\
D) 55
  \item 10,08 litr $\mathrm{H}_{2}$ va 12,32 litr Ar dan iborat gazlar aralashmasidagi $\mathrm{H}_{2}$ ning hajmiy ulushini \% da toping. (hajmlar bir xil sharoitda o'lchangan)\\
A) 65\\
B) 45\\
C) 35\\
D) 55
  \item $5 \mathrm{~g} \mathrm{H}_{2}$ va $5 \mathrm{~g} \mathrm{~N}_{2}$ dan iborat gazlar aralashmasidagi $\mathrm{H}_{2}$ ning massa ulushini $\%$ da toping.\\
A) 50\\
B) 40\\
C) 30\\
D) 20
  \item $30 \mathrm{~g} \mathrm{H}_{2} \mathrm{~S}$ va $70 \mathrm{~g} \mathrm{~N}_{2} \mathrm{O}$ dan iborat gazlar aralashmasidagi $\mathrm{H}_{2} \mathrm{~S}$ ning тиква ulushini \% da toping.\\
A) 20\\
B) 30\\
C) 25\\
D) 40
  \item 60 g NO va 40 g CO dan iborat gazlar aralashmasidagi CO ning massa ulushini \% da toping.\\
A) 70\\
B) 60\\
C) 25\\
D) 40
  \item $25 \mathrm{~g} \mathrm{Cl}_{2}$ va 75 g Ar dan íborat gazlar aralashmasidagi $\mathrm{Cl}_{2}$ ning massa ulushini \% da toping.\\
A) 20\\
B) 30\\
C) 25\\
D) 40
  \item $8 \mathrm{~g} \mathrm{NO}_{2}$ va $2 \mathrm{~g} \mathrm{~N}_{2}$ dan iborat gazlar aralashmasidagi $\mathrm{N}_{2}$ ning massa ulushini \% da toping.\\
A) 90\\
B) 10\\
C) 20\\
D) 80
  \item 4 g NO va $4 \mathrm{~g} \mathrm{~N}_{2} \mathrm{O}$ dan iborat gazlar aralashmasidagi $\mathrm{N}_{2} \mathrm{O}$ ning massa ulushini \% da toping.\\
A) 55,5\\
B) 17,77\\
C) 50\\
D) 20
  \item $5,5 \mathrm{~g} \mathrm{Kr}$ va $4,5 \mathrm{~g} \mathrm{~N}_{2}$ dan iborat gazlar aralashmasidagi $\mathrm{N}_{2}$ ning massa ulushini $\%$ da toping.\\
A) 55\\
B) 40\\
C) 60\\
D) 45
  \item $12,5 \mathrm{Ar}$ va $87,5 \mathrm{~g}$ He dan iborat gazlar aralashmasidagi He ning massa ulushini \% da toping.\\
A) 65,5\\
B) 27,77\\
C) 23,33\\
D) 87,5
  \item $7,84 \mathrm{~g} \mathrm{CO}$ va $14,56 \mathrm{~g} \mathrm{~N}_{2} \mathrm{O}$ dan iborat gazlar aralashmasidagi CO ning massa ulushini \% da toping.\\
A) 65\\
B) 45\\
C) 35\\
D) 55
  \item $10,08 \mathrm{gr} \mathrm{H}_{2}$ va $17,92 \mathrm{~g} \mathrm{Ar}$ dan iborat gazlar aralashmasidagi $\mathrm{H}_{2}$ ning massa ulushini \% da toping.\\
A) 65\\
B) 45\\
C) 35\\
D) 55
  \item $2 \mathrm{~g} \mathrm{H}_{2}$ va 89,6 litrda o'lchangan $\mathrm{N}_{2}$ dan iborat gazlar aralashmasidagi $\mathrm{H}_{2}$ ning hajmiy ulushini \% da toping.\\
A) 50\\
B) 40\\
C) 30\\
D) 20\\
  \item $13,6 \mathrm{~g} \mathrm{H}_{2} \mathrm{~S}$ va 13,44 litrda o'lchangan $\mathrm{N}_{2} \mathrm{O}$ dan iborat gazlar aralashmasidagi $\mathrm{H}_{2} \mathrm{~S}$ ning hajmiy ulushini\\
$\%$ da toping.\\
A) 20\\
B) 30\\
C) 25\\
D) 40
  \item 9 g NO va 15,68 litrda o'lchangan CO dan iborat gazlar aralashmasidagi CO ning hajmiy ulushini \% da toping.\\
A) 70\\
B) 60\\
C) 25\\
D) 40
  \item $17,75 \mathrm{~g} \mathrm{Cl}_{2}$ va 16,8 litrda o'lchangan Ar dan iborat gazlar aralashmasidagi $\mathrm{Cl}_{2}$ ning hajmiy ulushini $\%$ da toping.\\
A) 20\\
B) 30\\
C) 25\\
D) 40
  \item $41,4 \mathrm{~g} \mathrm{NO}_{2}$ va 2,24 litrda o'lchangan $\mathrm{N}_{2}$ dan iborat gazlar aralashmasidagi $\mathrm{N}_{2}$ ning hajmiy ulushini $\%$ da toping.\\
A) 90\\
B) 10\\
C) 20\\
D) 80
  \item 24 g NO va 4,48 litrda o'lchangan $\mathrm{N}_{2} \mathrm{O}$ dan iborat gazlar aralashmasidagi $\mathrm{N}_{2} \mathrm{O}$ ning hajmiy ulushini \% da toping.\\
A) 55,5\\
B) 17,77\\
C) 50\\
D) 20
  \item $46,2 \mathrm{~g} \mathrm{Kr}$ va 10,08 litrda o'lchangan $\mathrm{N}_{2}$ dan iborat gazlar aralashmasidagi $\mathrm{N}_{2}$ ning hajmiy ulushini \% da toping.\\
A) 55\\
B) 40\\
C) 60\\
D) 45
  \item 35 g Ar va 2,8 litrda o'lchangan He dan iborat gazlar aralashmasidagi He ning hajmiy ulushini \% da toping.\\
A) 65,5\\
B) 27,77\\
C) 23,33\\
D) 12,5
  \item $12,6 \mathrm{~g} \mathrm{CO}$ va 12,32 litrda o'lchangan $\mathrm{N}_{2} \mathrm{O}$ dan iborat gazlar aralashmasidagi CO ning hajmiy ulushini \% da toping.\\
A) 65\\
B) 45\\
C) 35\\
D) 55
  \item $0,9 \mathrm{~g} \mathrm{H}_{2}$ va 12,32 litrda o'lchangan Ar dan iborat gazlar aralashmasidagi $\mathrm{H}_{2}$ ning hajmiy ulushini \% da toping.\\
A) 65\\
B) 45\\
C) 35\\
D) 55
  \item 2 g Ho va 89,6 litrda o'lchangan $\mathrm{N}_{2}$ dan iborat gazlar aralashmasidagi H a ning massa ulushini \% da toping.\\
A) $1 / 57$\\
B) $1 / 40$\\
C) $1 / 30$\\
D) $1 / 25$\\
  \item $13,6 \mathrm{~g} \mathrm{H}_{2} \mathrm{~S}$ va 13,44 litrda o'lchangan $\mathrm{N}_{2} \mathrm{O}$ dan iborat gazlar aralashmasidagi $\mathrm{H}_{2} \mathrm{~S}$ ning massa ulushini \% da toping.\\
A) $6,8 / 20$\\
B) $30 / 40$\\
C) $25,2 / 40$\\
D) $14,4 / 40$
  \item 9 g NO va 15,68 litrda o'lchangan CO dan iborat gazlar aralashmasidagi CO ning massa ulushini \% da toping.\\
A) $15,4 / 35,2$\\
B) $14,4 / 30,2$\\
C) $19,6 / 28,6$\\
D) $40 / 60$
  \item $17,75 \mathrm{~g} \mathrm{Cl}_{2}$ va 16,8 litrda o'lchangan Ar dan iborat gazlar aralashmasidagi $\mathrm{Cl}_{2}$ ning massa ulushini \% da toping.\\
A) $17,75 / 47,75$\\
B) $17,75 / 38,5$\\
C) $17,75 / 35,45$\\
D) $17,75 / 75,2$
  \item $41,4 \mathrm{~g} \mathrm{NO}_{2}$ va 2,24 litrda o'lchangan $\mathrm{N}_{2}$ dan iborat gazlar aralashmasidagi $\mathrm{N}_{2}$ ning massa ulushini \% da toping.\\
A) $12,4 / 35,3$\\
B) $2,8 / 48$\\
C) $20 / 60$\\
D) $1,4 / 22,1$
  \item 24 g NO va 4,48 litrda o'lchangan $\mathrm{N}_{2} \mathrm{O}$ dan iborat gazlar aralashmasidagi $\mathrm{N}_{2} \mathrm{O}$ ning massa ulushini \% da toping.\\
A) $8,8 / 32,8$\\
B) $4,4 / 35,8$\\
C) $8,8 / 16,4$\\
D) $12,8 / 44,2$
  \item $46,2 \mathrm{~g} \mathrm{Kr}$ va 10,08 litrda o'lchangan $\mathrm{N}_{2}$ dan iborat gazlar aralashmasidagi $\mathrm{N}_{2}$ ning massa ulushini \% da toping.\\
A) $24 / 55$\\
B) $12,6 / 58,8$\\
C) $12,6 / 29,4$\\
D) $12,6 / 45$
  \item 35 g Ar va 2,8 litrda o'lchangan He dan iborat gazlar aralashmasidagi He ning massa ulushini \% da toping.\\
A) $6 / 55$\\
B) $7 / 77$\\
C) $2 / 33$\\
D) $1 / 71$
  \item $12,6 \mathrm{~g} \mathrm{CO}$ va 12,32 litrda o'lchangan $\mathrm{SO}_{2}$ dan iborat gazlar aralashmasidagi CO ning massa ulushini \% da toping.\\
A) $6,3 / 36$\\
B) $12,6 / 74$\\
C) $12,6 / 47,8$\\
D) $1 / 55$
  \item $1,3 \mathrm{~g} \mathrm{H}_{2}$ va 12,32 litrda o'lchangan Ar dan iborat gazlar aralashmasidagi $\mathrm{H}_{2}$ ning massa ulushini \% da toping.\\
A) $1,3 / 22$\\
B) $0,65 / 45$\\
C) $0,65 / 11,65$\\
D) $1,3 / 12,55$
  \item He ga nisbatan zichligi 2 ga teng bo'lgan $\mathrm{H}_{2}$ va $\mathrm{O}_{2}$ dan iborat gazlar aralashmasidagi $\mathrm{O}_{2}$ ning hajmiy ulushini \% da toping.\\
A) 20\\
B) 80\\
C) 60\\
D) 40\\
  \item $\mathrm{H}_{2}$ ga nisbatan zichligi 15 ga teng bo'lgan Ne va Ar dan iborat gazlar aralashmasidagi Ne ning hajmiy ulushini \% da toping.\\
A) 50\\
B) 30\\
C) 60\\
D) 40
  \item Ne ga nisbatan zichligi 1 ga teng bo'lgan He va $\mathrm{N}_{2} \mathrm{O}$ dan iborat gazlar aralashmasidagi He ning hajmiy ulushini \% da toping.\\
A) 50\\
B) 30\\
C) 60\\
D) 40
  \item HF ga nisbatan zichligi 1,5 ga teng bo'lgan $\mathrm{CO}_{2}$ va $\mathrm{CH}_{4}$ dan iborat gazlar aralashmasidagi $\mathrm{CH}_{4}$ ning hajmiy ulushini \% da toping.\\
A) 50\\
B) 30\\
C) 60\\
D) 40
  \item $\mathrm{H}_{2}$ ga nisbatan zichligi 5 ga teng bo'lgan HF va He dan iborat gazlar aralashmasidagi HF ning hajmiy ulushini $\%$ da toping.\\
A) 62,5\\
B) 37,5\\
C) 40\\
D) 50
  \item He ga nisbatan zichligi 6 ga teng bo'lgan Ar va $\mathrm{CH}_{4}$ dan iborat gazlar aralashmasidagi Ar ning hajmiy ulushini $\%$ da toping.\\
A) 66,6\\
B) 44,4\\
C) 33,3\\
D) 55,5
  \item CO ga nisbatan zichligi 1 ga teng bo'lgan HF va $\mathrm{C}_{3} \mathrm{H}_{4}$ dan iborat gazlar aralashmasidagi HF ning hajmiy ulushini \% da toping.\\
A) 50\\
B) 30\\
C) 60\\
D) 40
  \item $\mathrm{H}_{2}$ ga nisbatan zichligi 16 ga teng bo'lgan $\mathrm{N}_{2}$ va $\mathrm{C}_{3} \mathrm{H}_{4}$ dan iborat gazlar\\
aralashmasidagi $\mathrm{N}_{2}$ ning hajmiy ulushini $\%$ da toping.\\
A) 66,6\\
B) 44,4\\
C) 33,3\\
D) 55,5
  \item CO ga nisbatan zichligi 1 ga teng bo'lgan Ne va Ar dan iborat gazlar aralashmasidagi Ne ning hajmiy ulushini \% da toping.\\
A) 50\\
B) 30\\
C) 60\\
D) 40
  \item $\mathrm{CH}_{4}$ ga nisbatan zichligi 1,5 ga teng bo'lgan Ne va $\mathrm{N}_{2}$ dan iborat gazlar aralashmasidagi $\mathrm{N}_{2}$ ning hajmiy ulushini $\%$ da toping.\\
A) 50\\
B) 30\\
C) 60\\
D) 40
  \item He ga nisbatan zichligi 2 ga teng bo'lgan $\mathrm{H}_{2}$ va $\mathrm{O}_{2}$ dan iborat gazlar aralashmasidagi $\mathrm{O}_{2}$ ning massa ulushini \% da toping.\\
A) 20\\
B) 80\\
C) 60\\
D) 40
  \item $\mathrm{H}_{2}$ ga nisbatan zichligi 15 ga teng bo'lgan Ne va Ar dan iborat gazlar aralashmasidagi Ne ning massa ulushini \% da toping.\\
A) 44,4\\
B) 66,7\\
C) 55,5\\
D) 33,3
  \item Ne ga nisbatan zichligi 1 ga teng bo'lgan He va $\mathrm{N}_{2} \mathrm{O}$ dan iborat gazlar aralashmasidagi He ning massa ulushini \% da toping.\\
A) 12\\
B) 3\\
C) 6\\
D) 4
  \item HF ga nisbatan zichligi 1,5 ga teng bo'lgan $\mathrm{CO}_{2}$ va $\mathrm{CH}_{4}$ dan iborat gazlar aralashmasidagi $\mathrm{CH}_{4}$ ning massa ulushini \% da toping.\\
A) 53,24\\
B) 34,54\\
C) 62,5\\
D) 26,67
  \item $\mathrm{H}_{2}$ ga nisbatan zichligi 5 ga teng bo'lgan HF va He dan iborat gazlar aralashmasidagi HF ning massa ulushini \% da toping.\\
A) 65\\
B) 75\\
C) 40\\
D) 50
  \item He ga nisbatan zichligi 6 ga teng\\
bo'lgan Ar va $\mathrm{CH}_{4}$ dan iborat gazlar aralashmasidagi Ar ning massa ulushini \% da toping.\\
A) 66.6\\
B) 44.4\\
C) 33,3\\
D) 55,5
  \item CO ga nisbatan zichligi 1 ga teng bo'lgan HF va $\mathrm{C}_{3} \mathrm{H}_{4}$ dan iborat gazlar aralashmasidagi HF ning massa ulushini \% da toping.\\
A) 42,86\\
B) 33,75\\
C) 65,4\\
D) 47,85
  \item $\mathrm{H}_{2}$ ga nisbatan zichligi 16 ga teng bo'lgan $\mathrm{N}_{2}$ va $\mathrm{C}_{3} \mathrm{H}_{4}$ dan iborat gazlar aralashmasidagi $\mathrm{N}_{2}$ ning massa ulushini $\%$ da toping.\\
A) 56,6\\
B) 44,4\\
C) 58,3\\
D) 55,5
  \item CO ga nisbatan zichligi 1 ga teng bo'lgan Ne va Ar dan iborat gazlar aralashmasidagi Ne ning massa ulushini $\%$ da toping.\\
A) 42,86\\
B) 33,75\\
C) 65,4\\
D) 47,85
  \item $\mathrm{CH}_{4}$ ga nisbatan zichligi 1,5 ga teng bo'lgan Ne va $\mathrm{N}_{2}$ dan iborat gazlar aralashmasidagi $\mathrm{N}_{2}$ ning massa ulushini $\%$ da toping.\\
A) 56,6\\
B) 44,4\\
C) 58,3\\
D) 55,5
  \item Ma'lum hajmli idish $\mathrm{CO}_{2}$ bilan to'ldirib o'lchanganda massasi 32 g keldi. Shu sharoitda $\mathrm{N}_{2}$ solib o'lchanganda 24 g kelsa, $\mathrm{CO}_{2}$ ning massasini hisoblang.\\
A) 14\\
B) 22\\
C) 18\\
D) 33
  \item Ma'lum hajmli idish $\mathrm{SO}_{2}$ bilan to'ldirib o'lchanganda massasi 42 g keldi. Shu sharoitda $\mathrm{F}_{2}$ solib o'lchanganda 29 g kelsa, $\mathrm{SO}_{2}$ ning massasini hisoblang.\\
A) 32\\
B) 42\\
C) 28\\
D) 33
  \item Ma'lum hajmli idish CO bilan to'ldirib o'lchanganda massasi 24 g keldi. Shu sharoitda Ne solib o'lchanganda 20 g\\
kelsa, CO ning massasini hisoblang.\\
A) 14\\
B) 22\\
C) 18\\
D) 33
  \item Ma'lum hajmli idish $\mathrm{CH}_{4}$ bilan to'ldirib o'lchanganda massasi 18 g keldi. Shu sharoitda Ar solib o'lchanganda 30 g kelsa. Ar ning massasini hisoblang.\\
A) 30\\
B) 25\\
C) 20\\
D) 35
  \item Ma'lum hajmli idish $\mathrm{Br}_{2}$ bilan to'ldirib o'lchanganda massasi 90 g keldi. Shu sharoitda $\mathrm{Cl}_{2}$ solib o'lchanganda $45,5 \mathrm{~g}$ kelsa, $\mathrm{Br}_{2}$ ning massasini hisoblang.\\
A) 140\\
B) 160\\
C) 180\\
D) 80
  \item Ma'lum hajmli idish $\mathrm{H}_{2} \mathrm{~S}$ bilan to'ldirib o'lchanganda massasi 37 g keldi.\\
Shu sharoitda He solib o'lchanganda 22 g kelsa, $\mathrm{H}_{2} \mathrm{~S}$ ning massasini hisoblang.\\
A) 17\\
B) 12\\
C) 18\\
D) 13
  \item Ma'lum hajmli idish CO bilan to'ldirib o'lchanganda massasi 29 g keldi. Shu sharoitda HF solib o'lchanganda 25 g kelsa, HF ning massasini hisoblang.\\
A) 24\\
B) 10\\
C) 18\\
D) 37
  \item Ma'lum hajmli idish $\mathrm{C}_{2} \mathrm{H}_{2}$ bilan to'ldirib o'lchanganda massasi 28 g keldi. Shu sharoitda $\mathrm{CH}_{4}$ solib o'lchanganda 23 g kelsa, $\mathrm{CH}_{4}$ ning massasini hisoblang.\\
A) 4\\
B) 12\\
C) 8\\
D) 13
  \item Ma'lum hajmli idish HCl bilan to'ldirib o'lchanganda massasi 83 g keldi. Shu sharoitda $\mathrm{N}_{2}$ solib o'lchanganda 66 g kelsa, HCl ning massasini hisoblang.\\
A) 73\\
B) 40\\
C) 36,5\\
D) 18,25
  \item Ma'lum hajmli idish $\mathrm{CO}_{2}$ bilan to'ldirib o'lchanganda massasi 98 g keldi. Shu sharoitda $\mathrm{F}_{2}$ solib o'lchanganda 86 g kelsa, $\mathrm{CO}_{2}$ ning massasini hisoblang.\\
A) 44\\
B) 88\\
C) 11\\
D) 33
  \item Ma'lum hajmli idish $\mathrm{CO}_{2}$ bilan to'ldirib o'lchanganda massasi 32 g keldi. Shu sharoitda $\mathrm{N}_{2}$ solib o'lchanganda 24 g kelsa, idish hajmini toping. (Barcha gazlarda o'lchangan)\\
A) 22,4\\
B) 11,2\\
C) 5,6\\
D) 13,44
  \item Ma'lum hajmli idish $\mathrm{SO}_{2}$ bilan to'ldirib o'lchanganda massasi 74 g keldi. Shu sharoitda F2 solib o'lchanganda 48 g kelsa, idish hajmini toping. (Barcha gazlarda o'lchangan)\\
A) 22,4\\
B) 11,2\\
C) 5,6\\
D) 13,44
  \item Ma'lum hajmli idish $\mathrm{CH}_{4}$ bilan to'ldirib o'lchanganda massasi $17,6 \mathrm{~g}$ keldi. Shu sharoitda Ar solib o'lchanganda 32 g kelsa, idish hajmini toping. (Barcha gazlarda o'lchangan)\\
A) 22,4\\
B) 11,2\\
C) 5,6\\
D) 13,44
  \item Ma'lum hajmli idish $\mathrm{Br}_{2}$ bilan to'ldirib o'lchanganda massasi 172 g keldi. Shu sharoitda $\mathrm{Cl}_{2}$ solib o'lchanganda 83 g kelsa, idish hajmini toping. (Barcha gazlarda o'lchangan)\\
A) 22,4\\
B) 11,2\\
C) 5,6\\
D) 13,44
  \item Ma'lum hajmli idish $\mathrm{H}_{2} \mathrm{~S}$ bilan to'ldirib o'lchanganda massasi $26,5 \mathrm{~g}$ keldi. Shu sharoitda He solib o'lchanganda 19 g kelsa, idish hajmini toping. (Barcha gazlarda o'lchangan)\\
A) 22,4\\
B) 11,2\\
C) 5,6\\
D) 13,44
  \item Ma'lum hajmli idish CO bilan to'ldirib o'lchanganda massasi 35 g keldi. Shu sharoitda HF solib o'lchanganda 27 g kelsa, idish hajmini toping. (Barcha gazlarda o'lchangan)\\
A) 22,4\\
B) 11,2\\
C) 5,6\\
D) 13,44
  \item Ma'lum hajmli idish $\mathrm{C}_{2} \mathrm{H}_{2}$ bilan to'ldirib o'lchanganda massasi $28,6 \mathrm{~g}$ keldi. Shu sharoitda $\mathrm{CH}_{4}$ solib o'lchanganda 22,6 g kelsa, idish hajmini toping. (Barcha gazlarda o'lchangan)\\
A) 22,4\\
B) 11,2\\
C) 5,6\\
D) 13,44
  \item Ma'lum hajmli idish HCl bilan to'ldirib o'lchanganda massasi $51,5 \mathrm{~g}$ keldi. Shu sharoitda $\mathrm{N}_{2}$ solib o'lchanganda 43 g kelsa, idish hajmini toping. (Barcha gazlarda o'lchangan)\\
A) 22,4\\
B) 11,2\\
C) 5,6\\
D) 13,44
  \item Ma'lum hajmli idish $\mathrm{CO}_{2}$ bilan to'ldirib o'lchanganda massasi $66,4 \mathrm{~g}$ keldi.
Shu sharoitda $\mathrm{F}_{2}$ solib o'lchanganda 62.8 g kelsa, idish hajmini toping. (Barcha gazlarda o'lchangan)\\
A) 22,4\\
B) 11.2\\
C) 5.6\\
D) 13,44
  \item Ma'lum hajmli idish $\mathrm{CO}_{2}$ bilan to'ldirib o'lchanganda massasi 32 g keldi. Shu sharoitda $\mathrm{N}_{2}$ solib o'lchanganda 24 g kelsa, idish massasini toping.\\
A) 10\\
B) 7\\
C) 5\\
D) 13
  \item Ma'lum hajmli idish $\mathrm{SO}_{2}$ bilan to'ldirib o'lchanganda massasi 74 g keldi. Shu sharoitda $\mathrm{F}_{2}$ solib o'lchanganda 48 g kelsa, idish massasini toping.\\
A) 22\\
B) 11\\
C) 5\\
D) 10
  \item Ma'lum hajmli idish CO bilan to'ldirib o'lchanganda massasi 20 g keldi. Shu sharoitda Ne solib o'lchanganda 16 g kelsa, idish massasini toping.\\
A) 6\\
B) 11\\
C) 5\\
D) 7
  \item Ma'lum hajmli idish $\mathrm{CH}_{4}$ bilan to'ldirib o'lchanganda massasi $17,6 \mathrm{~g}$ keldi. Shu sharoitda Ar solib o'lchanganda 32 g kelsa, idish massasini toping.\\
A) 7\\
B) 9\\
C) 8\\
D) 10
  \item Ma'lum hajmli idish $\mathrm{Br}_{2}$ bilan to'ldirib o'lchanganda massasi 172 g keldi. Shu sharoitda $\mathrm{Cl}_{2}$ solib o'lchanganda 83 g kelsa, idish massasini toping.\\
A) 24\\
B) 12\\
C) 6\\
D) 14
  \item Ma'lum hajmli idish $\mathrm{H}_{2} \mathrm{~S}$ bilan to'ldirib o'lchanganda massasi $26,5 \mathrm{~g}$ keldi. Shu sharoitda He solib o'lchanganda 19 g kelsa, idish massasini toping.\\
A) 14\\
B) 11\\
C) 56\\
D) 18
  \item Ma'lum hajmli idish CO bilan to'ldirib o'lchanganda massasi 35 g keldi. Shu sharoitda HF solib o'lchanganda 27 g kelsa, idish massasini toping.\\
A) 5\\
B) 7\\
C) 6\\
D) 8
  \item Ma'lum hajmli idish $\mathrm{C}_{2} \mathrm{H}_{2}$ bilan\\
to'ldirib o'lchanganda massasi $28,6 \mathrm{~g}$ keldi. Shu sharoitda $\mathrm{CH}_{4}$ solib o'lchanganda 22,6 g kelsa, idish massasini toping.\\
A) 14\\
B) 11\\
C) 13\\
D) 12
  \item Malum hajmli idish HCl bilan to'ldirib o'lchanganda massasi $51,5 \mathrm{~g}$ keldi. Shu sharoitda $\mathrm{N}_{2}$ solib o'lchanganda 43 g kelsa, idish massasini toping.\\
A) 15\\
B) 11\\
C) 16\\
D) 14
  \item Ma'lum hajmli idish $\mathrm{CO}_{2}$ bilan to'ldirib o'lchanganda massasi $66,4 \mathrm{~g}$ keldi. Shu sharoitda $\mathrm{F}_{2}$ solib o'lchanganda $62,8 \mathrm{~g}$ kelsa, idish massasini toping.\\
A) 30\\
B) 65\\
C) 40\\
D) 55
  \item Ma'lum hajmli idish $\mathrm{CO}_{2}$ bilan to'ldirib o'lchanganda massasi 32 g keldi. Shu sharoitda $\mathrm{N}_{2}$ solib o'lchanganda 24 g kelsa, He solib o'lchansa necha gr keladi?\\
A) 12\\
B) 22\\
C) 18\\
D) 33
  \item Ma'lum hajmli idish $\mathrm{SO}_{2}$ bilan to'ldirib o'lchanganda massasi 42 g keldi. Shu sharoitda $\mathrm{F}_{2}$ solib o'lchanganda 29 g kelsa, Ne solib o'lchansa necha gr keladi?\\
A) 12\\
B) 42\\
C) 20\\
D) 33
  \item Ma'lum hajmli idish CO bilan to'ldirib o'lchanganda massasi 24 g keldi. Shu sharoitda Ne solib o'lchanganda 20 g kelsa, $\mathrm{CH}_{4}$ solib o'lchansa necha gr keladi?\\
A) 14\\
B) 22\\
C) 18\\
D) 33
  \item Ma'lum hajmli idish $\mathrm{CH}_{4}$ bilan to'ldirib o'lchanganda massasi 18 g keldi. Shu sharoitda Ar solib o'lchanganda 30 g kelsa, $\mathrm{H}_{2}$ solib o'lchansa necha gr keladi?\\
A) 14\\
B) 11\\
C) 15\\
D) 13
  \item Ma'lum hajmli idish $\mathrm{Br}_{2}$ bilan to'ldirib o'lchanganda massasi 90 g keldi. Shu sharoitda $\mathrm{Cl}_{2}$ solib o'lchanganda $45,5 \mathrm{~g}$ kelsa, $\mathrm{SiH}_{4}$ solib o'lchansa necha gr keladi?\\
A) 14\\
B) 24\\
C) 18\\
D) 26
  \item Ma'lum hajmli idish $\mathrm{H}_{2} \mathrm{~S}$ bilan to'ldirib o'lchanganda massasi 37 g keldi. Shu sharoitda He solib o'lchanganda 22 g kelsa, $\mathrm{C}_{2} \mathrm{H}_{4}$ solib o'lchansa necha gr keladi?\\
A) 34\\
B) 12\\
C) 28\\
D) 13
  \item Ma'lum hajmli idish CO bilan to'ldiríb o'lchanganda massasi 29 g keldi. Shu sharoitda HF solib o'lchanganda 25 g kelsa, $\mathrm{NH}_{3}$ solib o'lchansa necha gr keladi?\\
A) 24,5\\
B) 20,6\\
C) 23,5\\
D) 37,4
  \item Ma'lum hajmli idish $\mathrm{C}_{2} \mathrm{H}_{2}$ bilan to'ldirib o'lchanganda massasi 28 g keldi.\\
Shu sharoitda $\mathrm{CH}_{4}$ solib o'lchanganda 23 g kelsa, $\mathrm{C}_{2} \mathrm{H}_{6}$ solib o'lchansa necha gr keladi?\\
A) 24\\
B) 22\\
C) 30\\
D) 25
  \item Ma'lum hajmli idish HCl bilan to'ldirib o'lchanganda massasi 83 g keldi. Shu sharoitda $\mathrm{N}_{2}$ solib o'lchanganda 66 g kelsa, $\mathrm{C}_{2} \mathrm{H}_{2}$ solib o'lchansa necha gr keladi?\\
A) 62\\
B) 40\\
C) 26,5\\
D) 18,25
  \item Ma'lum hajmli idish $\mathrm{CO}_{2}$ bilan to'ldirib o'lchanganda massasi 98 g keldi. Shu sharoitda $\mathrm{F}_{2}$ solib o'lchanganda 86 g kelsa, He solib o'lchansa necha gr keladi?\\
A) 14\\
B) 18\\
C) 11\\
D) 13
  \item Ma'lum hajmli idish $\mathrm{CO}_{2}$ bilan to'ldirib o'lchanganda massasi 32 g keldi. Shu sharoitda $\mathrm{N}_{2}$ solib o'lchanganda 24 g keladi, noma'lum X gaz solib o'lchanganda 25 gr kelsa, noma'lum X gazni toping.\\
A) NO\\
B) $\mathrm{H}_{2} \mathrm{~S}$\\
C) $\mathrm{SO}_{2}$\\
D) $\mathrm{NH}_{3}$
  \item Ma'lum hajmli idish $\mathrm{SO}_{2}$ bilan to'ldirib o'lchanganda massasi 42 g keldi. Shu sharoitda $\mathrm{F}_{2}$ solib o'lchanganda 29 g\\
keladi, noma'lum X gaz solib o'lchanganda 33 gr kelsa, noma'lum X gazni toping.\\
A) $\mathrm{NO}_{2}$\\
B) $\mathrm{H}_{2} \mathrm{~S}$\\
C) $\mathrm{CO}_{2}$\\
D) $\mathrm{NH}_{3}$
  \item Ma'lum hajmli idish CO bilan to'ldirib o'lchanganda massasi 24 g keldi. Shu sharoitda Ne solib o'lchanganda 20 g keladi, noma'lum X gaz solib o'lchanganda 34 gr kelsa, noma'lum $X$ gazni toping.\\
A) $\mathrm{N}_{2} \mathrm{O}_{3}$\\
B) $\mathrm{Br}_{2}$\\
C) CO\\
D) $\mathrm{O}_{3}$
  \item Ma'lum hajmli idish $\mathrm{CH}_{4}$ bilan to'ldirib o'lchanganda massasi 18 g keldi. Shu sharoitda Ar solib o'lchanganda 30 g keladi, noma'lum X gaz solib o'lchanganda 42 gr kelsa, noma'lum X gazni toping.\\
A) $\mathrm{N}_{2} \mathrm{O}_{5}$\\
B) HF\\
C) $\mathrm{SO}_{2}$\\
D) $\mathrm{N}_{2} \mathrm{H}_{4}$
  \item Ma'lum hajmli idish $\mathrm{Br}_{2}$ bilan to'ldirib o'lchanganda massasi 90 g keldi. Shu sharoitda $\mathrm{Cl}_{2}$ solib o'lchanganda $45,5 \mathrm{~g}$ keladi, noma'lum X gaz solib o'lchanganda 26 gr kelsa, noma'lum $X$ gazni toping.\\
A) $\mathrm{N}_{2} \mathrm{O}_{5}$\\
B) HF\\
C) $\mathrm{SO}_{2}$\\
D) $\mathrm{N}_{2} \mathrm{H}_{4}$
  \item Ma'lum hajmli idish $\mathrm{H}_{2} \mathrm{~S}$ bilan to'ldirib o'lchanganda massasi 37 g keldi. Shu sharoitda He solib o'lchanganda 22 g keladi, noma'lum X gaz solib o'lchanganda 30 gr kelsa, noma'lum X gazni toping.\\
A) $\mathrm{N}_{2} \mathrm{O}$\\
B) HF\\
C) $\mathrm{SO}_{2}$\\
D) $\mathrm{N}_{2} \mathrm{H}_{4}$
  \item Ma'lum hajmli idish CO bilan to'ldirib o'lchanganda massasi 29 g keldi. Shu sharoitda HF solib o'lchanganda 25 g keladi, noma'lum X gaz solib o'lchanganda 35 gr kelsa, noma'lum X gazni toping.\\
A) $\mathrm{C}_{3} \mathrm{H}_{4}$\\
B) HI\\
C) $\mathrm{P}_{2} \mathrm{O}_{5}$\\
D) $\mathrm{C}_{2} \mathrm{H}_{4}$
  \item Ma'lum hajmli idish $\mathrm{C}_{2} \mathrm{H}_{2}$ bilan to'ldirib o'lchanganda massasi 28 g keldi.\\
Shu sharoitda $\mathrm{CH}_{4}$ solib o'lchanganda 23 g keladi, noma'lum $X$ gaz solib o'lchanganda 79 gr kelsa, noma'lum $X$ gazni toping.\\
A) $\mathrm{C}_{3} \mathrm{H}_{4}$\\
B) HI\\
C) $\mathrm{P}_{2} \mathrm{O}_{5}$\\
D) $\mathrm{C}_{2} \mathrm{H}_{4}$
  \item Ma’lum hajmli idish HCl bilan to'ldirib o'lchanganda massasi 83 g keldi. Shu sharoitda $\mathrm{N}_{2}$ solib o'lchanganda 66 g keladi, noma'lum X gaz solib o'lchanganda 90 gr kelsa, noma'lum $X$ gazni toping.\\
A) $\mathrm{C}_{3} \mathrm{H}_{4}$\\
B) HI\\
C) $\mathrm{P}_{2} \mathrm{O}_{5}$\\
D) $\mathrm{C}_{2} \mathrm{H}_{4}$
  \item Ma'lum hajmli idish $\mathrm{CO}_{2}$ bilan to'ldirib o'lchanganda massasi 98 g keldi. Shu sharoitda $\mathrm{F}_{2}$ solib o'lchanganda 86 g kelsa, keladi, noma'lum X gaz solib o'lchanganda 66 gr kelsa, noma'lum X gazni toping.\\
A) $\mathrm{C}_{3} \mathrm{H}_{4}$\\
B) HI\\
C) $\mathrm{P}_{2} \mathrm{O}_{5}$\\
D) $\mathrm{C}_{2} \mathrm{H}_{4}$
  \item $27^{\circ} \mathrm{C}$ haroratda o'lchangan 30 litr $\mathrm{H}_{2}$ gazi $127^{\circ} \mathrm{C}$ haroratda necha litr hajmni egallaydi? $(\mathrm{P}=$ const $)$\\
A) 35\\
B) 25\\
C) 28\\
D) 40\\
  \item $227^{\circ} \mathrm{C}$ haroratda o'lchangan 50 litr $\mathrm{H}_{2}$ gazi $127^{\circ} \mathrm{C}$ haroratda necha litr hajmni egallaydi? $(\mathrm{P}=$ const $)$\\
A) 35\\
B) 25\\
C) 28\\
D) 40
  \item $17^{\circ} \mathrm{C}$ haroratda o'lchangan 20 litr $\mathrm{H}_{2}$ gazi $162^{\circ} \mathrm{C}$ haroratda necha litr hajmni egallaydi? ( $\mathrm{P}=$ const)\\
A) 30\\
B) 25\\
C) 45\\
D) 40
  \item $7^{\circ} \mathrm{C}$ haroratda o'lchangan 12,5 litr $\mathrm{H}_{2}$ gazi $287^{\circ} \mathrm{C}$ haroratda necha litr hajmni egallaydi? ( $\mathrm{P}=$ const)\\
A) 35\\
B) 25\\
C) 28\\
D) 40
  \item $40^{\circ} \mathrm{C}$ haroratda o'lchangan 14 litr $\mathrm{H}_{2}$ gazi $353^{\circ} \mathrm{C}$ haroratda necha litr hajmni egallaydi? $(\mathrm{P}=$ const $)$\\
A) 35\\
B) 25\\
C) 28\\
D) 40
  \item $75^{\circ} \mathrm{C}$ haroratda o'lchangan 30 litr $\mathrm{H}_{2}$ gazi $249^{\circ} \mathrm{C}$ haroratda necha litr hajmni egallaydi? $(\mathrm{P}=$ const $)$\\
A) 35\\
B) 45\\
C) 38\\
D) 40
  \item $14{ }^{\circ} \mathrm{C}$ haroratda o'lchangan 10 litr $\mathrm{H}_{2}$ gazi $301^{\circ} \mathrm{C}$ haroratda necha litr hajmni egallaydi? $(\mathrm{P}=$ const $)$\\
A) 30\\
B) 50\\
C) 20\\
D) 40
  \item $27^{\circ} \mathrm{C}$ haroratda o'lchangan 30 litr $\mathrm{H}_{2}$ gazi $627^{\circ} \mathrm{C}$ haroratda necha litr hajmni egallaydi? ( $\mathrm{P}=$ const)\\
A) 90\\
B) 80\\
C) 70\\
D) 60
  \item $17^{\circ} \mathrm{C}$ haroratda o'lchangan 30 litr $\mathrm{H}_{2}$ gazi $887^{\circ} \mathrm{C}$ haroratda necha litr hajmni egallaydi? $(\mathrm{P}=$ const $)$\\
A) 120\\
B) 125\\
C) 150\\
D) 140
  \item $15^{\circ} \mathrm{C}$ haroratda o'lchangan 30 litr $\mathrm{H}_{2}$ gazi $303^{\circ} \mathrm{C}$ haroratda necha litr hajmni egallaydi? $(\mathrm{P}=$ const $)$\\
A) 90\\
B) 80\\
C) 70\\
D) 60
11. $15^{\circ} \mathrm{C}$ haroratda o'lchangan 30 litr $\mathrm{H}_{2}$ gazi necha ${ }^{\circ} \mathrm{C}$ haroratda 60 litr hajmni egallaydi? $(\mathrm{P}=$ const $)$\\
A) 303\\
B) 808\\
C) 707\\
D) 606\\

12. $45^{\circ} \mathrm{C}$ haroratda o'lchangan 25 litr $\mathrm{H}_{2}$ gazi necha ${ }^{\circ} \mathrm{C}$ haroratda 75 litr hajmni egallaydi? ( $\mathrm{P}=$ const )\\
A) 317\\
B) 818\\
C) 745\\
D) 681\\
13. $25^{\circ} \mathrm{C}$ haroratda o'lchangan 20 litr $\mathrm{H}_{2}$ gazi necha ${ }^{\circ} \mathrm{C}$ haroratda 30 litr hajmni egallaydi? ( $\mathrm{P}=$ const)\\
A) 245\\
B) 147\\
C) 174\\
D) 243\\
14. $55^{\circ} \mathrm{C}$ haroratda o'lchangan 30 litr $\mathrm{H}_{2}$ gazi necha ${ }^{\circ} \mathrm{C}$ haroratda 75 litr hajmni egallaydi? ( $\mathrm{P}=$ const)\\
A) 313\\
B) 547\\
C) 727\\
D) 475\\
15. $47^{\circ} \mathrm{C}$ haroratda o'lchangan 10 litr $\mathrm{H}_{2}$ gazi necha ${ }^{\circ} \mathrm{C}$ haroratda 15 litr hajmni egallaydi? ( $\mathrm{P}=$ const)\\
A) 203\\
B) 308\\
C) 207\\
D) 306\\
16. $35^{\circ} \mathrm{C}$ haroratda o'lchangan 35 litr $\mathrm{H}_{2}$ gazi necha ${ }^{\circ} \mathrm{C}$ haroratda 70 litr hajmni egallaydi? ( $\mathrm{P}=$ const )\\
A) 343\\
B) 848\\
C) 717\\
D) 616\\
17. $25^{\circ} \mathrm{C}$ haroratda o'lchangan 12 litr $\mathrm{H}_{2}$ gazi necha ${ }^{\circ} \mathrm{C}$ haroratda 36 litr hajmni egallaydi? $(\mathrm{P}=$ const $)$\\
A) 518\\
B) 621\\
C) 447\\
D) 813\\
18. $18^{\circ} \mathrm{C}$ haroratda o'lchangan 20 litr $\mathrm{H}_{2}$ gazi necha ${ }^{\circ} \mathrm{C}$ haroratda 10 litr hajmni egallaydi? $(\mathrm{P}=$ const $)$\\
A) $-127,5$\\
B) 127,5\\
C) $\cdot 707$\\
D) 707\\
20. $9{ }^{\circ} \mathrm{C}$ hammatta o'khangan 80 lite $\mathrm{H}_{ \pm}$ sawi necha ${ }^{2} \mathrm{C}$ hatoratda 20 lite hajmui egallaydi? (Paronst)\\
A) 78\\
B) 128\\
C) $\cdot 125$\\
D) 125
  \item 80 kpa bosimda o'lchangan 40 litr gar 60 kpa bosimda necha litt hajmmi egallaydi?\\
(T=const)\\
A) 20\\
B) 14\\
C) 17\\
D) 24\\
23. 15 kpa bosimda o'lchangan 45 litr gaz 45 kpa bosimda necha litr hajmni egallaydi?\\
( $\mathrm{T}=$ const)\\
A) 10\\
B) 12\\
C) 15\\
D) 24
  \item 20 kpa bosimda o'lchangan 40 litr gaz 10 kpa bosimda necha litr hajmni egallaydi?\\
( $\mathrm{T}=$ const)\\
A) 20\\
B) 80\\
C) 60\\
D) 14
  \item 14 kpa bosimda o'lchangan 10 litr gaz 7 kpa bosimda necha litr hajmni egallaydi?\\
( $\mathrm{T}=$ const)\\
A) 20\\
B) 14\\
C) 17\\
D) 24
  \item 35 kpa bosimda o'lchangan 50 litr gaz 70 kpa bosimda necha litr hajmni egallaydi?\\
( $\mathrm{T}=$ const)\\
A) 20\\
B) 14\\
C) 28\\
D) 25
  \item 42 kpa bosimda o'lchangan 30 litr gaz 14 kpa bosimda necha litr hajmni egallaydi?\\
( $\mathrm{T}=$ const)\\
A) 90\\
B) 60\\
C) 40\\
D) 55
22. 1.1 kpm bosimda o'lchangan 20 lite ga\% ga kpa bosimda nucha lite hajmni equllaydi?\\
( $1=$ const)\\
A) 10\\
B) $1 \cdot 4$\\
C) 17\\
1)) 27\\
30. 10 kpa bosim(a o'lehangan 90 lite ga\% 30 kpa bosimda neeha litr hajmni ognallaydi?\\
( $\mathrm{T}=$ const )\\
A) $\cdot 15$\\
B) 15\\
C) 30\\
D) 25
  \item 30 kpa bosimdn o'lchangan 40 litr ga\% hajmi nocha kpa bosimda 60 litr hajmni ogallaydi? ( $\mathrm{T}=$ const)\\
A) 20\\
B) 14\\
C) 17\\
D) 24\\
  \item 15 kpa bosimda o'lchangan 45 litr ga\% hajmi necha kpa bosimda 22.5 litr hajmni ogallaydi? (' $\mathrm{l}=$ const)\\
A) 30\\
B) 32\\
C) 35\\
J)) 24
  \item 20 kpa bosinda o'lchangan 40 litr gaz hajmi necha kpa bosimda 10 litr hajmni egallaydi? (' 1 =const)\\
A) 20\\
B) 80\\
C) 60\\
D) 14
  \item 14 kpa bosimda o'lchangan 10 litr gaz hajmi necha kpa bosimda 5 litr hajmni egallaydi? ( $\mathrm{T}=$ const)\\
A) 22\\
B) 24\\
C) 28\\
D) 25
  \item 35 kpa bosimda o'lchangan 50 litr gaz hajmi necha kpa bosimda 25 litr hajmni egallaydi? ( $\mathrm{T}=$ const)\\
A) 70\\
B) 45\\
C) 80\\
D) 50
  \item 42 kpa bosimda o'lchangan 30 litr gaz hajmi necha kpa bosimda 60 litr hajmni egallaydi? ( $\mathrm{T}=$ const)\\
A) 18\\
B) 14\\
C) 21\\
D) 25
  \item 30 kpa bosimda o'lchangan 8 litr gaz hajmi necha kpa bosimda 16 litr hajmni egallaydi? (T=const)\\
A) 20\\
B) 15\\
C) 17\\
D) 24
  \item 3 kpa bosimda o'lchangan 30 litr gaz hajmi necha kpa bosimda 45 litr hajmni egallaydi? (T=const)\\
A) 2\\
B) 3\\
C) 4\\
D) 5
  \item 14 kpa bosimda o'lchangan 20 litr gaz hajmi necha kpa bosimda 10 litr hajmni egallaydi? (T=const)\\
A) 30\\
B) 24\\
C) 28\\
D) 27
  \item 10 kpa bosimda o'lchangan 90 litr gaz hajmi necha kpa bosimda 45 litr hajmni egallaydi? $(T=$ const $)$\\
A) 20\\
B) 24\\
C) 27\\
D) 25
41. Gaz $17^{\circ} \mathrm{C}$ haroratda bosimi 30 kpa teng bolsa, $307^{\circ} \mathrm{C}$ haroratda necha kpa bosimga teng bo'ladi? ( $\mathrm{V}=$ const)\\
A) 60\\
B) 40\\
C) 70\\
D) 50
42. Gaz $27^{\circ} \mathrm{C}$ haroratda bosimi 10 kpa teng bolsa, $177^{\circ} \mathrm{C}$ haroratda necha kpa bosimga teng bo'ladi? ( $\mathrm{V}=$ const)\\
A) 18\\
B) 40\\
C) 15\\
D) 50\\
43. Gaz $18^{\circ} \mathrm{C}$ haroratda bosimi 25 kpa teng bo'lsa, $309^{\circ} \mathrm{C}$ haroratda necha kpa bosimga teng bo'ladi? ( $\mathrm{V}=$ const)\\
A) 60\\
B) 40\\
C) 70\\
D) 50\\
44. Gaz $40^{\circ} \mathrm{C}$ haroratda bosimi 12 kpa teng bo'lsa, $353^{\circ} \mathrm{C}$ haroratda necha kpa bosimga teng bo'ladi? (V=const)\\
A) 48\\
B) 24\\
C) 18\\
D) 16\\
45. Gaz $9^{\circ} \mathrm{C}$ haroratda bosimi 20 kpa teng bo'lsa, $150^{\circ} \mathrm{C}$ haroratda necha kpa bosimga teng bo'ladi? ( $\mathrm{V}=$ const)\\
A) 30\\
B) 40\\
C) 60\\
D) 50\\
46. $\mathrm{Gaz} 72^{\circ} \mathrm{C}$ haroratda bosimi 15 kpa teng bolsa, $762^{\circ} \mathrm{C}$ haroratda necha kpa bosimga teng bo'ladi? ( $\mathrm{V}=$ const)\\
A) 65\\
B) 45\\
C) 77\\
D) 55\\
47. $\mathrm{Gaz} 14{ }^{\circ} \mathrm{C}$ haroratda bosimi 18 kpa teng bolsa, $301^{\circ} \mathrm{C}$ haroratda necha kpa bosimga teng bo'ladi? ( $\mathrm{V}=$ const)\\
A) 46\\
B) 41\\
C) 36\\
D) 33\\
48. Gaz $4^{\circ} \mathrm{C}$ haroratda bosimi 10 kpa teng bo'lsa, $281^{\circ} \mathrm{C}$ haroratda necha kpa bosimga teng bo'ladi? ( $\mathrm{V}=$ const )\\
A) 20\\
B) 40\\
C) 30\\
D) 50\\
49. Gaz $5^{\circ} \mathrm{C}$ haroratda bosimi 5 kpa teng bo'lsa, $144^{\circ} \mathrm{C}$ haroratda necha kpa bosimga teng bo'ladi? ( $\mathrm{V}=$ const)\\
A) 6,5\\
B) 4\\
C) 7,5\\
D) 5.5\\
50. Gaz $1{ }^{\circ} \mathrm{C}$ haroratda bosimi 30 kpa teng bo'lsa, $138^{\circ} \mathrm{C}$ haroratda necha kpa bosimga teng bo'ladi? ( $\mathrm{V}=$ const)\\
A) 65\\
B) 40\\
C) 35\\
D) 45
  \item Gaz $17^{\circ} \mathrm{C}$ haroratda bosimi 30 kpa teng bo'lsa, necha ${ }^{\circ} \mathrm{C}$ haroratda necha 60 kpa bosimga teng bo'ladi? ( $\mathrm{V}=$ const)\\
A) 307\\
B) 403\\
C) 177\\
D) 504\\
  \item Gaz $27^{\circ} \mathrm{C}$ haroratda bosimi 10 kpa teng bo'lsa, necha ${ }^{\circ} \mathrm{C}$ haroratda necha 25 kpa bosimga teng bo'ladi? (V=const)\\
A) 277\\
B) 153\\
C) 477\\
D) 143
  \item Gaz $18^{\circ} \mathrm{C}$ haroratda bosimi 25 kpa teng bo'lsa, necha ${ }^{\circ} \mathrm{C}$ haroratda necha 50 kpa bosimga teng bo'ladi? (V=const)\\
A) 603\\
B) 409\\
C) 701\\
D) 309
  \item Gaz $40^{\circ} \mathrm{C}$ haroratda bosimi 12 kpa teng bo'lsa, necha ${ }^{\circ} \mathrm{C}$ haroratda necha 24 kpa bosimga teng bo'ladi? ( $\mathrm{V}=$ const)\\
A) 481\\
B) 243\\
C) 353\\
D) 167
  \item Gaz $9^{\circ} \mathrm{C}$ haroratda bosimi 20 kpa teng bo'lsa, necha ${ }^{\circ} \mathrm{C}$ haroratda necha 40 kpa bosimga teng bo'ladi? ( $\mathrm{V}=$ const)\\
A) 291\\
B) 140\\
C) 601\\
D) 150
  \item Gaz $72^{\circ} \mathrm{C}$ haroratda bosimi 15 kpa teng bo'lsa, necha ${ }^{\circ} \mathrm{C}$ haroratda necha 45 kpa bosimga teng bo'ladi? (V=const)\\
A) 651\\
B) 457\\
C) 762\\
D) 553
  \item Gaz $14{ }^{\circ} \mathrm{C}$ haroratda bosimi 18 kpa teng bo'lsa, necha ${ }^{\circ} \mathrm{C}$ haroratda necha 27 kpa bosimga teng bo'ladi? (V=const)\\
A) 461,2\\
B) 157,5\\
C) 368,2\\
D) $133,0 \overline{1}$
  \item Gaz $4^{\circ} \mathrm{C}$ haroratda bosimi 10 kpa teng bo'lsa, necha ${ }^{\circ} \mathrm{C}$ haroratda necha 20 kpa bosimga teng bo'ladi? ( $\mathrm{V}=$ const)\\
A) 281\\
B) 417\\
C) 327\\
D) 507
  \item Gaz $5^{\circ} \mathrm{C}$ haroratda bosimi 5 kpa teng\\
bo'lsn, necha ${ }^{\circ} \mathrm{C}$ haroratda nochn 15 kpa bosimga teng bo'ladi? ( $\mathrm{V}=$ const)\\
A) 561\\
B) 413\\
C) 754\\
D) 551
  \item Gaz $1{ }^{\circ} \mathrm{C}$ haroratda bosimi 30 kpa tong bolsa, necha ${ }^{\circ} \mathrm{C}$ haroratda necha 60 kpa bosimga teng bo'ladi? ( $\mathrm{V}=$ const)\\
A) 275\\
B) 403\\
C) 351\\
D) 458
  \item Massasi $3,2 \mathrm{~g}$ bo'lgan noma'lum gaz $127^{\circ} \mathrm{C}$ haroratda 83,14 litr bo'lgan idishda 32 kpa bosim hosil qilsa, noma'lum gazni toping.\\
A) $He$\\
B) $\mathrm{Cl}_{2}$\\
C) $\mathrm{O}_{2}$\\
D) $Ar$
  \setcounter{enumi}{61}
  \item Massasi 8 g bo'lgan noma'lum gaz 127 ${ }^{\circ} \mathrm{C}$ haroratda 83,14 litr bo'lgan idishda 10 kpa bosim hosil qilsa, noma'lum gazni toping.\\
A) He\\
B) $\mathrm{Cl}_{2}$\\
C) $\mathrm{O}_{2}$\\
D) Ar
  \item Massasi 20 g bo'lgan noma'lum gaz 27 ${ }^{\circ} \mathrm{C}$ haroratda 83, 14 litr bo'lgan idíshda 30 kpa bosim hosil qilsa, noma'lum gazní toping.\\
A) $\mathrm{SiH}_{4}$\\
B) $\mathrm{F}_{2}$\\
C) $\mathrm{O}_{3}$\\
D) Ne
  \item Massasi $9,5 \mathrm{~g}$ bo'lgan noma'lum gaz $127^{\circ} \mathrm{C}$ haroratda 83,14 litr bo'lgan idishda 10 kpa bosim hosil qilsa, noma'lum gazni toping.\\
A) $\mathrm{SiH}_{4}$\\
B) $\mathrm{F}_{2}$\\
C) $\mathrm{O}_{3}$\\
D) Ne
  \item Massasi 20 g bo'lgan noma'lum gaz 17 ${ }^{\circ} \mathrm{C}$ haroratda 83,14 litr bo'lgan idishda 290 kpa bosim hosíl qílsa, noma'lum gazní toping.\\
A) $\mathrm{H}_{2}$\\
B) $\mathrm{NO}_{2}$\\
C) $\mathrm{CO}_{2}$\\
D) HF
  \item Massasi 200 g bo'lgan noma'lum gaz 17 ${ }^{\circ} \mathrm{C}$ haroratda 83,14 litr bo'lgan idishda 290 kpa bosim hosil qilsa, noma'lum gazni toping.\\
A) $\mathrm{H}_{2}$\\
B) $\mathrm{NO}_{2}$\\
C) $\mathrm{CO}_{2}$\\
D) HF
  \item Massasí $14,2 \mathrm{~g}$ bo'lgan noma'lum gaz $27^{\circ} \mathrm{C}$ haroratda 83,14 litr bo'lgan idishda 6 kpa bosim hosil qilsa, noma'lum gazni toping.\\
A) He\\
B) $\mathrm{Cl}_{2}$\\
C) $\mathrm{O}_{2}$\\
D) Ar
  \item Маннияі 9,6 \% bo'lkan noma'lum қи\% $177^{\circ} \mathrm{C}$ haroratda K'3, 14 litr bolgan idiahda 9 kpa bosim hosil qila, noma'lum gayni toping.\\
A) $\mathrm{PH}_{3}$\\
B) $\mathrm{Cl}_{2}$\\
C) $\mathrm{O}_{3}$\\
D) $\Lambda_{r}$
  \item Мавнаві $6,8 \mathrm{~g}$ bo'lgan noma'lum цa\% $227^{\circ} \mathrm{C}$ haroratda 83,14 litr bolyan idishda 10 kpa bosim hosil qilsa, nomailum gazni toping.\\
A) $\mathrm{PH}_{3}$\\
B) $\mathrm{Cl}_{2}$\\
C) $\mathrm{O}_{3}$\\
D) $\Lambda r$
  \item Мавsаві necha g bo'lgan $\mathrm{O}_{2}$ gazi $227^{\circ} \mathrm{C}$ haroratda 83,14 litr bo'lgan idishda 100 kpa bosim hosil qíladi?\\
A) 64\\
B) 32\\
C) 16\\
D) 48
  \item Massasi necha g bo'lgan $\mathrm{N}_{2}$ gazi $-70,35$ ${ }^{\circ} \mathrm{C}$ haroratda 83,14 litr bo'lgan ídishda $101,325 \mathrm{kpa}$ bosim hosíl qiladi?\\
A) 140\\
B) 56\\
C) 70\\
D) 28
  \item Massasi necha g bo'lgan $\mathrm{NO}_{2}$ gazi 127 ${ }^{\circ} \mathrm{C}$ haroratda 16,628 lítr bo'lgan idishda 200 kpa bosim hosil qiladi?\\
A) 23\\
B) 92\\
C) 46\\
D) 184
  \item Massasi necha g bo'lgan $\mathrm{CO}_{2}$ gazi -53 ${ }^{\circ} \mathrm{C}$ haroratda 16,628 litr bo'lgan idíshda 300 kpa bosim hosil qiladi?\\
A) 80\\
B) 120\\
C) 160\\
D) 480
  \item Massasi necha g bo'lgan $\mathrm{SO}_{2}$ gazi 127 ${ }^{\circ} \mathrm{C}$ haroratda 8,314 litr bo'lgan idishda 200 kpa bosim hosil qiladi?\\
A) 64\\
B) 32\\
C) 16\\
D) 48
  \item Massasi necha g bo'lgan $\mathrm{N}_{2} \mathrm{O}$ gazí 167 ${ }^{\circ} \mathrm{C}$ haroratda 83,14 litr bolgan idishda 150 kpa bosim hosil qiladi?\\
A) 164\\
B) 320\\
C) 160\\
D) 150
  \item Massasi necha g bo'lgan Ar gazi $-33^{\circ} \mathrm{C}$ haroratda 16,628 litr bo'lgan idishda 120 kpa bosim hosil qiladi?\\
A) 60\\
B) 30\\
C) 16\\
D) 40
  \item Massasi necha g bolgan He gazi $127^{\circ} \mathrm{C}$ haroratda 88,14 litr bolgan idishda 150 kpa bosin hosil qiladi?\\
A) 14\\
B) 12\\
C) 15\\
D) 18
  \item Massasi necha g bo'lgan $\mathrm{SiH}_{4}$ gazi 227 ${ }^{\circ} \mathrm{C}$ haroratda 83.14 litr bo'lgan idishda 100 kpa bosim hosil qiladi?\\
A) 64\\
B) 32\\
C) 16 .\\
D) 48
  \item Massasi necha g bo'lgan $\mathrm{N}_{3} \mathrm{H}_{4}$ gazi 227 ${ }^{\circ} \mathrm{C}$ haroratda 83.14 litr bo'lgan idishda 100 kpa bosin hosil qiladi?\\
A) 64\\
B) 32\\
C) 16\\
D) 48
  \item Massasi 1 g bo'lgan $\mathrm{O}_{3}$ gazi $207^{\circ} \mathrm{C}$ haroratda 0.82 litr bolgan idishda necha atm bosim hosil qiladi?\\
A) 1\\
B) 3\\
C) 2\\
D) 4
  \item Massasi 5 g bolgan Os gazi $47^{\circ} \mathrm{C}$ haroratda 8.2 litr bo'lgan idishda necha $a t m$ bosim hosil qiladi?\\
A) 0,1\\
B) 0.6\\
C) 0,5\\
D) 0.4
  \item Massasi $6,8 \mathrm{~g}$ bo'lgan $\mathrm{H}_{2} \mathrm{~S}$ gazi $67^{\circ} \mathrm{C}$ haroratda 16,4 litr bo'lgan idishda necha atm bosim hosil qiladi?\\
A) 6.8\\
B) 0,34\\
C) 3,4\\
D) 0,68
  \item Massasi 6 g bo'lgan SO 2 gazi $47^{\circ} \mathrm{C}$ haroratda 0,82 litr bo'lgan idishda necha atm bosim hosil qiladi?\\
A) 1\\
B) 3\\
C) 2\\
D) 4
  \item Massasi 4 g bo'lgan $\mathrm{CO} \simeq$ gazi $167^{\circ} \mathrm{C}$ haroratda 0.82 litr bo'lgan idishda necha atm bosim hosil qiladi?\\
A) 1\\
B) 3\\
C) 2\\
D) 4
  \item Massasi 2 g bo'lgan NO gazi $27^{\circ} \mathrm{C}$ haroratda 8.2 litr bo'lgan idishda necha atm bosim hosil qiladi?\\
A) 0,2\\
B) 0,6\\
C) 0,5\\
D) 0,4
  \item Massasi 7 g bo'lgan $\mathrm{N}_{2} \mathrm{O}_{5}$ gazi $267^{\circ} \mathrm{C}$ haroratda 0,82 litr bo'lgan idishda necha atm bosim hosil qiladi?\\
A) 0,7\\
B) 3,5\\
C) 1,4\\
D) 14
  \item Massasi $2,5 \mathrm{~g}$ bo'lgan №Os gazi $107^{\circ} \mathrm{C}$ haroratda 0,82 litr bo'lgan idishda necha atm bosim hosil qiladi?\\
A) 5\\
B) 2,5\\
C) 1,25\\
D) 2.4
  \item Massasi 3 g bo'lgan HCl gazi $457^{\circ} \mathrm{C}$ haroratda 0.82 litr bo'lgan idishda necha atm bosim hosil qiladi?\\
A) 6\\
B) 3\\
C) 1.5\\
D) 7,5
  \item Massasi 40 g bo'lgan $\mathrm{O}_{3}$ gazi $687^{\circ} \mathrm{C}$ haroratda 8,2 litr bo'lgan idishda necha atm bosim hosil qiladi?\\
A) 4\\
B) 8\\
C) 16\\
D) 32
  \item Massasi 20 g bo'lgan $\mathrm{O}_{2}$ gazi necha ${ }^{\circ} \mathrm{C}$ haroratda 40 litr bo'lgan idishda 623,6 Hg.ust. bosim hosil qiladi?\\
A) 367\\
B) 817\\
C) 167\\
D) 327
  \item Massasi 32 g bo'lgan $\mathrm{SiH}_{4}$ gazi necha ${ }^{\circ} \mathrm{C}$ haroratda 50 litr bo'lgan idishda 249,44 Hg.ust. bosim hosil qiladi?\\
A) 37\\
B) -17\\
C) 67\\
D) -73
  \item Massasi 100 g bo'lgan $\mathrm{NO}_{2}$ gazi necha ${ }^{\circ} \mathrm{C}$ haroratda 500 litr bo'lgan idishda $187,08 \mathrm{Hg}$. ust. bosim hosil qiladi?\\
A) 467\\
B) 417\\
C) 267\\
D) 327
  \item Massasi 90 g bo'lgan $\mathrm{CO}_{2}$ gazi necha ${ }^{\circ} \mathrm{C}$ haroratda 360 litr bo'lgan idishda 124,72 Hg.ust. bosim hosil qiladi?\\
A) 79\\
B) 87\\
C) 67\\
D) 37
  \item Massasi 70 g bo'lgan SO gazi necha ${ }^{\circ} \mathrm{C}$ haroratda 210 litr bo'lgan idishda 187,08 Hg.ust. bosim hosil qiladi?\\
A) 307\\
B) 217\\
C) 303\\
D) 387
  \item Massasi 96 g bo'lgan $\mathrm{O}_{s}$ gazi necha ${ }^{\circ} \mathrm{C}$ haroratda 480 litr bo'lgan idishda 62,36 Hg.ust. bosim hosil qiladi?\\
A) 33\\
B) -47\\
C) 47\\
D) -33
  \item Massasi 34 g bo'lgan $\mathrm{NH}_{3}$ gazi necha ${ }^{\circ} \mathrm{C}$ haroratda 50 litr bo'lgan idishda 623,6 Hg.ust. bosim hosil qiladi?\\
A) -23\\
B) 17\\
C) -17\\
D) 23
  \item Massasi 20 g bo'lgan $\mathrm{N}_{2} \mathrm{H}_{4}$ gazi necha ${ }^{\circ} \mathrm{C}$ haroratda 50 litr bo'lgan idishda 623,6 Hg.ust. bosim hosil qiladi?\\
A) 367\\
B) 517\\
C) 167\\
D) 527
  \item Massasi 15 g bo'lgan NO gazi necha ${ }^{\circ} \mathrm{C}$ haroratda 25 litr bo'lgan idishda 124,72 Hg.ust. bosim hosil qiladi?\\
A) -367\\
B) 217\\
C) -173\\
D) 327
  \item ${ }^{23} \mathrm{Na}$ atomi tarkibidagi protonlar sonini aniqlang.\\
A) 11\\
B) 12\\
C) 13\\
D) 14
  \item ${ }^{27} \mathrm{Al}$ atomi tarkibidagi elektronlar sonini aniqlang.\\
A) 11\\
B) 12\\
C) 13\\
D) 14
  \item ${ }^{31} \mathrm{P}$ atomi tarkibidagi neytronlar sonini aniqlang.\\
A) 14\\
B) 15\\
C) 17\\
D) 16
  \item ${ }^{32} \mathrm{~S}$ atomi tarkibidagi protonlar sonini aniqlang.\\
A) 14\\
B) 15\\
C) 17\\
D) 16
  \item ${ }^{56} \mathrm{Fe}$ atomi tarkibidagi neytronlar sonini aniqlang.\\
A) 26\\
B) 30\\
C) 29\\
D) 31
  \item ${ }^{37} \mathrm{Cl}$ atomi tarkibidagi elektronlar sonini aniqlang.\\
A) 14\\
B) 15\\
C) 17\\
D) 16
  \item ${ }^{52} \mathrm{Cr}$ atomi tarkibidagi protonlar sonini aniqlang.\\
A) 24\\
B) 25\\
C) 26\\
D) 27
  \item ${ }^{65} \mathrm{Cu}$ atomi tarkibidagi neytronlar sonini aniqlang.\\
A) 32\\
B) 34\\
C) 36\\
D) 29
  \item ${ }^{55} \mathrm{Mn}$ atomi tarkibidagi elektronlar sonini aniqlang.\\
A) 25\\
B) 30\\
C) 26\\
D) 24
  \item ${ }^{48} \mathrm{Ti}$ atomi tarkibidagi protonlar sonini aniqlang.\\
A) 21\\
B) 22\\
C) 23\\
D) 24
  \item $\mathrm{Al}_{2}\left(\mathrm{SO}_{4}\right)_{3}$ molekulasi tarkibidagi protonlar sonini aniqlang.\\
A) 170\\
B) 172\\
C) 173\\
D) 174
  \item $\mathrm{H}_{2} \mathrm{O}$ molekulasi tarkibidagi elektronlar sonini aniqlang.\\
A) 10\\
B) 12\\
C) 13\\
D) 14
  \item $\mathrm{N}_{2} \mathrm{O}$ molekulasi tarkibidagi neytronlar sonini aniqlang.\\
A) 20\\
B) 22\\
C) 23\\
D) 24
  \item $\mathrm{H}_{2} \mathrm{SO}_{4}$ molekulasi tarkibidagi protonlar sonini aniqlang.\\
A) 54\\
B) 52\\
C) 50\\
D) 48
  \item $\mathrm{CaCO}_{3}$ molekulasi tarkibidagi protonlar sonini aniqlang.\\
A) 54\\
B) 52\\
C) 50\\
D) 48
  \item $\mathrm{Ca}(\mathrm{OH})_{2}$ molekulasi tarkibidagi elektronlar sonini aniqlang.\\
A) 40\\
B) 38\\
C) 43\\
D) 34
  \item $\mathrm{HNO}_{3}$ molekulasi tarkibidagi neytronlar sonini aniqlang.\\
A) 40\\
B) 31\\
C) 43\\
D) 34
  \item $\mathrm{H}_{2} \mathrm{O}_{2}$ molekulasi tarkibidagi protonlar sonini aniqlang.\\
A) 18\\
B) 19\\
C) 13\\
D) 14
  \item KOH molekulasi tarkibidagi elektronlar sonini aniqlang.\\
A) 20\\
B) 22\\
C) 23\\
D) 28
  \item HCl molekulasi tarkibidagi protonlar sonini aniqlang.\\
A) 18\\
B) 19\\
C) 13\\
D) 14
  \item $\mathrm{SO}_{4}{ }^{-2}$ ioni tarkibidagi elektronlar sonini aniqlang.\\
A) 50\\
B) 52\\
C) 54\\
D) 53
22. $\mathrm{CO}_{3}{ }^{-2}$ ioni tarkibidagi protonlar sonini aniqlang.\\
A) 40\\
B) 32\\
C) 34\\
D) 30\\
23. $\mathrm{CrO}_{4}{ }^{-2}$ ioni tarkibidagi neytronlar sonini aniqlang.\\
A) 50\\
B) 60\\
C) 44\\
D) 53\\
24. $\mathrm{PO}_{4}{ }^{-3}$ ioni tarkibidagi elektronlar sonini aniqlang.\\
A) 50\\
B) 60\\
C) 44\\
D) 53\\
25. $\mathrm{SiO}_{3}{ }^{\cdot 2}$ ioni tarkibidagi protonlar sonini aniqlang.\\
A) 50\\
B) 32\\
C) 38\\
D) 53\\
26. $\mathrm{MnO}_{4}{ }^{-2}$ ioni tarkibidagi elektronlar sonini aniqlang.\\
A) 30\\
B) 32\\
C) 59\\
D) 33\\
27. $\mathrm{Ca}^{+2}$ ioni tarkibidagi elektronlar sonini aniqlang.\\
A) 18\\
B) 20\\
C) 14\\
D) 13\\
28. $\mathrm{Al}(\mathrm{OH})_{2}{ }^{+}$ioni tarkibidagi elektronlar sonini aniqlang.\\
A) 50\\
B) 20\\
C) 34\\
D) 30\\
29. $\mathrm{CrCl}^{+2}$ ioni tarkibidagi protonlar sonini aniqlang.\\
A) 40\\
B) 41\\
C) 44\\
D) 43\\
30. $\mathrm{Mg}^{+2}$ ioni tarkibidagi neytronlar sonini aniqlang.\\
A) 12\\
B) 10\\
C) 11\\
D) 13
  \item ${ }^{23} \mathrm{Na}$ atomi tarkibidagi barcha zarrachalarining necha foizi protonlardan iborat?\\
A) $550 / 17$\\
B) $1200 / 34$\\
C) $1100 / 27$\\
D) $1400 / 17$
  \item ${ }^{27} \mathrm{Al}$ atomi tarkibidagi barcha zarrachalarining necha foizi elektronlardan iborat?\\
A) $1300 / 20$\\
B) $650 / 40$\\
C) $325 / 10$\\
D) $1300 / 10$
  \item ${ }^{31} \mathrm{P}$ atomi tarkibidagi barcha zarrachalarining necha foizi neytronlardan iborat?\\
A) $800 / 23$\\
B) $150 / 24$\\
C) $1600 / 23$\\
D) $160 / 23$
  \item ${ }^{32} \mathrm{~S}$ atomi tarkibidagi barcha zarrachalarining necha foizi protonlardan iborat?\\
A) $1600 / 24$\\
B) $200 / 6$\\
C) $160 / 24$\\
D) $160 / 22$
  \item ${ }^{56} \mathrm{Fe}$ atomi tarkibidagi barcha zarrachalarining necha foizi neytronlardan iborat?\\
A) $300 / 8$\\
B) $300 / 12$\\
C) $3000 / 84$\\
D) $1500 / 41$
  \item ${ }^{37} \mathrm{Cl}$ atomi tarkibidagi barcha zarrachalarining necha foizi elektronlardan iborat?\\
A) $1700 / 50$\\
B) $1500 / 34$\\
C) $850 / 27$\\
D) $160 / 3$
  \item ${ }^{52} \mathrm{Cr}$ atomi tarkibidagi barcha zarrachalarining necha foizi protonlardan iborat?\\
A) $600 / 19$\\
B) $300 / 19$\\
C) $2400 / 38$\\
D) $1200 / 76$
  \item ${ }^{65} \mathrm{Cu}$ atomi tarkibidagi barcha zarrachalarining necha foizi neytronlardan iborat?\\
A) $3600 / 24$\\
B) $340 / 17$\\
C) $360 / 47$\\
D) $1800 / 47$
  \item ${ }^{55} \mathrm{Mn}$ atomi tarkibidagi barcha zarrachalarining necha foizi protonlardan iborat?\\
A) $1250 / 40$\\
B) $2500 / 40$\\
C) $2600 / 34$\\
D) $2400 / 20$
  \item ${ }^{48} \mathrm{Ti}$ atomi tarkibidagi barcha zarrachalarining necha foizi neyronlardan iborat?\\
A) $2100 / 40$\\
B) $2200 / 44$\\
C) $2300 / 36$\\
D) $1300 / 35$
  \item $\mathrm{Al}_{2}\left(\mathrm{SO}_{4}\right)_{3}$ molekulasi tarkibidagi barcha zarrachalarining necha foizi protonlardan iborat?\\
A) $2125 / 64$\\
B) $1720 / 25$\\
C) $1700 / 512$\\
D) $1740 / 42$\\
  \item $\mathrm{H}_{2} \mathrm{O}$ molekulasi tarkibidagi barcha zarrachalarining necha foizi elektronlardan iborat?\\
A) $100 / 18$\\
B) $300 / 20$\\
C) $500 / 14$\\
D) $200 / 7$
  \item $\mathrm{N}_{2} \mathrm{O}$ molekulasi tarkibidagi barcha zarrachalarining necha foizi neytronlardan iborat?\\
A) $2200 / 66$\\
B) $2200 / 33$\\
C) $2300 / 46$\\
D) $2400 / 12$
  \item $\mathrm{H}_{2} \mathrm{SO}_{4}$ molekulasi tarkibidagi barcha zarrachalarining necha foizi protonlardan iborat?\\
A) $5000 / 74$\\
B) $2500 / 74$\\
C) $50 / 70$\\
D) $48 / 7$
  \item $\mathrm{CaCO}_{3}$ molekulasi tarkibidagi barcha zarrachalarining necha foizi elektronlardan iborat?\\
A) $5400 / 270$\\
B) $5200 / 26$\\
C) $5000 / 150$\\
D) $5000 / 75$
  \item $\mathrm{Ca}(\mathrm{OH})_{2}$ molekulasi tarkibidagi barcha zarrachalarining necha foizi elektronlardan iborat?\\
A) $380 / 14$\\
B) $475 / 14$\\
C) $3800 / 140$\\
D) $4750 / 14$
  \item $\mathrm{HNO}_{3}$ molekulasi tarkibidagi barcha zarrachalarining necha foizi neytronlardan iborat?"\\
A) $3100 / 95$\\
B) $380 / 7$\\
C) $4300 / 15$\\
D) $340 / 14$
  \item $\mathrm{H}_{2} \mathrm{O}$ 2 molekulasi tarkibidagi barcha zarrachalarining necha foizi protonlardan iborat?\\
A) $1800 / 26$\\
B) $1900 / 95$\\
C) $1800 / 52$\\
D) $1400 / 72$
  \item KOH molekulasi tarkibidagi barcha zarrachalarining necha foizi elektronlardan iborat?\\
A) $700 / 21$\\
B) $1400 / 84$\\
C) $2300 / 31$\\
D) $2800 / 41$
  \item H1 molekulasi tarkibidagi barcha zarrachalarining necha foizi protonlardan iborat?\\
A) $4800 / 600$\\
B) $1900 / 45$\\
C) $5400 / 182$\\
D) $1400 / 74$
  \item $\mathrm{SO}_{4}{ }^{-2}$ ioni tarkibidagi barcha zarrachalarining necha foizi elektronlardan iborat?\\
A) $2500 / 73$\\
B) $2500 / 146$\\
C) $5000 / 73$\\
D) $5000 / 74$
52. $\mathrm{CO}_{3}^{-2}$ ioni tarkibidagi barcha zarrachalarining necha foizi protonlardan iborat?\\
A) $1500 / 48$\\
B) $3000 / 48$\\
C) $1500 / 92$\\
D) $3000 / 92$\\
53. $\mathrm{CrO}_{4}{ }^{-2}$ ioni tarkibidagi barcha zarrachalarining necha foizi neytronlardan iborat?\\
A) $3000 / 92$\\
B) $6000 / 174$\\
C) $1500 / 174$\\
D) $3000 / 214$\\
54. $\mathrm{PO}_{4} \cdot{ }^{3}$ ioni tarkibidagi barcha zarrachalarining necha foizi elektronlardan iborat?\\
A) $5000 / 148$\\
B) $2500 / 72$\\
C) $5000 / 145$\\
D) $2500 / 145$\\
55. $\mathrm{SiO}_{3}{ }^{-2}$ ioni tarkibidagi barcha zarrachalarining necha foizi protonlardan iborat?\\
A) $3800 / 118$\\
B) $1900 / 116$\\
C) $950 / 29$\\
D) $950 / 35$\\
56. $\mathrm{MnO}_{4}{ }^{-2}$ ioni tarkibidagi barcha zarrachalarining necha foizi elektronlardan iborat?\\
A) $5900 / 178$\\
B) $5900 / 94$\\
C) $5900 / 80$\\
D) $5900 / 112$\\
57. $\mathrm{Ca}^{+2}$ ioni tarkibidagi barcha zarrachalarining necha foizi neytronlardan iborat?\\
A) $18 / 58$\\
B) $2000 / 29$\\
C) $2000 / 58$\\
D) $1800 / 58$\\
58. $\mathrm{Al}(\mathrm{OH})_{2}{ }^{+}$ioni tarkibidagi barcha zarrachalarining necha foizi elektronlardan iborat?\\
A) $3000 / 92$\\
B) $3000 / 82$\\
C) $3000 / 81$\\
D) $3000 / 91$\\
59. $\mathrm{CrF}^{+2}$ ioni tarkibidagi barcha zarrachalarining necha foizi protonlardan iborat?\\
A) $3300 / 104$\\
B) $2200 / 102$\\
C) $3300 / 102$\\
D) $2200 / 104$\\
60. $\mathrm{Mg}^{+2}$ ioni tarkibidagi barcha zarrachalarining necha foizi neytronlardan iborat?\\
A) $1200 / 34$\\
B) $1000 / 24$\\
C) $1200 / 24$\\
D) $1200 / 12$
  \item O'zaro izoelektron bo'lgan zarrachalarni tanlang.\\
A) $\mathrm{H}_{2} \mathrm{O} ; \mathrm{NH}_{3}$\\
B) $\mathrm{S} \cdot 2 ; \mathrm{S}$\\
C) $\mathrm{HNO}_{3} ; \mathrm{SiH}_{4}$\\
D) $\mathrm{CaCO}_{3} ; \mathrm{Ca}\left(\mathrm{NO}_{3}\right)_{2}$
  \item O'zaro izoelektron bo'lgan zarrachalarni tanlang.\\
A) $\mathrm{NH}_{3} ; \mathrm{SiH}_{4}$\\
B) $\mathrm{S}^{2 ;} ; \mathrm{Ar}$\\
C) $\mathrm{HNO}_{3} ; \mathrm{Ca}\left(\mathrm{NO}_{3}\right)_{2}$\\
D) $\mathrm{SiH}_{4} ; \mathrm{S}$
  \item O'zaro izoelektron bo'lgan zarrachalarni tanlang.\\
A) $\mathrm{NaCl} ; \mathrm{KCl}$\\
B) $\mathrm{K}_{2} \mathrm{~S} ; \mathrm{H}_{2} \mathrm{~S}$\\
C) $\mathrm{H}_{2} \mathrm{~S} \quad$;Ar\\
D) $\mathrm{H}_{2} \mathrm{O} ; \mathrm{H}_{2} \mathrm{~S}$
  \item O'zaro izoelektron bo'lgan zarrachalarni tanlang.\\
A) $\mathrm{HF} ; \mathrm{NH}_{3}$\\
B) $\mathrm{S} \div \mathrm{S}$\\
C) $\mathrm{HNO}_{3} ; \mathrm{SiH}_{4}$\\
D) $\mathrm{CaCO}_{3} ; \mathrm{Ca}\left(\mathrm{NO}_{3}\right)_{2}$
  \item O'zaro izoelektron bo'lgan zarrachalarni tanlang.\\
A) $\mathrm{H}_{2} \mathrm{~S} ; \mathrm{NH}_{4}{ }^{+}$\\
B) $\mathrm{S} ; \mathrm{NH}_{3}$\\
C) $\mathrm{HNO}_{3} ; \mathrm{SiH}_{4}$\\
D) $\mathrm{Ca}^{+2} ; \mathrm{HCl}$
  \item O'zaro izoelektron bo'lgan zarrachalarni tanlang.\\
A) $\mathrm{H}_{2} \mathrm{O}_{2} ; \mathrm{NH}_{3}$\\
B) $\mathrm{S}^{2} \cdot \mathrm{SO}_{2}$\\
C) $\mathrm{HNO}_{2} ; \mathrm{SiH}_{4}$\\
D) $\mathrm{CaCO}_{3} ; \mathrm{H}_{2} \mathrm{SO}_{4}$
  \item O'zaro izoelektron bo'lgan zarrachalarni tanlang.\\
A) $\mathrm{H}_{2} \mathrm{O} ; \mathrm{N}_{2} \mathrm{H}_{4}$\\
B) $\mathrm{HF} ; \mathrm{NH}_{3}$\\
C) $\mathrm{H}_{2} \mathrm{SO}_{3} ; \mathrm{SiH}_{4}$\\
D) $\mathrm{CaCO}_{3} ; \mathrm{Ca}\left(\mathrm{NO}_{3}\right)_{2}$
  \item O'zaro izoelektron bo'lgan zarrachalarni tanlang.\\
A) $\mathrm{H}_{2} \mathrm{O} ; \mathrm{N}_{2} \mathrm{H}_{4}$\\
B) $\mathrm{S}^{2 ;} ; \mathrm{Ar}$\\
C) $\mathrm{H}_{2} \mathrm{SO}_{3} ; \mathrm{SiH}_{4}$\\
D) $\mathrm{CaCO}_{3} ; \mathrm{Ca}\left(\mathrm{NO}_{3}\right)_{2}$
  \item O'zaro izoelektron bo'lgan zarrachalarni tanlang.\\
A) $\mathrm{H}_{2} \mathrm{O} ; \mathrm{HNO}_{3}$\\
B) $S^{-2} ; S$\\
C) $\mathrm{HNO}_{3} ; \mathrm{SiH}_{4}$\\
D) $\mathrm{CaCO}_{3} ; \mathrm{H}_{3} \mathrm{PO}_{4}$
  \item O'zaro izoelektron bo'lgan zarrachalarni tanlang.\\
A) $\mathrm{CH}_{4} ; \mathrm{NH}_{3}$\\
B) $\mathrm{S} \cdot 2 ; \mathrm{S}$\\
C) $\mathrm{HNO}_{3} ; \mathrm{SiH}_{4}$\\
D) $\mathrm{CaCO}_{3} ; \mathrm{Ca}\left(\mathrm{NO}_{3}\right)_{2}$
  \item Quyidagi yadro reaksiyasidan qaysi element izotopi hosil bo'ladi?\\
( ${ }_{7}^{14} \mathrm{~N}+{ }_{-1}^{0} \beta \rightarrow \mathrm{X}$ )\\
A) ${ }_{9}^{18} \mathrm{~F}$\\
B) ${ }_{8}^{18} \mathrm{O}$\\
C) ${ }_{5}^{10} \mathrm{~B}$\\
D) ${ }_{6}^{14} \mathrm{C}$
  \item Quyidagi yadro reaksiyasidan qaysi element izotopi hosil bo'ladi?\\
( ${ }_{7}^{14} \mathrm{~N}+{ }_{1}^{11} \mathrm{P} \rightarrow \mathrm{X}$ )\\
A) ${ }_{8}^{18} \mathrm{O}$\\
B) ${ }_{8}^{17} \mathrm{O}$\\
C) ${ }_{8}^{15} \mathrm{O}$\\
D) ${ }_{6}^{14} \mathrm{C}$
  \item Quyidagi yadro reaksiyasidan qaysi element izotopi hosil bo'ladi?\\
( ${ }_{7}^{14} \mathrm{~N}+{ }_{0}^{1} \mathrm{n} \rightarrow \mathrm{X}$ )\\
A) ${ }_{9}^{18} \mathrm{O}$\\
B) ${ }_{7}^{15} \mathrm{~N}$\\
C) ${ }_{7}^{13} \mathrm{~N}$\\
D) ${ }_{6}^{15} \mathrm{C}$
  \item Quyidagi yadro reaksiyasidan qaysi element izotopi hosil bo'ladi?\\
( ${ }_{7}^{14} \mathrm{~N}+{ }_{1}^{2} \mathrm{D} \rightarrow \mathrm{X}$ )\\
A) ${ }_{8}^{18} \mathrm{O}$\\
B) ${ }_{8}^{17} \mathrm{O}$\\
C) ${ }_{8}^{15} \mathrm{O}$\\
D) ${ }_{8}^{16} \mathrm{O}$
  \item Quyidagi yadro reaksiyasidan qaysi element izotopi hosil bo'ladi?\\
$\left({ }_{20}^{40} \mathrm{Ca}+{ }_{-1}^{0} \overline{\mathrm{e}} \rightarrow \mathrm{X}\right)$\\
A) ${ }_{21}^{40} \mathrm{Sc}$\\
B) ${ }_{19}^{40} \mathrm{~K}$\\
C) ${ }_{18}^{36} \mathrm{Ar}$\\
D) ${ }_{19}^{39} \mathrm{~K}$
  \item Quyidagi yadro reaksiyasidan qaysi element izotopi hosil bo'ladi?\\
( ${ }_{7}^{14} \mathrm{~N}+{ }_{1}^{3} \mathrm{~T} \rightarrow \mathrm{X}$ )\\
A) ${ }_{8}^{18} \mathrm{O}$\\
B) ${ }_{8}^{17} 0$\\
C) ${ }_{8}^{15} \mathrm{O}$\\
D) ${ }_{6}^{14} \mathrm{C}$
  \item Quyidagi yadro reaksiyasidan qaysi element izotopi hosil bo'ladi?\\
( ${ }_{7}^{14} \mathrm{~N}+{ }_{-1}^{0} \mathrm{e} \rightarrow \mathrm{X}$ )\\
A) ${ }_{6}^{14} \mathrm{C}$\\
B) ${ }_{8}^{18} \mathrm{O}$\\
C) ${ }_{5}^{10} \mathrm{~B}$\\
D) ${ }_{9}^{18} \mathrm{~F}$
  \item Quyidagi yadro reaksiyasidan qaysi element izotopi hosil bo'ladi?\\
( ${ }_{7}^{15} \mathrm{~N}+{ }_{+}^{0} \beta \rightarrow \mathrm{X}$ )\\
A) ${ }_{9}^{18} \mathrm{~F}$\\
B) ${ }_{8}^{15} \mathrm{O}$\\
C) ${ }_{5}^{10} \mathrm{~B}$\\
D) ${ }_{6}^{14} \mathrm{C}$
  \item Quyidagi yadro reaksiyasidan qayși element izotopi hosil bo'ladi?\\
$\left({ }_{20}^{40} \mathrm{Ca}+{ }_{2}^{4} \alpha \rightarrow \mathrm{X}\right)$\\
A) ${ }_{21}^{40} \mathrm{Sc}$\\
B) ${ }_{22}^{44} \mathrm{Ti}$\\
C) ${ }_{18}^{36} \mathrm{Ar}$\\
D) ${ }_{19}^{39} \mathrm{~K}$
  \item Quyidagi yadro reaksiyasidan qaysi element izotopi hosil bo'ladi?\\
( ${ }_{20}^{40} \mathrm{Ca} \rightarrow \mathrm{X}+{ }_{-1}^{0} \beta$ )\\
A) ${ }_{21}^{40} \mathrm{Sc}$\\
B) ${ }_{22}^{44} \mathrm{Ti}$\\
C) ${ }_{18}^{36} \mathrm{Ar}$\\
D) ${ }_{19}^{39} \mathrm{~K}$
  \item Quyidagi yadro reaksiyasidan qaysi element izotopi hosil bo'ladi?\\
( ${ }_{20}^{40} \mathrm{Ca} \rightarrow \mathrm{X}+{ }_{+}{ }_{+1}^{0} \beta$ )\\
A) ${ }_{21}^{40} \mathrm{Sc}$\\
B) ${ }_{22}^{44} \mathrm{Ti}$\\
C) ${ }_{18}^{36} \mathrm{Ar}$\\
D) ${ }_{19}^{40} \mathrm{~K}$
  \item Quyidagi yadro reaksiyasidan qaysi element izotopi hosil bo'ladi?\\
( ${ }_{20}^{40} \mathrm{Ca} \rightarrow \mathrm{X}+{ }_{1}^{1} \mathrm{P}$ )\\
A) ${ }_{21}^{41} \mathrm{Sc}$\\
B) ${ }_{22}^{44} \mathrm{Ti}$\\
C) ${ }_{18}^{38} \mathrm{Ar}$\\
D) ${ }_{19}^{39} \mathrm{~K}$
  \item Quyidagi yadro reaksiyasidan qaysi element izotopi hosil bo'ladi?\\
( ${ }_{20}^{40} \mathrm{Ca} \rightarrow \mathrm{X}+{ }_{0}^{1} \mathrm{n}$ )\\
A) ${ }_{21}^{40} \mathrm{Sc}$\\
B) ${ }_{20}^{99} \mathrm{Ca}$\\
C) ${ }_{18}^{36} \mathrm{Ar}$\\
D) ${ }_{19}^{39} \mathrm{~K}$
  \item Quyidagi yadro reaksiyasidan qaysi element izotopi hosil bo'ladi?\\
( ${ }_{20}^{40} \mathrm{Ca} \rightarrow \mathrm{X}+{ }_{1}^{2} \mathrm{D}$ )\\
A) ${ }_{21}^{40} \mathrm{Sc}$\\
B) ${ }_{22}^{44} \mathrm{Ti}$\\
C) ${ }_{19}^{38} \mathrm{~K}$\\
D) ${ }_{19}^{39} \mathrm{~K}$
  \item Quyidagi yadro reaksiyasidan qaysi element izotopi hosil bo'ladi?\\
$\left({ }_{20}^{40} \mathrm{Ca} \rightarrow \mathrm{X}+{ }_{1}^{3} \mathrm{~T}\right)$\\
A) ${ }_{21}^{42} \mathrm{Sc}$\\
B) ${ }_{19}^{37} \mathrm{~K}$\\
C) ${ }_{18}^{40} \mathrm{Ar}$\\
D) ${ }_{19}^{39} \mathrm{~K}$
  \item Quyidagi yadro reaksiyasidan qaysi element izotopi hosil bo'ladi?\\
$\left({ }_{20}^{40} \mathrm{Ca} \rightarrow \mathrm{X}+{ }_{-1}^{0} \mathrm{e}\right)$\\
A) ${ }_{21}^{40} \mathrm{Sc}$\\
B) ${ }_{19}^{40} \mathrm{~K}$\\
C) ${ }_{18}^{36} \mathrm{Ar}$\\
D) ${ }_{19}^{39} \mathrm{~K}$
  \item Quyidagi yadro reaksiyasidan qaysi element izotopi hosil bo'ladi?\\
( ${ }_{7}^{14} \mathrm{~N} \rightarrow \mathrm{X}+{ }_{1}^{3} \mathrm{~T}$ )\\
A) ${ }_{8}^{18} \mathrm{O}$\\
B) ${ }_{8}^{17} \mathrm{O}$\\
C) ${ }_{8}^{15} 0$\\
D) ${ }_{6}^{11} \mathrm{C}$
  \item Quyidagi yadro reaksiyasidan qaysi element izotopi hosil bo'ladi?\\
( ${ }^{15} \mathrm{~N} \rightarrow \mathrm{X}+{ }_{-1}^{0} \mathrm{e}$ )\\
A) ${ }_{6}^{14} \mathrm{C}$\\
B) ${ }_{8}^{15} \mathrm{O}$\\
C) ${ }_{5}^{10} \mathrm{~B}$\\
D) ${ }_{9}^{18} \mathrm{~F}$
  \item Quyidagi yadro reaksiyasidan qaysi element izotopi hosil bo'ladi?\\
( ${ }_{7}^{15} \mathrm{~N} \rightarrow \mathrm{X}+{ }_{0}^{1} \mathrm{n}$ )\\
A) ${ }_{8}^{18} \mathrm{O}$\\
B) ${ }_{7}^{14} \mathrm{~N}$\\
C) ${ }_{7}^{13} \mathrm{~N}$\\
D) ${ }_{6}^{15} \mathrm{C}$
21. Quyidagi yadro reaksiyasida qatnashgan X element izotopini aniqlang. $\mathrm{X}+{ }_{2}^{4} \mathrm{He} \rightarrow{ }_{17}^{34} \mathrm{Cl}+{ }_{1}^{2} \mathrm{D}$\\
A) ${ }_{14}^{30} \mathrm{Si}$\\
B) ${ }_{15}^{31} \mathrm{P}$\\
C) ${ }_{16}^{32} \mathrm{~s}$\\
D) ${ }_{13}^{28} \mathrm{Al}$
  \item Quyidagi yadro reaksiyasida qatnashgan X element izotopini aniqlang. $\mathrm{X}+{ }_{1}^{1} \mathrm{P} \rightarrow{ }_{19}^{40} \mathrm{~K}+{ }_{+}{ }_{1}^{0} \beta$\\
A) ${ }_{14}^{30} \mathrm{Si}$\\
B) ${ }_{15}^{31} \mathrm{P}$\\
C) ${ }_{16}^{32} \mathrm{~S}$\\
D) ${ }_{19}^{39} \mathrm{~K}$
  \item Quyidagi yadro reaksiyasida qatnashgan X element izotopini aniqlang. $\mathrm{X}+{ }_{0}^{1} \mathrm{n} \rightarrow{ }_{22}^{44} \mathrm{Ti}+{ }_{1}^{2} \mathrm{D}$\\
A) ${ }_{22}^{44} \mathrm{Ti}$\\
B) ${ }_{18}^{40} \mathrm{Ar}$\\
C) ${ }_{23}^{45} \mathrm{~V}$\\
D) ${ }_{21}^{41} \mathrm{Sc}$
  \item Quyidagi yadro reaksiyasida qatnashgan $X$ element izotopini aniqlang. $\mathrm{X}+{ }_{-1}^{0} \mathrm{e} \rightarrow{ }_{19}^{39} \mathrm{~K}+{ }_{0}^{1} \mathrm{n}$\\
A) ${ }_{21}^{40} \mathrm{Sc}$\\
B) ${ }_{20}^{40} \mathrm{Ca}$\\
C) ${ }_{18}^{36} \mathrm{Ar}$\\
D) ${ }_{19}^{39} \mathrm{~K}$
  \item Quyidagi yadro reaksiyasida qatnashgan $X$ element izotopini aniqlang. $\mathrm{X}+{ }_{1}^{3} \mathrm{~T} \rightarrow{ }_{15}^{31} \mathrm{P}+{ }_{1}^{2} \mathrm{D}$\\
A) ${ }_{14}^{30} \mathrm{Si}$\\
B) ${ }_{15}^{30} \mathrm{P}$\\
C) ${ }_{16}^{32} \mathrm{~s}$\\
D) ${ }_{13}^{28} \mathrm{Al}$
  \item Quyidagi yadro reaksiyasida qatnashgan $X$ element izotopini aniqlang. $\mathrm{X}+{ }_{1}^{1} \mathrm{P} \rightarrow{ }_{26}^{56} \mathrm{Fe}+{ }_{-1}^{0} \mathrm{e}$\\
A) ${ }_{24}^{55} \mathrm{Cr}$\\
B) ${ }_{24}^{52} \mathrm{Cr}$\\
C) ${ }_{25}^{55} \mathrm{Mn}$\\
D) ${ }_{25}^{56} \mathrm{Mn}$
  \item Quyidagi yadro reaksiyasida qatnashgan X element izotopini aniqlang. $\mathrm{X}+{ }_{0}^{1} \mathrm{n} \rightarrow{ }_{14}^{30} \mathrm{Si}+{ }_{1}^{2} \mathrm{D}$\\
A) ${ }_{14}^{30} \mathrm{Si}$\\
B) ${ }_{15}^{31} \mathrm{P}$\\
C) ${ }_{16}^{32} \mathrm{~s}$\\
D) ${ }_{13}^{28} \mathrm{Al}$
  \item Quyidagi yadro reaksiyasida qatnashgan X element izotopini aniqlang. $\mathrm{X}+{ }_{1}^{1} \mathrm{P} \rightarrow{ }_{13}^{28} \mathrm{Al}+{ }_{1}^{0}$ e\\
A) ${ }_{21}^{40} \mathrm{Sc}$\\
B) ${ }_{19}^{40} \mathrm{~K}$\\
C) ${ }_{16}^{32} \mathrm{~s}$\\
D) ${ }_{11}^{27} \mathrm{Na}$
  \item Quyidagi yadro reaksiyasida qatnashgan X element izotopini aniqlang. $\mathrm{X}+{ }_{2}^{4} \mathrm{He} \rightarrow{ }_{9}^{18} \mathrm{~F}+{ }_{1}^{2} \mathrm{D}$\\
A) ${ }_{8}^{18} \mathrm{O}$\\
B) ${ }_{8}^{17} \mathrm{O}$\\
C) ${ }_{8}^{16} 0$\\
D) ${ }_{6}^{11} \mathrm{C}$
  \item Quyidagi yadro reaksiyasida qatnashgan X element izotopini aniqlang. $\mathrm{X}+{ }_{2}^{4} \mathrm{He} \rightarrow{ }_{7}^{13} \mathrm{~N}+{ }_{0}^{1} \mathrm{n}$\\
A) ${ }_{6}^{14} \mathrm{C}$\\
B) ${ }_{8}^{15} \mathrm{O}$\\
C) ${ }_{5}^{10} \mathrm{~B}$\\
D) ${ }_{9}^{18} \mathrm{~F}$
  \item ${ }_{87}^{223} \mathrm{Fr} \rightarrow{ }_{81}^{203} \mathrm{Tl}+\mathrm{x}_{2}^{4} \mathrm{He}+\mathrm{y}_{-1}^{0}$ è ushbu yadro reaksiyasidagi $x$ va y qiymatlarini ko'rsating..\\
A) 5,4\\
B) 7,4\\
C) 8,3\\
D) 6,3
  \item ${ }_{87}^{223} \mathrm{Fr} \rightarrow{ }_{81}^{199} \mathrm{Tl}+\mathrm{x}_{2}^{4} \mathrm{He}+\mathrm{y}_{-1}^{0}$ ē ushbu yadro reaksiyasidagi x va y qiymatlarini ko'rsating.\\
A) 5,4\\
B) 4,4\\
C) 8,4\\
D) 6,6
  \item ${ }_{87}^{220} \mathrm{Fr} \rightarrow{ }_{81}^{200} \mathrm{Tl}+\mathrm{x}_{2}^{4} \mathrm{He}+\mathrm{y}_{-1}^{0}$ ē ushbu yadro reaksiyasidagi $x$ va y qiymatlarini ko'rsating.\\
A) 7,4\\
B) 5,4\\
C) 7,3\\
D) 5,3
  \item ${ }_{87}^{222} \mathrm{Fr} \rightarrow{ }_{81}^{194} \mathrm{Tl}+\mathrm{x}_{2}^{4} \mathrm{He}+\mathrm{y}_{-1}^{0}$ e ushbu yadro reaksiyasidagi $x$ va y qiymatlarini ko'rsating.\\
A) 5,4\\
B) 7,8\\
C) 8,3\\
D) 6,3
  \item ${ }_{87}^{243} \mathrm{Fr} \rightarrow{ }_{81}^{235} \mathrm{Tl}+\mathrm{x}_{2}^{4} \mathrm{He}+\mathrm{y}_{+}{ }_{1}^{0} \beta$ ushbu yadro reaksiyasidagi $x$ va y qiymatlarini ko'rsating?\\
A) 5,5\\
B) 4,4\\
C) 3,3\\
D) 2,2
  \item ${ }_{87}^{228} \mathrm{Fr} \rightarrow{ }_{81}^{192} \mathrm{Tl}+\mathrm{x}_{2}^{4} \mathrm{He}+\mathrm{y}_{-1}^{0}$ ē ushbu yadro reaksiyasidagi $x$ va y qiymatlarini ko'rsating.\\
A) 9,12\\
B) 8,7\\
C) 8,9\\
D) 6,7
  \item ${ }_{87}^{229} \mathrm{Fr} \rightarrow{ }_{81}^{197} \mathrm{Tl}+\mathrm{x}_{2}^{4} \mathrm{He}+\mathrm{y}_{-1}^{0}$ ē ushbu yadro reaksiyasidagi x va y qiymatlarini ko'rsating.\\
A) 4,7\\
B) 7,4\\
C) 8,10\\
D) 10,8
  \item ${ }_{87}^{242} \mathrm{Fr} \rightarrow{ }_{81}^{222} \mathrm{Tl}+\mathrm{x}_{2}^{4} \mathrm{He}+\mathrm{y}_{-1}^{0}$ ë ushbu yadro reaksiyasidagi $x$ va y qiymatlarini ko'rsating.\\
A) 5,4\\
B) 7,4\\
C) 8,3\\
D) 6,3
  \item ${ }_{87}^{236} \mathrm{Fr} \rightarrow{ }_{81}^{188} \mathrm{Tl}+\mathrm{x}_{2}^{4} \mathrm{He}+\mathrm{y}_{-1}^{0}$ è ushbu yadro reaksiyasidagi $x$ va y qiymatlarini ko'rsating.\\
A) 5,8\\
B) 8,5\\
C) 12,18\\
D) 9,12
  \item ${ }_{87}^{223} \mathrm{Fr} \rightarrow{ }_{81}^{187} \mathrm{Tl}+\mathrm{x}_{2}^{4} \mathrm{He}+\mathrm{y}_{-1}^{0}$ ē ushbu yadro reaksiyasidagi $x$ va y qiymatlarini ko'rsating.\\
A) 9,12\\
B) 6,9\\
C) 8,7\\
D) 7,8
  \item ${ }_{102}^{254} \mathrm{No} \rightarrow{ }_{96}^{238} \mathrm{Cm}+\mathrm{x}_{2}^{4} \mathrm{He}+\mathrm{y}_{-1}^{0} \mathrm{e}$ e ushbu yadro reaksiyasida $12,7 \mathrm{~g}$ nobeliy yemirilishidan hosil bo'lgan elektronlar sonini hisoblang.\\
A) $12,04 \cdot 10^{23}$\\
B) $6,02 \cdot 10^{22}$\\
C) $18,06 \cdot 10^{20}$\\
D) $36,12 \cdot 10^{22}$
  \item ${ }_{94}^{242} \mathrm{Pu} \rightarrow{ }_{90}^{230} \mathrm{Th}+\mathrm{x}_{2}^{4} \mathrm{He}+\mathrm{y}_{-1}^{0} \mathrm{e}$ u ushbu yadro reaksiyasida $72,6 \mathrm{~g}$ pulutoniy yemirilishidan hosil bo'lgan elektronlar sonini hisoblang.\\
A) $3,612 \cdot 10^{23}$\\
B) $6,02 \cdot 10^{23}$\\
C) $18,06 \cdot 10^{22}$\\
D) $36,12 \cdot 10^{23}$
  \item ${ }_{102}^{254} \mathrm{No} \rightarrow{ }_{96}^{238} \mathrm{Cm}+\mathrm{x}_{2}^{4} \mathrm{He}+\mathrm{y}_{-1}^{0} \mathrm{e}$ e ushbu yadro reaksiyasida $25,4 \mathrm{~g}$ nobeliy yemirilishidan hosil bo'lgan elektronlar sonini hisoblang.\\
A) $12,04 \cdot 10^{22}$\\
B) $6,02 \cdot 10^{22}$\\
C) $18,06 \cdot 10^{20}$\\
D) $36,12 \cdot 10^{22}$
  \item ${ }_{87}^{223} \mathrm{Fr} \rightarrow{ }_{81}^{187} \mathrm{Tl}+\mathrm{x}_{2}^{4} \mathrm{He}+\mathrm{y}_{-1}^{0}$ ē ushbu yadro reaksiyasida $44,6 \mathrm{~g}$ fransiy yemirílishidan hosil bo'lgan elektronlar sonini hisoblang.\\
A) $12,04 \cdot 10^{23}$\\
B) $6,02 \cdot 10^{22}$\\
C) $18,06 \cdot 10^{23}$\\
D) $14,448 \cdot 10^{23}$
  \item ${ }_{94}^{240} \mathrm{Pu} \rightarrow{ }_{90}^{224} \mathrm{Th}+\mathrm{x}_{2}^{4} \mathrm{He}+\mathrm{y}_{-1}^{0} \mathrm{e}$ ushbu yadro reaksiyasida 48 g pulutoniy yemirilishidan hosil bo'lgan elektronlar sonini hisoblang.\\
A) $4,816 \cdot 10^{23}$\\
B) $6,02 \cdot 10^{23}$\\
C) $7,224 \cdot 10^{22}$\\
D) $36,12 \cdot 10^{23}$
  \item ${ }_{102}^{250} \mathrm{No} \rightarrow{ }_{96}^{230} \mathrm{Cm}+\mathrm{x}_{2}^{4} \mathrm{He}+\mathrm{y}_{-1}^{0} \mathrm{e}$ e ushbu yadro reaksiyasida 75 g nobeliy yemirilishidan hosil bo'lgan elektronlar sonini hisoblang.\\
A) $9,632 \cdot 10^{23}$\\
B) $6,02 \cdot 10^{23}$\\
C) $7,224 \cdot 10^{23}$\\
D) $36,12 \cdot 10^{23}$
  \item ${ }_{87}^{228} \mathrm{Fr} \rightarrow{ }_{81}^{208} \mathrm{Tl}+\mathrm{x}_{2}^{4} \mathrm{He}+\mathrm{y}_{-1}^{0}$ ē ushbu yadro reaksiyasida $22,8 \mathrm{~g}$ fransiy yemirilishidan hosil bo'lgan elektronlar sonini hisoblang.\\
A) $2,408 \cdot 10^{23}$\\
B) $6,02 \cdot 10^{22}$\\
C) $6,02 \cdot 10^{23}$\\
D) $7,224 \cdot 10^{23}$
  \item ${ }_{94}^{248} \mathrm{Pu} \rightarrow{ }_{90}^{232} \mathrm{Th}+\mathrm{x}_{2}^{4} \mathrm{He}+\mathrm{y}_{-1}^{0}$ ē ushbu yadro reaksiyasida $74,4 \mathrm{~g}$ pulutoniy yemirilishidan hosil bo'lgan elektronlar sonini hisoblang.\\
A) $9,632 \cdot 10^{23}$\\
B) $6,02 \cdot 10^{23}$\\
C) $7,224 \cdot 10^{23}$\\
D) $36,12 \cdot 10^{23}$
  \item ${ }_{102}^{262} \mathrm{No} \rightarrow{ }_{96}^{246} \mathrm{Cm}+\mathrm{x}_{2}^{4} \mathrm{He}+\mathrm{y}_{-1}^{0} \mathrm{e}$ e ushbu yadro reaksiyasida $13,1 \mathrm{~g}$ nobeliy yemirilishidan hosil bo'lgan elektronlar sonini hisoblang.\\
A) $12,04 \cdot 10^{22}$\\
B) $6,02 \cdot 10^{22}$\\
C) $18,06 \cdot 10^{20}$\\
D) $36,12 \cdot 10^{22}$
  \item ${ }_{87}^{229} \mathrm{Fr} \rightarrow{ }_{81}^{205} \mathrm{Tl}+\mathrm{x}_{2}^{4} \mathrm{He}+\mathrm{y}_{-1}^{0}$ ē ushbu yadro reaksiyasida $22,9 \mathrm{~g}$ fransiy yemirilishidan hosil bo'lgan elektronlar sonini hisoblang.\\
A) $12,04 \cdot 10^{22}$\\
B) $6,02 \cdot 10^{22}$\\
C) $18,06 \cdot 10^{20}$\\
D) $36,12 \cdot 10^{22}$
  \item ${ }_{102}^{254} \mathrm{No} \rightarrow{ }_{96}^{238} \mathrm{Cm}+\mathrm{x}_{2}^{4} \mathrm{He}+\mathrm{y}_{-1}^{0}$ ē ushbu yadro reaksiyasida necha g nobeliy yemirilishidan $6,02 \cdot 10^{22}$ dona elektron hosil bo'ladi?\\
A) 12,7\\
B) 127\\
C) 25,4\\
D) 254
  \item ${ }_{94}^{242} \mathrm{Pu} \rightarrow{ }_{90}^{230} \mathrm{Th}+\mathrm{x}_{2}^{4} \mathrm{He}+\mathrm{y}_{-1}^{0} \mathrm{e}$ ushbu yadro reaksiyasida necha g pulutoniy yemirilishidan $36,12 \cdot 10^{23}$ dona elektron hosil bo'ladi?\\
A) 242\\
B) 24,2\\
C) 726\\
D) 72,6
  \item ${ }_{102}^{254} \mathrm{No} \rightarrow{ }_{96}^{238} \mathrm{Cm}+\mathrm{x}_{2}^{4} \mathrm{He}+\mathrm{y}_{-1}^{0}$ ē ushbu yadro reaksiyasida necha g nobeliy yemirilishidan $12,04 \cdot 10^{22}$ dona elektron hosil bo'ladi?\\
A) 12,7\\
B) 127\\
C) 25,4\\
D) 254
  \item ${ }_{87}^{223} \mathrm{Fr} \rightarrow{ }_{81}^{187} \mathrm{Tl}+\mathrm{x}_{2}^{4} \mathrm{He}+\mathrm{y}_{-1}^{0}$ ē ushbu yadro reaksiyasida necha $g$ fransiy\\
yemirilishidan $7,224 \cdot 10^{23}$ dona elektron hosil bo'ladi?\\
A) 22.3\\
B) 12,7\\
C) 46,6\\
D) 25,4
  \item ${ }_{94}^{240} \mathrm{Pu} \rightarrow{ }_{90}^{224} \mathrm{Th}+\mathrm{x}_{2}^{4} \mathrm{He}+\mathrm{y}_{-1}^{0}$ é ushbu yadro reaksiyasida necha g pulutoniy yemirilishidan $4,816 \cdot 10^{23}$ dona elektron hosil bo'ladi?\\
A) 24\\
B) 48\\
C) 96\\
D) 12
  \item ${ }_{102}^{250} \mathrm{No} \rightarrow{ }_{96}^{230} \mathrm{Cm}+\mathrm{x}_{2}^{4} \mathrm{He}+\mathrm{y}_{-1}^{0}$ ē ushbu yadro reaksiyasida necha $g$ nobelliy yemirilishidan $12,04 \cdot 10^{23}$ dona elektron hosil bo'ladi?\\
A) 124\\
B) 480\\
C) 196\\
D) 125
  \item ${ }_{87}^{228} \mathrm{Fr} \rightarrow{ }_{81}^{208} \mathrm{Tl}+\mathrm{x}_{2}^{4} \mathrm{He}+\mathrm{y}_{-1}^{0}$ ē ushbu yadro reaksiyasida necha g fransiy yemirilishidan $12,04 \cdot 10^{23}$ dona elektron hosil bo'ladi?\\
A) 57\\
B) 11,4\\
C) 114\\
D) 22,8
  \item ${ }_{94}^{248} \mathrm{Pu} \rightarrow{ }_{90}^{232} \mathrm{Th}+\mathrm{x}_{2}^{4} \mathrm{He}+\mathrm{y}_{-1}^{0}$ ē ushbu yadro reaksiyasida necha g pulutoniy yemirilishidan $2,408 \cdot 10^{23}$ dona elektron hosil bo'ladi?\\
A) 24,8\\
B) 12,4\\
C) 9,6\\
D) 12,8
  \item ${ }_{102}^{262} \mathrm{No} \rightarrow{ }_{96}^{246} \mathrm{Cm}+\mathrm{x}_{2}^{4} \mathrm{He}+\mathrm{y}_{-1}^{0}$ ē ushbu yadro reaksiyasida necha $g$ nobelliy yemirilishidan $6,02 \cdot 10^{23}$ dona elektron hosil bo'ladi?\\
A) 262\\
B) 124\\
C) 13,1\\
D) 131
  \item ${ }_{87}^{229} \mathrm{Fr} \rightarrow{ }_{81}^{205} \mathrm{Tl}+\mathrm{x}_{2}^{4} \mathrm{He}+\mathrm{y}_{\sim 1}^{0}$ ē ushbu yadro reaksiyasida necha g fransiy yemirilishidan $18,06 \cdot 10^{23}$ dona elektron hosil bo'ladi?\\
A) 57\\
B) 11,45\\
C) 114,5\\
D) 22,8
  \item ${ }_{91}^{227} \mathrm{~Pa} \rightarrow{ }_{86}^{A} \mathrm{Rn} \rightarrow+\mathrm{x}_{2}^{4} \mathrm{He}+\mathrm{y}_{-1}{ }_{-1}^{0}$ ē ushbu yadro reaksiyasi asosida 681 g protaktiniy yemirilishidan $54,18 \cdot 10^{23}$ ta elektron ajralgan bo'lsa, reaksiya natijasida hosil bo'lgan radon izotopining.nisbiy atom massasini hisoblang.\\
A) 210\\
B) 120\\
C) 211\\
D) 12
62. ${ }_{102}^{262} \mathrm{No} \rightarrow{ }_{96}^{\mathrm{A}} \mathrm{Cm}+\mathrm{x}_{2}^{4} \mathrm{He}+\mathrm{y}_{-1}^{0}$ ē ushbu yadro reaksiyasi asosida 131 g nobelliy yemirilishidan $6.02 \cdot 10^{23}$ ta elektron ajralgan bo'lsa, reaksiya natijasida hosil bo'lgan kyuriy izotopining nisbiy atom massasini hisoblang.\\
A) 246\\
B) 220\\
C) 241\\
D) 224\\
63. ${ }_{94}^{240} \mathrm{Pu} \rightarrow{ }_{90}^{A} \mathrm{Th}+\mathrm{x}_{2}^{4} \mathrm{He}+\mathrm{y}_{-1}^{0}$ é ushbu yadro reaksiyasi asosida 48 g pulutoniy yemirilishidan $4,816 \cdot 10^{23}$ ta elektron ajralgan bo'lsa, reaksiya natijasida hosil bo'lgan toriy izotopining nisbiy atom massasini hisoblang.\\
A) 246\\
B) 220\\
C) 241\\
D) 224\\
64. ${ }_{102}^{254} \mathrm{No} \rightarrow{ }_{96}^{A} \mathrm{Cm}+\mathrm{x}_{2}^{4} \mathrm{He}+\mathrm{y}_{-1}^{0}$ ē ushbu yadro reaksiyasi asosida $12,7 \mathrm{~g}$ nobelliy yemirilishidan $6,02 \cdot 10^{22}$ ta elektron ajralgan bo'lsa, reaksiya natijasida hosil bo'lgan kyuriy izotopining nisbiy atom massasini hisoblang.\\
A) 238\\
B) 246\\
C) 231\\
D) 324\\
65. ${ }_{94}^{242} \mathrm{Pu} \rightarrow{ }_{90}^{A} \mathrm{Th}+\mathrm{x}_{2}^{4} \mathrm{He}+\mathrm{y}_{-1}^{0}$ ēushbu yadro reaksiyasi asosida 726 g pulutoniy yemirilishidan $36,12 \cdot 10^{23}$ ta elektron ajralgan bo'lsa, reaksiya natijasida hosil bo'lgan toriy izotopining nisbiy atom massasini hisoblang.\\
A) 210\\
B) 230\\
C) 211\\
D) 224\\
66. ${ }_{87}^{228} \mathrm{Fr} \rightarrow{ }_{81}^{A} \mathrm{Tl}+\mathrm{x}_{2}^{4} \mathrm{He}+\mathrm{y}_{-1}^{0}$ ē ushbu yadro reaksiyasi asosida 114 g fransiy yemirilishidan $12,04 \cdot 10^{23}$ ta elektron ajralgan bo'lsa, reaksiya natijasida hosil bo'lgan talliy izotopining nisbiy atom massasini hisoblang.\\
A) 210\\
B) 220\\
C) 208\\
D) 204\\
67. ${ }_{87}^{223} \mathrm{Fr} \rightarrow{ }_{8}^{A} \mathrm{Tl}+\mathrm{x}_{2}^{4} \mathrm{He}+\mathrm{y}_{-1}^{0}$ ē ushbu yadro reaksiyasi asosida $23,3 \mathrm{~g}$ fransiy yemirilishidan $7,224 \cdot 10^{23}$ ta elektron ajralgan bo'lsa, reaksiya natijasida hosil bo'lgan talliy izotopining nisbiy atom massasini hisoblang.\\
A) 210\\
B) 187\\
C) 211\\
D) 124\\
  \item Cr elementining elektron konfiguratsiyasini aniqlang.\\
A) $\ldots 4 \mathrm{~s}^{1} 3 \mathrm{~d}^{5}$\\
B) $\ldots 4 \mathrm{~s}^{2} 3 \mathrm{~d}^{5}$\\
C) $\ldots 4 \mathrm{~s}^{2} 3 \mathrm{~d}^{4}$\\
D) $\ldots 4 \mathrm{~s}^{1} 3 \mathrm{~d}^{4}$
2. Ca elementining elektron konfiguratsiyasini aniqlang.\\
A) $\ldots 4 s^{1}$\\
B) $\ldots 3 \mathrm{~s}^{2}$\\
C) $\ldots 4 s^{2}$\\
D) $\ldots 3 \mathrm{~s}^{1}$\\
3. Ar elementining elektron konfiguratsiyasini aniqlang.\\
A) $\ldots 3 s^{2} 3 p^{5}$\\
B) $\ldots 3 \mathrm{~s}^{2} 3 \mathrm{p}^{6}$\\
C) $\ldots 4 \mathrm{~s}^{2} 3 \mathrm{~d}^{4}$\\
D).. $.3 s^{2} 3 p^{1}$\\
4. Mn elementining elektron konfiguratsiyasini aniqlang.\\
A) $\ldots 4 s^{1} 3 d^{5}$\\
B) $\ldots 4 \mathrm{~s}^{2} 3 \mathrm{~d}^{6}$\\
C) $\ldots 4 s^{2} 3 d^{4}$\\
D) $\ldots 4 s^{1} 3 d^{4}$\\
5. S elementining olektron konfiguratsiyasini aniqlang.\\
A) ... $3 s^{2} 3 p^{5}$\\
B) ... $3 \mathrm{~s}^{2} 3 \mathrm{p}^{6}$\\
C) $\ldots 4 s^{2} 3 d^{4}$\\
D) $\ldots 3 s^{3} 3 p^{4}$\\
6. P elementining elektron konfiguratsiyasini aniqlang.\\
A) ... $3 \mathrm{~s}^{2} 3 \mathrm{p}^{3}$\\
B) $\ldots 3 s^{2} 3 p^{5}$\\
C) $\ldots 4 s^{3} 3 d^{1}$\\
D) ... $3 s^{3} 3 p^{4}$\\
7. C elementining elektron konfiguratsiyasini aniqlang.\\
A) ... $2 \mathrm{~s}^{2} 2 \mathrm{p}^{3}$\\
B) ... $2 s^{2} 2 p^{2}$\\
C) $\ldots 4 \mathrm{~s}^{2} 3 \mathrm{~d}^{1}$\\
D) $\ldots 3 s^{2} 3 p^{6}$\\
8. Ne elementining elektron konfiguratsiyasini aniqlang.\\
A) ... $2 s^{2} 2 p^{4}$\\
B) $\ldots 2 \mathrm{~s}^{2} 2 \mathrm{p}^{5}$\\
C) $\ldots 4 s^{2} 3 d^{7}$\\
D) $\ldots 2 s^{2} 2 p^{6}$\\
9. Al elementining elektron konfiguratsiyasini aniqlang.\\
A) $\ldots 3 s^{2} 3 p^{1}$\\
B) ... $2 \mathrm{~s}^{2} 2 \mathrm{p}^{2}$\\
C) $\ldots 4 s^{2} 3 d^{1}$\\
D) $\ldots 3 s^{2} 3 p^{4}$\\
10. Mg elementining elektron konfiguratsiyasini aniqlang.\\
A) $\ldots 3 s^{2} 3 p^{1}$\\
B) $\ldots 2 \mathrm{~s}^{2}$\\
C) $\ldots 4 s^{2} 3 d^{1}$\\
D) $\ldots 3 \mathrm{~s}^{2}$
  \item Elektron konfiguratsiyasi ... $4 \mathrm{~s}^{2} 3 \mathrm{~d}^{5}$ bo'lgan elementni aniqlang.
A) Mn\\
B) S\\
C) Cr\\
D) Cu\\
12. Elektron konfiguratsiyasi ... $4 \mathrm{~s}^{1} 3 \mathrm{~d}^{5}$ bo'lgan elementni aniqlang.\\
A) Mn\\
B) S\\
C) Cr\\
D) Cu\\
13. Elektron konfiguratsiyasi ... $4 \mathrm{~s}^{2} 3 \mathrm{~d}^{7}$ bo'lgan elementni aniqlang.\\
A) Mo\\
B) Sn\\
C) Co\\
D) Cu\\
14. Elektron konfiguratsiyasi ... $3 \mathrm{~s}^{2} 3 \mathrm{p}^{5}$ bo'lgan elementni aniqlang.\\
A) Ne\\
B) Cl\\
C) C\\
D) Ar\\
15. Elektron konfiguratsiyasi ... $4 \mathrm{~s}^{2} 3 \mathrm{~d}^{2}$ bo'lgan elementni aniqlang.\\
A) Pb\\
B) Se\\
C) Cu\\
D) Ti\\
16. Elektron konfiguratsiyasi ... $4 \mathrm{~s}^{1} 3 \mathrm{~d}^{10}$ bo'lgan elementni aniqlang.\\
A) Mn\\
B) S\\
C) Cr\\
D) Cu\\
17. Elektron konfiguratsiyasi ... $4 \mathrm{~s}^{2}$ bo'lgan elementni aniqlang.\\
A) Mg\\
B) Na\\
C) Ca\\
D) C\\
18. Elektron konfiguratsiyasi ... $3 \mathrm{~s}^{2}$ bo'lgan elementni aniqlang.\\
A) Mg\\
B) Na\\
C) Ca\\
D) C\\
19. Elektron konfiguratsiyasi ...3s ${ }^{1}$ bo'lgan elementni aniqlang.\\
A) Mg\\
B) Na\\
C) Ca\\
D) C\\
20. Elektron konfiguratsiyasi ... $4 \mathrm{~s}^{2} 3 \mathrm{~d}^{6}$ bo'lgan elementni aniqlang.\\
A) Mn\\
B) Zn\\
C) Cr\\
D) Fe
  \item $\mathrm{Ca}^{+2}$ ionining elektron konfiguratsiyasini aniqlang\\
A) $\ldots 3 s^{2} 3 p^{5}$\\
B) $\ldots 3 s^{2} 3 p^{6}$\\
C) $\ldots 4 \mathrm{~s}^{2} 3 \mathrm{~d}^{4}$\\
D) $\ldots 3 \mathrm{~s}^{2} 3 \mathrm{p}^{4}$\\
  \item $\mathrm{S}^{-2}$ ionining elektron konfiguratsiyasini aniqlang\\
A) $\ldots 3 s^{2} 3 p^{5}$\\
B) $\ldots 3 s^{2} 3 p^{6}$\\
C) $\ldots 4 \mathrm{~s}^{2} 3 \mathrm{~d}^{4}$\\
D) $\ldots 3 s^{2} 3 p{ }^{4}$
  \item $\mathrm{Mg}^{+2}$ ionining elektron konfiguratsiyasini aniqlang\\
A) $\ldots 2 s^{2} 2 p^{5}$\\
B) $\ldots 3 s^{2} 3 p^{6}$\\
C) $\ldots 4 \mathrm{~s}^{2} 3 \mathrm{~d}^{4}$\\
D) $\ldots 2 \mathrm{~s}^{2} 2 \mathrm{p}^{6}$
  \item $\mathrm{Al}^{+3}$ ionining elektron konfiguratsiyasini aniqlang\\
A) $\ldots 2 s^{2} 3 p^{6}$\\
B) $\ldots 3 s^{2} 3 p^{6}$\\
C) $\ldots 4 \mathrm{~s}^{2} 3 \mathrm{~d}^{4}$\\
D) $\ldots 3 s^{2} 3 p{ }^{4}$
  \item $\mathrm{Cl}^{-}$ionining elektron konfiguratsiyasini aniqlang\\
A) $\ldots 3 s^{2} 3 p^{5}$\\
B) $\ldots 3 s^{2} 3 p^{6}$\\
C) $\ldots 4 \mathrm{~s}^{2} 3 \mathrm{~d}^{4}$\\
D) $\ldots 3 s^{2} 3 p^{4}$
  \item $\mathrm{O}^{+2}$ ionining elektron konfiguratsiyasini aniqlang\\
A) $\ldots 3 s^{2} 3 p^{2}$\\
B) $\ldots 3 s^{2} 3 p^{3}$\\
C) $\ldots 4 s^{2} 3 d^{4}$\\
D).. $.2 s^{2} 2 p^{2}$
  \item F- ionining elektron konfiguratsiyasini aniqlang\\
A) $\ldots 2 s^{2} 2 p^{6}$\\
B) $\ldots 3 \mathrm{~s}^{2} 3 \mathrm{p}^{6}$\\
C) $\ldots 4 s^{2} 3 d^{4}$\\
D) $\ldots 3 \mathrm{~s}^{2} 3 \mathrm{p}^{4}$
  \item $\mathrm{Na}^{+}$ionining elektron konfiguratsiyasini aniqlang\\
A) $\ldots 2 s^{2} 2 p^{6}$\\
B) $\ldots 3 s^{2} 3 p^{6}$\\
C) $\ldots 4 s^{2} 3 d^{4}$\\
D) $\ldots 3 \mathrm{~s}^{2} 3 \mathrm{p}^{4}$
  \item $\mathrm{Cr}^{+2}$ ionining elektron konfiguratsiyasini aniqlang\\
A) $\ldots 4 s^{1} 3 d^{3}$\\
B) $\ldots 4 \mathrm{~s}^{0} 3 \mathrm{~d}^{5}$\\
C) $\ldots 4 s^{0} 3 d^{4}$\\
D) $\ldots 3 s^{2} 3 p^{4}$
  \item $\mathrm{C}^{+2}$ ionining elektron konfiguratsiyasini aniqlang\\
A) $\ldots 2 \mathrm{~s}^{2}$\\
B) $\ldots 3 \mathrm{~s}^{2}$\\
C) $\ldots 4 s^{2}$\\
D) $\ldots 3 p^{2}$
  \item $\mathrm{MnO}_{2}$ tarkibidagi Mn ning elektron konfiguratsiyasini aniqlang.\\
A) $\ldots 4 s^{2} 3 d^{1}$\\
B) $\ldots 4 \mathrm{~s}^{2} 3 \mathrm{~d}^{6}$\\
C) $\ldots 4 s^{0} 3 d^{3}$\\
D) $\ldots 3 s^{2} 3 p^{4}$\\
  \item $\mathrm{CO}_{2}$ tarkibidagi C ning elektron konfiguratsiyasini aniqlang.\\
A) $\ldots 4 \mathrm{~s}^{0} 3 \mathrm{~d}^{1}$\\
B) $\ldots 2 \mathrm{~s}^{0} 2 \mathrm{p}^{0}$\\
C) $\ldots 4 \mathrm{~s}^{0} 3 \mathrm{~d}^{3}$\\
D) $\ldots 3 \mathrm{~s}^{0} 3 \mathrm{p}^{1}$
  \item $\mathrm{N}_{2} \mathrm{O}_{5}$ tarkibidagi N ning elektron konfiguratsiyasini aniqlang.\\
A) $\ldots 4 s^{0} 3 d^{1}$\\
B) $\ldots 2 \mathrm{~s}^{0} 2 \mathrm{p}^{0}$\\
C) $\ldots 4 \mathrm{~s}^{0} 3 \mathrm{~d}^{3}$\\
D) $\ldots 3 \mathrm{~s}^{0} 3 \mathrm{p}^{1}$
  \item $\mathrm{HMnO}_{4}$ tarkibidagi Mn ning elektron konfiguratsiyasini aniqlang.\\
A) $\ldots 4 \mathrm{~s}^{2} 3 \mathrm{~d}^{1}$\\
B) $\ldots 4 \mathrm{~s}^{2} 3 \mathrm{~d}^{6}$\\
C) $\ldots 4 s^{0} 3 d^{0}$\\
D) $\ldots 3 \mathrm{~s}^{2} 3 \mathrm{p}^{4}$
  \item $\mathrm{H}_{2} \mathrm{CO}_{3}$ tarkibidagi C ning elektron konfiguratsiyasini aniqlang.\\
A) $\ldots 4 s^{0} 3 d^{1}$\\
B) $\ldots 2 \mathrm{~s}^{0} 2 \mathrm{p}^{0}$\\
C) $\ldots 4 \mathrm{~s}^{0} 3 \mathrm{~d}^{3}$\\
D) $\ldots 3 \mathrm{~s}^{0} 3 \mathrm{p}^{1}$
  \item $\mathrm{H}_{2} \mathrm{O}$ tarkibidagi O ning elektron konfiguratsiyasini aniqlang.\\
A) $\ldots 2 \mathrm{~s}^{0} 2 \mathrm{p}^{0}$\\
B) $\ldots 4 \mathrm{~s}^{2} 3 \mathrm{~d}^{6}$\\
C) $\ldots 3 s^{0} 3 p^{1}$\\
D) $\ldots 2 \mathrm{~s}^{2} 2 \mathrm{p}^{6}$
  \item $\mathrm{Cr}_{2} \mathrm{O}_{3}$ tarkibidagi Cr ning elektron konfiguratsiyasini aniqlang.\\
A) $\ldots 4 \mathrm{~s}^{2} 3 \mathrm{~d}^{1}$\\
B) $\ldots 4 \mathrm{~s}^{2} 3 \mathrm{~d}^{6}$\\
C) $\ldots 4 \mathrm{~s}^{0} 3 \mathrm{~d}^{3}$\\
D) $\ldots 3 s^{2} 3 p^{4}$
  \item $\mathrm{Fe}_{2} \mathrm{O}_{3}$ tarkibidagi Fe ning elektron konfiguratsiyasini aniqlang.\\
A) $\ldots 4 s^{2} 3 d^{1}$\\
B) $\ldots 4 s^{2} 3 d^{6}$\\
C) $\ldots 4 s^{0} 3 d^{5}$\\
D) $\ldots 3 s^{2} 3 p^{4}$
  \item $\mathrm{NH}_{3}$ tarkibidagi N ning elektron konfiguratsiyasini aniqlang.\\
A) $\ldots 2 \mathrm{~s}^{2} 2 \mathrm{p}^{4}$\\
B) $\ldots 2 \mathrm{~s}^{2} 2 \mathrm{p}^{5}$\\
C) $\ldots 2 \mathrm{~s}^{2} 2 \mathrm{p}^{2}$\\
D) $\ldots 2 \mathrm{~s}^{2} 2 \mathrm{p}^{6}$
  \item MnO tarkibidagi Mn ning elektron konfiguratsiyasini aniqlang.\\
A) $\ldots 4 \mathrm{~s}^{2} 3 \mathrm{~d}^{1}$\\
B) $\ldots 4 \mathrm{~s}^{2} 3 \mathrm{~d}^{6}$\\
C) $\ldots 4 s^{0} 3 d^{5}$\\
D) $\ldots 3 s^{2} 3 p^{4}$
  \item Cr elementida nechta s elektron bor?\\
A) 7\\
B) 6\\
C) 8\\
D) 9
  \item Ca elementida nechta p elektron bor?\\
A) 9\\
B) 10\\
C) 12\\
D) 7
  \item Cu elementida nechta d elektron bor?\\
A) 9\\
B) 10\\
C) 12\\
D) 7
  \item C elementida nechta s elektron bor?\\
A) 3\\
B) 2\\
C) 4\\
D) 5
  \item Si elementida nechta p elektron bor?\\
A) 7\\
B) 6\\
C) 8\\
D) 9
  \item Al elementida nechta s elektron bor?\\
A) 7\\
B) 6\\
C) 8\\
D) 9
  \item Fe elementida nechta s elektron bor?\\
A) 5\\
B) 6\\
C) 7\\
D) 8
  \item Ar elementida nechta s elektron bor?\\
A) 7\\
B) 6\\
C) 8\\
D) 9
  \item W elementida nechta s elektron bor?\\
A) 12\\
B) 16\\
C) 8\\
D) 9
  \item Sr elementida nechta s elektron bor?\\
A) 7\\
B) 8\\
C) 9\\
D) 10
  \item Ca elementida nechta elektron bilan to'lgan qavat va qavatcha mavjud?\\
A) $2: 6$\\
B) $3 ; 8$\\
C) $4 ; 5$\\
D) $2 ; 8$\\
  \item Fe elementida nechta elektron bilan to'lgan qavat va qavatcha mavjud?\\
A) $2 ; 6$\\
B) $3 ; 8$\\
C) $4 ; 5$\\
D) $2 ; 8$
  \item S elementida nechta elektron bilan to'lgan qavat va qavatcha mavjud?\\
A) $2 ; 6$\\
B) $3 ; 8$\\
C) $2 ; 5$\\
D) $2 ; 4$
  \item Br elementida nechta elektron bilan to'lgan qavat va qavatcha mavjud?\\
A) $3 ; 6$\\
B) $3 ; 7$\\
C) $4 ; 5$\\
D) $2 ; 8$
  \item Cl elementida nechta elektron bilan to'lgan qavat va qavatcha mavjud?\\
A) $2 ; 6$\\
B) $3 ; 8$\\
C) $2 ; 5$\\
D) $2 ; 4$
  \item N elementida nechta elektron bilan to'lgan qavat va qavatcha mavjud?\\
A) $1 ; 2$\\
B) $1 ; 3$\\
C) $1 ; 4$\\
D) $1 ; 8$
  \item Si elementida nechta elektron bilan to'lgan qavat va qavatcha mavjud?\\
A) $2 ; 6$\\
B) $3 ; 8$\\
C) $2 ; 5$\\
D) $2 ; 4$
  \item Cr elementida nechta elektron bilan to'lgan qavat va qavatcha mavjud?\\
A) $2 ; 6$\\
B) $3 ; 8$\\
C) $2 ; 5$\\
D) $2 ; 4$
  \item Ar elementida nechta elektron bilan to'lgan qavat va qavatcha mavjud?\\
A) $2 ; 6$\\
B) $3 ; 8$\\
C) $2 ; 5$\\
D) $2 ; 4$
  \item Ne elementida nechta elektron bilan to'lgan qavat va qavatcha mavjud?\\
A) $2 ; 2$\\
B) $2 ; 3$\\
C) $1 ; 4$\\
D) $1 ; 8$
  \item Cr elementida nechta toq elektron bor?\\
A) 7\\
B) 6\\
C) 8\\
D) 9\\
  \item Mn elementida nechta toq elektron bor?\\
A) 5\\
B) 10\\
C) 2\\
D) 7
  \item Cu elementida nechta toq elektron bor?\\
A) 4\\
B) 3\\
C) 2\\
D) 1
  \item C elementida nechta toq elektron bor?\\
A) 3\\
B) 2\\
C) 4\\
D) 5
  \item Si elementida nechta toq elektron bor?\\
A) 3\\
B) 2\\
C) 4\\
D) 5
  \item Al elementida nechta toq elektron bor?\\
A) 4\\
B) 3\\
C) 2\\
D) 1
  \item Fe elementida nechta toq elektron bor?\\
A) 4\\
B) 3\\
C) 2\\
D) 1
  \item Ar elementida nechta toq elektron bor?\\
A) 7\\
B) 1\\
C) 0\\
D) 9
  \item W elementida nechta toq elektron bor?\\
A) 4\\
B) 3\\
C) 2\\
D) 1
  \item Sr elementida nechta toq elektron bor?\\
A) 7\\
B) 1\\
C) 0\\
D) 9
  \item Mg atomining valent elektronlarini magnit kvanť sonini aniqlang.\\
A) 7\\
B) 1\\
C) 0\\
D) 9
72. C atomining valent elektronlarini magnit kvant sonini aniqlang.\\
A) $0,-1,0$\\
B) $1,0,2$\\
C) $0,1,2$\\
D) $2,0,-1$\\
73. Al atomining valent elektronlarini bosh kvant sonini aniqlang.\\
A) $3,3,3$\\
B) $2,2,3$\\
C) $1,1,1$\\
D) $2,2,2$\\
74. Li atomining valent elektronlarini spin kvant sonini aniqlang.\\
A) $+1 / 2,-1 / 2$\\
B) $+1 / 2$\\
C) $-1 / 2,+1 / 2$\\
D) $-1 / 2$\\
75. In atomining valent elektronlarini magnit kvant sonini aniqlang.\\
A) $1,-1$\\
B) 1,0\\
C) 0,0\\
D) $0, \cdot 1$\\
76. Cs atomining valent elektronlarini bosh kvant sonini aniqlang.\\
A) 6\\
B) 1\\
C) 4\\
D) 3\\
77. Mg atomining valent elektronlarini bosh kvant sonini aniqlang.\\
A) 2,2\\
B) 1, 1\\
C) 0,0\\
D) 3,3\\
78. N atomining valent elektronlarining spin kvant sonini aniqlang.\\
A) $+1 / 2,-1 / 2,+1 / 2,+1 / 2+1 / 2$\\
B) $+1 / 2,-1 / 2,-1 / 2,+1 / 2+1 / 2$\\
C) $+1 / 2,-1 / 2,+1 / 2,+1 / 2-1 / 2$\\
D) $+1 / 2,-1 / 2,-1 / 2,-1 / 2+1 / 2$\\
79. B atomining valent elektronlarining magnit kvant sonini aniqlang.\\
A) $1,-1$\\
B) 1,0\\
C) 0,0\\
D) $0,-1$\\
80. Na atomining valent elektronlarining magnit kvant sonini aniqlang.\\
A) 6\\
B) 1\\
C) 4\\
D) 0
  \item Quyida berilgan kvant sonlar qaysi element atomiga tegishli?\\
A) Fe\\
B) Co\\
C) Ni\\
D) Cr\\
82. Quyida berilgan kvant sonlar qaysi element atomiga tegishli?\\
$\mathrm{n}=4 ; \mathrm{l}=3 ; \mathrm{m}_{\mathrm{I}}=0 ; \mathrm{m}_{\mathrm{S}}=+1 / 2$\\
A) $U$\\
B) Pm\\
C) Cm\\
D) Pu\\
83. Quyida berilgan kvant sonlar qaysi element atomiga tegishli?\\
$\mathrm{n}=5 ; \mathrm{l}=0 ; \mathrm{m}_{1}=0 ; \mathrm{ms}=+1 / 2$.\\
A) Ba\\
B) Cs\\
C) Rb\\
D) Pb\\
84. Quyida berilgan kvant sonlar qaysi element atomiga tegishli?\\
$\mathrm{n}=6 ; \mathrm{l}=0 ; \mathrm{m}_{1}=0 ; \mathrm{m}_{\mathrm{S}}=-1 / 2$\\
A) Cs\\
B) Rb\\
C) Ra\\
D) Ba\\
85. Quyida berilgan kvant sonlar qaysi element atomiga tegishli?\\
$\mathrm{n}=3 ; \mathrm{l}=1 ; \mathrm{m}_{1}=-1 ; \mathrm{ms}=-1 / 2$\\
A) S\\
B) P\\
C) Cl\\
D) O\\
86. Quyida berilgan kvant sonlar qaysi element atomiga tegishli?\\
$\mathrm{n}=3 ; \mathrm{l}=1 ; \mathrm{m}_{1}=+1 ; \mathrm{ms}_{\mathrm{S}}=-1 / 2$\\
A) O\\
B) Ar\\
C) S\\
D) Br\\
87. Quyida berilgan kvant sonlar qaysi element atomiga tegishli?\\
$\mathrm{n}=2 ; \mathrm{l}=1 ; \mathrm{m}_{1}=+1 ; \mathrm{m}_{\mathrm{S}}=+1 / 2$\\
A) Ar\\
B) Ne\\
C) N\\
D) B\\
88. Quyida berilgan kvant sonlari qaysi element atomiga tegishli?\\
$\mathrm{n}=4 ; \mathrm{l}=1 ; \mathrm{m}_{\mathrm{l}}=0 ; \mathrm{m}_{\mathrm{S}}=-1 / 2$\\
A) Br\\
B) Ge\\
C) Ga\\
D) S\\
89. Quyida berilgan kvant sonlar qaysi element atomiga tegishli?\\
$\mathrm{n}=3 ; \mathrm{l}=2 ; \mathrm{m}_{\mathrm{l}}=0 ; \mathrm{m}_{\mathrm{S}}=+1 / 2$\\
A) Sc\\
B) V\\
C) Ni\\
D) Zn\\
90. Quyida berilgan kvant sonlar qaysi element atomiga tegishli?\\
$\mathrm{n}=4 ; \mathrm{l}=2 ; \mathrm{m}_{1}=0 ; \mathrm{m}_{\mathrm{S}}=-1 / 2$\\
A) Rh\\
B) Mo\\
C) Pd\\
D) W
  \item Formulasi $4 \mathrm{f}^{8}$ bo'lgan elektronning kvant sonlarini aniqlang.
A) $\mathrm{n}=4 ; \mathrm{l}=3 ; \mathrm{m}_{1}=-3 ; \mathrm{ms}_{\mathrm{s}}=-1 / 2$\\
B) $\mathrm{n}=4 ; \mathrm{l}=3 ; \mathrm{m}_{\mathrm{l}}=+2 ; \mathrm{ms}=-1 / 2$\\
C) $\mathrm{n}=4 ; \mathrm{l}=3 ; \mathrm{m}_{\mathrm{l}}=-2 ; \mathrm{ms}_{\mathrm{s}}=+1 / 2$\\
D) $\mathrm{n}=4 ; \mathrm{l}=3 ; \mathrm{m}_{\mathrm{l}}=-2 ; \mathrm{m}_{\mathrm{S}}=-1 / 2$\\
92. Formulasi $4 \mathrm{p}^{4}$ bo'lgan elektronning kvant sonlarini aniqlang.\\
A) $\mathrm{n}=4 ; \mathrm{l}=1 ; \mathrm{m}_{\mathrm{l}}=+1 ; \mathrm{ms}^{=-1 / 2}$\\
B) $\mathrm{n}=4 ; \mathrm{l}=1 ; \mathrm{m}_{\mathrm{l}}=-1 ; \mathrm{m}_{\mathrm{S}}=-1 / 2$\\
C) $\mathrm{n}=4 ; \mathrm{l}=1 ; \mathrm{m}_{\mathrm{l}}=-1 ; \mathrm{m}_{\mathrm{S}}=+1 / 2$\\
D) $\mathrm{n}=4 ; \mathrm{l}=1 ; \mathrm{m}_{\mathrm{l}}=+1 ; \mathrm{ms}_{\mathrm{S}}=+1 / 2$\\
93. Formulasi $4 p^{2}$ bo'lgan elektronning kvant sonlarini aniqlang.\\
A) $\mathrm{n}=4 ; \mathrm{l}=1 ; \mathrm{m}_{\mathrm{l}}=0 ; \mathrm{m}_{\mathrm{S}}=-1 / 2$\\
B) $\mathrm{n}=4 ; \mathrm{l}=1 ; \mathrm{m}_{1}=-1 ; \mathrm{m}_{\mathrm{S}}=+1 / 2$\\
C) $\mathrm{n}=4 ; \mathrm{l}=1 ; \mathrm{m}_{1}=0 ; \mathrm{m}_{\mathrm{S}}=+1 / 2$\\
D) $\mathrm{n}=4 ; \mathrm{l}=1 ; \mathrm{m}_{\mathrm{l}}=+1 ; \mathrm{m}_{\mathrm{S}}=+1 / 2$\\
94. Formulasi $4 \mathrm{f}^{9}$ bo'lgan elektronning kvant sonlarini aniqlang.\\
A) $\mathrm{n}=4 ; \mathrm{l}=3 ; \mathrm{m}_{\mathrm{l}}=-2 ; \mathrm{m}_{\mathrm{S}}=-1 / 2$\\
B) $\mathrm{n}=4 ; \mathrm{l}=3 ; \mathrm{m}_{\mathrm{l}}=+2 ; \mathrm{ms}_{\mathrm{s}}=+1 / 2$\\
C) $\mathrm{n}=4 ; \mathrm{l}=3 ; \mathrm{m}_{\mathrm{l}}=-1 ; \dot{\mathrm{m}}_{\mathrm{S}}=+1 / 2$\\
D) $\mathrm{n}=4 ; \mathrm{l}=3 ; \mathrm{m}_{\mathrm{l}}=+1 ; \mathrm{m}_{\mathrm{S}}=-1 / 2$\\
95. Formulasi $4 f^{4}$ bo'lgan elektronning kvant sonlarini aniqlang.\\
A) $\mathrm{n}=4 ; \mathrm{l}=3 ; \mathrm{m}_{\mathrm{l}}=0 ; \mathrm{m}_{\mathrm{S}}=-1 / 2$\\
B) $\mathrm{n}=4 ; \mathrm{l}=3 ; \mathrm{m}_{\mathrm{l}}=-1 ; \mathrm{m}_{\mathrm{S}}=-1 / 2$\\
C) $\mathrm{n}=4 ; \mathrm{l}=3 ; \mathrm{m}_{\mathrm{l}}=0 ; \mathrm{m}_{\mathrm{s}}=+1 / 2$\\
D) $\mathrm{n}=4 ; \mathrm{l}=3 ; \mathrm{m}_{\mathrm{l}}=-1 ; \mathrm{m}_{\mathrm{S}}=+1 / 2$\\
96. Formulasi $4 \mathrm{~d}^{3}$ bo'lgan elektronning kvant sonlarini aniqlang.\\
A) $\mathrm{n}=4 ; \mathrm{l}=2 ; \mathrm{m}_{\mathrm{l}}=0 ; \mathrm{m}_{\mathrm{S}}=-1 / 2$\\
B) $\mathrm{n}=4 ; \mathrm{l}=2 ; \mathrm{m}_{\mathrm{l}}=0 ; \mathrm{m}_{\mathrm{S}}=+1 / 2$\\
C) $\mathrm{n}=4 ; \mathrm{l}=2 ; \mathrm{m}_{\mathrm{l}}=-1 ; \mathrm{m}_{\mathrm{S}}=-1 / 2$\\
D) $\mathrm{n}=4 ; \mathrm{l}=2 ; \mathrm{m}_{\mathrm{l}}=-1 ; \mathrm{ms}_{\mathrm{s}}=+1 / 2$\\
97. Formulasi $4 \mathrm{~d}^{5}$ bo'lgan elektronning kvant\\
sonlarini aniqlang.\\
A) $\mathrm{n}=4 ; \mathrm{l}=2 ; \mathrm{m}_{\mathrm{l}}=-2 ; \mathrm{m}_{\mathrm{S}}=-1 / 2$\\
B) $\mathrm{n}=4 ; \mathrm{l}=2 ; \mathrm{m}_{\mathrm{l}}=+2 ; \mathrm{m}_{\mathrm{S}}=-1 / 2$\\
C) $\mathrm{n}=4 ; \mathrm{l}=2 ; \mathrm{m}_{\mathrm{l}}=+2 ; \mathrm{m}_{\mathrm{s}}=+1 / 2$\\
D) $\mathrm{n}=4 ; \mathrm{l}=2 ; \mathrm{m}_{\mathrm{l}}=-2 ; \mathrm{ms}_{\mathrm{s}}=+1 / 2$\\
98. Formulasi $4 f^{7}$ bo'lgan elektronning kvant sonlarini aniqlang.\\
A) $\mathrm{n}=4 ; \mathrm{l}=3 ; \mathrm{m}_{\mathrm{l}}=0 ; \mathrm{m}_{\mathrm{s}}=-1 / 2$\\
B) $\mathrm{n}=4 ; \mathrm{l}=3 ; \mathrm{m}_{\mathrm{l}}=-3 ; \mathrm{m}_{\mathrm{S}}=-1 / 2$\\
C) $\mathrm{n}=4 ; \mathrm{l}=3 ; \mathrm{m}_{\mathrm{l}}=-1 ; \mathrm{m}_{\mathrm{S}}=+1 / 2$\\
D) $\mathrm{n}=4 ; \mathrm{l}=3 ; \mathrm{m}_{1}=+3 ; \mathrm{m}_{\mathrm{S}}=+1 / 2$\\
99. Formulasi $3 s^{2}$ bo'lgan elektronning kvant sonlarini aniqlang.\\
A) $\mathrm{n}=3 ; \mathrm{l}=0 ; \mathrm{m}_{\mathrm{l}}=0 ; \mathrm{m}_{\mathrm{S}}=-1 / 2$\\
B) $\mathrm{n}=3 ; \mathrm{l}=0 ; \mathrm{m}_{\mathrm{l}}=0 ; \mathrm{m}_{\mathrm{S}}=+1 / 2$\\
C) $\mathrm{n}=3 ; \mathrm{l}=0 ; \mathrm{m}_{\mathrm{I}}=+1 ; \mathrm{m}_{\mathrm{S}}=+1 / 2$\\
D) $\mathrm{n}=3 ; \mathrm{l}=0 ; \mathrm{m}_{\mathrm{l}}=+1 ; \mathrm{m}_{\mathrm{s}}=-1 / 2$\\
100. Formulasi $4 \mathrm{~s}^{1}$ bo'lgan elektronning kvant sonlarini aniqlang.\\
A) $\mathrm{n}=4 ; \mathrm{l}=0 ; \mathrm{m}_{\mathrm{l}}=0 ; \mathrm{m}_{\mathrm{S}}=-1 / 2$\\
B) $\mathrm{n}=4 ; \mathrm{l}=0 ; \mathrm{m}_{1}=0 ; \mathrm{m}_{\mathrm{S}}=+1 / 2$\\
C) $\mathrm{n}=4 ; \mathrm{l}=0 ; \mathrm{m}_{\mathrm{l}}=+1 ; \mathrm{m}_{\mathrm{S}}=-1 / 2$\\
D) $\mathrm{n}=4 ; \mathrm{l}=0 ; \mathrm{m}_{\mathrm{l}}=+1 ; \mathrm{ms}_{\mathrm{s}}=+1 / 2$